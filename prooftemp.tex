\SetTitle{23}{The Deutsch--Jozsa problem}{Demonstrates asymptotic quantum advantage}{23}

\begin{frame}{test}

\begin{ScrollingProof}
\Next{\Three}{hi \alert<\Curr>{there}}
\Next{\Two}{and now}
\Next{\One}{for something different}
\end{ScrollingProof}
    
\end{frame}

{
\def\V{\ColorTwo{\ensuremath{v}}}
\def\W{\ColorOne{\ensuremath{w}}}
\def\Reason#1#2{\only<#1>{\textit{#2}}}
\begin{frame}{Proof}{\TensSupProd{\Hadamard}{n} on states other than \TensSupProd{0}{n}}

\Vskip{-3em}\begin{theorem}
Given a basis state $\ket{\W}, \AllBits{\W}{n}$
\[
P(n): \TensSupProd{\Hadamard}{n}\ket{\W} = \RootTwoN{n}\SumBV{\V}{n} \NegOneExp{\DotP{\V}{\W}} \ket{\V}
\]
\end{theorem}
\begin{itemize}[<+->]
    \item Given any basis state \W, its Hadamard transform is the uniform superposition of all basis states of $n$~qubits.
    \item The coefficent of each state~\V{} is $\pm 1$, determined by its dot product with~\W.
    \item If we change any bit of \W, then at least one term in the sum will flip its coefficient.  The signature of $\pm 1$ on each term \ket{\V} is thus unique for a given~\W{}. This is true also because \TensSupProd{\Hadamard}{n} is its own inverse.
    \item Our proof proceeds by induction on $n$.
\end{itemize}
    
\end{frame}

\begin{frame}{Proof}{Base case}

When $n=0$
\begin{align*}
    \TensSupProd{\Hadamard}{0}\ket{\W} &= \RootTwoN{0}\SumBV{\V}{0} \NegOneExp{\DotP{\V}{\W}} \ket{\V} \\[1em]
    &= \frac{\NegOneExp{{0}}}{1} \\[1em]
    &= 1
\end{align*}
which is the \href{https://en.wikipedia.org/wiki/Hadamard_transform}{definition} of \TensSupProd{\Hadamard}{0}.
\end{frame}

\begin{frame}{Proof}{Induction step}
Sketch to show \Implies{P(k)}{P(k+1)}:
\begin{itemize}
    \item We will assume $P(k)$, which tells us how to compute \TensSupProd{\Hadamard}{k}\ket{\W} when \W{} is $k$~qubits long.
    \item We then consider the tensor product of \ket{\W} with a single-qubit state~\QState{}.
    \item Because we deal only with basis states, we must have exactly one of the following:
    \begin{itemize}
        \item $\QState{}=\QZero{}$
        \item $\QState{}=\QOne{}$
    \end{itemize}
    \item By cases we show that the formula correctly computes \TensSupProd{\Hadamard}{k+1}.
\end{itemize}

\end{frame}

\begin{frame}{Induction step}

\vrule width 0pt depth 1em
\Reason{2}{Induction hypothesis}
\Reason{3}{Substitution}
\Reason{4-7}{\Implies{\QState{} = \QZero{}}{\ColorTwo{\Hadamard\QState{}}=\ColorThree{\QPlus{}}}}
\Reason{8-11}{\Implies{\QState{} = \QOne{}}{\Hadamard\QState{}=\QMinus{}}}
\Vskip{-1em}\begin{align*}
\only<1-3>{%
    P(k)\longrightarrow \visible<2->{\ColorOne{\sqrt{2^n}\  \TensSupProd{\Hadamard}{k}\ket{w}} &= \ColorOne{\SumBV{v}{k} \NegOneExp{\DotP{v}{w}} \ket{v}} \\}}
    \only<3-4>{%
    \visible<3->{\ColorOne{\sqrt{2^n}}\ \TensProd{\ColorOne{\TensSupProd{\Hadamard}{k}\ket{w}}}{\ColorTwo{\Hadamard\QState{}}} &= \TensProd{\ColorOne{\SumBV{v}{k} \NegOneExp{\DotP{v}{w}} \ket{v}}}{\ColorTwo{\Hadamard\QState{}}} \\}}
\only<4->{%
    \ColorThree{\sqrt{2}}\ColorOne{\sqrt{2^n}}\ \TensProd{\ColorOne{\TensSupProd{\Hadamard}{k}\ket{w}}}{\ColorTwo{\Hadamard\QState{}}} &= \TensProd{\ColorOne{\SumBV{v}{k} \NegOneExp{\DotP{v}{w}} \ket{v}}}{\ColorThree{(\QZero{}+\QOne{})}} \\}
\only<5->{%
    &= \TensProd{\ColorOne{\SumBV{v}{k} \NegOneExp{\DotP{v}{w}} \ket{v}}}{\ColorThree{(\NegOneExp{\DotP{0}{0}}\QZero{}+\NegOneExp{\DotP{1}{0}}\QOne{})}} \\
    &= \ColorOne{\SumBV{v}{k} \NegOneExp{\DotP{v}{w}}\ColorThree{\NegOneExp{\DotP{0}{0}}} \ket{v}}\ColorThree{\ket{0}} 
    \\ & +
    \ColorOne{\SumBV{v}{k} \NegOneExp{\DotP{v}{w}} \ColorThree{\NegOneExp{\DotP{1}{0}}} \ket{v}}\ColorThree{\ket{1}} \\
   \ColorOne{\sqrt{2^{n\ColorThree{+1}}}}\ \ColorOne{\TensSupProd{\Hadamard}{k\ColorThree{+1}}\ket{w}\ColorTwo{\QZero{}}} &= \ColorOne{\SumBV{v}{k\ColorThree{+1}} \NegOneExp{\DotP{v}{w\ColorTwo{0}}} \ket{v}}
    }
\end{align*}
    
\end{frame}
}
