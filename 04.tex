\SetTitle{4}{The physics}{Becoming a believer}{04}

\section{Overview}
\begin{frame}{Overview}{What will we study here?}
\begin{itemize}
    \item We have some experience with polarized light, so we begin by investigating some theories about how a succession of polarizing filters behaves.  This will serve as our introduction to \href{https://en.wikipedia.org/wiki/Quantum_superposition}{superposition}.
    \item We must look at the classic \href{https://en.wikipedia.org/wiki/Double-slit_experiment}{double-slit} experiment which involves both superposition and \href{https://en.wikipedia.org/wiki/Wave_interference}{wave interference}.
    \item We return to polarized light by playing some \href{https://lab.quantumflytrap.com/lab}{quantum games} that you did for fun since last time. \alert{As time allows, I'll ask you to present your solutions.}
\end{itemize}
    
\end{frame}
\section{Polarized light}

\begin{frame}{Polarized light}{Experiments you can do with commonly found polarizing filters}
\begin{itemize}
    \item The experiments described here can be carried out using \href{https://en.wikipedia.org/wiki/Polarizer}{polarizers}.
    \item Examples include
    \begin{description}
        \item[sunglasses] These are polarized to diminish reflected light from surfaces such as water.  Such light is typically polarized horizontally, so these sunglasses admit only vertically polarized light.
        \item[computer screen]  If you are watching these slides on an \href{https://en.wikipedia.org/wiki/Liquid-crystal_display}{LCD computer screen}, the light reaching your eyeballs is (most likely) linearly polarized.
    \end{description}
    \item You can thus do these experiments with your computer screen and $2-3$ pairs of polarizing sunglasses.
    
\end{itemize}
\OnlyRemark{2}{In the slides that follow, light is emitted from the screen and heading toward you.    Each successive filter is placed in front of the previous one.}
\end{frame}

\begin{frame}{A bad theory for polarization}{Based on the classical world}
\TwoUnequalColumns{0.65\textwidth}{0.35\textwidth}{%
\Vskip{-3em}\begin{itemize}
    \item<1-> Light is a collection of waves, each due to one \href{https://en.wikipedia.org/wiki/Photon}{photon}.
    \item<2-> Unpolarized light might then be many photons, each with an arbitrary angle of polarization.
    \item<3-> What happens when the photons strike a filter as shown here?
    \item<4-> Only those waves polarized vertically, or nearly so, would pass through. 
\end{itemize}
}{%
\only<2->{%
\Vskip{-3em}\begin{center}
\begin{TIKZP}
\RadiantArrows{22.5}{->,color=BrickRed,thick}
\end{TIKZP}
\end{center}
}%
\only<3->{%
\begin{center}
\begin{TIKZP}[overlay]
\PFilter{-1}{0.0}{1}{2.5}{20}
\end{TIKZP}
\end{center}
}%
\only<4->{%
\begin{center}
\begin{TIKZP}[scale=0.5]

  \draw[<->,thick,color=BrickRed] (0,-1) -- (0,1);

\end{TIKZP}
\end{center}
}
}
\OnlyRemark{5}{If this is true, we should see a very small fraction of the light passing through. But we see $\frac{1}{2}$ of the light passing through.  How can this be?}
\end{frame}

\begin{frame}{Polarizing filters in succession}{Observations about this experiment}
\TwoColumns{%
\begin{itemize}
    \item<1-> Start with unpolarized light.
    \item<2-> Impose a \textcolor{\RCtwo}{vertically polarizing filter}.
    \item<3-> You see half of the light passing through.
    \item<4-> Impose a second \textcolor{\RCthree}{identical filter}.
    \item<5-> No more light is filtered out this time.
    \item<6-> Impose \textcolor{\RCone}{another filter} at \Degrees{45}.
    \item<7> You now see half of the previous light passing through, now polarized according to this \textcolor{\RCone}{last filter}.
\end{itemize}
}{%
\begin{center}
\begin{TIKZP}
\visible<1-2>{\RadiantArrows{22.5}{->,color=BrickRed,thick}}
\visible<2>{\begin{scope}[draw=\RCtwo]\PFilter{-1}{-1.2}{1}{1.2}{20}\end{scope}}
\visible<3-4,5,6>{%
\foreach \x in {-1, -0.75,...,1} {
\draw[<->,thick,color=BrickRed] (\x,-1) -- (\x,1);
}
}
\visible<4>{\begin{scope}[draw=\RCthree]\PFilter{-1}{-1.2}{1}{1.2}{20}\end{scope}}

\visible<6>{\begin{scope}[draw=\RCone,rotate=45]\PFilter{-1}{-1.2}{1}{1.2}{20}\end{scope}}
\visible<7>{%
\begin{scope}[rotate=45]
\foreach \x in {-1, -0.5,...,1} {
\draw[<->,thick,color=BrickRed] (\x,-1) -- (\x,1);
}
\end{scope}
}
\end{TIKZP}
\end{center}
}
\end{frame}



\begin{frame}{Orthogonal filters}{They can eliminate all of the light}
\TwoColumns{%
\begin{itemize}
    \item<1-> Start with unpolarized light.
    \item<2-> Impose a \textcolor{\RCtwo}{vertically polarizing filter}.
    \item<3-> You see half the light passing through.
    \item<4-> Add a second \textcolor{\RCthree}{identical filter}, behind the first one, rotated \Degrees{90}.
    \item<5-> No light passes through now.  We should expect this.
\end{itemize}
}{%
\begin{center}
\begin{TIKZP}
\visible<1-2>{\RadiantArrows{22.5}{->,color=BrickRed,thick}}
\visible<2,4>{\begin{scope}[draw=\RCtwo]\PFilter{-1}{-1.2}{1}{1.2}{20}\end{scope}}
\visible<3-4>{%
\foreach \x in {-1, -0.75,...,1} {
\draw[<->,thick,color=BrickRed] (\x,-1) -- (\x,1);
}
}
\visible<4>{\begin{scope}[draw=\RCthree,rotate=90]\PFilter{-1}{-1.2}{1}{1.2}{20}\end{scope}}
\end{TIKZP}
\end{center}
}
\end{frame}

\begin{frame}{How much light passes through two filters?}{Getting quantitative}
\TwoUnequalColumns{.70\textwidth}{.30\textwidth}{%
\begin{itemize}
    \item<1-> When the filters align, so the angle between them is~\Degrees{0}, there is no change in the light passed through the second filter.
    \item<2-> When the filters are orthogonal, so the angle between them is~\Degrees{90}, no light passes through the second filter.
    \item<3-> If you experiment with various angles and measure the fraction of light that enters the first filter and passes through the second, you will see that it is related to $\cos^{2}\theta$, where $\theta$ is the angle between the filters' relative orientations.\LinkArrow{https://en.wikipedia.org/wiki/Polarizer\#Malus's_law_and_other_properties}
\end{itemize}
}{%
\begin{center}
\begin{TIKZP}
\visible<1-3>{\begin{scope}[draw=\RCtwo,rotate=90]\PFilter{-1}{-1.2}{1}{1.2}{20}\end{scope}
\draw[->,thick] (0,0) -- (1,0) ;
}
\visible<2>{\begin{scope}[draw=\RCthree]\PFilter{-1}{-1.2}{1}{1.2}{20}\end{scope}
\draw[->,thick] (0,0) -- (1,0) ;
\draw[->,thick] (0,0) -- (0,1);
}
\visible<3>{\begin{scope}[draw=Sepia,rotate=-20]\PFilter{-1}{-1.2}{1}{1.2}{20}\end{scope}
\draw[->,thick] (0,0) -- (70:1);
\draw (0.2,0) node[above right] {$\theta$};
}
\end{TIKZP}
\end{center}
}
\end{frame}

\begin{frame}{And now a surprise}{A filter can actually cause light to emerge}
\TwoUnequalColumns{0.65\textwidth}{0.35\textwidth}{%
\begin{itemize}
    \item<1-> Start with unpolarized light
    \item<2-> Impose a \textcolor{\RCtwo}{vertically polarizing filter}.
    \item<3-> You see half the light passing through.
    \item<4-> Add a second \textcolor{\RCthree}{identical filter}, in front the first one, rotated \Degrees{90}.
    \item<5-> You see no light, like previously.
    \item<6-> Impose \textcolor{\RCone}{another filter} at \Degrees{45} sandwiched \emph{between} the \textcolor{\RCtwo}{first} and \textcolor{\RCthree}{second} filters.
    \item<7> You now see 25\% of the light passing through and it is polarized according to the \textcolor{\RCthree}{last} filter.
\end{itemize}
}{%
\begin{center}
\begin{TIKZP}
\visible<1-2>{\RadiantArrows{22.5}{->,color=BrickRed,thick}}
\visible<2,4,6,7>{\begin{scope}[draw=\RCtwo]\PFilter{-1}{-1.2}{1}{1.2}{20}\end{scope}}
\visible<3-4>{%
\foreach \x in {-1, -0.75,...,1} {
\draw[<->,thick,color=BrickRed] (\x,-1) -- (\x,1);
}
}
\visible<6,7>{\begin{scope}[draw=\RCone,rotate=45]\PFilter{-1}{-1.2}{1}{1.2}{20}\end{scope}}
\visible<4,6,7>{\begin{scope}[draw=\RCthree,rotate=90]\PFilter{-1}{-1.2}{1}{1.2}{20}\end{scope}}
\visible<7>{%
\begin{scope}[rotate=90]
\foreach \x in {-0.5,0, ...,0.5} {
\draw[<->,thick,color=BrickRed] (\x,-1) -- (\x,1);
}
\end{scope}
}

\end{TIKZP}
\end{center}
}
\end{frame}

\begin{frame}{Quantum interpretation}{A theory that yields precise measurements}
\begin{itemize}
    \item An unpolarized photon is in a \emph{superposition} of polarizations.  Because our personal experiences differ so greatly from that of the photon, we lack the language to describe its behavior.  We will express this concept mathematically, as a linear combination of possible outcomes.  Here we can say that each outcome is equally likely.
    \item When the photon hits the polarizing filter, it interacts with the filter so as to either pass through or not.
    \begin{itemize}
        \item In the \href{https://en.wikipedia.org/wiki/Copenhagen_interpretation}{Copenhagen interpretation}, we say that the superposition \emph{collapsed} by this interaction, which constituted a \emph{measurement}.
        \item In the \href{https://en.wikipedia.org/wiki/Many-worlds_interpretation}{Many Worlds} interpretation, we became entangled with the measurement outcome we saw.  There is a different world where we saw the measurement turn out the other way.
    \end{itemize}
    \item The outcome is consistent with any number of subsequent, identical measurements.
    
\end{itemize}
\end{frame}

\begin{frame}{Reinterpretation of the surprise}{We have to analyze the effects of each filter in turn}
\TwoUnequalColumns{0.65\textwidth}{0.35\textwidth}{%
\begin{itemize}
    \item<1-> Start with unpolarized light
    \item<2-> Impose a \textcolor{\RCtwo}{vertically polarizing filter}.
    \item<3-> You see half the light passing through.
.
    \item<4-> The \textcolor{\RCone}{another filter} at \Degrees{45} sandwiched is next.
    \item<5-> Its angle will cause $\cos^2(\pi/4)=1/2$ of the light to pass through.
        \item<6-> Add the final filter \textcolor{\RCthree}{identical filter}, rotated \Degrees{90}.
    \item<6-> Its angle with the previous filter is also \Degrees{45}, so that the light is cut down by another factor of 2.
    
    \item<7> You thus see $1/4$ of the original light, polarized according to the \textcolor{\RCthree}{last} filter.
\end{itemize}
}{%
\begin{center}
\begin{TIKZP}
\visible<1>{\RadiantArrows{22.5}{->,color=BrickRed,thick}}
\visible<2->{\begin{scope}[draw=\RCtwo]\PFilter{-1}{-1.2}{1}{1.2}{20}\end{scope}}
\visible<2-3>{%
\foreach \x in {-1, -0.75,...,1} {
\draw[<->,thick,color=BrickRed] (\x,-1) -- (\x,1);
}
}
\visible<4->{\begin{scope}[draw=\RCone,rotate=45]\PFilter{-1}{-1.2}{1}{1.2}{20}\end{scope}}
\visible<4-5>{%
\begin{scope}[rotate=45]
\foreach \x in {-0.5,-0.25,...,0.5} {
\draw[<->,thick,color=BrickRed] (\x,-1) -- (\x,1);
}
\end{scope}}
\visible<6->{\begin{scope}[draw=\RCthree,rotate=90]\PFilter{-1}{-1.2}{1}{1.2}{20}\end{scope}}
\visible<6-7>{%
\begin{scope}[rotate=90]
\foreach \x in {-0.5,0.5} {
\draw[<->,thick,color=BrickRed] (\x,-1) -- (\x,1);
}
\end{scope}
}

\end{TIKZP}
\end{center}
}
\end{frame}

\section{Double-slit experiment}

\begin{frame}{Double-slit experiment}
\begin{itemize}
\item No introduction to quantum behavior is complete without some coverage of the \href{https://en.wikipedia.org/wiki/Double-slit_experiment}{double-slit experiment}.  

\item Let's look at (some of) \href{https://en.wikipedia.org/wiki/Umesh_Vazirani}{Umesh Vazirani}'s videos on this subject.
\begin{itemize}
    \item \href{https://www.youtube.com/watch?v=1X7CDd1lvR0}{Video 1} (7.5 minutes)
    \item \href{https://www.youtube.com/watch?v=pHmRp2eGETk}{Video 2} (13 minutes)
    \item \href{https://www.youtube.com/watch?v=SzC_O13IH2w}{Video 3} (9.5 minutes)
\end{itemize}
\item A more mathematical view would be satisfied with probabilities based on $2$-norms and not care so much about the physics.
\item However, it may be worthwhile to consider why probabilities are based on $2$-norms from a physics point of view, and this has to do with the energy required to create a wave of a given amplitude.  We look at that next.
\end{itemize}
\end{frame}

\def\GrowingWave#1{%
\SineWave{#1}
\draw[->] (1,0) -- (1,#1) node[above] {x};
}
\begin{frame}{The intensity of a wave}{This is related to the probability of seeing a given outcome}
\TwoUnequalColumns{0.5\textwidth}{0.5\textwidth}{%
\begin{itemize}
    \item<1-> The \href{https://en.wikipedia.org/wiki/Intensity_(physics)}{intensity of a wave} is a function of its amplitude~$d$.
    \item<2-> The force to push a spring distance $x$ is: \[  F = k\times x \] for some constant $k$ (\href{https://en.wikipedia.org/wiki/Hooke\%27s_law}{Hooke's law}).
    \item<3-> We have to keep pushing~$x$ to achieve the amplitude~$d$.
    \item<6> Thus the energy is proportional to the \emph{square} of the amplitude.
 
\end{itemize}
}{%
\begin{center}
\begin{TIKZP}
\visible<1>{
\SineWave{1.0}
\draw[->] (1,0) -- (1,1);
\draw (1.2,0.2) node[above] {$d$};}
\visible<2>{
\GrowingWave{0.25}}
\visible<3>{
\GrowingWave{0.50}}
\visible<4>{
\GrowingWave{0.75}}
\visible<5->{
\GrowingWave{1.0}
\draw (1.2,0.2) node[above] {$d$};
}

\end{TIKZP}
\end{center}
   \only<5->{ To achieve amplitude $d$ we must push continually from $0$ to $d$, so that the total energy is 
    \[
    \int_{0}^{d} kx \, dx = \frac{1}{2} k d^{2}
    \]}
}
\end{frame}

\begin{frame}{Quantum games}{Beam splitters and reflectors}

\begin{itemize}
    \item As time permits we will go over some of the \href{https://quantumflytrap.com/}{quantum games site} puzzles.
\end{itemize}


    
\end{frame}
