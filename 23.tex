\SetTitle{23}{The Deutsch--Jozsa problem}{Demonstrates asymptotic quantum advantage}{23}

\section*{Overview}

\begin{frame}{Overview}{What will we study here?}

\begin{itemize}[<+->]
    \item This problem and its solution are due to
    \href{https://en.wikipedia.org/wiki/David_Deutsch}{David Deutsch} and \href{https://en.wikipedia.org/wiki/Richard_Jozsa}{Richard Jozsa}.
    \item This problem is similar to Deutsch's problem, which we have studied.
    \item But here, $f(x)$ takes an input of $n$~bits.
    \item We care, as before, whether $f(x)$ is \emph{constant} or \emph{balanced}.
    \item This problem distinguishes the complexity classes
    \begin{description}
        \item[\href{https://complexityzoo.net/Complexity_Zoo:P}{\CompClass{P}}] problems that can be solved in polynomial time on a classical computer
        \item[\href{https://complexityzoo.net/Complexity_Zoo:E\#eqp}{\CompClass{EQP}}] problems that can be solved \textbf{e}xactly on a \textbf{q}uantum computer in \textbf{p}olynomial time.
    \end{description}
    \item The phase-kickback trick we have just studied is the key to solving this problem efficiently on a quantum computer.
\end{itemize}
    
\end{frame}

\begin{frame}{Exact?}{What's up with that?}
\begin{itemize}[<+->]
    \item There are two sources of inaccuracy in quantum computing:
    \begin{itemize}
        \item Some algorithms are inherently probabilistic, such as the Elitzur--Vaidman bomb we have studied. While you are likely to get a desired result, it may take effort to overcome the probability of obtaining the wrong answer.  And you may be unlucky to the point that it takes a long time to get the right answer.
        \item Quantum computing devices are subject to noise and other interference that causes \href{https://en.wikipedia.org/wiki/Quantum_decoherence}{decoherence}.  Results from a system experiencing decoherence are unpredictable and invalid.
    \end{itemize}
    \item Simulation (emulation) provides results with no artifacts of decoherence, but those are limited in the number of quantum bits (\href{https://qiskit.org/documentation/tutorials/simulators/6_extended_stabilizer_tutorial.html}{currently}, 63~qubits) that can be accommodated in reasonable time.
    \item So, \emph{exact quantum polynomial} means that on an ideal quantum device, we obtain the answer exactly, with probability 100\%, in polynomial time.
\end{itemize}
\end{frame}

\begin{frame}{On the pronunciation of Jozsa}{From the author himself}

Thanks for your interesting question!  I generally pronounce my surname as~\Quote{joe--zuh}.
\SmallSkip{}
But this is an anglicised\footnote{%
Ron notes that \emph{anglici\alert{s}ed} is the anglicized version of \emph{anglici\alert{z}ed}}
version---the name is Hungarian and the correct Hungarian pronunciation is  \Quote{your--zho} with \emph{zh} pronounced as the \emph{s} in \emph{pleasure} and \emph{o} as the \emph{o} in \emph{hot}.
\SmallSkip{}
I hope that gives enough of an indication, and I wish you and your students all the best for your QC course!
\BigSkip{}
Best wishes,  Richard.
\end{frame}

\section*{Problem}

\begin{frame}{Deutsch--Jozsa}{Problem statement}
\begin{itemize}[<+->]
    \item We are given a function $f(x): \Set{0,1}^{n}\mapsto \Set{0,1}$.  It takes in a value expressed in $n$ bits and returns \False{}~($0$) or \True~($1$).
    \item We are \href{https://en.wikipedia.org/wiki/Promise_problem}{promised} that $f(x)$ is one of the following:
    \begin{description}
        \item[constant]  \Forall{x\in \Set{0,1}^n}{f(x) = k}, where $k=0$ or $k=1$.
        \item[balanced]  For exactly half of its inputs, $f(x) = 0$ and for the other half $f(x) = 1$.
    \end{description}
    \item We assume that the promise holds, but we eventually consider the consequences of breaking this promise.
\end{itemize}
\end{frame}

\subsection*{Classical solution}

\begin{frame}{Classical approach}{Takes $\Theta(2^{n})$ evaluations of the function}

\begin{itemize}[<+->]
    \item The function $f(x)$ accepts an input \AllBits{x}{n}.  Each input is $n$~bits long, and there are $2^n$ such strings in the domain.
    \item Let's assume $f(x)$ takes $O(1)$ time
    to evaluate its result for a given input~$x$.
    \item With $2^{n}$ possible inputs, the function $f(x)$ could conspire to make the first $\frac{2^n}{2}=2^{n-1}$ results the same.  Let's suppose they all evaluate to $0$.
    \item At that point, a classical approach lacks the necessary information to decide whether $f(x)$ is constant or balanced.
    \item The next evaluation will either yield
    \begin{description}
      \item[$0$] implying $f(x)$ is constant
      \item[$1$] implying $f(x)$ is balanced
    \end{description}
    \item Classically it can take $2^{n-1}+1$ evaluations to determine the nature of $f(x)$.
\end{itemize}
    
\end{frame}

\begin{frame}{Sneak preview}{Of the secret sauce}
\Vskip{-3.5em}\begin{itemize}[<+->]
    \item The function $f(x)$ accepts an input \AllBits{x}{n}.  Each input is $n$~bits long, and \alert<10->{there are $2^n$ such strings} in the domain.
    \item For each such $x$, $f(x)$ evaluates to~$0$ or~$1$.  \only<10->{\alert{Then \NegF{x} is $\pm 1$.}}
    \item If we compute $\SumBV{x}{n} \alt<9->{\alert{\NegF{x}}}{f(x)}$ then we should obtain:
    \begin{description}
        \item[\alt<11->{\alert{$2^n$}}{$0$}]  We obtain this sum if \Forall{x}{f(x)=0}.
        \item[\alt<12->{\alert{$-2^{n}$}}{$2^n$}]  We obtain this sum if \Forall{x}{f(x)=1}.
        \item[\alt<13->{\alert{$0$}}{$\frac{2^n}{2}=2^{n-1}$}]  \textcolor<14->{\RCtwo}{We obtain this sum if exactly half of the function evaluations are~$1$ and the other half are~$0$.}
    \end{description}
    \item As described, we require $\Theta(2^n)$ function evaluations classically.  We hope to use a quantum computer to accomplish that work with a single evaluation.
    \item But first we are going to make \alert{one change} to our summation to leverage \textcolor<14->{\RCtwo}{interference} in quantum systems.
\end{itemize}
    
\end{frame}

\subsection*{Quantum solution}

\begin{frame}{Deutsch--Jozsa solution}{Compared to the Deutsch circuit}

\TwoUnequalColumns{0.55\textwidth}{0.45\textwidth}{%
\begin{center}
\Vskip{-3em}\adjustbox{valign=t, width=\textwidth}{\begin{quantikz}
\lstick{\ket{\TensSupProd{0}{n}}} & \qwbundle{\alert<1>{n}}\slice{\QState{0}} &  \gate{\alert<3>{\TensSupProd{\Hadamard}{n}}}\slice{\QState{1}} & \gate[wires=2][5em]{\mbox{$U_f$}}\gateinput{$x$}\gateoutput{$x$}\slice{\QState{2}} &\gate{\alert<3>{\TensSupProd{\Hadamard}{n}}}\slice{\QState{3}} & \meter{} \\
\lstick{\alert<2>{\QZero{}}} & \gate{\alert<2>{\PauliX}} &   \gate{\alert<2>{\Hadamard}}   &  \qw\gateinput{$y$}\gateoutput{\Xor{y}{f(x)}} & \qw & \qw
\end{quantikz}} \only<1-3>{\\[1em]
Deutsch--Jozsa Problem, $n$ qubits}
\end{center}
}{%
\only<1-3>{%
\begin{center}
\Vskip{-3em}\adjustbox{valign=t, width=\textwidth}{\begin{quantikz}
\lstick{\QZero{}} & \qw\slice{\QState{0}} &  \gate{\Hadamard}\slice{\QState{1}} & \gate[wires=2][5em]{\mbox{$U_f$}}\gateinput{$x$}\gateoutput{$x$}\slice{\QState{2}} &\gate{\Hadamard}\slice{\QState{3}} & \meter{0/1} \\
\lstick{\QZero{}} & \gate{\PauliX} &   \gate{\Hadamard}   &  \qw\gateinput{$y$}\gateoutput{\Xor{y}{f(x)}} & \qw & \qw
\end{quantikz}} \only<1-3>{\\[1.2em]
Deutsch Problem, single qubit}
\end{center}%
}%
}
\only<1-3>{%
\BigSkip{}
\begin{itemize}
    \item<1-> The Deutsch--Jozsa input for $x$ is now $n$ qubits.
    \item<2-> The input for $y$ is still a single qubit, whose value will be \ket{-} going into $U_f$.
    \item<3-> We apply an $n$-way Hadamard operation on either side of $U_f$ to the top $n$~qubits~($x$).  This is realized by a single Hadamard gate applied to each qubit for the input~$x$.
\end{itemize}
}

\end{frame}

\section*{Hadamard, n-way}

\begin{frame}{The $n$-way \href{https://en.wikipedia.org/wiki/Hadamard_transform}{Hadamard transform}}{One \Hadamard{} gate applied to each qubit}

\begin{itemize}[<+->]
    \item By convention we define $\TensSupProd{\Hadamard}{0}=1$.
    \item We have seen the single-qubit Hadamard transform \Hadamard{} defined as:
    \[
        \Hadamard = \TensSupProd{\Hadamard}{1} = \HMatrix{}
    \]
    \item We can then define
    \Vskip{-1.5em}\[  \TensSupProd{\Hadamard}{n} = \TensProd{\Hadamard}{\TensSupProd{\Hadamard}{n-1}} = \SQBG{\RootTwo}{\TensSupProd{\Hadamard}{n-1}}{\TensSupProd{\Hadamard}{n-1}}{\TensSupProd{\Hadamard}{n-1}}{-\TensSupProd{\Hadamard}{n-1}}
    \]
    \item When $n$ is clear in context, we may drop the superscript, for example:
    \[
       \TensSupProd{\Hadamard}{2} \ket{00} \mbox{ may be written as }
       \Hadamard \ket{00}
    \]
    \item If the input to \TensSupProd{\Hadamard}{n} is not entangled, then analysis can consider each qubit separately.
\end{itemize}
    
\end{frame}

\section*{Analysis}

\begin{frame}{Deutsch--Jozsa Solution}{Up to \QState{2}, prior to the final \TensSupProd{\Hadamard}{n}}

\Vskip{-4em}\TwoUnequalColumns{0.51\textwidth}{0.49\textwidth}{%
\begin{center}
\Vskip{-3em}\adjustbox{valign=t, width=\textwidth}{\begin{quantikz}
\lstick{\ket{\textcolor<5->{\RCone}{\TensSupProd{0}{n}}}} & \qwbundle{n}\slice{\alert<5>{\QState{0}}} &  \gate{\alert<2>{\TensSupProd{\Hadamard}{n}}}\slice{\alert<6-7>{\QState{1}}} & \gate[wires=2][5em]{\mbox{$U_f$}}\gateinput{$x$}\gateoutput{$x$}\slice{\alert<8>{\QState{2}}} &\gate{\alert<3-4,8>{\TensSupProd{\Hadamard}{n}}}\slice{\alert<9->{\QState{3}}} & \meter{} \\
\lstick{\textcolor<5->{\RCtwo}{\QZero{}}} & \gate{\textcolor<5->{\RCtwo}{\PauliX}} &   \gate{\Hadamard}   &  \qw\gateinput{$y$}\gateoutput{\Xor{y}{f(x)}} &  & 
\end{quantikz}}
\end{center}
\only<8>{%
\begin{itemize}
\item This is the phase-kickback result.
\item As drawn, \QState{2} includes only the top $n$ qubits, omitting the bottom one.
\item The complete state developed on output of $U_f$ is \TensProd{\QState{2}}{\ColorTwo{\ket{-}}}.
\item The bottom unmeasured qubit, \ColorTwo{\ket{-}}, is tensor factorable from the top $n$~qubits. We drop it as of \QState{2}.
\end{itemize}}
}{%
\only<5-13>{%
\Vskip{-2.5em}\begin{align*}
   \only<5-7>{ \QState{0} &= \ket{\ColorOne{\TensSupProd{0}{n}}\,\ColorTwo{1}} \\}
   \only<5-8>{%
   \visible<6->{ \QState{1} &= \TensSupProd{\Hadamard}{n+1} \QState{0} \\}
    \visible<7-> {& =  \ColorOne{\RootTwoN{n}}\TensProd{\ColorOne{\SumBV{w}{n}\ket{w}}}{\ColorTwo{\ket{-}}} \\}}
    \only<8-10>{%
    \visible<8->{\QState{2} &= \ColorOne{\RootTwoN{n}}\ColorOne{\SumBV{w}{n} \NegF{w}\ket{w}}\\}
    \visible<9->{\QState{3} &= \TensSupProd{\Hadamard}{n}(\QState{2}) \\}}
    \visible<10->{%
       \only<11->{\QState{3}} & = \frac{\TensSupProd{\Hadamard}{n}
       \left(
       \ColorOne{\SumBV{w}{n} \NegF{w}\ket{w}}
       \right)}{\ColorOne{\sqrt{2^n}}}
    }
\end{align*}
}%
}%
\only<1-4>{%
\begin{itemize}
    \item<1-> We have seen
    \Vskip{-3em}\[ \TensSupProd{\Hadamard}{n}\ket{\TensSupProd{0}{n}} = 
    \RootTwoN{n}\SumBV{w}{n} \ket{w} = \RootTwoN{n}\DQB{1}{\vdots}{\vdots}{1}
    \]
    \item<2-> So we know the result of this $n$-way \Hadamard{} operation.
    \item<3-> But what about \alert<3>{this} one?  
    \item<4-> Its input is probably not \ket{\TensSupProd{0}{n}}, so some deeper analysis is needed.  But first we build up to \QState{2} by analyzing states~\QState{0} and~\QState{1}.
\end{itemize}
}%
\only<11-13>{%
\BigSkip{}
\begin{itemize}
\item<11-> We show some \ColorOne{examples of \QState{2}} to develop intuition.
\item<12-> We then turn to computing \QState{3}, applying \TensSupProd{\Hadamard}{n}(\ket{w}), where \ket{w} is a basis state (but not necessarily~\ket{\TensSupProd{0}{n}}).
\item<13-> It suffices to study basis states, as linearity applies to \ColorOne{superpositions}.
\end{itemize}
}%
\end{frame}

\begin{frame}{A closer look at \QState{2}}{Case $f(x)$ is constant: \Forall{x}{f(x)=0}}

\TwoUnequalColumns{0.5\textwidth}{0.5\textwidth}{%
\only<1-12>{\begin{itemize}
  \item<2-> Consider $n=3$
  \item<4-> \Forall{w}{\ColorThree{f(w)}=0}.
  \item<5-> Then \Forall{\ColorThree{w}}{\ColorThree{\NegF{w}}=\ColorThree{1}}.
  \item<8-> \Hadamard{} is its own inverse.  
  \item<10-> The state whose Hadamard transformation is the uniform superposition shown here is \ket{000}.
  \item<11-> Measurement of \QState{3} will yield \ket{000} always.
  \item<12-> This argument holds for any $n$.
\end{itemize}}%
\only<13->{%
\begin{itemize}
\item Another way to say this is that we have arranged for constructive interference on the~\ket{000} term  in~\QState{3} and destructive interference on all other terms.

\item With all amplitude on the \ket{000} term, measurement will yield $000$.

\item We just have to be sure we see \ket{000} if \emph{and only if} the function $f(x)$ is constant.
\end{itemize}
}
}{%
\only<1-8,10->{
\begin{align*}
   \visible<1->{ \ColorOne{\sqrt{2^{\alt<1-2>{n}{3}}}}\QState{2} &= \ColorOne{\SumBV{w}{\alt<1-2>{n}{3}}} \ColorThree{\alt<1-4>{\NegF{w}}{\NegOneExp{0}}} \ColorOne{\ket{w}} \\}
   \visible<6->{ &= \ColorThree{1} \ColorOne{\ket{000}} +
    \ColorThree{1} \ColorOne{\ket{001}} \\
    & + \ColorThree{1} \ColorOne{\ket{010}} +
    \ColorThree{1} \ColorOne{\ket{011}} \\
    & + \ColorThree{1} \ColorOne{\ket{100}} +
     \ColorThree{1} \ColorOne{\ket{101}} \\
   & + \ColorThree{1} \ColorOne{\ket{110}} +
    \ColorThree{1} \ColorOne{\ket{111}}} \\
    \visible<7->{\QState{3} &= \TensSupProd{\Hadamard}{3}\QState{2} \\}
    \visible<10->{ &= \ket{000}}
\end{align*}}
\only<9>{%
\begin{align*}
    \Hadamard{}^{2} &= \HMatrix{} \times \HMatrix{} \\[1em]
    &= \IMatrix{}
\end{align*}
}
}
    
\end{frame}

\begin{frame}{Closer look at \QState{2}}{Case $f(x)$ is (the other) constant: \Forall{x}{f(x)=1}}

\TwoColumns{%
\only<1-9>{%
\begin{itemize}
    \item<1-> Same as previous case, except $\ldots$
    \item<2-> Now our function returns~$1$ for any input instead of~$0$.
    \item<5-> This is a global phase, present on each term, so this state is equivalent to our previous case.
    \item<8-> We will measure $000$ in this case as well.
    \item<9-> To achieve our desired result we must show that if $f(x)$ is balanced, we \emph{never} measure $000$.
\end{itemize}}%
\only<10->{%
\begin{itemize}
    \item<10-> It takes some analysis to prove how a balanced $f(x)$ measures after \QState{3}.
    \item<11-> When $f(x)$ is balanced, a global phase that switches the sign of each term will have no effect on the measurement outcome.
    \item<12-> In other words, the function $1-f(x)$ is balanced iff $f(x)$ is balanced. 
    \item<13-> It is helpful to view \ColorThree{\NegF{x}} as determining the \ColorThree{sign $\pm 1$} of each \ColorOne{term}.
\end{itemize}
}
}{%
\begin{align*}
   \visible<1->{ \ColorOne{\sqrt{2^{\alt<1>{n}{3}}}}\QState{2} &= \ColorOne{\SumBV{w}{\alt<1>{n}{3}}} \ColorThree{\alt<1>{\NegF{w}}{\NegOneExp{1}}} \ColorOne{\ket{w}} \\}
   \visible<3->{ &\alt<6->{\equiv}{=} \ColorThree{\only<1-5>{-}1} \ColorOne{\ket{000}} +
    \ColorThree{\only<1-5>{-}1} \ColorOne{\ket{001}} \\
    & + \ColorThree{\only<1-5>{-}1} \ColorOne{\ket{010}} +
    \ColorThree{\only<1-5>{-}1} \ColorOne{\ket{011}} \\
    & + \ColorThree{\only<1-5>{-}1} \ColorOne{\ket{100}} +
     \ColorThree{\only<1-5>{-}1} \ColorOne{\ket{101}} \\
   & + \ColorThree{\only<1-5>{-}1} \ColorOne{\ket{110}} +
    \ColorThree{\only<1-5>{-}1} \ColorOne{\ket{111}}} \\
    \visible<4->{\QState{3} &= \TensSupProd{\Hadamard}{3}\QState{2} \\}
    \visible<7->{ &= \ket{000}}
\end{align*}
}

\end{frame}

\begin{frame}{Hadamard of an arbitrary basis state}{Superpositions follow using linearity}
\begin{itemize}
    \item We have seen $\TensSupProd{\Hadamard}{n}\ket{\TensSupProd{0}{n}}=\RootTwoN{n}\SumBV{v}{n}\ket{v}$.
    \item How do we compute the $n$-way Hadamard of a basis state \ket{w} of $n$ qubits?
    \[
    \TensSupProd{\Hadamard}{n}\ket{w} = \mbox{?}
    \]
\end{itemize}
\end{frame}

\section*{Dot product}

\begin{frame}{Defining the operation \DotP{v}{w}}{A Boolean dot (inner) product of two bit vectors}

\TwoColumns{%
\Vskip{-4em}\begin{center}
\begin{TIKZP}
\draw (0,0) rectangle ++(0.5,0.8) node[pos=0.5] {$v_1$};
\draw (0.5,0) rectangle ++(0.5,0.8) node[pos=0.5] {$v_2$};
\draw (1,0) rectangle ++(0.5,0.8) node[pos=0.5] {$\cdot$};
\draw (1.5,0) rectangle ++(0.5,0.8) node[pos=0.5] {$\cdot$};
\draw (2,0) rectangle ++(0.5,0.8) node[pos=0.5] {$\cdot$};
\draw (2.5,0) rectangle ++(0.5,0.8) node[pos=0.5] {$v_n$};
\draw (0,-1.2) rectangle ++(0.5,0.8) node[pos=0.5] {$w_1$};
\draw (0.5,-1.2) rectangle ++(0.5,0.8) node[pos=0.5] {$w_2$};
\draw (1,-1.2) rectangle ++(0.5,0.8) node[pos=0.5] {$\cdot$};
\draw (1.5,-1.2) rectangle ++(0.5,0.8) node[pos=0.5] {$\cdot$};
\draw (2,-1.2) rectangle ++(0.5,0.8) node[pos=0.5] {$\cdot$};
\draw (2.5,-1.2) rectangle ++(0.5,0.8) node[pos=0.5] {$w_n$};
\end{TIKZP}\end{center}
\begin{itemize}
    \item<2-> \ColorThree{Exclusive-or of the \emph{and} of each pair}
    \item<3-> \ColorFour{Sum of the product of each pair}
\end{itemize}
}{%
\begin{align*}
   \visible<2->{ \DotP{v}{w} & \ColorThree{= \oplus_{i=1}^{n}\ {v_i}{w_i}} \\}
        \visible<3->{ & \ColorFour{= \sum_{i=1}^{n}\ {v_i}{w_i}\ \mbox{(mod 2)}}}
\end{align*}
\only<4->{\Vskip{-1em}The above are equivalent (proof later).}
}
\only<5->{%
\Vskip{-1em}\TwoColumns{%
Examples
\begin{itemize}
    \item<5-> $\DotP{\alert{1}010}{\alert{1}101} = 1$
    \item<6-> $\DotP{1\alert{11}0}{0\alert{11}0}=0$
    \item<7-> $\DotP{1110}{\ColorTwo{0000}}=0$
    \item<8-> $\DotP{\alert{111}0}{\ColorThree{1111}}=1$
\end{itemize}
}{%
Generally
\begin{itemize}
    \item<7-> $\Forall{\AllBits{v}{n}}{ \DotP{v}{\ColorTwo{0^n}}=0}$
    \item<8-> $\Forall{\AllBits{v}{n}}{ \DotP{v}{\ColorThree{1^n}} = \mbox{\href{https://en.wikipedia.org/wiki/Parity_function}{parity}}(v)}$
    \item<9-> Is the number of \alert{common \Quote{1}s} even or odd?
\end{itemize}
}}
\end{frame}

\begin{frame}{\TensSupProd{\Hadamard}{n}\ket{w}, where $w$ is an arbitrary $n$-qubit basis state}{Superpositions follow from linearity}
\Vskip{-3em}\[
P(n):  \TensSupProd{\Hadamard}{n}\ket{\alt<3-4>{\alert{\TensSupProd{0}{n}}}{w}}
  = \RootTwoN{n}\SumBV{\ColorThree{v}}{n} \NegOneExp{\alt<5>{\ColorTwo{0}}{\ColorTwo{\DotP{\ColorThree{v}}{\alt<4>{0^n}{w}}}}} \ket{\ColorThree{v}}
\]

\only<1-6>{%
\begin{itemize}
\item<1-> We prove $P(n)$ later but accept it as true for now.
\item<2-> \TensSupProd{\Hadamard}{n}\ket{\alert{\TensSupProd{0}{n}}} provides the correct result, because \Implies{w=0^n}{\Forall{v}{\ColorTwo{\DotP{\ColorThree{v}}{w}=0}}}.  \visible<5->{The weight present on each~\ColorThree{\ket{v}} is~$1$.}
\item<6-> We can use this formula to compute the result for any basis state \ket{w}, by computing $w$'s \ColorTwo{dot product} with every basis vector \ColorThree{$v$}.
\end{itemize}}%
\only<7->{%
\TwoColumns{%
\begin{HTable}{\alert<12-15>{1}0\alert<9,11,13,15>{1}}{v}{3}
\visible<8->{\TE{000}{0}{+}{000} \\}
\visible<9->{\TE{00\alert<9>{1}}{1}{-}{001} \\}
\visible<10->{\TE{000}{0}{+}{010} \\}
\visible<11->{\TE{00\alert<11>{1}}{1}{-}{011} \\}
\visible<12->{\TE{\alert<12>{1}00}{1}{-}{100} \\}
\visible<13->{\TE{\alert<13>{1}0\alert<13>{1}}{0}{+}{101} \\}
\visible<14->{\TE{\alert<14>{1}00}{1}{-}{110} \\}
\visible<15->{\TE{\alert<15>{1}0\alert<15>{1}}{0}{+}{111}}
\end{HTable}
}{%
\begin{HTable}{0\alert<18,19,22,23>{1}\alert<17,19,21,23>{1}}{v}{3}
\visible<16->{\TE{000}{0}{+}{000} \\}
\visible<17->{\TE{00\alert<17>{1}}{1}{-}{001} \\}
\visible<18->{\TE{0\alert<18>{1}0}{1}{-}{010} \\}
\visible<19->{\TE{0\alert<19>{11}}{0}{+}{011} \\}
\visible<20->{\TE{000}{0}{+}{100} \\}
\visible<21->{\TE{00\alert<21>{1}}{1}{-}{101} \\}
\visible<22->{\TE{0\alert<22>{1}0}{1}{-}{110} \\}
\visible<23->{\TE{0\alert<23>{11}}{0}{+}{111}}
\end{HTable}
}}
\end{frame}

\begin{frame}{Complete two-qubit example}{And the superposition of those results}

\only<1-11>{%
\Vskip{-5em}\TwoColumns{%
{\small
\begin{center}
\visible<1->{\begin{HTable}{\ColorOne{00}}{v}{2}
\TE{00}{0}{\alert<6>{+}}{00} \\
\TE{00}{0}{\alert<7>{+}}{01} \\
\TE{00}{0}{\alert<8>{+}}{10} \\
\TE{00}{0}{\alert<9>{+}}{11}
\end{HTable} \\[1em]}
\visible<2->{\begin{HTable}{\ColorTwo{01}}{v}{2}
\TE{00}{0}{\alert<6>{+}}{00} \\
\TE{01}{1}{\alert<7>{-}}{01} \\
\TE{00}{0}{\alert<8>{+}}{10} \\
\TE{01}{1}{\alert<9>{-}}{11}
\end{HTable}}\end{center}}
}{%
{\small
\begin{center}
\visible<3->{\begin{HTable}{\ColorThree{10}}{v}{2}
\TE{00}{0}{\alert<6>{+}}{00} \\
\TE{00}{0}{\alert<7>{+}}{01} \\
\TE{10}{1}{\alert<8>{-}}{10} \\
\TE{10}{1}{\alert<9>{-}}{11}
\end{HTable} \\[1em]}
\visible<4->{\begin{HTable}{\ColorFour{11}}{v}{2}
\TE{00}{0}{\alert<6>{+}}{00} \\
\TE{01}{1}{\alert<7>{-}}{01} \\
\TE{10}{1}{\alert<8>{-}}{10} \\
\TE{11}{0}{\alert<9>{+}}{11}
\end{HTable}}\end{center}}
}
\only<5-9>{%
\BigSkip{}
\visible<5->{Notice what happens if we sum these tables.} \visible<6->{\alert<6>{Constructive} interference on term~\ket{00}.}  \visible<7->{\alert<7-9>{Destructive} interference on the the other three terms.}}}
\only<12->{%

We knew 
\[
\Hadamard(\ket{00}) = \frac{1}{2}\DQB{1}{1}{1}{1}
\]
and now we see, in another way, that \Hadamard{} is its own inverse:
\[
\Hadamard\left(\frac{1}{2}\DQB{1}{1}{1}{1}\right) = \ket{00}
\]
}
\only<10->{%
\SmallSkip{}
\[
\frac{\alt<10>{\Hadamard(\ket{\ColorOne{00}}) +
\Hadamard(\ket{\ColorTwo{01}}) +
\Hadamard(\ket{\ColorThree{10}}) +
\Hadamard(\ket{\ColorFour{11}})}{%
\Hadamard(
\ket{\ColorOne{00}} + \ket{\ColorTwo{01}} + \ket{\ColorThree{10}} + \ket{\ColorFour{11}}
)}}{2} 
= \ket{00}
\]}
    
\end{frame}

\begin{frame}{Deutsch--Jozsa Solution}{After \QState{2}}
\only<1-4>{%
\Vskip{-4em}\TwoUnequalColumns{0.51\textwidth}{0.49\textwidth}{%
\begin{center}
\Vskip{-3em}\adjustbox{valign=t, width=\textwidth}{\begin{quantikz}
\lstick{\ket{\TensSupProd{0}{n}}} & \qwbundle{n}\slice{\QState{0}} &  \gate{\TensSupProd{\Hadamard}{n}}\slice{\QState{1}} & \gate[wires=2][5em]{\mbox{$U_f$}}\gateinput{$x$}\gateoutput{$x$}\slice{\QState{2}} &\gate{\ColorThree{\TensSupProd{\Hadamard}{n}}}\slice{\QState{3}} & \meter{} \\
\lstick{\QZero{}} & \gate{\PauliX} &   \gate{\Hadamard}   &  \qw\gateinput{$y$}\gateoutput{\Xor{y}{f(x)}} &  & 
\end{quantikz}}
\end{center}
}{%
\Vskip{-2em}\begin{align*}
     \QState{2} &= \ColorOne{\RootTwoN{n}}\ColorOne{\SumBV{w}{n} \NegF{w}\ket{w}} \\
     \QState{3} &= \ColorThree{\TensSupProd{\Hadamard}{n}}(\QState{2}) 
\end{align*}
}%
\BigSkip{}
\[
P(n):  \ColorThree{\TensSupProd{\Hadamard}{n}\ket{w}
  = \RootTwoN{n}\SumBV{v}{n} \NegOneExp{\DotP{v}{w}} \ket{v}}
\]}
\Vskip{-2em}\only<5->{\Vskip{-3em}}\begin{align*}
    \visible<2->{\ColorOne{\sqrt{2^n}} \QState{3} & = \ColorThree{\TensSupProd{\Hadamard}{n}}
       \left(
       \ColorOne{\SumBV{w}{n} \NegF{w}\ket{w}}
       \right)} \visible<3->{= \ColorOne{\SumBV{w}{n} \NegF{w}} \ColorThree{\TensSupProd{\Hadamard}{n}} \ColorOne{\ket{w}} \\}
       \visible<4,5->{%
       \ColorThree{\sqrt{2^n}}\ColorOne{\sqrt{2^n}} \QState{3} &= \ColorOne{\SumBV{w}{n}\NegF{w}}
       \ColorThree{\SumBV{v}{n} \NegOneExp{\DotP{v}{\ColorOne{w}}} \ket{v}}\ \mbox{substitute \ColorThree{\TensSupProd{\Hadamard}{n}}} \\
       }
       \only<6->{%
    \visible<6,7->{   & = 
       \ColorThree{\SumBV{v}{n}}\ 
       \textcolor<6>{white}{\fbox{\ColorOne{\SumBV{w}{n}}
       \ColorOne{\NegF{w}}
       \ColorThree{\NegOneExp{\DotP{v}{\ColorOne{w}}}}}} \ \ColorThree{\ket{v}}
        \ \mbox{move \ColorThree{$\Sigma$} left}\\}
     \visible<8,9->{  &= \alert<10>{\fbox{\ColorOne{\SumBV{w}{n}\NegF{w}} \only<8>{\ColorThree{\NegOneExp{\DotP{0^n}{\ColorOne{w}}}}}}} \ColorThree{\ket{\TensSupProd{0}{n}}}
     + \fbox{\fontsize{3.5}{4}\selectfont\ColorOne{\SumBV{w}{n}}
       \ColorOne{\NegF{w}}
       \ColorThree{\NegOneExp{\DotP{0^{n}1}{\ColorOne{w}}}}} \ColorThree{\ket{\TensSupProd{0}{n-1}1}}
     \\ & + \ldots + 
     \fbox{\fontsize{3.5}{4}\selectfont\ColorOne{\SumBV{w}{n}}
       \ColorOne{\NegF{w}}
       \ColorThree{\NegOneExp{\DotP{1^{n}}{\ColorOne{w}}}}} \ColorThree{\ket{\TensSupProd{1}{n}}}
     }}
\end{align*}
\end{frame}

\section*{Measurement}

\begin{frame}{Analysis of the \ket{\TensSupProd{0}{n}} term}{Under the three conditions of interest}
\Vskip{-3em}\[
\ColorOne{\SumBV{w}{n}\NegF{w}} \ColorThree{\ket{00...0}}
\]
\begin{description}
  \visible<2->{\item[\Forall{w}{f(w)=0}] $\ColorOne{\SumBV{w}{n}\NegF{w}} = +2^{n}$}
  \visible<3->{\item[\Forall{w}{f(w)=1}] $\ColorOne{\SumBV{w}{n}\NegF{w}} = -2^{n}$}
\end{description}%
\visible<4->{%
In the above two cases, all weight is on the \ColorThree{\ket{00..0}} term and all other terms have no weight.  We developed this result before: we can only measure \ColorThree{\ket{00..0}}.}
\visible<5->{
\begin{description}
  \item[f(w) half 0, half 1] $\ColorOne{\SumBV{w}{n}\NegF{w}} = 2^{n}/2 - 2^{n}/2 = 0$
\end{description}}
\visible<6->{%
There is \emph{no weight} on the \ColorThree{\ket{00..0}} term so the weight must be on other terms.  It is impossible to measure \ColorThree{\ket{00..0}}}
    
\end{frame}

\begin{frame}{Deutsch--Jozsa}{Putting it all together}

\begin{center}
\Vskip{-3em}\adjustbox{valign=t, width=0.7\textwidth}{\begin{quantikz}
\lstick{\ket{\TensSupProd{0}{n}}} & \qwbundle{n}\slice{\QState{0}} &  \gate{\TensSupProd{\Hadamard}{n}}\slice{\QState{1}} & \gate[wires=2][5em]{\mbox{$U_f$}}\gateinput{$x$}\gateoutput{$x$}\slice{\QState{2}} &\gate{\TensSupProd{\Hadamard}{n}}\slice{\QState{3}} & \meter{} \\
\lstick{\QZero{}} & \gate{\PauliX} &   \gate{\Hadamard}   &  \qw\gateinput{$y$}\gateoutput{\Xor{y}{f(x)}} &  & 
\end{quantikz}}
\end{center}
\begin{itemize}
    \item Run the above circuit and measure the top $n$~qubits.
    \item If we measure \TensSupProd{0}{n} we declare $f(x)$ to be constant.
    \item Otherwise we declare $f(x)$ to be balanced.
\end{itemize}
    
\end{frame}

\section*{Proofs}

\subsection*{Dot product}

\begin{frame}{Theorem}{Equivalence of \DotP{v}{w} definitions}
    

\Vskip{-3em}\begin{theorem}
\TwoColumns{%
Given bit strings $v$ and $w$, each of length~$n$:
\Vskip{-2.5em}\begin{align*}
    v &= v_{1} v_{2} \ldots v_{n} \\
    w &= w_{1} w_{2} \ldots w_{n}
\end{align*}
}{%
Define P(n):  \Vskip{-1em}\[\oplus_{i=1}^{n}\ {v_i}{w_i} = \sum_{i=1}^{n}\ v_{i}{w_i}\mbox{ (mod 2)}\]
}
\[\Forall{n\geq 0}{P(n)}\]
\end{theorem}
\only<2>{%
\begin{description}
  \item[Base case $P(0)$:]
  \[\oplus_{i=1}^{0}\ {v_i}{w_i} = 0 =  \sum_{i=1}^{0}\ v_{i}{w_i}\mbox{ (mod 2)}\]
\end{description}}%
\end{frame}

\begin{frame}{Proof inductive step, show \Implies{P(k-1)}{P(k)}}{Equivalence of \DotP{v}{w} definitions}
$P(n):\  \oplus_{i=1}^{n}\ {v_i}{w_i} = \sum_{i=1}^{n}\ v_{i}{w_i}\mbox{ (mod 2)}$
\BigSkip{}
\only<1-6>{%
Case $v_{k}w_{k}=0$
\Vskip{-2em}\begin{align*}
\visible<1->{\alert<1-2,6>{P(k-1)  \longrightarrow} }
\visible<2->{\oplus_{i=1}^{k-1}\ {v_i}{w_i} &= \sum_{i=1}^{k-1}\ v_{i}{w_i}\mbox{ (mod 2)} \\}
\visible<3->{0 + \oplus_{i=1}^{k-1}\ {v_i}{w_i} &= \sum_{i=1}^{k-1}\ v_{i}{w_i}\mbox{ (mod 2)} + 0 \\}
\visible<4->{v_{k}w_{k} + \oplus_{i=1}^{k-1}\ {v_i}{w_i} &= \sum_{i=1}^{k-1}\ v_{i}{w_i}\mbox{ (mod 2)} + v_{k}w_{k} \\}
\visible<5->{\oplus_{i=1}^{k}\ {v_i}{w_i} &= \sum_{i=1}^{k}\ v_{i}{w_i}\mbox{ (mod 2)}}
\visible<6->{
\alert<6>{\longrightarrow P(k)}}
\end{align*}}
\only<7-11>{%
Case $v_{k}w_{k}=1$
\Vskip{-2em}\begin{align*}
\visible<7->{\alert<7,11>{P(k-1)  \longrightarrow}
\oplus_{i=1}^{k-1}\ {v_i}{w_i} &= \sum_{i=1}^{k-1}\ v_{i}{w_i}\mbox{ (mod 2)} \\}
\visible<8->{\Not{\oplus_{i=1}^{k-1}\ {v_i}{w_i}} &= 
\sum_{i=1}^{k-1}\ v_{i}{w_i} + 1\mbox{ (mod 2)} \\}
\visible<9->{\Xor{1}{\left(\oplus_{i=1}^{k-1}\ {v_i}{w_i}\right)} &= \sum_{i=1}^{k-1}\ v_{i}{w_i} + v_{k}w_{k}\mbox{ (mod 2)} \\}
\visible<10->{\oplus_{i=1}^{k}\ {v_i}{w_i} &= \sum_{i=1}^{k}\ v_{i}{w_i}\mbox{ (mod 2)}}
\visible<11->{\alert<11>{\longrightarrow P(k)}}
\end{align*}}
\only<12->{%
We have shown \Forall{n\geq 0}{P(n)}
\QED{}}
\end{frame}

\subsection*{Hadamard}

{
\def\V{\ColorTwo{\ensuremath{v}}}
\def\W{\ColorOne{\ensuremath{w}}}
\begin{frame}{Proof}{\TensSupProd{\Hadamard}{n} on states other than \TensSupProd{0}{n}}

\Vskip{-3em}\begin{theorem}
Given a basis state $\ket{\W}, \AllBits{\W}{n}$
\[
P(n): \TensSupProd{\Hadamard}{n}\ket{\W} = \RootTwoN{n}\SumBV{\V}{n} \NegOneExp{\DotP{\V}{\W}} \ket{\V}
\]
\end{theorem}
\begin{itemize}[<+->]
    \item Given any basis state \W, its Hadamard transform is the uniform superposition of all basis states of $n$~qubits.
    \item The coefficent of each state~\V{} is $\pm 1$, determined by its dot product with~\W.
    \item If we change any bit of \W, then at least one term in the sum will flip its coefficient.  The signature of $\pm 1$ on each term \ket{\V} is thus unique for a given~\W{}. This is true also because \TensSupProd{\Hadamard}{n} is its own inverse.
    \item Our proof proceeds by induction on $n$.
\end{itemize}
    
\end{frame}

\begin{frame}{Proof}{Base case}

When $n=0$
\begin{align*}
    \TensSupProd{\Hadamard}{0}\ket{\W} &= \RootTwoN{0}\SumBV{\V}{0} \NegOneExp{\DotP{\V}{\W}} \ket{\V} \\[1em]
    &= \frac{\NegOneExp{{0}}}{1} \\[1em]
    &= 1
\end{align*}
which is the \href{https://en.wikipedia.org/wiki/Hadamard_transform}{definition} of \TensSupProd{\Hadamard}{0}.
\end{frame}

\begin{frame}{Proof}{Induction step}
Sketch to show \Implies{P(k)}{P(k+1)}:
\begin{itemize}
    \item We will assume $P(k)$, which tells us how to compute \TensSupProd{\Hadamard}{k}\ket{\W} when \W{} is $k$~qubits long.
    \item We then consider the tensor product of \ket{\W} with a single-qubit state~\QState{}.
    \item Because we deal only with basis states, we must have exactly one of the following:
    \begin{itemize}
        \item $\QState{}=\QZero{}$
        \item $\QState{}=\QOne{}$
    \end{itemize}
    \item By cases we show that the formula correctly computes \TensSupProd{\Hadamard}{k+1}.
\end{itemize}

\end{frame}

\begin{frame}{Induction step, $\QState{}=\QZero{}$}

\begin{Reasoning}
\Reason{1}{Induction hypothesis}
\Reason{2}{Include one more qubit, case $\QState{}=\QZero{}$}
\Reason{3}{Substitution of $\Hadamard(\QZero{})=\QPlus{}$}
\Reason{4}{Scroll up}
\Reason{6}{Introduce coefficients with value $1$}
\Reason{7}{Scroll up}
\Reason{9}{Apply tensor product}
\Reason{10}{Rewrite to prove $P(k+1)$ when $\QState{}=\QZero{}$}
\end{Reasoning}

\ScrollProof{1}{4}{%
   \Next{\Four}{P(k)\longrightarrow \ColorOne{\sqrt{2^n}\  \TensSupProd{\Hadamard}{k}\ket{w}} &= \ColorOne{\SumBV{v}{k} \NegOneExp{\DotP{v}{w}} \ket{v}} \\}
    \Next{\Three}{%
    \ColorOne{\sqrt{2^n}}\ \TensProd{\ColorOne{\TensSupProd{\Hadamard}{k}\ket{w}}}{\ColorTwo{\Hadamard\QState{}}} &= \TensProd{\ColorOne{\SumBV{v}{k} \NegOneExp{\DotP{v}{w}} \ket{v}}}{\ColorTwo{\Hadamard\QState{}}} \\}
    \gdef\CarryOne{\ColorThree{\sqrt{2}}\ColorOne{\sqrt{2^n}}\ \TensProd{\ColorOne{\TensSupProd{\Hadamard}{k}\ket{w}}}{\ColorTwo{\Hadamard\QState{}}} &= \TensProd{\ColorOne{\SumBV{v}{k} \NegOneExp{\DotP{v}{w}} \ket{v}}}{\ColorThree{(\QZero{}+\QOne{})}} \\}
    \Last{\CarryOne}
}
\ScrollProof{5}{7}{%
\Next{\Three}{\CarryOne}
\gdef\CarryTwo{\ColorThree{\sqrt{2}}\ColorOne{\sqrt{2^n}}\ \TensProd{\ColorOne{\TensSupProd{\Hadamard}{k}\ket{w}}}{\ColorTwo{\Hadamard\QState{}}}&= \TensProd{\ColorOne{\SumBV{v}{k} \NegOneExp{\DotP{v}{w}} \ket{v}}}{\ColorThree{(\NegOneExp{\DotP{0}{0}}\QZero{}+\NegOneExp{\DotP{1}{0}}\QOne{})}} \\}
\Last{\CarryTwo}
}
\ScrollProof{8}{10}{%
\Next{\Three}{\CarryTwo}
\Next{\Two}{%
    &= \ColorOne{\SumBV{v}{k} \NegOneExp{\DotP{v}{w}}\ColorThree{\NegOneExp{\DotP{0}{0}}} \ket{v}}\ColorThree{\ket{0}} 
    \\ & +
    \ColorOne{\SumBV{v}{k} \NegOneExp{\DotP{v}{w}} \ColorThree{\NegOneExp{\DotP{1}{0}}} \ket{v}}\ColorThree{\ket{1}} \\}
\Next{\One}{%
   \ColorOne{\sqrt{2^{n\ColorThree{+1}}}}\ \ColorOne{\TensSupProd{\Hadamard}{k\ColorThree{+1}}\ket{w}\ColorTwo{\QZero{}}} &= \ColorOne{\SumBV{v}{k\ColorThree{+1}} \NegOneExp{\DotP{v}{w\ColorTwo{0}}} \ket{v}}}
}

    
\end{frame}

\begin{frame}{Induction step, $\QState{}=\QOne{}$}

\begin{Reasoning}
\Reason{1}{Induction hypothesis}
\Reason{2}{Include one more qubit, case $\QState{}=\QOne{}$}
\Reason{3}{Substitution of $\Hadamard(\QOne{})=\QMinus{}$}
\Reason{4}{Scroll up}
\Reason{6}{Introduce coefficients with value $1$}
\Reason{7}{Scroll up}
\Reason{9}{Apply tensor product}
\Reason{10}{Rewrite to prove $P(k+1)$ when $\QState{}=\QOne{}$}
\end{Reasoning}

\ScrollProof{1}{4}{%
   \Next{\Four}{P(k)\longrightarrow \ColorOne{\sqrt{2^n}\  \TensSupProd{\Hadamard}{k}\ket{w}} &= \ColorOne{\SumBV{v}{k} \NegOneExp{\DotP{v}{w}} \ket{v}} \\}
    \Next{\Three}{%
    \ColorOne{\sqrt{2^n}}\ \TensProd{\ColorOne{\TensSupProd{\Hadamard}{k}\ket{w}}}{\ColorTwo{\Hadamard\QState{}}} &= \TensProd{\ColorOne{\SumBV{v}{k} \NegOneExp{\DotP{v}{w}} \ket{v}}}{\ColorTwo{\Hadamard\QState{}}} \\}
    \gdef\CarryOne{\ColorThree{\sqrt{2}}\ColorOne{\sqrt{2^n}}\ \TensProd{\ColorOne{\TensSupProd{\Hadamard}{k}\ket{w}}}{\ColorTwo{\Hadamard\QState{}}} &= \TensProd{\ColorOne{\SumBV{v}{k} \NegOneExp{\DotP{v}{w}} \ket{v}}}{\ColorThree{(\QZero{}-\QOne{})}} \\}
    \Last{\CarryOne}
}
\ScrollProof{5}{7}{%
\Next{\Three}{\CarryOne}
\gdef\CarryTwo{\ColorThree{\sqrt{2}}\ColorOne{\sqrt{2^n}}\ \TensProd{\ColorOne{\TensSupProd{\Hadamard}{k}\ket{w}}}{\ColorTwo{\Hadamard\QState{}}}&= \TensProd{\ColorOne{\SumBV{v}{k} \NegOneExp{\DotP{v}{w}} \ket{v}}}{\ColorThree{(\NegOneExp{\DotP{0}{1}}\QZero{}+\NegOneExp{\DotP{1}{1}}\QOne{})}} \\}
\Last{\CarryTwo}
}
\ScrollProof{8}{10}{%
\Next{\Three}{\CarryTwo}
\Next{\Two}{%
    &= \ColorOne{\SumBV{v}{k} \NegOneExp{\DotP{v}{w}}\ColorThree{\NegOneExp{\DotP{0}{1}}} \ket{v}}\ColorThree{\ket{0}} 
    \\ & +
    \ColorOne{\SumBV{v}{k} \NegOneExp{\DotP{v}{w}} \ColorThree{\NegOneExp{\DotP{1}{1}}} \ket{v}}\ColorThree{\ket{1}} \\}
\Next{\One}{%
   \ColorOne{\sqrt{2^{n\ColorThree{+1}}}}\ \ColorOne{\TensSupProd{\Hadamard}{k\ColorThree{+1}}\ket{w}\ColorTwo{\QOne{}}} &= \ColorOne{\SumBV{v}{k\ColorThree{+1}} \NegOneExp{\DotP{v}{w\ColorTwo{1}}} \ket{v}}}
}

    
\end{frame}
}

\begin{frame}{Summary}{What have we learned?}

\begin{itemize}[<+->]
    \item Given a Deutsch--Jozsa problem of $n$~bits, it can take exponential time on a classical computer to determine if the function is constant or balanced.
    \item This same problem can be solved using a single query on a quantum computer.
    \item When the function is as promised, we obtain an exact solution in polynomial (constant) time.
    \item This problem serves to separate the complexity class \CompClass{P} from \CompClass{EQP}.
    \item The circuit that solves this problem develops full or zero amplitude on the \TensSupProd{0}{n} state if the function is constant or balanced, respectively.
    \item You will explore on your own the consequences of a broken promise.
    \item We learned how to compute the Hadamard transform of an arbitrary basis state of $n$~qubits.
\end{itemize}
    
\end{frame}

    
