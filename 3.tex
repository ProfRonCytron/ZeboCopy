\SetTitle{3}{Overview of computatonal complexity}{No energy lost or gained}{reverse}

\section{Overview}

\begin{frame}{Modeling computation}{Turing machines and complexity classes}
\begin{itemize}
    \item A \href{https://en.wikipedia.org/wiki/Turing_machine}{Turing machine}~(TM) is a simple conceptual device that formalizes the state-to-state nature of computation.
    \item Nobody really builds such a machine, but we say that anything that is computable can be computed on a TM.   
    \only<2>{
    \SmallSkip{}
    Well some people do build such machines:
    \begin{itemize}
        \item \href{https://www.youtube.com/watch?v=E3keLeMwfHY}{Using 35mm film}
        \item \href{https://www.youtube.com/watch?v=5_Hj5x6OWTM}{One with a connection to Washington University.}
    \end{itemize}}
    \item<3-> Here we can think of computation as a program we write. \only<3>{We have to make sure the language we use is clear about how many steps a given statement takes.}
    \item<4-> We can then talk about the number of steps it takes to solve a problem of some size~$n$.  You should already be familiar with \href{https://en.wikipedia.org/wiki/Asymptotic_computational_complexity}{asymptotic notation to express complexity}, especially the time an algorithm or problem takes.
\end{itemize}

\OnlyRemark{5}{We will use asymptotic complexity to describe the time and area needed for computation on a quantum computer.}
    
\end{frame}

\begin{frame}{What input provides a certain output?}{A common problem to consider}

\TwoColumns{%
\begin{center}

\only<1-3>{
\begin{GateBox}{2.5}{1}{1}
\BoxLabel{$f(x)$}
\Input{0}{$x$}
\Output{0}{$y$}

\end{GateBox}
}
\only<4-5>{
\begin{GateBox}{2.5}{1}{1}
\BoxLabel{$f(x)=2x+5$}
\Input{0}{?}
\Output{0}{105}

\end{GateBox}
}

\end{center}
\SmallSkip{}
\begin{itemize}
    \item What input value for $x$ produces a $y$ of interest?
    \item If we have the inverse of f we could compute $x=f^{-1}(y)$, but not all functions are invertible.
    \item Also some functions are invertible, but may take a long time to compute $x$.
    
\end{itemize}
}{

Consider the function $f(x)=2x+5$.  We can use algebra to compute $x=50$ when $f(x)=105$.

}
\end{frame}

\section{Factoring}

\begin{frame}{Famous problem for quantum computing}{Factoring a number that is the product of two large primes}
A common problem related to cryptography is: given an integer $n$, compute integers $p$ and $q$ such that $pq=n$.  Computing $n$ from $p$ and $q$ is easy, but computing $p$ and $q$ from $n$ is \emph{apparently} hard.

We don't know yet the true difficulty of this problem, but the best solutions on classical computers must try an exponential number of possibilties for $p$ and $q$.

As we will study, a quantum computer can solve this problem in polynomial time.
\end{frame}