% Copyright 2007 by Till Tantau
%
% This file may be distributed and/or modified
%
% 1. under the LaTeX Project Public License and/or
% 2. under the GNU Public License.
%
% See the file doc/licenses/LICENSE for more details.

%
% DO NOT USE THIS FILE AS A TEMPLATE FOR YOUR OWN TALKS¡!!
%
% Use a file in the directory solutions instead.
% They are much better suited.
%


\lecture[1]{Syntax versus Semantik}{lecture-text}

\subtitle{Text und seine Bedeutung}

\date{27. Oktober 2006}


\begin{document}

\begin{frame}
  \maketitle
\end{frame}


\section*{Ziele und Inhalt}

\begin{frame}{Die Lernziele der heutigen Vorlesung und der Übungen.} 
  \begin{enumerate}
  \item Die Begriffe Syntax und Semantik erklären können
  \item Syntaktische und semantische Elemente natürlicher Sprachen und
    von Programmiersprachen benennen können
  \item Die Begriffe Alphabet und Wort kennen
  \item Objekte als Worte kodieren können
  \end{enumerate}
\end{frame}

\begin{frame}\frametitle<presentation>{Gliederung}
  \tableofcontents
\end{frame}


\section{Was ist Syntax?}

\begin{frame}{Die zwei Hauptbegriffe der heutigen Vorlesung.}
  \begin{block}{Grobe Definition (Syntax)}
    Unter einer \alert{Syntax} verstehen wir \alert{Regeln}, nach denen
    Texte \alert{strukturiert} werden dürfen. 
  \end{block}
  \begin{block}{Grobe Definition (Semantik)}
    Unter einer \alert{Semantik} verstehen wir die Zuordnung von
    \alert{Bedeutung} zu Text.
  \end{block}
\end{frame}


\subsection[Syntax \protect\\ natürlicher Sprachen]{Syntax natürlicher Sprachen}

\begin{frame}{Beobachtungen zu einem ägyptischen Text.}
  \includegraphicscopyright[width=6cm]{beamerexample-lecture-pic3.jpg}
  {Copyright by Guillaume Blanchard, GNU Free Documentation License, Low Resultion}

  \begin{block}{Beobachtungen}
    \begin{itemize}
    \item Wir haben keine Ahnung, was der Text bedeutet.
    \item Es gibt aber \alert{Regeln}, die offenbar eingehalten wurden,
      wie »Hieroglyphen stehen in Zeilen«.
    \item Solche Regeln sind \alert{syntaktische Regeln} -- man kann sie
      überprüfen, ohne den Inhalt zu verstehen.
    \end{itemize}
  \end{block}
\end{frame}


\begin{frame}{Beobachtungen zu einem kyrillischen Text.}

  \includegraphicscopyright[width=6.75cm]{beamerexample-lecture-pic4.jpg}
  {Copyright by Cristian Chirita, GNU Free Documentation License, Low Resultion}

  \begin{block}{Beobachtungen}
    \begin{itemize}
    \item Wir haben keine Ahnung, was der Text bedeutet.
    \item Es gibt aber \alert{Regeln}, die offenbar eingehalten wurden.
    \item Wir kennen mehr Regeln als bei den Hieroglyphen.
    \end{itemize}
  \end{block}

  \begin{block}{Zur Diskussion}
    Welche syntaktischen Regeln fallen Ihnen ein, die bei dem Text
    eingehalten wurden?
  \end{block}
\end{frame}



\begin{frame}{Beobachtungen zu einem deutschen Text.}
  \begin{quotation}
    Informatiker lieben Logiker.
  \end{quotation}

  \bigskip
  \begin{block}{Beobachtungen}
    \begin{itemize}
    \item Auch hier werden viele syntaktische Regeln eingehalten.
    \item Es fällt uns aber \alert{schwerer}, diese zu erkennen.
    \item Der Grund ist, dass wir \alert{sofort über die Bedeutung
        nachdenken}. 
    \end{itemize}
  \end{block}
\end{frame}

\begin{frame}{Zur Syntax von natürlichen Sprachen.}
  \begin{itemize}
  \item 
    Die \alert{Syntax} einer natürlichen Sprache ist die Menge an
    \alert{Regeln}, nach denen Sätze gebildet werden dürfen. 
  \item 
    Die \alert{Bedeutung} oder der \alert{Sinn} der gebildeten Sätze
    ist dabei unerheblich.
  \item
    Jede Sprache hat ihre eigene Syntax; die Syntax verschiedener
    Sprachen ähneln sich aber oft.
  \item
    Es ist nicht immer klar, ob eine Regel noch zur Syntax gehört
    oder ob es schon um den Sinn geht.

    \ExampleInline{Substantive werden groß geschrieben.}
  \end{itemize}
\end{frame}

\subsection{Syntax von Programmiersprachen}

\begin{frame}[fragile]{Beobachtungen zu einem Programmtext.}
  
\begin{verbatim}
\def\pgfpointadd#1#2{%
  \pgf@process{#1}%
  \pgf@xa=\pgf@x%
  \pgf@ya=\pgf@y%
  \pgf@process{#2}%
  \advance\pgf@x by\pgf@xa%
  \advance\pgf@y by\pgf@ya}
\end{verbatim}
  \begin{block}{Beobachtungen}
    \begin{itemize}
    \item Der Programmtext sieht sehr kryptisch aus.
    \item Trotzdem gibt es offenbar wieder Regeln.
    \item So scheint einem Doppelkreuz eine Ziffer zu folgen und
      Zeilen muss man offenbar mit Prozentzeichen beenden.
    \end{itemize}
  \end{block}
\end{frame}


\begin{frame}[fragile]{Beobachtungen zu einem weiteren Programmtext.}
  
\begin{verbatim}
for (int i = 0; i < 100; i++)
  a[i] = a[i];
\end{verbatim}
  \begin{block}{Beobachtungen}
    \begin{itemize}
    \item Wieder gibt es Regeln, die eingehalten werden.
    \item Wieder fällt es uns \alert{schwerer}, diese zu erkennen, da
      wir \alert{sofort über den Sinn nachdenken}.
    \end{itemize}
  \end{block}
\end{frame}


\begin{frame}{Zur Syntax von Programmiersprachen}
  \begin{itemize}
  \item Die \alert{Syntax} einer Programmiersprache ist die
    \alert{Menge von Regeln}, nach der Programmtexte gebildet werden 
    dürfen.
  \item Die \alert{Bedeutung} oder der \alert{Sinn} der Programmtexte
    ist dabei egal.
  \item
    Jede Programmiersprache hat ihre eigene Syntax; die Syntax
    verschiedener Sprachen ähneln sich aber oft.
  \end{itemize}  
\end{frame}

\begin{frame}{5-Minuten-Aufgabe}
  Welche der folgenden Regeln sind Syntax-Regeln?
  \begin{enumerate}
  \item Bezeichner dürfen nicht mit einer Ziffer anfangen.
  \item Programme müssen in endlicher Zeit ein Ergebnis produzieren.
  \item Öffnende und schließende geschweifte Klammern  müssen
    »balanciert« sein. 
  \item Methoden von Null-Objekten dürfen nicht aufgerufen werden.
  \item Variablen müssen vor ihrer ersten Benutzung deklariert werden.
  \end{enumerate}  
\end{frame}


\subsection[Syntax\protect\\ logischer Sprachen]{Syntax logischer Sprachen}

\begin{frame}{Beobachtungen zu einer logischen Formel.}
  \begin{quotation}
    $p \to q \land \neg q$
  \end{quotation}

  \bigskip
  \begin{block}{Beobachtungen}
    \begin{itemize}
    \item Auch logische Formeln haben eine syntaktische Struktur.
    \item So wäre es \alert{syntaktisch falsch}, statt einem Pfeil zwei
      Pfeile zu benutzen.
    \item Es wäre aber \alert{syntaktisch richtig}, statt einem
      Negationszeichen zwei Negationszeichen zu verwenden.
    \end{itemize}
  \end{block}
\end{frame}

\begin{frame}{Zur Syntax von logischen Sprachen}
  \begin{itemize}
  \item Die \alert{Syntax} einer logischen Sprache ist die
    \alert{Menge von Regeln}, nach der Formeln gebildet werden 
    dürfen.
  \item Die \alert{Bedeutung} oder der \alert{Sinn} der Formeln
    ist dabei egal.
  \item
    Jede logische Sprache hat ihre eigene Syntax; die Syntax
    verschiedener Sprachen ähneln sich aber oft.
  \end{itemize}  
\end{frame}



\section{Was ist Semantik?}

\subsection[Semantik\protect\\ natürlicher Sprachen]{Semantik natürlicher Sprachen}

\begin{frame}{Was bedeutet ein Satz?}

  \begin{quotation}
    Der Hörsaal ist groß.
  \end{quotation}

  \bigskip
  \begin{itemize}
  \item Dieser Satz hat eine \alert{Bedeutung}.
  \item Eine \alert{Semantik} legt solche Bedeutungen fest.
  \item Syntaktisch falschen Sätzen wird im Allgemeinen keine
    Bedeutung zugewiesen.
  \end{itemize}
\end{frame}

\begin{frame}{Ein Satz, zwei Bedeutungen.}
  \begin{quotation}
    Steter Tropfen höhlt den Stein.
  \end{quotation}

  \bigskip
  \begin{itemize}
  \item Ein Satz kann \alert{mehrere Bedeutungen haben}, welche durch
    \alert{unterschiedliche Semantiken} gegeben sind.
  \item In der \alert{wortwörtlichen Semantik} sagt der Satz aus, dass
    Steine ausgehöhlte werden, wenn man jahrelang Wasser auf
    sie tropft.
  \item In der \alert{übertragenen Semantik} sagt der Satz aus, dass
    sich Beharrlichkeit auszahlt.
  \end{itemize}
\end{frame}

\begin{frame}{Die Semantik der Hieroglyphen}
  \includegraphicscopyright[height=8cm]{beamerexample-lecture-pic5.jpg}
  {Unknown Author, Public Domain, Low Resolution}
\end{frame}


\subsection{Semantik von Programmiersprachen}

\begin{frame}[fragile]{Was bedeutet ein Programm?}
\begin{verbatim}
for (int i = 0; i < 100; i++)
  a[i] = a[i];
\end{verbatim}
  \begin{itemize}
  \item Auch dieser Programmtext »bedeutet etwas«, wir »meinen etwas«
    mit diesem Text.
  \item Die \alert{Semantik der Programmiersprache} legt fest,
    was mit dem Programmtext gemeint ist.
  \end{itemize}
\end{frame}

\begin{frame}[fragile]{Ein Programm, zwei Bedeutungen.}
\begin{verbatim}
for (int i = 0; i < 100; i++)
  a[i] = a[i];
\end{verbatim}
  \begin{itemize}
  \item Ein Programmtext kann \alert{mehrere Bedeutungen haben},
    welche durch \alert{unterschiedliche Semantiken} gegeben sind.
  \item In der \alert{operationalen Semantik} bedeutet der
    Programmtext, dass die ersten einhundert Elemente eines Arrays
    \verb!a! nacheinander ihren eigenen Wert zugewiesen bekommen.
  \item In der \alert{denotationellen Semantik} bedeutet der
    Programmtext, dass nichts passiert.
  \end{itemize}
\end{frame}


\subsection[Semantik\protect\\ logischer Sprachen]{Semantik logischer Sprachen}




\section{Grundlage der Syntax: Text}

\begin{frame}{Eine mathematische Sicht auf Text.}
  \begin{itemize}
  \item Viele (aber nicht alle!) syntaktische Systeme bauen auf
    \alert{Text} auf.
  \item Auch solche Systeme, die nicht auf Text aufbauen, lassen sich
    trotzdem durch Text beschreiben. 
  \item Es ist deshalb nützlich, auf Text \text{Methoden der
      Mathematik} anwenden zu können.
  \item Im Folgenden wird deshalb die \alert{mathematische Sicht} auf
    Text eingeführt, die \alert{in der gesamten Theoretischen
      Informatik} genutzt wird.
  \end{itemize}  
\end{frame}


\subsection{Alphabete}

\begin{frame}{Formale Alphabete}
  \begin{definition}[Alphabet]
    Ein \alert{Alphabet} ist eine nicht-leere, endliche Menge von
    \alert{Symbolen} (auch \alert{Buchstaben} genannt). 
  \end{definition}
  
  \begin{itemize}
  \item Alphabete werden häufig mit griechischen Großbuchstaben
    bezeichnet, also $\Gamma$ oder~$\Sigma$. Manchmal auch mit
    lateinischen Großbuchstaben, also $N$ oder~$T$.
  \item Ein Symbol oder »Buchstabe« kann auch ein komplexes oder
    komisches »Ding« sein wie ein Pointer oder ein Leerzeichen.
  \end{itemize}

  \begin{examples}
    \begin{itemize}
    \item Die Groß- und Kleinbuchstaben
    \item Die Menge $\{0,1\}$ (bei Informatikern beliebt)
    \item Die Menge $\{A,C,G,T\}$ (bei Biologen beliebt)
    \item Die Zeichenmenge des UNICODE.
    \end{itemize}
  \end{examples}
\end{frame}


\subsection{Worte}


\begin{frame}{Formale Worte}
  \begin{definition}[Wort]
    Ein \alert{Wort} ist eine (endliche) Folge von Symbolen. 
  \end{definition}
  \begin{itemize}
    \item »Worte« sind im Prinzip dasselbe wie
      Strings. Insbesondere können in Worten Leerzeichen als Symbole
      auftauchen.
    \item Die Menge aller Worte über einem Alphabet $\Sigma$ hat einen
      besonderen Namen: $\Sigma^*$.
    \item 
      Deshalb schreibt man oft: »Sei $w \in \Sigma^*$, \dots«
    \item Es gibt auch ein \alert{leeres Wort}, abgekürzt
      $\epsilon$ oder $\lambda$, das dem String
      \texttt{\char`\"\char`\"} entspricht. 
    \end{itemize}
   
  \begin{examples}
    \begin{itemize}
    \item \texttt{Hallo}
    \item \texttt{TATAAAATATTA}
    \item $\epsilon$
    \item \texttt{Hallo Welt.}
    \end{itemize}
  \end{examples} 
\end{frame}


\begin{frame}{5-Minuten-Aufgabe}
  Die folgenden Aufgaben sind nach Schwierigkeit sortiert. Lösen Sie
  \alert{eine} der Aufgaben. 
  \begin{enumerate}
  \item
    Schreiben Sie alle Worte der Länge höchstens $2$ über dem Alphabet
    $\Sigma = \{0,1,*\}$ auf.
  \item
    Wie viele Worte der Länge $n$ über dem Alphabet $\Sigma =
    \{0,1,*\}$  gibt es?
  \item
    Wie viele Worte der Länge höchstens $n$ über einem Alphabet mit
    $q$ Buchstaben gibt es?
  \end{enumerate}
\end{frame}


\subsection{Sprachen}

\begin{frame}{Formale Sprachen}{Definition}
  \begin{itemize}
  \item Natürlichen Sprachen sind komplexe Dinge, bestehend aus
    Wörtern, ihrer Ausprache, einer Grammatik, Ausnahmen, Dialekten,
    und vielem mehr.
  \item Bei \alert{formalen Sprachen} vereinfacht man radikal.
  \item Formale Sprachen müssen weder sinnvoll noch interessant sein.      
  \end{itemize}

  \begin{definition}[Formale Sprache]
    Eine \alert{formale Sprache} ist eine (oft unendliche!) Menge von
    Worten für ein festes Alphabet.
  \end{definition}

  \begin{itemize}
  \item Statt \frqq formale Sprache\flqq\ sagt man einfach \frqq Sprache\flqq.
  \item Als Menge von Worten ist eine Sprache eine Teilmenge von
    $\Sigma^*$.
  \item 
    Deshalb schreibt man oft: \frqq Sei $L \subseteq \Sigma^*$,
    \dots\flqq
  \end{itemize}
\end{frame}

\begin{frame}{Formale Sprachen}{Einfache Beispiele}
  \begin{examples}
    \begin{itemize}
    \item Die Menge $\{AAA, AAC, AAT\}$ (endliche Sprache).
    \item Die Menge aller Java-Programmtexte (unendliche Sprache).
    \item Die Menge aller Basensequenzen, die \texttt{TATA} enthalten
      (unendliche Sprache).
    \end{itemize}
  \end{examples} 
\end{frame}

\begin{frame}{Formale Sprachen in der Medieninformatik}
  \begin{itemize}
  \item Ein Renderer produziert 3D-Bilder.
  \item Dazu erhält er eine \alert{Szenerie} als Eingabe.
  \item Diese Szenerie ist als \alert{Text}, also als ein \alert{Wort} gegeben.
  \item Eine \alert{Syntax} beschreibt die (formale) Sprache, die alle
    \alert{syntaktisch korrekten Szenerien}  enthält.
  \item Eine \alert{Semantik} beschreibt, was diese Beschreibungen bedeuten.
  \end{itemize}
\end{frame}

\begin{frame}[fragile]{Formale Sprachen in der Medieninformatik}{Das
    »Wort«, das eine Szenerie beschreibt\dots}
\only<presentation>{\scriptsize}
\begin{verbatim*}
global_settings { assumed_gamma 1.0 }

camera {
  location  <10.0, 10, -10.0>
  direction 1.5*z
  right     x*image_width/image_height
  look_at   <0.0, 0.0,  0.0>
}

sky_sphere { pigment { color rgb <0.6,0.7,1.0> } }

light_source {
  <0, 0, 0>            // light's position (translated below)
  color rgb <1, 1, 1>  // light's color
  translate <-30, 30, -30>
  shadowless
}

#declare i = 0; 
#declare Steps = 30;
#declare Kugel = sphere{<0,0,0>,0.5 pigment{color rgb<1,0,0>}};

#while(i<Steps)
    object{Kugel  translate<3,0,0> rotate <0,i * 360 / Steps, 0> }
  #declare i = i + 1;
#end
\end{verbatim*}
\end{frame}


\begin{frame}{Formale Sprachen in der Medieninformatik}{\dots\ und was es bedeutet.}
  \includegraphicscopyright[width=9.5cm]{beamerexample-lecture-pic2.jpg}
  {Copyright Matthias Kabel, GNU Free Documentation License, Low Resolution}
\end{frame}

\begin{frame}{Formale Sprachen in der Medieninformatik}{Komplexeres Beispielbild, das ein Renderer produziert.}
  \includegraphicscopyright[width=9.5cm]{beamerexample-lecture-pic1.jpg}
  {Copyright Giorgio Krenkel and Alex Sandri, GNU Free Documentation License, Low Resolution}
\end{frame}


\begin{frame}{Formale Sprachen in der Bioinformatik}
  \begin{itemize}
  \item In der Bioinformatik untersucht man unter anderem Proteine.
  \item Dazu erhält man \alert{Molekülbeschreibungen} als Eingabe.
  \item Eine solche ist auch ein \alert{Wort}.
  \item Eine \alert{Syntax} beschreibt die (formale) Sprache, die alle
    \alert{syntaktisch korrekten Molkülbeschreibungen}  enthält.
  \item Eine \alert{Semantik} beschreibt, was diese Beschreibungen bedeuten.
  \end{itemize}
\end{frame}


\begin{frame}[fragile]{Formale Sprachen in der Bioinformatik}
  {Das »Wort«, das ein Protein beschreibt\dots}
\only<presentation>{\tiny}  
\only<article>{\footnotesize}  
\begin{verbatim}
HEADER    HYDROLASE                               25-JUL-03   1UJ1              
TITLE     CRYSTAL STRUCTURE OF SARS CORONAVIRUS MAIN PROTEINASE                 
TITLE    2 (3CLPRO)
COMPND    MOL_ID: 1;                                                            
COMPND   2 MOLECULE: 3C-LIKE PROTEINASE;                                        
COMPND   3 CHAIN: A, B;                                                         
COMPND   4 SYNONYM: MAIN PROTEINASE, 3CLPRO;                                    
COMPND   5 EC: 3.4.24.-;                                                        
COMPND   6 ENGINEERED: YES                                                      
SOURCE    MOL_ID: 1;                                                            
SOURCE   2 ORGANISM_SCIENTIFIC: SARS CORONAVIRUS;                               
SOURCE   3 ORGANISM_COMMON: VIRUSES;                                            
SOURCE   4 STRAIN: SARS;                                                        
...
REVDAT   1   18-NOV-03 1UJ1    0                                                
JRNL        AUTH   H.YANG,M.YANG,Y.DING,Y.LIU,Z.LOU,Z.ZHOU,L.SUN,L.MO,          
JRNL        AUTH 2 S.YE,H.PANG,G.F.GAO,K.ANAND,M.BARTLAM,R.HILGENFELD,          
JRNL        AUTH 3 Z.RAO                                                        
JRNL        TITL   THE CRYSTAL STRUCTURES OF SEVERE ACUTE RESPIRATORY           
JRNL        TITL 2 SYNDROME VIRUS MAIN PROTEASE AND ITS COMPLEX WITH            
JRNL        TITL 3 AN INHIBITOR                                                 
JRNL        REF    PROC.NAT.ACAD.SCI.USA         V. 100 13190 2003              
JRNL        REFN   ASTM PNASA6  US ISSN 0027-8424                               
....
ATOM      1  N   PHE A   3      63.478 -27.806  23.971  1.00 44.82           N  
ATOM      2  CA  PHE A   3      64.607 -26.997  24.516  1.00 42.13           C  
ATOM      3  C   PHE A   3      64.674 -25.701  23.723  1.00 41.61           C  
ATOM      4  O   PHE A   3      65.331 -25.633  22.673  1.00 40.73           O  
ATOM      5  CB  PHE A   3      65.912 -27.763  24.358  1.00 44.33           C  
ATOM      6  CG  PHE A   3      67.065 -27.162  25.108  1.00 44.20           C  
ATOM      7  CD1 PHE A   3      67.083 -27.172  26.496  1.00 43.35           C  
ATOM      8  CD2 PHE A   3      68.135 -26.595  24.422  1.00 43.49           C  
ATOM      9  CE1 PHE A   3      68.140 -26.631  27.187  1.00 43.21           C  
ATOM     10  CE2 PHE A   3      69.210 -26.046  25.108  1.00 42.91           C  
ATOM     11  CZ  PHE A   3      69.216 -26.062  26.493  1.00 43.22           C  
ATOM     12  N   ARG A   4      64.007 -24.666  24.228  1.00 34.90           N  
ATOM     13  CA  ARG A   4      63.951 -23.376  23.543  1.00 37.71           C  
...
\end{verbatim}
\end{frame}

\begin{frame}[fragile]{Formale Sprachen in der Bioinformatik}
  {\dots\ und das Protein, das beschrieben wird.}

  \includegraphicscopyright[width=9.5cm]{beamerexample-lecture-pic6.jpg}
  {Copyright Till Tantau, Low Resultion}
\end{frame}


\section<article>{Zusammenfassung}
\section<presentation>*{Zusammenfassung}

\begin{frame}{Zusammenfassung}
  \begin{enumerate}
  \item Ein \alert{Wort} ist eine Folge von Symbolen aus einem
    \alert{Alphabet}. 
  \item Eine \alert{Syntax} besteht aus Regeln, nach denen
    Worte (Texte) gebaut werden dürfen.
  \item Eine \alert{Semantik} legt fest, was Worte \alert{bedeuten}.
  \item Eine \alert{formale Sprache} ist eine Menge von Worten
    über einem Alphabet.
  \end{enumerate}
\end{frame}

\end{document}





