\SetTitle{24}{Simon's problem}{Harder probabilistically than Deutsch--Jozsa}{24}

\begin{frame}{Overview}{What will we study?}

\begin{itemize}
    \item Review of Deutsch--Jozsa
    \item A good probabilistic algorithm for Deutsch--Jozsa
    \item Simon's problem
    \item Difficulty of a classical solution
    \item Quantum computing solution
\end{itemize}

\end{frame}

\begin{frame}{Deutsch--Jozsa revisited}{We can solve this efficiently if we give up an exact solution}

\TwoColumns{%
\only<1-13>{%
\begin{itemize}[<+->]
    \item Given $f(x): \BinAlph{n}\mapsto \Set{0,1}$. 
    \item Is $f(x)$
    \begin{description}
        \item[\ColorOne{constant}] $f(x)$ same result for all inputs
        \item[\ColorTwo{balanced}]  $f(x)$ same result for half of its inputs
    \end{description}
    \item Exact solution in $\Theta(2^{n})$ time classically
    \item Exact solution in $\Theta(1)$ time on a quantum computer
\end{itemize}
}%
\only<14-16>{%
\ColorTwo{Suppose $f(x)$ is balanced}
\begin{itemize}
    \item<14-> \alert<14>{Half the time we determine this by $f(x)\not=f(y)$}
    \item<15-> \alert<15>{The other half of the time we are right one-third of the time.}
\end{itemize}
\[ \visible<14->{\frac{1}{2}\left(1\right)}\visible<15->{+ \frac{1}{2}\left(\frac{1}{3}\right)}  \visible<16->{= \frac{2}{3}}\]
}%
\only<17->{%
\ColorOne{Suppose $f(x)$ is constant}
\begin{itemize}
    \item<17-> \alert<17>{We never see $f(x)\not=f(y)$}
    \item<18-> \alert<15>{We correctly say \ColorOne{constant} $\frac{2}{3}$ of the time.}
\end{itemize}
\only<19->{%
This algorithm produces the correct answer with probability~$\frac{2}{3}$.
\QED{}}
}
}{%
\begin{itemize}[<+->]
    \item We can do well if we are willing to take a chance.
    \item Choose $x$ and $y$ randomly from \BinAlph{n}.
    \item \alert<14>{If $f(x)\not=f(y)$ say \ColorTwo{balanced}}
    \item Otherwise
    \begin{itemize}
        \item \alert<15,17>{With probability $\frac{1}{3}$ say \ColorTwo{balanced}}
        \item \alert<18>{With probability $\frac{2}{3}$ say \ColorOne{constant}}
    \end{itemize}
    \item Claim: This produces the correct answer $\frac{2}{3}$ of the time.
\end{itemize}
}
\end{frame}

\begin{frame}{For Deutsch--Jozsa, can we do better?}{We classically improve our chances exponentially with each query}

\begin{itemize}[<+->]
    \item Suppose we are willing to evaluate $f(x)$ $k$ times.
    \item We then say the function is constant iff all evaluations return the same result.
    \item To simplify analysis each $x$ is drawn randomly and uniformly from \BinAlph{n}.
    \item If $f(x)$ is \ColorOne{balanced} what is the probability that after $k$ queries we have seen only the same result?
    \begin{itemize}
        \item After the first query, each subsequent query has probability $\frac{1}{2}$ of having the same result.
        \item The probability of seeing $k-1$ same results is then \[\left(\frac{1}{2}\right)^{k-1} =\ \frac{1}{2^{k-1}}\]
    \end{itemize}
    \item Our certainty that $f(x)$ is constant improves exponentially with each query that returns the same value.
\end{itemize}
    
\end{frame}

\begin{frame}{Simon's problem}{Is difficult probabilistically}
\begin{itemize}[<+->]
    \item Our joy at separating \CompClass{P} from \CompClass{EQP} is tempered when we realize we can solve Deutsch--Jozsa with high probability using a small number of queries.
    \item Simon's problem considers 
    \begin{itemize}
        \item a function $f: \BinAlph{n}\mapsto \BinAlph{n-1}$
        \item a secret $\AllBits{s}{n} \not= 0^{n}$
    \end{itemize}
    \item with the following property:
    \[\Forall{\AllBits{x}{n}}{f(x)=f(\Xor{x}{s})} \]
    \item In other words, each \AllBits{x}{n} has a \emph{buddy}~\Xor{x}{s} whose value in $f$ is identical.
    \item Note that the buddy of $x$'s buddy is $x$, because
    $\Forall{x}{\Xor{(\Xor{x}{s})}{s}=x}$.
    \item \alert{Simon's problem is to determine the value of~$s$.}
\end{itemize}
\end{frame}

{
\def\R#1#2#3{\alert<#1>{#2} & \alert<#1>{#3}}
\begin{frame}{Simon's problem}{Example $n=3$}
What is $s$?  \visible<4->{101}
\TwoColumns{%
\begin{center}
    \begin{tabular}{c|c}
        $x$ & $f(x)$ \\ \hline
         \R{2-3,12}{000}{01} \\
        \R{4-5,11}{001}{00} \\
         \R{6-7,14}{010}{11} \\
         \R{8-9,13}{011}{10} \\
         \R{5,11}{100}{00} \\
         \R{3,12}{101}{01} \\
         \R{9,13}{110}{10} \\
         \R{7,14}{111}{11}
    \end{tabular}
\end{center}
}{%
\only<2-9>{%
\begin{itemize}
    \item<3-> $\Xor{000}{101}=101$
    \item<4-> $\Xor{001}{101}=100$
    \item<6-> $\Xor{010}{101}=111$
    \item<8-> $\Xor{011}{101}=110$
\end{itemize}
}%
\only<10->{%
\begin{center}
Buddies \\[1em]
    \begin{tabular}{c|cccc}
      $f(x)$   & \alert<11>{00}  & \alert<12>{01} & \alert<13>{10} & \alert<14>{11}  \\ \hline
       $x$     & 001 & 000 & 011 & 010 \\
       buddy   & 100 & 101 & 110 & 111
    \end{tabular}
\end{center}
}
}
\end{frame}}

\begin{frame}{Before we go further, is this problem difficult probabilistically?}{It is related to the birthday problem}

\begin{itemize}[<+->]
    \item Consider a year of $d=2^{n-1}$ days (function values).
    \item A room of $2^{n}$ people (function inputs).
    \item In the room exactly one pair (buddies) share the same birthday.
    \item We need to find two people with the same birthday.   Once we know their \Quote{names} we can compute the secret~$s$.
    \item For an exact answer, we may need to examine $2^{n-1}+1$ people worst-case, as with the Deutsch--Jozsa problem.
    \item For a probabilistic approach, the \href{https://en.wikipedia.org/wiki/Birthday_problem\#Probability_of_a_shared_birthday_(collision)}{shared-birthday problem} defines $f(p,d)$  as the approximate number of queries needed to find~2 identical values among~$d$ possible values with probability $p$:
    \[ f(p,d)= \sqrt{2d\times \ln\left(\frac{1}{1-p}\right)}\]
\end{itemize}
\end{frame}

\begin{frame}{How many queries?}{To find a pair of buddies}
\Vskip{-3.5em}\begin{itemize}[<+->]
\item Given $d$ possible values, we need approximately this many queries to find a pair of buddies with probability $p$:
 \Vskip{-1em}\[ f(p,d)= \sqrt{2d\times \ln\left(\frac{1}{1-p}\right)}\]
\item We have $2^{n-1}$ possible values, so
\Vskip{-1em}\[ f(p,2^{n-1})= \sqrt{2^{n} \ln\left(\frac{1}{1-p}\right)}\]
\item For any fixed probability $p$, or more generally when $p$ is not a function of $n$, this becomes~$\Omega(2^n)$.
\item We thus expect we will need an \emph{exponential} number of queries to find one pair of buddies for this problem to find the solution with probability~$p$.
\end{itemize}

\end{frame}