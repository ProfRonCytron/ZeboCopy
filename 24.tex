\SetTitle{24}{Simon's problem}{Harder probabilistically than Deutsch--Jozsa}{24}

\begin{frame}{Overview}{What will we study?}

\begin{itemize}[<+->]
    \item Review of Deutsch--Jozsa
    \item A good probabilistic algorithm for Deutsch--Jozsa
    \item Simon's problem
\end{itemize}

\end{frame}

\begin{frame}{Deutsch--Jozsa revisited}{We can solve this efficiently if we give up an exact solution}

\TwoColumns{%
\only<1-13>{%
\begin{itemize}[<+->]
    \item Given $f(x): \BinAlph{n}\mapsto \Set{0,1}$. 
    \item Is $f(x)$
    \begin{description}
        \item[\ColorOne{constant}] $f(x)$ same result for all inputs
        \item[\ColorTwo{balanced}]  $f(x)$ same result for half of its inputs
    \end{description}
    \item Exact solution in $\Theta(2^{n})$ time classically
    \item Exact solution in $\Theta(1)$ time in a quantum computer
\end{itemize}
}%
\only<14-16>{%
\ColorTwo{Suppose $f(x)$ is balanced}
\begin{itemize}
    \item<14-> \alert<14>{Half the time we determine this by $f(x)\not=f(y)$}
    \item<15-> \alert<15>{The other half of the time we are right one-third of the time.}
\end{itemize}
\[ \visible<14->{\frac{1}{2}\left(1\right)}\visible<15->{+ \frac{1}{2}\left(\frac{1}{3}\right)}  \visible<16->{= \frac{2}{3}}\]
}%
\only<17->{%
\ColorOne{Suppose $f(x)$ is constant}
\begin{itemize}
    \item<17-> \alert<17>{We never see $f(x)\not=f(y)$}
    \item<18-> \alert<15>{We correctly say \ColorOne{constant} $\frac{2}{3}$ of the time.}
\end{itemize}
\only<19->{%
This algorithm produces the correct answer with probability~$\frac{2}{3}$.
\QED{}}
}
}{%
\begin{itemize}[<+->]
    \item We can do well if we are willing to take a chance.
    \item Choose $x$ and $y$ randomly from \BinAlph{n}.
    \item \alert<14>{If $f(x)\not=f(y)$ say \ColorTwo{balanced}}
    \item Otherwise
    \begin{itemize}
        \item \alert<15,17>{With probability $\frac{1}{3}$ say \ColorTwo{balanced}}
        \item \alert<18>{With probability $\frac{2}{3}$ say \ColorOne{constant}}
    \end{itemize}
    \item Claim: This produces the correct answer $\frac{2}{3}$ of the time.
\end{itemize}
}
\end{frame}

\begin{frame}{For Deutsch--Jozsa, can we do better?}{We classically improve our chances exponentially with each query}

\begin{itemize}[<+->]
    \item Suppose we are willing to evaluate $f(x)$ $k$ times.
    \item We then say the function is constant iff all evaluations return the same result.
    \item To simplify analysis each $x$ is drawn randomly and uniformly from \BinAlph{n}.
    \item If $f(x)$ is \ColorOne{balanced} what is the probability that after $k$ queries we have seen only the same result?
    \begin{itemize}
        \item After the first query, each subsequent query has probability $\frac{1}{2}$ of having the same result.
        \item The probability of seeing $k-1$ same results is then \[\left(\frac{1}{2}\right)^{k-1} =\ \frac{1}{2^{k-1}}\]
    \end{itemize}
    \item Our certainty that $f(x)$ is constant improves exponentially with each query that returns the same value.
\end{itemize}
    
\end{frame}

\begin{frame}{Simon's problem}{Is difficult probabilistically}
\begin{itemize}[<+->]
    \item Our joy at separating \CompClass{P} from \CompClass{EQP} is tempered when we realize we can solve Deutsch--Jozsa with high probability using a small number of queries.
    \item Simon's problem considers 
    \begin{itemize}
        \item a function $f: \BinAlph{n}\mapsto \BinAlph{n-1}$
        \item a secret $\AllBits{s}{n} \not= 0^{n}$
    \end{itemize}
    \item with the following property:
    \[\Forall{\AllBits{x}{n}}{f(x)=f(\Xor{x}{s})} \]
    \item In other words, each \AllBits{x}{n} has a \emph{buddy}~\Xor{x}{s} whose value in $f$ is identical.
    \item Note that the buddy of $x$'s buddy is $x$, because
    $\Forall{x}{\Xor{(\Xor{x}{s})}{s}=x}$.
    \item \alert{Simon's problem is to determine the value of~$s$.}
\end{itemize}
\end{frame}

\begin{frame}{Simon's problem}{Example}
\end{frame}