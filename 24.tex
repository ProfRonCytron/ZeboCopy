\SetTitle{24}{Simon's problem}{Harder probabilistically than Deutsch--Jozsa}{24}

\begin{frame}{Overview}{What will we study?}

\begin{itemize}
    \item Review of Deutsch--Jozsa
    \item A good probabilistic algorithm for Deutsch--Jozsa
    \item Simon's problem
    \item Difficulty of a classical solution
    \item Quantum computing solution
    \item An efficient solution to this problem involves both a quantum computer and some processing that would typically be performed on a classical computer.
    \item It has the typical rotate--compute--rotate form of a quantum computation.
    \item It features intentional partial measurement.
    \item It is highly likely to yield results quickly.
    \item However, it is not \emph{guaranteed} to provide an answer in any finite amount of time.
\end{itemize}

\end{frame}

\begin{frame}{Deutsch--Jozsa revisited}{We can solve this efficiently if we give up an exact solution}

\TwoColumns{%
\only<1-13>{%
\begin{itemize}[<+->]
    \item Given $f(x): \BinAlph{n}\mapsto \Set{0,1}$. 
    \item Is $f(x)$
    \begin{description}
        \item[\ColorOne{constant}] $f(x)$ same result for all inputs
        \item[\ColorTwo{balanced}]  $f(x)$ same result for half of its inputs
    \end{description}
    \item Exact solution in $\Theta(2^{n})$ time classically
    \item Exact solution in $\Theta(1)$ time on a quantum computer
\end{itemize}
}%
\only<14-16>{%
\ColorTwo{Suppose $f(x)$ is balanced}
\begin{itemize}
    \item<14-> \alert<14>{Half the time we determine this by $f(x)\not=f(y)$}
    \item<15-> \alert<15>{The other half of the time we are right one-third of the time.}
\end{itemize}
\[ \visible<14->{\frac{1}{2}\left(1\right)}\visible<15->{+ \frac{1}{2}\left(\frac{1}{3}\right)}  \visible<16->{= \frac{2}{3}}\]
}%
\only<17->{%
\ColorOne{Suppose $f(x)$ is constant}
\begin{itemize}
    \item<17-> \alert<17>{We never see $f(x)\not=f(y)$}
    \item<18-> \alert<15>{We correctly say \ColorOne{constant} $\frac{2}{3}$ of the time.}
\end{itemize}
\only<19->{%
This algorithm produces the correct answer with probability~$\frac{2}{3}$.
\QED{}}
}
}{%
\begin{itemize}[<+->]
    \item We can do well if we are willing to take a chance.
    \item Choose $x$ and $y$ randomly from \BinAlph{n}.
    \item \alert<14>{If $f(x)\not=f(y)$ say \ColorTwo{balanced}}
    \item Otherwise
    \begin{itemize}
        \item \alert<15,17>{With probability $\frac{1}{3}$ say \ColorTwo{balanced}}
        \item \alert<18>{With probability $\frac{2}{3}$ say \ColorOne{constant}}
    \end{itemize}
    \item Claim: This produces the correct answer $\frac{2}{3}$ of the time.
\end{itemize}
}
\end{frame}

\begin{frame}{For Deutsch--Jozsa, can we do better?}{We classically improve our chances exponentially with each query}

\begin{itemize}[<+->]
    \item Suppose we are willing to evaluate $f(x)$ $k$ times.
    \item We then say the function is constant iff all evaluations return the same result.
    \item To simplify analysis each $x$ is drawn randomly and uniformly from \BinAlph{n}.
    \item If $f(x)$ is \ColorOne{balanced} what is the probability that after $k$ queries we have seen only the same result?
    \begin{itemize}
        \item After the first query, each subsequent query has probability $\frac{1}{2}$ of having the same result.
        \item The probability of seeing $k-1$ same results is then \[\left(\frac{1}{2}\right)^{k-1} =\ \frac{1}{2^{k-1}}\]
    \end{itemize}
    \item Our certainty that $f(x)$ is constant improves exponentially with each query that returns the same value.
\end{itemize}
    
\end{frame}

\begin{frame}{Simon's problem}{Is difficult probabilistically}
\begin{itemize}[<+->]
    \item Our joy from separating \CompClass{P} from \CompClass{EQP} is tempered when we realize we can solve Deutsch--Jozsa with high probability using a small number of queries.
    \item Simon's problem considers 
    \begin{itemize}
        \item a function $f: \BinAlph{n}\mapsto \BinAlph{n-1}$
        \item a secret $\AllBits{s}{n} \not= 0^{n}$
    \end{itemize}
    \item with the following property:
    \[\Forall{\AllBits{x}{n}}{f(x)=f(\Xor{x}{s})} \]
    \item In other words, each \AllBits{x}{n} has a \emph{buddy}~\Xor{x}{s} whose value in $f$ is identical.
    \item Note that the buddy of $x$'s buddy is $x$, because
    $\Forall{x}{\Xor{(\Xor{x}{s})}{s}=x}$.
    \item \alert{Simon's problem is to determine the value of~$s$.}
\end{itemize}
\end{frame}

{
\def\R#1#2#3{\alert<#1>{#2} & \alert<#1>{#3}}
\begin{frame}{Simon's problem}{Example $n=3$}
What is $s$?  \visible<4->{101}
\TwoColumns{%
\begin{center}
    \begin{tabular}{c|c}
        $x$ & $f(x)$ \\ \hline
         \R{2-3,12}{000}{01} \\
        \R{4-5,11}{001}{00} \\
         \R{6-7,14}{010}{11} \\
         \R{8-9,13}{011}{10} \\
         \R{5,11}{100}{00} \\
         \R{3,12}{101}{01} \\
         \R{9,13}{110}{10} \\
         \R{7,14}{111}{11}
    \end{tabular}
\end{center}
}{%
\only<2-9>{%
\begin{itemize}
    \item<3-> $\Xor{000}{101}=101$
    \item<4-> $\Xor{001}{101}=100$
    \item<6-> $\Xor{010}{101}=111$
    \item<8-> $\Xor{011}{101}=110$
\end{itemize}
}%
\only<10->{%
\begin{center}
Buddies \\[1em]
    \begin{tabular}{c|cccc}
      $f(x)$   & \alert<11>{00}  & \alert<12>{01} & \alert<13>{10} & \alert<14>{11}  \\ \hline
       $x$     & 001 & 000 & 011 & 010 \\
       buddy   & 100 & 101 & 110 & 111
    \end{tabular}
\end{center}
}
}
\end{frame}}

\begin{frame}{Before we go further, is this problem difficult probabilistically?}{It is related to the birthday problem}

\begin{itemize}[<+->]
    \item Consider a year of $d=2^{n-1}$ days (function values).
    \item A room of $2^{n}$ people (function inputs).
    \item In the room exactly one pair (buddies) share the same birthday.
    \item We need to find two people with the same birthday.   Once we know their \Quote{names} we can compute the secret~$s$.
    \item For an exact answer, we may need to examine $2^{n-1}+1$ people worst-case, as with the Deutsch--Jozsa problem.
    \item For a probabilistic approach, the \href{https://en.wikipedia.org/wiki/Birthday_problem\#Probability_of_a_shared_birthday_(collision)}{shared-birthday problem} defines $f(p,d)$  as the \emph{approximate} number of queries needed to find~2 identical values among~$d$ possible values with probability $p$:
    \[ f(p,d)= \sqrt{2d\times \ln\left(\frac{1}{1-p}\right)}\]
\end{itemize}
\end{frame}

\begin{frame}{How many queries?}{To find a pair of buddies}
\Vskip{-3.5em}\begin{itemize}[<+->]
\item Given $d$ possible values, we need approximately this many queries to find a pair of buddies with probability $p$:
 \Vskip{-1em}\[ f(p,d)= \sqrt{2d\times \ln\left(\frac{1}{1-p}\right)}\]
\item We have $2^{n-1}$ possible values, so
\Vskip{-1em}\[ f(p,2^{n-1})= \sqrt{2^{n} \ln\left(\frac{1}{1-p}\right)}\]
\item For any fixed probability $p$, or more generally when $p$ is not a function of $n$, this becomes~$\Omega(2^n)$.
\item We thus expect we will need an \emph{exponential} number of queries to find one pair of buddies for this problem to find the solution with probability~$p$.
\end{itemize}

\end{frame}

\begin{frame}{Oracle}{Wider inputs and outputs than before}
\Vskip{-4em}\TwoColumns{%
\Vskip{-3em}\begin{center}
    \begin{GateBox}{2}{4}{8}
    \draw[dashed,blue] (-1,2) -- ++(5,0);
    \BoxLabel{\raisebox{2em}{$U_f$}}
    \visible<1->{%
    \Input{0}{$x_1$}\Output{0}{$x_1$}
    \Input{1}{$x_2$}\Output{1}{$x_2$}
    \Input{2}{\RVDots}\Output{2}{\RVDots}
    \Input{3}{$x_n$}\Output{3}{$x_n$}%
    }%
    \visible<2->{%
    \Input{4}{$b_1$}
    \Input{5}{$b_2$}
    \Input{6}{\RVDots}
    \Input{7}{$b_{n-1}$}}
    \visible<3->{%
    \Output{4}{$\Xor{b_1}{f(x)_{1}}$}
    \Output{5}{$\Xor{b_2}{f(x)_{2}}$}
    \Output{6}{\RVDots}
    \Output{7}{$\Xor{b_{n-1}}{f(x)_{n-1}}$}}
    \end{GateBox}
\end{center}
}{%
\begin{itemize}[<+->]
    \item There are $n$ qubits for the input $x$, which $U_f$ sends to the top $n$~qubits of the output.
    \item There are also $n-1$ qubits for an input $b$.
    \item The oracle places \Xor{b}{f(x)} on the bottom $n-1$~qubits.
\end{itemize}
}

\BigSkip{}
\begin{itemize}
    \item<1-> Create the superposition of all inputs for $x$.
    \item<2-> Supply \QZero{} for each qubit of $b$.
    \item<4-> Result is the superposition of each \AllBits{x}{n} and $f(x)$!
\end{itemize}
\end{frame}

\begin{frame}{Solution for Simon's problem}{Circuit and analysis}
\begin{center}
\Vskip{-3em}\adjustbox{valign=t, width=0.7\textwidth}{\begin{quantikz}
\lstick{\ket{\TensSupProd{0}{n}}} & \qwbundle{\alert<1>{n}} & \qw\slice{\alert<4>{\QState{0}}}& \gate{\alert<3>{\TensSupProd{\Hadamard}{n}}}\slice{\alert<5-9>{\QState{1}}} & \gate[wires=2][5em]{\mbox{$U_f$}}\gateinput{$x$}\gateoutput{$x$}\slice{\alert<10-18>{\QState{2}}} &\qw\slice{\alert<19>{\QState{3}}}  & \gate{\alert<3>{\TensSupProd{\Hadamard}{n}}}\slice{\QState{4}} & \meter{} & \qw \\
\lstick{\alert<2>{\ket{\TensSupProd{0}{n-1}}}} &  \qwbundle{\alert<1>{n-1}} & \qw & \qw   &  \qw\gateinput{$b$}\gateoutput{\Xor{b}{f(x)}} & \meter{} & \qw & \qw & \qw
\end{quantikz}}
\end{center}
\only<1-3>{%
\begin{itemize}
    \item<1-> There are $n$~qubits at the top and $n-1$ qubits at the bottom.
    \item<2-> The bottom inputs are all \QZero{}.
    \item<3-> We perform an $n$-way Hadamard transformation on the top qubits before and after the oracle.
\end{itemize}
}%
\Vskip{-2.5em}
\ScrollProof{4}{8}{%
\Next{\Four}{\QState{0} &= \ket{\TensSupProd{0}{n}}\ket{\TensSupProd{0}{n-1}} \\}
\Next{\Three}{\QState{1} &= \TensProd{\TensSupProd{\Hadamard}{n}\ket{\TensSupProd{0}{n}}}{\ket{\TensSupProd{0}{n-1}}} \\}
\Next{\Two}{ &= \RootTwoN{n}\TensProd{\SumBV{x}{n}\ket{x}}{\ket{\TensSupProd{0}{n-1}}} \\}
\gdef\CarryOne{\sqrt{2^n}\QState{1} &= \SumBV{x}{n}\ket{x}\ket{\TensSupProd{0}{n-1}}}
\Last{\CarryOne}
}
\ScrollProof{9}{12}{%
\Next{\Four}{\CarryOne{} \\}
\Next{\Three}{\sqrt{2^n}\QState{2} = \sqrt{2^n}U_{f}(\QState{1}) &= U_{f}\left(\SumBV{x}{n}\ket{x}\ket{\TensSupProd{0}{n-1}}\right) \\}
\gdef\CarryTwo{\sqrt{2^n}\QState{2} &= \SumBV{x}{n}\ U_{f}\left(\ket{x}\ket{\TensSupProd{0}{n-1}}\right)}
\Last{\CarryTwo}
}
\ScrollProof{13}{16}{%
\Next{\Four}{\CarryTwo \\}
\Next{\Three}{ &= \SumBV{}{n}\ \ket{x}\ket{\Xor{0^{n-1}}{f(x)}} \\}
\gdef\CarryThree{\QState{2} &= \RootTwoN{n}\SumBV{x}{n}\ \ket{x}\ket{f(x)}}
\Last{\CarryThree}
}
\ScrollProof{17}{19}{%
\Next{\Three}{\CarryThree}
}
\only<17->{%
\Vskip{-2.5em}\begin{itemize}
    \item<17-> This is a remarkable state in that it contains the uniform superposition of \emph{every possible} input $x$ (the top~$n$ qubits) and its evaluation $f(x)$ (the bottom~$n-1$ qubits).
    \item<18-> Measurement of all qubits will yield a randomly chosen \TensProd{x}{f(x)}, but this doesn't tell us anything about $x$'s buddy.
    \item<19-> What happens at \QState{3}, where we measure \only{only} the bottom qubits?
\end{itemize}
}
\end{frame}
{
\def\R#1#2#3{\alert<#1>{#2} & \alert<#1>{#3}}
\begin{frame}{\vrule width0pt depth2.2em Example measuring $\QState{2} = \RootTwoN{n}\SumBV{x}{n}\ \ket{x}\ket{f(x)}$}

\Vskip{-4em}\TwoUnequalColumns{0.2\textwidth}{0.8\textwidth}{%
\begin{center}
    \begin{tabular}{c|c}
        $x$ & $f(x)$ \\ \hline
         \R{8-9}{000}{01} \\
         \R{1,10}{001}{00} \\
         \R{5}{010}{11} \\
         \R{7}{011}{10} \\
         \R{3,10}{100}{00} \\
         \R{8-9}{101}{01} \\
         \R{7}{110}{10} \\
         \R{2,4}{111}{11}
    \end{tabular}
\end{center}
}{%
\only<1-5>{%
\begin{itemize}
  \item<1-> Each measurement of \QState{2} will yield a random $\ket{x}\ket{f(x)}$.
  \item<2-> We run this circuit many times, but each outcome is random.
  \item<4-> And we may see a measurement we have seen before.
  \item<5-> This is no better than classically evaluating $f(x)$.
\end{itemize}}%
\only<6->{%
\begin{itemize}
    \item<6-> What happens if we measure only the \emph{bottom}~$n-1$ qubits?
    The system must collapse into a state consistent with what we observe.  
    
    \item<7-> Thus if we measure \ket{\alert<7>{10}} in the bottom qubits, then the top qubits collapse into \TwoSup{\alert<7>{011}}{\alert<7>{110}}.
    \item<8-> As another example, measuring \ket{\alert<8>{01}} at the bottom yields \TwoSup{\alert<8>{000}}{\alert<8>{101}} at the top.
    \item<9-> These partial measurements don't help:  Sampling the top qubits will yield \ket{\alert<9>{000}} or \ket{\alert<9>{101}}.
    \item<10-> Running the circuit again will yield another random measurement on the bottom qubits, with a concomitant collapse of the top qubits' state.
\end{itemize}}
}
    
\end{frame}}

\begin{frame}{Analysis from \QState{3}}{The partial measurement does help, partially}
\begin{center}
\Vskip{-3em}\adjustbox{valign=t, width=0.7\textwidth}{\begin{quantikz}
\lstick{\ket{\TensSupProd{0}{n}}} & \qwbundle{n} & \qw\slice{\QState{0}}& \gate{\TensSupProd{\Hadamard}{n}}\slice{\QState{1}} & \gate[wires=2][5em]{\mbox{$U_f$}}\gateinput{$x$}\gateoutput{$x$}\slice{\alert<5>{\QState{2}}} &\qw\slice{\alert<6>{\QState{3}}}  & \gate{\TensSupProd{\Hadamard}{n}}\slice{\alert<7->{\QState{4}}} & \meter{} & \qw \\
\lstick{\ket{\TensSupProd{0}{n-1}}} &  \qwbundle{n-1} & \qw & \qw   &  \qw\gateinput{$b$}\gateoutput{\Xor{b}{f(x)}} & \meter{} & \qw & \qw & \qw
\end{quantikz}}
\end{center}
\only<1-4>{%
\begin{itemize}
    \item<1-> Measuring the bottom qubits yields some state \ket{f(x)} that we will ignore.
    \item<2-> Of greater importance, the measurement places the top qubits into the state~\TwoSup{x}{\Xor{x}{s}} at \QState{3}.
    \item<3-> The computation finishes with an $n$-way Hadamard transformation of \QState{3}, which will provide clues about the secret~$s$.
    \item<4-> We restrict our state analysis to the top~$n$ qubits going forward.
\end{itemize}
}%
\Vskip{-2.5em}
\ScrollProof{5}{8}{%
\Next{\Four}{\sqrt{2^n}\QState{2} &= \RootTwoN{n}\SumBV{x}{n}\ \ket{x}\ket{f(x)} \\}
\Next{\Three}{\QState{3} &= \TwoSup{x}{\Xor{x}{s}} \\}
\Last{\sqrt{2}\QState{4} &= \TensSupProd{\Hadamard}{n}\left(\ket{x} + \ket{\Xor{x}{s}}\right) = \TensSupProd{\Hadamard}{n}(\ket{x}) + \TensSupProd{\Hadamard}{n}(\ket{\Xor{x}{s}})}
}%
\ScrollProof{9}{12}{%
\Next{\Four}{\sqrt{2}\QState{4} &= \ColorOne{\TensSupProd{\Hadamard}{n}(\ket{x})} + \ColorTwo{\TensSupProd{\Hadamard}{n}(\ket{\Xor{x}{s}})} \\}
\Next{\Three}{\mbox{Recall: }\TensSupProd{\Hadamard}{n}\ket{z} &= \RootTwoN{n}\NHadamard{z}{n}{w} \\}
\gdef\CarryOne{\sqrt{2^n}\sqrt{2}\QState{4} &= \ColorOne{\NHadamard{x}{n}{w}} + \ColorTwo{\NHadamard{(\Xor{x}{s})}{n}{w}}}
\Last{\CarryOne}
}%
\ScrollProof{13}{16}{%
\Next{\Four}{\CarryOne \\}
\Next{\Three}{ &= \SumBV{w}{n} \ColorOne{\NegOneExp{\DotP{x}{w}}} + \ColorTwo{\NegOneExp{\DotP{(\Xor{x}{s})}{w}}}\ \ket{w} \\}
\gdef\CarryTwo{\sqrt{2^{n+1}}\QState{4} &= \SumBV{w}{n} \ColorOne{\NegOneExp{\DotP{x}{w}}} + \ColorTwo{\NegOneExp{\Xor{(\DotP{x}{w})}{(\ColorFour{\DotP{s}{w}})}}}\ \ket{w}}
\Last{\CarryTwo}
}%
\only<17->{%
\begin{align*}
    \CarryTwo{}
\end{align*}
}
\only<17-20>{%
\Vskip{-2em}\begin{itemize}
    \item<18-> All dot products and \Xor{}{} operations are either~$0$ or~$1$.  The \ColorOne{first} and \ColorTwo{second} terms in the sum are thus $\pm 1$.
    \item<19-> The value of \ColorOne{\DotP{x}{w}} is of course identical to \ColorTwo{\DotP{x}{w}}. However, if \ColorFour{\DotP{s}{w}} is~$1$ then
    $\ColorOne{\NegOneExp{\DotP{x}{w}}} + \ColorTwo{\NegOneExp{\Xor{(\DotP{x}{w})}{\ColorFour{1}}}}=0$, canceling the basis state \ket{w} in \QState{4}.
    \item<20-> This depends on the secret~$s$ and on the
    particular value of term~$w$.
\end{itemize}}
\only<21-23>{%
\Vskip{-2em}%
Two cases:
\begin{description}
   \item<21->[$\ColorFour{\DotP{s}{w}}=1$] There is no amplitude on the \ket{w}.
   \item<22->[$\ColorFour{\DotP{s}{w}}=0$] The coefficient on \ket{w} is $\pm 2$.  Its probability is then $\left(\frac{\pm 2}{\sqrt{2^{n+1}}}\right)^{2}=\frac{1}{2^{n-1}}$
\end{description}
\visible<23->{For any $s\not=\TensSupProd{0}{n}$, the two cases occur with equal frequency:  half of \AllBits{w}{n}.  We can rewrite \QState{4} to reflect only observable outcomes.}
}%
\only<24->{%
\[
\QState{4} \equiv \RootTwoN{n-1}\sum_{w\ |\ \DotP{s}{w}=0}\ \ket{w}
\]
We are equally likely to see any outcome \ket{w} such that its dot product with~$s$ is~$0$.
}
    
\end{frame}

\begin{frame}{How to find the secret $s$}{It takes multiple runs of the circuit, but each can provide a clue}

\begin{itemize}
    \item Each time we run the circuit, we observe a \ket{w} such that $\DotP{w}{s}=0$.
    \item We first use an example to build intuition about the clues such \ket{w} observables provide.
    \item We then turn to formalizing the solution to Simon's problem and its efficiency on a quantum computer.
\end{itemize}
    
\end{frame}
