\SetTitle{21}{Deutsch's Problem}{Is $f(x)$ constant or balanced?}{21}

\begin{frame}{Deutsch's problem}{Formulation}

\begin{itemize}[<+->]
    \item We are given an oracle $f(x): \Set{0,1} \mapsto \Set{0,1}$
    \item We can characterize $f(x)$ as follows:
    \begin{description}
        \item[constant]   \Forall{x}{f(x)=k}, where $k$ is $0$ or $1$
        \item[balanced]   For half of its inputs, $f(x)=1$ and for the other half $f(x)=0$
    \end{description}
    \item Promise: given the domain and range of $f(x)$, the function must be one of the above.
    \item For this problem, it is impossible to break the promise.
    \item This is not a particularly expensive problem to solve classically.
    \item But we shall see that we can solve it more quickly on a quantum computer.
    \item And it is the beginning of a series of problems of increasing difficulty for a classical computer.
\end{itemize}
    
\end{frame}

\section*{Possible functions}

\begin{frame}{Possible functions}{There are only four}

\begin{itemize}
    \item Our function $f(x)$ has only two choices concerning where to send each of its two inputs $x$.  
    \item There are therefore only four possible  functions, two of which are \alert{constant} and two of which are \alert{balanced}.
\end{itemize}
\Vskip{-4em}\TwoColumns{%
\visible<2->{\begin{center}\alert{constant}\end{center}}
\visible<3->{\SBitTable{A}{0}{0}{1}{0}}
\visible<4->{\SBitTable{B}{0}{1}{1}{1}}
}{%
\visible<5->{\begin{center}\alert{balanced}\end{center}}
\visible<6->{\SBitTable{C}{0}{0}{1}{1}}
\visible<7->{\SBitTable{D}{0}{1}{1}{0}}
}
    
\end{frame}

\begin{frame}{Form of the oracle}{It has to be reversible}
\Vskip{-4em}\TwoUnequalColumns{0.6\textwidth}{0.4\textwidth}{%
\begin{itemize}[<+->]
    \item We consider a quantum gate $U_f$ that implements $f(x)$.
    \item It accepts $x$ and an \href{https://en.wikipedia.org/wiki/Ancilla_bit}{ancillary} input $y$.
    \item It copies $x$ onto the top output qubit.
    \item It computes \alert{\Xor{y}{f(x)}} on the bottom output qubit.
    \item When $y=\QZero{}$ the bottom qubit is \Vskip{-1em}\[\Xor{0}{f(x)}=f(x)\]
    \item When $y=\QOne{}$ the bottom qubit is \Vskip{-1em}\[\Xor{1}{f(x)}=\Not{f(x)}\]
\end{itemize}
}{%
\BigSkip{}
\only<1>{%
\begin{Oracle}[scale=0.75]{$U_f$}{\relax}{\relax}{\relax}{\relax}
\end{Oracle}}%
\only<2>{%
\begin{Oracle}[scale=0.75]{$U_f$}{$x$}{$y$}{\relax}{\relax}
\end{Oracle}}%
\only<3>{%
\begin{Oracle}[scale=0.75]{$U_f$}{$x$}{$y$}{$x$}{\relax}
\end{Oracle}}%
\only<4>{%
\begin{Oracle}[scale=0.75]{$U_f$}{$x$}{$y$}{$x$}{\alert{\Xor{y}{f(x)}}}
\end{Oracle}}%
\only<5>{%
\begin{Oracle}[scale=0.75]{$U_f$}{$x$}{\QZero{}}{$x$}{$f(x)$}
\end{Oracle}}%
\only<6>{%
\begin{Oracle}[scale=0.75]{$U_f$}{$x$}{\QOne{}}{$x$}{\Not{f(x)}}
\end{Oracle}}%
}    
\end{frame}



\begin{frame}{Realizing the oracle for \Quote{A}: $f(x)=0$}{We design a circuit that meets the classical specification}

\TwoColumns{%
\Vskip{-3em}\begin{center}
\begin{TTable}{A}{$x$}{$f(x)$}
\TRow{ }{0}{0}
\TRow{ }{1}{0}
\end{TTable}
\end{center}
\BigSkip{}
\visible<2->{%
\begin{center}
\begin{UTable}{$U_A$}{$x$}{$y$}{$f(x)$}{$x$}{\alt<3->{\alert<3>{$y$}}{\alert<2>{\Xor{y}{f(x)}}}}
\URow{}{0}{0}{0}{0}{0}
\URow{}{0}{1}{0}{0}{1}
\URow{}{1}{0}{0}{1}{0}
\URow{}{1}{1}{0}{1}{1}
\end{UTable}\end{center}}
}{%
\begin{Oracle}[scale=0.75]{$U_A$}{$x$}{$y$}{$x$}{\Xor{y}{f(x)}}
\end{Oracle}

\BigSkip{}
\visible<3->{%
\Forall{x}{f(x)=0}, so $\alert<3>{\Xor{y}{f(x)}}=\Xor{y}{0}=\alert<3>{y}$.  
\MedSkip{}
Thus, $U_A$ is just the identity function, so our quantum circuit doesn't have to do anything with its inputs to realize $U_A$.}

\BigSkip{}
\visible<4>{%
\begin{center}
\adjustbox{valign=t}{\begin{quantikz}
\lstick{\ket{x}} &  \qw & \qw \rstick{\ket{x}} \\
\lstick{\ket{y}} &  \qw & \qw \rstick{\ket{\Xor{y}{f(x)}}}
\end{quantikz}}\end{center}
}
}
    
\end{frame}

\begin{frame}{Realizing the oracle for \Quote{B}: $f(x)=1$}{We design a circuit that meets the classical specification}

\TwoColumns{%
\Vskip{-3em}\begin{center}
\begin{TTable}{B}{$x$}{$f(x)$}
\TRow{ }{0}{1}
\TRow{ }{1}{1}
\end{TTable}
\end{center}
\BigSkip{}
\visible<2->{%
\begin{center}
\begin{UTable}{$U_B$}{$x$}{$y$}{$f(x)$}{$x$}{\alt<3->{\alert<3>{\Not{y}}}{\alert<2>{\Xor{y}{f(x)}}}}
\URow{}{0}{0}{1}{0}{1}
\URow{}{0}{1}{1}{0}{0}
\URow{}{1}{0}{1}{1}{1}
\URow{}{1}{1}{1}{1}{0}
\end{UTable}\end{center}}
}{%
\begin{Oracle}[scale=0.75]{$U_B$}{$x$}{$y$}{$x$}{\Xor{y}{f(x)}}
\end{Oracle}

\BigSkip{}
\visible<3->{%
\Forall{x}{f(x)=1}, so $\alert<3>{\Xor{y}{f(x)}}=\Xor{y}{1}=\alert<3>{\Not{y}}$.  
\MedSkip{}
Thus, $U_B$ is realized by complementing $y$.}

\BigSkip{}
\visible<4>{%
\begin{center}
\adjustbox{valign=t}{\begin{quantikz}
\lstick{\ket{x}} &  \qw & \qw \rstick{\ket{x}} \\
\lstick{\ket{y}} &  \gate{X} & \qw \rstick{\ket{\Xor{y}{f(x)}}}
\end{quantikz}}\end{center}
}
}
    
\end{frame}

\begin{frame}{Realizing the oracle for \Quote{C}: $f(x)=x$}{We design a circuit that meets the classical specification}

\TwoColumns{%
\Vskip{-3em}\begin{center}
\begin{TTable}{C}{$x$}{$f(x)$}
\TRow{ }{0}{0}
\TRow{ }{1}{1}
\end{TTable}
\end{center}
\BigSkip{}
\visible<2->{%
\begin{center}
\begin{UTable}{$U_C$}{$x$}{$y$}{$f(x)$}{$x$}{\alt<3->{\alert<3>{\Xor{y}{x}}}{\alert<2>{\Xor{y}{f(x)}}}}
\URow{}{0}{0}{0}{0}{0}
\URow{}{0}{1}{0}{0}{1}
\URow{}{1}{0}{1}{1}{1}
\URow{}{1}{1}{1}{1}{0}
\end{UTable}\end{center}}
}{%
\begin{Oracle}[scale=0.75]{$U_C$}{$x$}{$y$}{$x$}{\Xor{y}{f(x)}}
\end{Oracle}

\BigSkip{}
\visible<3->{%
\Forall{x}{f(x)=x}, so $\alert<3>{\Xor{y}{f(x)}}=\alert<3>{\Xor{y}{x}}$.}%

\BigSkip{}
\only<3>{%
The matrix may look familiar...
\[U_{C} = \CNOTMatrix{} \]
}%
\only<4>{%
Thus, a \NamedGate{CNOT} gate will flip $y$ when $x$ is~$1$:
\begin{center}
\adjustbox{valign=t}{\begin{quantikz}
\lstick{\ket{x}} &  \ctrl{1} & \qw \rstick{\ket{x}} \\
\lstick{\ket{y}} &  \targ{} & \qw \rstick{\ket{\Xor{y}{f(x)}}}
\end{quantikz}}\end{center}
}
}
    
\end{frame}

\begin{frame}{Realizing the oracle for \Quote{D}: $f(x)=\Not{x}$}{We design a circuit that meets the classical specification}

\TwoColumns{%
\Vskip{-3em}\begin{center}
\begin{TTable}{D}{$x$}{$f(x)$}
\TRow{ }{0}{1}
\TRow{ }{1}{0}
\end{TTable}
\end{center}
\BigSkip{}
\visible<2->{%
\begin{center}
\begin{UTable}{$U_D$}{$x$}{$y$}{$f(x)$}{$x$}{\alt<3->{\alert<3>{\Not{\Xor{y}{x}}}}{\alert<2>{\Xor{y}{f(x)}}}}
\URow{}{0}{0}{1}{0}{1}
\URow{}{0}{1}{1}{0}{0}
\URow{}{1}{0}{0}{1}{0}
\URow{}{1}{1}{0}{1}{1}
\end{UTable}\end{center}}
}{%
\begin{Oracle}[scale=0.75]{$U_D$}{$x$}{$y$}{$x$}{\Xor{y}{f(x)}}
\end{Oracle}

\BigSkip{}
\visible<3->{%
\Forall{x}{f(x)=\Not{x}}, so $\alert<3>{\Xor{y}{f(x)}}=\Xor{y}{\Not{x}}=\alert<3>{\Not{\Xor{y}{x}}}$.}%

\BigSkip{}
\only<3>{%
This is just like the previous case, except the output is the complement of what we had before.
}%
\only<4>{%
Thus, we add an \PauliX{} gate after the \NamedGate{CNOT} gate:
\begin{center}
\adjustbox{valign=t}{\begin{quantikz}
\lstick{\ket{x}} &  \ctrl{1} & \qw & \qw \rstick{\ket{x}} \\
\lstick{\ket{y}} &  \targ{} & \gate{X} & \qw \rstick{\ket{\Xor{y}{f(x)}}}
\end{quantikz}}\end{center}
}
}
    
\end{frame}

\begin{frame}{Is $f(x)$ constant or balanced?}{Classical approach}

\TwoColumns{%
\begin{itemize}[<+->]
    \item If we perform one evaluation of $f$, say on $0$, then we do not know if $f(x)$ is always $f(0)$ or if $f(1)$ may be different.
    \item Classically it takes \emph{two} queries of $f$ to determine the answer to this question.
    \visible<3-4>{Here $f(x)$ is \alt<3>{\ColorOne{constant}}{\ColorTwo{balanced}}.}
\end{itemize}
}{%
\begin{align*}
    f(0) &= 0 \\
    \visible<2->{f(1) &= \alt<4>{\ColorTwo{1}}{\alt<3>{\ColorOne{0}}{\ ?}}}
\end{align*}
}
\BigSkip{}
\visible<5>{We will see that a quantum circuit for $f$ can reveal whether the function is constant or balanced with \emph{a single} query!}
\end{frame}

\section*{Using superposition}

\begin{frame}{Using superposition}{We can arrange for the quantum circuit to produce all possible answers}

\Vskip{-3em}\begin{center}
\begin{GateBox}[scale=1.0]{2}{1.2}{2}
\BoxLabel{$U_f$}
\Input{0}{\alt<1>{\QPlus{}}{\ket{+}}}
\Input{1}{\alt<16->{\alert{\ket{-}}}{\QZero{}}}
\ComputeMyY{0}\draw (0,\MyY) node[right] {$x$};
\ComputeMyY{1}\draw (0,\MyY) node[right] {$y$};
\Output{0}{\alert<12-14>{$x$}}
\Output{1}{\alert<12-14>{\Xor{y}{f(x)}}}
\end{GateBox}
\end{center}
\TwoColumns{%
\only<1-3>{%
\begin{itemize}
    \item<1-> The input is the superposition of the two possible basis states.
    \item<2-> We know this state as \ket{+}.
    \item<3-> Recall we can create $\ket{+}=\QPlus{}$ using a Hadamard gate.
\end{itemize}}
}{%
\only<3>{%
\begin{center}
\adjustbox{valign=t}{\begin{quantikz}
\lstick{\QZero{}} &  \gate{H} &  \qw \rstick{\ket{+}}
\end{quantikz}}%
\end{center}}%
}%

\only<4-6>{%
    $U_f$ produces
        $\TensProd{x}{(\Xor{y}{f(x)})}  =
        \TensProd{x}{(\Xor{0}{f(x)})} 
        = \TensProd{x}{f(x)}
        = \alert{\ket{x\,f(x)}}$

\SmallSkip{}
By linearity, with $x$ in the superposition \QPlus{} and $y=\QZero{}$
    \begin{align*}
    U_f(\TensProd{\QPlus}{\QZero}) &= \RootTwo{}U_f(\ket{00}+\ket{10}) \\
    \visible<5->{ &= \RootTwo{}\left(
        U_f(\ket{00}) + U_f(\ket{10})\right) \\}
       \visible<6->{ &= \frac{\ket{0\,f(0)} +
        \ket{1\,f(1)}}{\sqrt{2}}}
    \end{align*}}%
\only<7->{%
\visible<7-16>{
\[ U_{f}(\ket{+\,0}) = \TwoSup{0\,\textcolor<14>{\RCone}{f(0)}}{1\,\textcolor<14>{\RCtwo}{f(1)}} \]}}
\only<7-11>{%
\Vskip{-2em}\begin{itemize}
    \item<7-> We forced $U_f$ to produce \emph{both} outputs.
    \item<8-> But what happens if we measure?
    \item<9-> We randomly and uniformly see \emph{either} \ket{0\,f(0)} or \ket{1\,f(1)}.
    \item<10-> It could take \emph{more} than two evaluations to see both answers!
    \item<11-> Even if we measure only the top qubit, the bottom qubit collapses too.
\end{itemize}
}%
\only<12-14>{%
\Vskip{-2em}\begin{itemize}
    \item<12-> It is misleading to see the \alert<12>{output} of $U_f$ labeled to look like a tensor product.  $U_f$ will generate the outputs as specified, but the result is not necessarily expressible as \TensProd{x}{(\Xor{y}{f(x)})}.
    \item<13-> Without analysis, we cannot know if the output is tensor-factorable.
    \item<14-> Consider \textcolor{\RCone}{$f(0)=0$} and \textcolor{\RCtwo}{$f(1)=1$}.  Then $U_f$ creates \alert<14>{\TwoSup{00}{11}} as the (entangled) output, which we know cannot be tensor factored!
\end{itemize}
}%
\only<15->{%
\begin{itemize}
    \item<15-> While the information is in the output, we cannot recover it by measurement.
    \item<16-> We need to try something different, using \alert{interference} to get the result we need.
\end{itemize}
}
\end{frame}

\begin{frame}{Creating \ket{-}}{A few approaches}

\TwoUnequalColumns{0.6\textwidth}{0.4\textwidth}{%
\Vskip{-3em}\begin{itemize}[<+->]
    \item Recall 
    \begin{itemize}
        \item $\ket{+}=\QPlus{}$
        \item $\ket{-}=\QMinus{}$
    \end{itemize}
    \item Use \PauliZ{} on \ket{+}
    \[
    \ZMatrix{}\PPlus{} = \PMinus{}=\ket{-}\]
    \item Use \Hadamard{} on \QOne{}
    \[ \HMatrix{}\POne{} = \PMinus{}=\ket{-}
    \]
\end{itemize}
}{%
\BigSkip{}\BigSkip{}
\visible<4>{%
\begin{center}
\adjustbox{valign=t}{\begin{quantikz}
\lstick{\QZero{}} &  \gate{H} & \gate{Z}& \qw \rstick{\ket{-}}
\end{quantikz}}\end{center}}%
\BigSkip{}
\BigSkip{}
\BigSkip{}
\visible<5>{%
\begin{center}
\adjustbox{valign=t}{\begin{quantikz}
\lstick{\QOne{}} &  \gate{H} & \qw & \qw \rstick{\ket{-}}
\end{quantikz}}\end{center}}%
}
    
\end{frame}

\section*{Solution}

\begin{frame}{Deutsch's algorithm}{Uses interference to find constant vs. balanced functions}

\Vskip{-4em}\TwoUnequalColumns{0.5\textwidth}{0.5\textwidth}{%
\begin{center}
\Vskip{-3em}\adjustbox{valign=t, width=\textwidth}{\begin{quantikz}
\lstick{\QZero{}} & \qw\slice{\alert<2>{\QState{0}}} &  \gate{\Hadamard}\slice{\alert<3-5>{\QState{1}}} & \gate[wires=2][5em]{\mbox{$U_f$}}\gateinput{$x$}\gateoutput{$x$}\slice{\alert<6-8>{\QState{2}}} &\gate{\Hadamard}\slice{\alert<9-10>{\QState{3}}} & \meter{\alt<30>{\alert{1}}{\alt<25>{\alert{0}}{0/1}}} \\
\lstick{\QZero{}} & \gate{\PauliX} &   \gate{\Hadamard}   &  \qw\gateinput{$y$}\gateoutput{\Xor{y}{f(x)}} & \qw & \qw
\end{quantikz}}%
\end{center}
}{%
\only<1-7>{%
    \begin{align*}
        \visible<2->{\QState{0} &= \ket{01} \\}
        \only<3-5>{
        \visible<3->{\QState{1} &= \ket{+-} \visible<4->{= \frac{1}{2}\DQB{1}{-1}{1}{-1}}} \\}
        \visible<5->{\only<6->{\QState{1}}&= \frac{\ket{00}-\ket{01}+\ket{10}-\ket{11}}{2}}
    \end{align*}}%
\only<8-12>{%
\begin{align*}
    U_{f}(\ket{\textcolor{\RCone}{x}\textcolor{\RCtwo}{y}}) & = \ket{\textcolor{\RCone}{x}\,\Xor{\textcolor{\RCtwo}{y}}{f(\textcolor{\RCone}{x})}} \\
    \only<9>{U_{f}(\ket{\textcolor{\RCone}{0}\textcolor{\RCtwo}{0}}) & = \ket{\textcolor{\RCone}{0}\,\Xor{\textcolor{\RCtwo}{0}}{f(\textcolor{\RCone}{0})}}}
    \only<10>{U_{f}(\ket{\textcolor{\RCone}{0}\textcolor{\RCtwo}{1}}) & = \ket{\textcolor{\RCone}{0}\,\Xor{\textcolor{\RCtwo}{1}}{f(\textcolor{\RCone}{0})}}}
    \only<11>{U_{f}(\ket{\textcolor{\RCone}{1}\textcolor{\RCtwo}{0}}) & = \ket{\textcolor{\RCone}{1}\,\Xor{\textcolor{\RCtwo}{0}}{f(\textcolor{\RCone}{1})}}}
    \only<12>{U_{f}(\ket{\textcolor{\RCone}{1}\textcolor{\RCtwo}{1}}) & = \ket{\textcolor{\RCone}{1}\,\Xor{\textcolor{\RCtwo}{1}}{f(\textcolor{\RCone}{1})}}}
\end{align*}}
\only<13-18>{%
\begin{align*}
    \visible<13-15>{\Xor{0}{f(a)} &= f(a)} \\
    \visible<16->{\Xor{1}{f(a)} &= \Not{f(a)}}
\end{align*}}%
\only<22,27>{%
\BigSkip{}
\alert{Recall \[\ket{a\,b} = \ket{a}\ket{b} \]}
}%
\only<23-24>{%
\[ \alert{\ket{+} = \QPlus{}} \]
}%
\only<28-30>{%
\[ \alert{\ket{-} = \QMinus{}} \]
}%
\only<31>{%
\begin{center}
    \begin{tabular}{ccc}
        & \QState{2} & \QState{3} \\ 
        $f(x)$ & \multicolumn{2}{c}{(top qubit)}\\ \hline
    constant & \ket{+} & \QZero{} \\
    balanced & \ket{-} & \QOne{}
    \end{tabular}
\end{center}
}
}
\only<6-30>{
\Vskip{-2em}\begin{align*}
    \only<6-7>{\QState{2} &= U_f(\frac{1}{2}\left[\ket{00}-\ket{01}+\ket{10}-\ket{11}\right]) \\[1em]}
     \only<7->{\visible<8->{\QState{2}}   &= \frac{%
        \alt<14->{\ket{0\,f(0)}}{\alt<9->{\ket{0\,\alert<13>{\Xor{0}{f(0)}}}}{U_f(\ket{00})}}-\alt<17->{\ket{0\,\Not{f(0)}}}{\alt<10->{\ket{0\,\alert<16>{\Xor{1}{f(0)}}}}{U_f(\ket{01})}}  + \alt<15->{\ket{1\,f(1)}}{\alt<11->{\ket{1\,\alert<13-14>{\Xor{0}{f(1)}}}}{U_f(\ket{10})}}-\alt<18->{\ket{1\,\Not{f(1)}}}{\alt<12->{\ket{1\,\alert<16-17>{\Xor{1}{f(1)}}}}{U_f(\ket{11})}}}{2}}
\end{align*}}
\only<18-20>{%
\begin{itemize}
    \item<18-> For any $f(x)$, \QState{2} has 4 distinct terms: \ket{00}, \ket{01}, \ket{10}, and \ket{11}.
    \item<19-> $U_f$ determines which terms are positive and which are negative.
    \item<20-> If only the numerator contained \ket{0\,f(0)} - \ket{0\,f(1)}, then those terms would cancel if $f(0)$ is constant.
\end{itemize}
}
\only<21-25>{%
\Vskip{-2em}\TwoUnequalColumns{0.3\textwidth}{0.7\textwidth}{%
Suppose $f(0)=f(1)$.  \only<23-24>{%
Tensor-factored, this state tells us the \textcolor{\RCone}{top} and \textcolor{\RCtwo}{bottom} qubits' states.
}
\only<25>{Then the top qubit at \QState{2} is \ket{+}.  Measuring at \QState{3} yields \alert{\ket{0}}.}
}{%
\Vskip{-4em}\begin{align*}
    2\times\QState{2} &= \ket{0\,f(0)} - \ket{0\,\Not{f(0)}} + \ket{1\,\alert<21>{f(0)}} - \ket{1\,\Not{\alert<21>{f(0)}}} \\
  \visible<22->{  & = \TensProd{(\ket{0}+\ket{1})}{\ket{f(0)}} - \TensProd{(\ket{0}+\ket{1})}{\ket{\Not{f(0)}}} \\}
  \visible<23->{  & = \TensProd{\textcolor<23-24>{\RCone}{(\ket{0}+\ket{1})}}{\textcolor<23-24>{\RCtwo}{(\ket{f(0)}-\ket{\Not{f(0)}})}} \\}
  \visible<24->{ \QState{2} &= \TensProd{\textcolor<24>{\RCone}{\ket{+}}}{\textcolor<24>{\RCtwo}{(\ket{f(0)}-\ket{\Not{f(0)}})\frac{1}{\sqrt{2}}}} }
\end{align*}
}
}
\only<26-30>{%
\Vskip{-2em}\TwoUnequalColumns{0.3\textwidth}{0.7\textwidth}{%
Suppose $f(0) = \Not{f(1)}$. 
 \only<28-29>{%
Tensor-factored, this state tells us the \textcolor{\RCone}{top} and \textcolor{\RCtwo}{bottom} qubits' states.
}
\only<30>{Then the top qubit at \QState{2} is \ket{-}.  Measuring at \QState{3} yields \alert{\ket{1}}.}
}{%
\Vskip{-4em}\begin{align*}
    2\times\QState{2} &= \ket{0\,f(0)} - \ket{0\,\alert<26>{f(1)}} + \ket{1\,f(1)} - \ket{1\,\alert<26>{f(0)}} \\
  \visible<27->{  & = \TensProd{(\ket{0}-\ket{1})}{\ket{f(0)}} - \TensProd{(\ket{0}-\ket{1})}{\ket{f(1)}} \\}
  \visible<28->{  & = \TensProd{\textcolor<28-29>{\RCone}{(\ket{0}-\ket{1})}}{\textcolor<28-29>{\RCtwo}{(\ket{f(0)}-\ket{f(1)})}} \\}
  \visible<29->{ \QState{2} &= \TensProd{\textcolor<29>{\RCone}{\ket{-}}}{\textcolor<29>{\RCtwo}{(\ket{f(0)}-\ket{f(1)})\frac{1}{\sqrt{2}}}} }
\end{align*}
}
}
\only<31->{%
\MedSkip{}
\begin{itemize}
    \item<31-> We can determine if $f(x)$ is constant or balanced with a single query of $U_f$.  Classically two queries were needed.
    \item<32-> We perform a \href{https://physics.stackexchange.com/questions/184242/partial-measurement-and-the-math-behind-it}{partial measurement} of the system, measuring only the top qubit.
    \item<33-> This quantum circuit never produced entanglement at \QState{2}:  that state could be tensor factored into the product of single qubit states.
    \item<34-> The effect of sending $y=\ket{-}$ into $U_f$ is manifested primarily on the top qubit.   We call this \emph{phase kickback} and study it next.
\end{itemize}
}

\end{frame}

\begin{frame}{What about the bottom qubit?}{Homework problem but let's take a quick look}

\Vskip{-4em}\begin{align*} 
\invisible<5>{f(0) = f(1) \longrightarrow &  \QState{2} = \TensProd{\textcolor{\RCone}{\ket{+}}}{\textcolor{\RCtwo}{(\ket{f(0)}-\ket{\Not{f(0)}})\frac{1}{\sqrt{2}}}} \\}
\invisible<4>{f(0) \not= f(1) \longrightarrow & \QState{2} = \TensProd{\textcolor{\RCone}{\ket{-}}}{\textcolor{\RCtwo}{(\ket{f(0)}-\ket{f(1)})\frac{1}{\sqrt{2}}}}}
\end{align*}
\Vskip{-2em}\begin{itemize}[<+->]
  \item Tensor factored, we can see the values present on the \textcolor{\RCone}{top} and \textcolor{\RCtwo}{bottom} qubits.
  \item You will analyze the \textcolor{\RCtwo}{bottom qubit's} value for each of the above cases.
  \item We note here that the input of \ket{-} for $y$ on the \textcolor{\RCtwo}{bottom qubit} has the effect of changing phase on the \textcolor{\RCone}{\emph{top} qubit's} value on output:
  \begin{description}
     \item[$f(0)=f(1)$] $\textcolor{\RCone}{\ket{+}}=\TwoSupOp{\ket{0}}{\alert{\ket{1}}}{\alert{+}}$
     \item[$f(0)\not=f(1)$] $\textcolor{\RCone}{\ket{-}}=\TwoSupOp{\ket{0}}{\alert{\ket{1}}}{\alert{-}}$
  \end{description}
  \item This is called \emph{phase kickback}.
\end{itemize}
\end{frame}

\begin{frame}{Summary}{What have we learned?}

\begin{itemize}
    \item We have seen how to implement some simple oracles using direct logic instead of matrices.
    \item The logic is based on inspection of the required outputs for each possible input.
    \item Deutsch's algorithm can tell whether the oracle's function $f(x)$ is constant or balanced with a single query.
    \item Classically it takes two queries to make that determination.
\end{itemize}
    
\end{frame}
