\SetTitle{6}{A single qubit}{Representation and some quantum gates}{06}

\section{Overview}
\begin{frame}{Overview}{What we will cover}
\begin{itemize}
    \item We develop an efficient representation for a single quantum bit (qubit), parameterized by two real values.
    \item We introduce the Bloch sphere: a compelling visualization of a single qubit.
    \item We introduce the Hadamard gate that transforms \ket{0} into a uniform superposition of \ket{0} and \ket{1}.
    \item We introduce the \emph{Pauli} gates and their associated unitary matrices.
    \item We develop a general approach to rotating a qubit around each axis of the Bloch sphere.
\end{itemize}
\end{frame}

\section{Efficient characterization}

\begin{frame}{State space}{Definition}
\Vskip{-3em}\Postulate{State Space}{The state of a quantum system is a unit vector in a complex \href{https://en.wikipedia.org/wiki/Hilbert_space}{Hilbert space}}
\begin{itemize}
    \item We need a Hilbert space, rather than the simpler \href{https://en.wikipedia.org/wiki/Euclidean_space}{Euclidean space}, because we use complex values to represent state.
    \item We have used two dimensions to specify a single qubit, one each for $\alpha$ and $\beta$: $\ket{\psi}=\SQB{\alpha}{\beta}$, where
    $\Prob{\alpha}+\Prob{\beta}=1$.
  
    \item The state of an $n$-qubit system is a unit column vector (ket)  of $\TwoToThe{n}$ complex values.
\end{itemize}
\OnlyRemark{1}{%
We next examine a more concise and visually compelling definition of a qubit.
}
\end{frame}

\begin{frame}{Coordinates for a qubit}{Cartesian and polar}
\Vskip{-5em}\TwoUnequalColumns{0.4\textwidth}{0.6\textwidth}{%
\ColHead{Cartesian}
\begin{itemize}
    \item $\ket{\psi}=\alpha\ket{0} + \beta\ket{1}$ with $\Mag{\alpha}^2+\Mag{\beta}^2=1$.
    \item Letting
    \begin{itemize}
        \item $\alpha=\Complex{a_{0}}{b_{0}}$
        \item $\beta=\Complex{a_{1}}{b_{1}}$
    \end{itemize}
    \item This requires \alert{four real parameters}
\end{itemize}
}{%
\ColHead{Polar}
\begin{itemize}
    \item Write $\alpha = r_{a}\left(\cos\theta_{a}+\NiceI \sin\theta_{a}\right) = \Polar{r_{a}}{\theta_{a}}$
    \item Write $\beta = r_{b}\left(\cos\theta_{b}+\NiceI \sin\theta_{b}\right) = \Polar{r_{b}}{\theta_{b}}$
    \item Then \begin{eqnarray*} \ket{\psi} & = & \Polar{r_{a}}{\theta_{a}}\ket{0} + \Polar{r_{b}}{\theta_{b}}\ket{1} \\
    & = & \visible<1>{\ExpPhase{\theta_{a}}}\left( \alert<4>{r_{a}}\ket{0} + \Polar{\alert<4>{r_{b}}}{
    \only<1-2>{(\theta_{b}-\theta_{a})}
    \only<3->{\alert<4>{\phi}}
    }\ket{1} \right)
    \end{eqnarray*}
    \item<2-> Discard $\ExpPhase{\theta_{a}}$ as a global phase
    \item<3-> Let $\phi=\theta_{b}-\theta_{a}$
    \item<4-> So we need just \alert{three real parameters}
\end{itemize}
}
\end{frame}

\def\Bloch#1#2{%
\ensuremath{%
\cos{\frac{\textcolor{\RCtwo}{#1}}{2}}\ket{0} + \Polar{}{\textcolor{\RCone}{#2}}\sin{\frac{\textcolor{\RCtwo}{#1}}{2}}\ket{1}}}

\section{Bloch sphere}

\begin{frame}{Can we do better?}{Yes!  It takes only two real parameters to denote the state of one qubit}
\begin{itemize}
   \item We start with three real parameters:
   \( \ket{\psi}= r_{a} \ket{0} + r_{b}\ExpPhase{\phi}\ket{1} \)
   \item We have not yet accounted for the unit nature of \ket{\psi}.   
   \item We require $r_{a}^{2}+r_{b}^{2}=1$ which implies \And{r_{a}^{2} \leq 1}{r_{b}^{2} \leq 1}. 
   \item We can show $\Implies{x^{2}\leq 1}{-1 \leq x \leq 1}$.
   \item<2-> We therefore substitute for $r_{a}$ and $r_{b}$ as follows:
   \[
     r_{a} = \cos{\frac{\theta}{2}}, 
     r_{b} = \sin{\frac{\theta}{2}} \]
   The use of $\theta/2$ is arbitrary for now but will work out well later.
   \item<3-> We can thus characterize $\ket{\psi}$ using only \alert{two (angular) real parameters}:
   \[ \ket{\psi} = \Bloch{\theta}{\phi}
   \]
\end{itemize}

    
\end{frame}

\begin{frame}{From polarizing filters to a sphere}{The angular separation between the basis states changes}
\Vskip{-3em}\TwoColumns{%
\begin{itemize}
    \item<1-> With polarizing filters, our bases were horiziontal \visible<2->{and vertical, so they were separated by \Degrees{90}.}
    \item<3-> We imposed a \Degrees{90} angular separation between these bases logically as well.
    \item<4-> We are about to use a \emph{sphere} to capture quantum state.  To use the entire sphere, we locate \ket{0} at the north pole and \ket{1} at the south pole.
    \item<5-> In that depiction, the basis states are separated by \Degrees{180}.
\end{itemize}
}{%
\Vskip{-3em}\begin{center}
\begin{TIKZP}
\visible<3->{
\draw[->,ultra thick] (0,0) -- (1,0) node[right] {\ \ket{0}};
\draw[->,ultra thick] (0,0) -- (0,1) node[above] {\ket{1}};
}
\visible<1->{\begin{scope}[draw=\RCtwo,rotate=90]\PFilter{-1}{-1.2}{1}{1.2}{20}\end{scope}}
\visible<2->{\begin{scope}[draw=\RCthree]\PFilter{-1}{-1.2}{1}{1.2}{20}\end{scope}}
\end{TIKZP}
\end{center}
\visible<4->{%
\begin{center}
    \begin{BlochSphere}[scale=1]{60}{115}
    \BlochSpherePhiTheta{}
     \TZPoint{0,0,1}{\ket{0}}{right}
     \TZPoint{0,0,-1}{\ket{1}}{below}
    \end{BlochSphere}
\end{center}}
}
\end{frame}

\begin{frame}{Our two-parameter characterization}{It meshes well with our studies so far}
\[ \Bloch{\theta}{\phi} \]
\begin{itemize}
    \item The parameter \textcolor{\RCtwo}{$\theta$} trades wave amplitude between \ket{0} and \ket{1}.
    \item Only the relative phase between \ket{0} and \ket{1} matters for interactions.  
    \item This formulation ascribes the phase difference to \ket{1}, using a walk of \textcolor{\RCone}{$\phi$} radians counterclockwise around the complex unit circle, starting at 0~radians.
    \item This formulation of state makes clear that while only two real-valued parameters are needed, the number of states is \href{https://en.wikipedia.org/wiki/Uncountable_set}{uncountably infinite}.
    
\end{itemize}
    
\end{frame}

\begin{frame}{Recall the beam splitter}{We characterize two of its states using our new formula}
\[ \Bloch{\theta}{\phi} \]
    \begin{itemize}
        \item The state of the photon was initially $\Bloch{0}{0} = \SQB{1}{0}$
        \item After one beam splitter we obtained \[ \Bloch{\pi/2}{\pi/2} = \frac{1}{\sqrt{2}} \SQB{1}{i} \]
    \end{itemize} 
\end{frame}

\begin{frame}{The Bloch sphere}{Our two angular parameters determine a point on the unit sphere}
\TwoColumns{%
\[ \ket{\psi} = \Bloch{\theta}{\phi} \]
\begin{itemize}

\item \ket{\psi} is located 
\begin{itemize}
    \item  \textcolor{\RCone}{$\phi$} radians off the $x$~axis and 
    \item \textcolor{\RCtwo}{$\theta$} radians off the $z$~axis.
\end{itemize}
\item We locate states \emph{on} the unit sphere, but the distance from the center of the sphere has no meaning here.  

A more general interpretation of the sphere ascribes meaning to interior points.
\end{itemize}
}{%
\Vskip{-3em}\begin{BlochSphere}[scale=2.5]{60}{115}
    
    \BlochSpherePhiTheta{}
    \TZPoint{theta}{\ket{\psi}}{above right}
\end{BlochSphere}
}
\end{frame}

\begin{frame}{Correspondence between the formula and the sphere}{We see why the formula cuts the angle in half}
\TwoUnequalColumns{0.5\textwidth}{0.5\textwidth}{%
\begin{itemize}
  \item<2-> We locate \ket{0} at \PhiTheta{0}{0} using \Bloch{0}{0}.  The choice of \textcolor{\RCone}{$\phi$} doesn't matter when $\textcolor{\RCtwo}{\theta}=0$.
  \item<3-> We locate \ket{1} at $\PhiTheta{0}{\pi}$ using $\Bloch{\pi}{0}$.
  \item<4-> From the picture, \textcolor{\RCone}{$\phi$} doesn't matter when $\textcolor{\RCtwo}{\theta}=\pi$.  How can this be?
\end{itemize}
}{%
\only<1-4>{%
\[ \ket{\psi} = \Bloch{\theta}{\phi} \]}
\only<5>{%
\[ \ket{\psi} =  \ExpPhase{\textcolor{\RCone}{\phi}} \ket{1} \]
}%
\only<1-4>{%
\begin{BlochSphere}[scale=1.5]{60}{115}
    
    \BlochSpherePhiTheta{}
    \visible<2->{\TZPoint{0,0,1}{$\ket{0}$}{right}}
    \visible<3->{\TZPoint{0,0,-1}{$\ket{1}$}{right}}
    
\end{BlochSphere}}%
\only<5>{%
\begin{itemize}
\item With $0$ as the amplitude on \ket{0}, any phase on \ket{1} with $\textcolor{\RCone}{\phi}\not=0$ can be canceled with an appropriate global phase of 
  \ExpPhase{(-\textcolor{\RCone}{\phi})} on \ket{\psi}.
\item Global phases can be ignored as they do not affect measurements or interactions.
\end{itemize}
}
}
\end{frame}

\begin{frame}{The computational basis}{Measurement outcomes are related to proximity to the north or south pole}
\TwoUnequalColumns{0.6\textwidth}{0.4\textwidth}{%
\begin{itemize}
  \item<1-> The north and south poles, \ket{0} and \ket{1}, respectively, are our \emph{standard} or \emph{computational basis}.
   All other points on the sphere are superpositions of that basis.
   \item<2-> The probability of measuring \ket{0} or \ket{1} is related to how close the state is to the north or south pole, respectively, which is represented by $\textcolor{\RCtwo}{\theta}$.
   \item<3-> The probability of measuring \ket{0} is $\cos^{2}(\textcolor{\RCtwo}{\theta}/2)$.  At the equator, we are equally likely to measure \ket{0} or \ket{1}.
\end{itemize}
}{%
\[ \ket{\psi} = \Bloch{\theta}{\phi} \]
\begin{BlochSphere}[scale=1.5]{60}{115}
    
    \BlochSpherePhiTheta{}
    \visible<2->{\TZPoint{0,0,1}{$\ket{0}$}{right}}
    \visible<3->{\TZPoint{0,0,-1}{$\ket{1}$}{right}}
    
\end{BlochSphere}
}
\end{frame}

{%
\def\Sket#1{%
\ket{\hbox to 4ex{\hfil\ensuremath{#1}\hfil}}}%
\def\R#1{\mbox{\textcolor{\RCone}{#1}}}%
\def\B#1{\mbox{\textcolor{\RCtwo}{#1}}}%
\begin{frame}{Navigating the Bloch sphere}{Other points of interest}


\Vskip{-4em}\TwoUnequalColumns{0.6\textwidth}{0.4\textwidth}{%
\begin{center}
    \begin{tabular}{ccc}
        State & \textcolor{\RCone}{$\phi$} & \textcolor{\RCtwo}{$\theta$} \\
        \visible<1->{\Sket{0} & \R{$0$} & \B{$0$} \\}%
        \visible<1->{\Sket{1} & \R{$0$} & \B{$\pi$} \\}%
        \visible<2->{\Sket{+x} & \R{$0$} & \B{$\pi/2$} \\}%
        \visible<3->{\Sket{-x} & \R{$0$} & \B{$-\pi/2$} \\}%
        \only<3-4>{\Sket{-x} & \R{$\pi$} & \B{$\pi/2$} \\}%
        \visible<5->{\Sket{+y} & \R{$\pi/2$} & \B{$\pi/2$} \\}%
        \visible<6->{\Sket{-y} & \R{$-\pi/2$} & \B{$\pi/2$} \\}%
    \end{tabular}
\end{center}
\only<4>{\Vskip{-5em}States \ket{+x} and \ket{-x} are also called \ket{+} and \ket{-}, respectively.  They are more familiar as:
\begin{eqnarray*} \ket{+} & = & \frac{1}{\sqrt{2}} \SQB{1}{1} \\
\ket{-} & = & \frac{1}{\sqrt{2}} \SQB{1}{-1}
\end{eqnarray*}
}%

}{%
\[ \ket{\psi} = \Bloch{\theta}{\phi} \]
\begin{BlochSphere}[scale=1.5]{60}{115}
    
    \BlochSpherePhiTheta{}
    \visible<1>{\TZPoint{0,0,1}{$\ket{0}$}{right}}
    \visible<1>{\TZPoint{0,0,-1}{$\ket{1}$}{right}}
    \visible<2-3>{\TZPoint{1,0,0}{$\ket{+x}$}{right}}
    \visible<3>{\TZPoint{-1,0,0}{$\ket{-x}$}{right}}
    \visible<5-6>{\TZPoint{0,1,0}{$\ket{+y}$}{above}}
    \visible<6>{\TZPoint{0,-1,0}{$\ket{-y}$}{right}}
\end{BlochSphere}
}
\Exercise{y}{What are +y and -y, expressed in the standard basis?}
\end{frame}
}

\begin{frame}{Switching between bases}{We consider here the standard and Hadamard bases}
\TwoUnequalColumns{0.7\textwidth}{0.3\textwidth}{%
\begin{itemize}
    \item Typically, our computations begin and end (with measurements) in the standard basis.
    \item We often create a uniform superposition of the standard basis states, \ket{+} or \ket{-}, which together form the \href{https://en.wikipedia.org/wiki/Quantum_logic_gate\#Hadamard_gate}{Hadamard} basis in quantum computing.
    \item<1-> How?  We need a transformation (unitary matrix) \Hadamard{} that sends $\ket{0}\mapsto \ket{+}$ and $\ket{1}\mapsto \ket{-}$.
    \item<3-> From linear algebra, the columns of \Hadamard{} are the transformed versions of \textcolor<4->{\RCtwo}{\ket{0}} and \textcolor<5->{\RCthree}{\ket{1}}.
    \item<6-> Because it is symmetric and its entries are real, \mbox{\Hadamard{} = \Conj{\Hadamard{}}}  $\rightarrow$ \Hadamard{} is also its own inverse.
    
\end{itemize}
}{%
\visible<2->{%
\begin{eqnarray*}
    \SQB{1}{0} & \mapsto & \textcolor<3->{\RCtwo}{\frac{1}{\sqrt{2}}\SQB{1}{1}} \\
    \SQB{0}{1} & \mapsto & \textcolor<3->{\RCthree}{\frac{1}{\sqrt{2}}\SQB{1}{-1}}
    \end{eqnarray*}}
\visible<4->{%
\[
\Hadamard{} = \HMatrix{}
\]
\begin{TIKZP}[overlay]
  \draw[white] (0,0) circle(1pt);  % anchor
  \draw<4-5>[\RCtwo,thick] (2.25,0.7) ellipse(0.3 and 0.6);
  \draw<5>[\RCthree,thick] (3, 0.7) ellipse (0.3 and 0.6);
\end{TIKZP}
}
}
\end{frame}

\begin{frame}{Physics and the Bloch sphere}{Models electron spin in space}
\Vskip{-4em}\TwoUnequalColumns{0.8\textwidth}{0.2\textwidth}{%
\begin{itemize}
\item The labeled axes on the Bloch sphere are often used in physics to model spin in each of the three physical dimensions:
\begin{description}
\item[z axis] spin up or down
\item[x axis] spin in or out
\item[y axis] spin left or right
\end{description}
\item In quantum computing we almost always use the standard basis of \ket{0} and \ket{1} (the z axis).
\item The named axes are mutually orthogonal. \Exercise{Orthogonal Axes}{Prove orthogonality of the x, y, and z axes}
\item While we can characterize two of the dimensions using only real values (the z--x plane, where $\textcolor{\RCone}{\phi}=0$), we require complex values to admit a third mutually orthogonal dimension (y).

\end{itemize}
}{%
\begin{center}
    \begin{BlochSphere}[scale=1]{60}{115}
    
    \BlochSpherePhiTheta{}
    \end{BlochSphere}
\end{center}
}

\end{frame}

\begin{frame}{What other bases are there?}{Quite a few}
\Definition{Two points on the Bloch sphere are \emph{antipodal} if they determine a line that passes through the center of the sphere.}
\TwoColumns{%
\begin{itemize}
    \item Every pair of antipodal points determines an orthogonal basis for the quantum state of a single qubit. \Exercise{Antipodal points determine a basis}{Prove that antipodal points on the Bloch sphere are orthogonal.}
    \item Using linear algebra, we can express any state \ket{\psi} in any basis. \LinkArrow{https://www.youtube.com/watch?v=P2LTAUO1TdA}
\end{itemize}
}{%
\begin{center}
\begin{BlochSphere}[scale=1.5]{60}{115}
    \BlochSpherePhiTheta{}
\end{BlochSphere}
\end{center}
}
\end{frame}

%% \begin{frame}{Antipodal points are orthogonal}{Leave as exercise?}
%% \ToDo{include or not?}
%% \end{frame}

\section{The Pauli matrices}

\begin{frame}{Overview of the Pauli matrices}{We use these as gates in quantum computation}

\begin{itemize}
    \item There are three \href{https://en.wikipedia.org/wiki/Pauli_matrices}{Pauli matrices} that are named for the three axes of the Bloch sphere.
    \item Each matrix has the property of rotating a state by $\pi$ radians about its associated axis.
    \item Each gate is its own inverse.
    \item We will develop a universality argument:  any quantum gate can be realized using a linear combination of the Pauli matrices.
    \item The gates we study here are quantum gates, used in the middle of a quantum circuit.  We can also think of these gates as \emph{measurement operators} but we study that later.
\end{itemize}
    
\end{frame}

\begin{frame}{Pauli \PauliZ{}}{A gate that changes the sign of \ket{1}}

\TwoUnequalColumns{0.6\textwidth}{0.4\textwidth}{%
\[
\PauliZ{} = \ZMatrix{}
\]
\begin{itemize}
    \item<1-> This matrix transforms a state as follows: \[ \ApplyGate{\PauliZ}{\SQB{\alpha}{\beta}} =  \SQB{\alpha}{-\beta} \]
    \item<2-> However, two states are unaffected by \PauliZ{}:
    \begin{itemize}
       \item $\ApplyGate{\PauliZ}{\left(\ket{0}=\SQB{1}{0}\right)} = \ket{0}$
       \item $\ApplyGate{\PauliZ}{\left(\ket{1}=\SQB{0}{1}\right)} = -\ket{1}\equiv \ket{1}$
    \end{itemize}
\end{itemize}
}{%
\Vskip{-4em}\begin{center}
    \begin{BlochSphere}[scale=1]{60}{115}
    \BlochSpherePhiTheta{}
     \TZPoint{0,0,1}{\ket{0}}{right}
     \TZPoint{0,0,-1}{\ket{1}}{below}
    \end{BlochSphere}
\end{center}
\only<2>{%
\Vskip{-2.5em}Such states are called eigenstates (or \href{https://en.wikipedia.org/wiki/Eigenvalues_and_eigenvectors}{eigenvectors}) of \PauliZ{}.  A matrix transforms an eigenstate into a multiple $\lambda$ of itself;  $\lambda$ is called an eigenvalue of the matrix.  Here, $\PauliZ{}\ket{0} = +1\ket{0}$ and $\PauliZ{}\ket{1}=-1\ket{1}$.
}
\only<3>{%
\Vskip{-2em}\PauliZ{} affects all other states, off the z axis, for example
\begin{eqnarray*}
\ApplyGate{\PauliZ}{\ket{+}} & = & \ket{-} \\
\ApplyGate{\PauliZ}{\ket{-}}  & = & \ket{+}
\end{eqnarray*}
rotations of $\pi$ radians around the z~axis.
}}
\end{frame}

\begin{frame}{Pauli \PauliX{}}{Also called the \NamedGate{NOT} gate}

\TwoUnequalColumns{0.6\textwidth}{0.4\textwidth}{%
\[
\PauliX{} = \XMatrix{}
\]
\begin{itemize}
    \item<1-> This transformation exchanges the amplitudes: \[ \ApplyGate{\PauliX}{\SQB{\alpha}{\beta}} =  \SQB{\beta}{\alpha} \]
    \item<2-> The two eigenstates are \ket{+x} and \ket{-x}. \Exercise{unaffected points}{Show that X doesn't affect +x or -x.}
    \item<3-> This gate maps $\ket{0}\mapsto\ket{1}$ and \VV{}.
\end{itemize}
}{%
\begin{center}
    \begin{BlochSphere}[scale=1]{60}{115}
    \BlochSpherePhiTheta{}
    \TZPoint{1,0,0}{\ket{x}}{left}
    \TZPoint{-1,0,0}{\ket{-x}}{right}
    \end{BlochSphere}
\end{center}
\PauliX{} generally
rotates a state  by $\pi$ radians around the x~axis.
}
\end{frame}

\begin{frame}{Pauli \PauliY{}}{Exchanges amplitudes and imposes a sign change}

\TwoUnequalColumns{0.6\textwidth}{0.4\textwidth}{%
\[
\PauliY{} = \YMatrix{}
\]
\begin{itemize}
    \item<1-> This transformation exchanges the amplitudes and imposes a relative sign change: \[ \ApplyGate{\PauliY}{\SQB{\alpha}{\beta}} =  \SQB{-\NiceI\beta}{i\alpha} = -\NiceI\SQB{\beta}{-\alpha} \equiv \SQB{\beta}{-\alpha} \]
    \item<2-> The two eigenstates are \ket{+y} and \ket{-y}. \Exercise{unaffected points}{Define +y and -y in the standard basis and show that Paul Y doesn't affect +y or -y.}
    \item<3-> This gate maps $\ket{0}\mapsto\ket{1}$ and \VV{}.
\end{itemize}
}{%
\begin{center}
    \begin{BlochSphere}[scale=1]{60}{115}
    \BlochSpherePhiTheta{}
    \TZPoint{0,1,0}{\ket{y}}{above}
    \TZPoint{0,-1,0}{\ket{-y}}{left}
    \end{BlochSphere}
\end{center}
\PauliY{} generally
rotates a state  by $\pi$ radians around the y~axis.  
\visible<3>{\MedSkip{}This gate also maps $\ket{+} \mapsto \ket{-}$ and \VV{}.}
}
\end{frame}

\begin{frame}{Properties of Pauli gates}{These should be apparent from our study so far}
\begin{center}
$\PauliX{}=\XMatrix{}$
$\PauliY{}=\YMatrix{}$
$\PauliZ{}=\ZMatrix{}$
\end{center}
\begin{itemize}
    \item Each matrix is \href{https://en.wikipedia.org/wiki/Involutory_matrix}{involutory} (each is its own inverse).
    \item These relations can be verified by computing the appropriate products:
    \begin{center}
    { \setlength{\tabcolsep}{3pt}%
    \begin{tabular}{rcrl@{\hspace{3em}}rcrl@{\hspace{3em}}rcrl} \setlength{\tabcolsep}{1pt}
        \PauliX{}\PauliY{} & = & \NiceI &\PauliZ{} & \PauliZ{}\PauliX{} & = & \NiceI &\PauliY{} & \PauliY{}\PauliZ{} & = & \NiceI &\PauliX{} \\
        \PauliY{}\PauliX{} & = & $-\NiceI$ &\PauliZ{} &  \PauliX{}\PauliZ{} & = & $-\NiceI$ &\PauliY{} & \PauliZ{}\PauliY{} & = & $-\NiceI$ &\PauliX{}
    \end{tabular}}
    \end{center}
    \item Each matrix has \href{https://en.wikipedia.org/wiki/Eigenvalues_and_eigenvectors}{eigenvalues} $\pm 1$.
    \item Each Pauli matrix rotates all states by $\pi$ radians about its associated axis, except the two states at the intercept of its axis with the unit Bloch sphere.  The eigenstates are therefore those intercepts.
\end{itemize}
\end{frame}

%%
%% This material might move to module 12
%%
\begin{frame}{Eigenvalues of the Pauli gates}{And their corresponding eigenvectors}
\only<1-5>{%
\begin{itemize}
    \item<1-> In quantum computing, a \href{https://en.wikipedia.org/wiki/Measurement_in_quantum_mechanics}{measurement of a quantum system} is modeled by an operator $M$ called an \href{https://en.wikipedia.org/wiki/Observable}{observable}.  We can generally view the measurement as yielding $+1$ or $-1$.
    \item<2-> After observation, the system is in a state $\psi$ such that subsequent applications of $\psi$ certainly yield the same measurement.
    \item<3-> If a system has been measured by an operator $M$ and it entered some state $\psi$, then we must have \[ M\psi = \lambda\psi \] where $\lambda=\pm 1$ is the measured result.
    \item<4-> The above equation is the \href{https://en.wikipedia.org/wiki/Eigenvalues_and_eigenvectors\#Formal_definition}{definition} of an eigenvalue~$\lambda$ and its eigenvector~$\psi$.
    \item<5-> For qubits, measurement operators yield $\pm 1$ and there will be a distinct eigenvector for each of those eigenvalues.
\end{itemize}}%
\only<6->{%
\TwoColumns{%
\only<6-12>{%
\begin{itemize}
    \item<6-> Each Pauli matrix is an observable.
    \item<7-> The \PauliZ{} matrix corresponds to the \href{https://en.wikipedia.org/wiki/Qubit\#Standard_representation}{computational basis}, because its eigenvectors are \visible<8->{\alert<8>{\ket{0}}} \visible<9->{and \alert<9>{\ket{1}}.}
    \item<10-> When an \alert<10>{arbitrary state~$\psi$}\visible<11->{ is measured as \ket{0} or \ket{1}, subsequent applications of \PauliZ{} produce the same eigenvalue and eigenvector.}  
    \item<12-> If $\psi$ happens to be \ket{0} or \ket{1}, then re-measurement must yield same state.
\end{itemize}}%
\only<13->{%
\begin{itemize}
    \item<13-> Consider the observable \PauliX{} with eigenvectors \visible<14->{\alert<14>{\ket{+}}} \visible<15->{and \alert<15>{\ket{-}}.}
\end{itemize}
}
}{%
\only<8-12>{%
\begin{alignat*}{5}
    \ZMatrix{} & \alert<8>{\PZero{}} & \mbox{ = } & +1  & \alert<8>{\PZero{}}  \\
    \visible<9->{\ZMatrix{} & \alert<9>{\POne{}} & \mbox{ = } & -1  & \alert<9>{\POne{}} }
\end{alignat*}}%
\only<13->{%
\begin{alignat*}{5}
    \XMatrix{} & \alert<14>{\PHad{}} & \mbox{ = } & +1  & \alert<14>{\PHad{}}  \\
    \visible<15->{\XMatrix{} & \alert<15>{\PHadm{}} & \mbox{ = } & -1  & \alert<15>{\PHadm{}} }
\end{alignat*}}
\begin{center}
\begin{TIKZP}
\visible<8->{
\draw[->,ultra thick] (0,0) -- (1,0) node[right] {\ \alert<8>{\ket{0}}};
\draw[->,ultra thick] (0,0) -- (0,1) node[above] {\alert<9>{\ket{1}}};
\visible<10>{\alert{\draw[->,thick] (0,0) -- (30:1) node[above] {$\psi$};}}
\visible<14>{\alert{\draw[->,thick] (0,0) -- (45:1) node[above] {\ket{+}};}}
\visible<15>{\alert{\draw[->,thick] (0,0) -- (-45:1) node[right] {\ket{-}};}}
}
\visible<8>{\begin{scope}[draw=\RCtwo,rotate=90]\PFilter{-1}{-1.2}{1}{1.2}{20}\end{scope}}
\visible<9>{\begin{scope}[draw=\RCthree]\PFilter{-1}{-1.2}{1}{1.2}{20}\end{scope}}
\visible<14>{\begin{scope}[draw=\RCtwo,scale=0.7,rotate=-45]\PFilter{-1}{-1.2}{1}{1.2}{20}\end{scope}}
\visible<15>{\begin{scope}[draw=\RCthree,scale=0.7,rotate=45]\PFilter{-1}{-1.2}{1}{1.2}{20}\end{scope}}
\end{TIKZP}
\end{center}
}
}

\end{frame}
%%
%% End of stuff that might move to module 12
%%

\section{General rotations}
{%
\def\Redify#1{\textcolor{RedOrange}{#1}}%
\def\Bluify#1{\textcolor{\RCtwo}{#1}}%
\def\Term#1{\frac{i^{#1}\theta^{#1}A^{#1}}{#1!}}%
\def\TermA#1{\frac{i\theta^{#1}A}{#1!}}%
\def\TermI#1{\frac{\theta^{#1}\Identity{}}{#1!}}%
\def\TermF#1{\frac{\theta^{#1}}{#1!}}%
\begin{frame}{A useful theorem for expressing rotations}{A matrix extension of \href{https://en.wikipedia.org/wiki/Euler\%27s_formula}{Euler's formula}}


\Vskip{-3em}
\only<1-6>{%
\[  \alert<2>{A^{2}=\Identity{}} \rightarrow \ExpPhase{\theta A} = \Complex{\cos(\theta)\ \Identity{}}{\sin(\theta)\ A} \]}
\TwoColumns{%
Recall the following \href{https://en.wikipedia.org/wiki/Taylor_series}{Taylor series}:
\begin{align*}
\Exp{x} = & 1 + \frac{x}{1!} + \frac{x^2}{2!} + \frac{x^3}{3!} + \cdots \\
\Redify{\cos(x)} = & \Redify{1 - \frac{x^2}{2!} + \frac{x^4}{4!} - \frac{x^6}{6!} +  \cdots} \\ 
\Bluify{\sin(x)} = & \Bluify{\frac{x}{1!} - \frac{x^3}{3!} + \frac{x^5}{5!} - \frac{x^7}{7!} + \cdots}
\only<2>{%
\\ i^{2} = & -1 \\
   i^{3} = & - i \\
   i^{4} = & \ 1
}
\end{align*}
}{%
\Vskip{-4.3em}\begin{align*}
\ExpPhase{\theta A}
\only<1-2>{%
= {} &
\Identity{} + \Term{1} + \alert<2>{\Term{2}} + \Term{3} \\[0.5em] & + \alert<2>{\Term{4}} + \Term{5} + \cdots \\[1em]
}%
\only<2-3>{%
  = {} & 1\,\Identity{} + 
  \TermA{1} - 
  \alert<2>{\TermI{2}} - 
  \TermA{3} \\[0.5em] & + 
  \alert<2>{\TermI{4}} + 
  \TermA{5} +\cdots\\[1em]}%
\only<4-5>{%
  = {} & \Redify{1\,\Identity{}} + 
  \Bluify{\TermA{1}} - 
  \Redify{\TermI{2}} - 
  \Bluify{\TermA{3}} \\[0.5em] & + 
  \Redify{\TermI{4}} + 
  \Bluify{\TermA{5}}+\cdots\\[1em]}%
\only<5-6>{%
  = {} & \Identity{}\left[ \Redify{1} - 
   \Redify{\TermF{2}}  
  + \Redify{\TermF{4}} - \cdots\right]
  \\[0.5em] &
  + i\,\left[
  \Bluify{\TermF{1}}
  - \Bluify{\TermF{3}} 
   + 
  \Bluify{\TermF{5}} - \cdots\right]\,A\\[1em]
  }%
\only<6>{%
= {} & \Identity{}\left[ \Redify{\cos(\theta)} \right]
  \\[0.5em] &
  + i\,\left[
  \Bluify{\sin(\theta)}\right]\,A \QEDsym{}
}
\end{align*}
}
\end{frame}
}


\begin{frame}{Arbitrary rotations}{The Pauli matrices rotate by $\pi$ radians;  how can we rotate by $\theta$ radians?}
We can use the extended Euler formula to express rotation about the Bloch x, y, or z axes:
\begin{align*}
R_{x}(\theta) = & \ExpNegPhase{\left(\frac{\theta}{2}\right)\,\PauliX{}}\\[0.7em]
R_{y}(\theta) = & \ExpNegPhase{\left(\frac{\theta}{2}\right)\,\PauliY{}}\\[0.7em]
R_{z}(\theta) = & \ExpNegPhase{\left(\frac{\theta}{2}\right)\,\PauliZ{}}
\end{align*}
where the matrices in the exponents are the Pauli matrices.

When $\theta=\pi$, $\cos(-\pi/2)=0$ and $\sin(-\pi/2)=-1$ $\rightarrow$ $R_a(\pi) = -i\ A \equiv A$, matching the Pauli gates in terms of their rotation by $\pi$ radians about their respective axes.
\end{frame}

\def\EulerP#1{%
\ComplexDiff{\cos\left(\frac{\theta}{2}\right)\Identity}{\sin\left(\frac{\theta}{2}\right)#1}}

\begin{frame}{Matrix formulation of rotating gates}{We use the matrix form of Euler's formula}

\begin{align*}
R_{x}(\theta) = & \ExpNegPhase{\left(\frac{\theta}{2}\right)\,\PauliX{}}%
\visible<2-3>{= \EulerP{\PauliX{}}}%
\visible<3>{= \SQBG{\relax}{\cos\left(\frac{\theta}{2}\right)}{\ImaginaryM{\sin\left(\frac{\theta}{2}\right)}}{\ImaginaryM{\sin\left(\frac{\theta}{2}\right)}}{\cos\left(\frac{\theta}{2}\right)}}\\[0.7em]
%
%
\alert<4>{R_{y}(\theta)} = & \ExpNegPhase{\left(\frac{\theta}{2}\right)\,\PauliY{}}%
\visible<2-3>{= \EulerP{\PauliY{}}}
\visible<3-4>{= \alert<4>{\SQBG{\relax}{\cos\left(\frac{\theta}{2}\right)}{-\sin\left(\frac{\theta}{2}\right)}{\sin\left(\frac{\theta}{2}\right)}{\cos\left(\frac{\theta}{2}\right)}}}\\[0.7em]
%
%
R_{z}(\theta) = & \ExpNegPhase{\left(\frac{\theta}{2}\right)\,\PauliZ{}}%
\visible<2-3>{= \EulerP{\PauliZ{}}}%
\visible<3>{= \SQBG{\relax}{\ExpNegPhase{\frac{\theta}{2}}}{0}{0}{\ExpPhase{\frac{\theta}{2}}}
\equiv \SQBG{\relax}{1}{0}{0}{\ExpPhase{\theta}}}
\end{align*}
\visible<4>{\alert<4>{Given a state with real amplitudes ($\phi=0$ on the Bloch sphere), this rotation shifts probability between \ket{0} and \ket{1} while keeping the amplitudes real.}}
\end{frame}

\section{Mixed states}

\begin{frame}{Interior points of the Bloch sphere}{Representing mixed states}
\Vskip{-2.2em}We now draw a distinction between the following two scenarios.  \visible<2->{On the left, we have an unpolarized photon.}  \visible<3->{On the right, we have a photon that is passed randomly through a horizontal or vertical polarizing filter.}
\Vskip{-3em}\TwoColumns{%
\begin{center}
\begin{TIKZP}[scale=1.0]
\visible<1->{\RadiantArrows{22.5}{->,color=BrickRed,thick}}
\end{TIKZP}\end{center}
\visible<2->{\ColorOne{\[\QState{} = \QPlus{}\]}}
}{%
\begin{center}
\begin{TIKZP}[scale=0.64]
\visible<1->{
\draw[->,ultra thick] (0,0) -- (1,0) node[right] {\ \ket{0}};
\draw[->,ultra thick] (0,0) -- (0,1) node[above] {\ket{1}};
}
\visible<1->{\begin{scope}[draw=\RCtwo,rotate=90]\PFilter{-1}{-1.2}{1}{1.2}{20}\end{scope}}
\visible<1->{\begin{scope}[draw=\RCthree]\PFilter{-1}{-1.2}{1}{1.2}{20}\end{scope}}
\end{TIKZP}\end{center}
\visible<3->{\Vskip{-3em}\ColorTwo{\begin{align*}
    \QState{} &= \QZero{} &\mbox{ with probability 0.5} \\
    &= \QOne{} &\mbox{ with probability 0.5}
\end{align*}}}
}%
\Vskip{-3.5em}
\only<3>{%

These states are indeed different.  Consider applying \Hadamard{} to the \ColorOne{left photon}.  You will get \QZero{}.  But applying \Hadamard{} to the \ColorTwo{right photon} yields \ket{+} or \ket{-}, each with probability~0.5.}%
\only<4->{

The \ColorOne{left state} is \href{https://en.wikipedia.org/wiki/Quantum_state\#Pure_states}{pure}, and is found on the surface of the Bloch sphere as \ket{+x}.   The \ColorTwo{right state} is \href{https://en.wikipedia.org/wiki/Quantum_state\#Mixed_states}{mixed}. Recalling \QZero{} and \QOne{} are the north and south poles, respectively, the \ColorTwo{right state} is the vector sum from the origin of~0.5\QZero{} and~0.5\QOne{}, which places it statistically at the center of the sphere.}
    
\end{frame}

\begin{frame}{The Bloch sphere}{Inside and out}
\TwoColumns{%
\begin{itemize}[<+->]
    \item A \ColorOne{pure state} can be written as a single wave function equation.  
    \item Such states are found \ColorOne{on the surface} of the Bloch sphere.
    \item A \ColorTwo{mixed state} is a probability distribution of pure states.  
    \item The vector sum of those states from the sphere center, each weighted by its probability in the mixture, locates the state \ColorTwo{somewhere inside} the Bloch sphere.
\end{itemize}
}{%
\begin{center}
    \begin{BlochSphere}[scale=1.2]{60}{115}
    \visible<2>{\draw[ball color = \RCone, opacity = 0.5] (0,0,0) circle (1cm);}
    \visible<4>{%
    \fill[ball color = \RCtwo, opacity = 1.0] (0,0,0) circle (1cm);
    }
    \end{BlochSphere}
\end{center}
\visible<5->{%
\BigSkip{}
Thus, a mixed state of just a single pure state is located on the sphere's surface. 
}
}
\end{frame}

\begin{frame}{Summary}

\begin{itemize}
    \item Two real parameters suffice for specifying the state of a single quantum bit.
    \item The Bloch sphere is a convenient visualization of a single qubit's state.
    \item We have studied four gates of interest so far:
    \begin{itemize}
        \item The Hadamard gate (\Hadamard) creates a uniform superposition of \ket{0} and \ket{1} given the input \ket{0}.
        \item The \PauliX{}, \PauliY{}, and \PauliZ{} gates rotate state by $\pi$ radians about their respective axes.
    \end{itemize}
    \item There are matrices that describe arbitrary rotations about the Bloch axes. 
    \item Interior points of the Bloch sphere represent mixed states.
\end{itemize}
\end{frame}
