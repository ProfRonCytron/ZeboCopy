\def\OtherAuthors{, Jeffery Chai, and Mitch Oldham}
\SetTitle{33}{Error Correction}{Using syndrome measurements}{33}

% Slide: Encoding and Decoding with an Error
\begin{frame}{Encoding and Decoding with an Error}
    \begin{columns}
        \begin{column}{0.5\textwidth}
            \centering
            Encoding:
            \begin{align*}
                \ket{0} &\rightarrow \ket{000} \\
                \ket{1} &\rightarrow \ket{111}
            \end{align*}
        \end{column}
        \begin{column}{0.5\textwidth}
            \centering
            Decoding with an Error:
            \begin{quantikz}
                & \lstick{$\ket{0}$} & \qw & \qw & \qw & \rstick{$\ket{0}$} \qw \\
                & \lstick{$\ket{0}$} & \qw & \gate{X} & \qw & \rstick{$\ket{1}$} \qw \\ % Middle qubit flip from 0 to 1
                & \lstick{$\ket{0}$} & \qw & \qw & \qw & \rstick{$\ket{0}$} \qw \\
            \end{quantikz}
        \end{column}
    \end{columns}
\end{frame}

\begin{frame}{Encoding and Decoding with an Error}
    \begin{align*}
        \text{No Error} & : 1-P \\
        \text{Error} & : P \\
        \text{No Error} & : (1-P)^3 \\
        \text{1 Error} & : 3P(1-P)^2 \\
        \text{2 Error} & : 3P^2(1-P) \\
        \text{3 Error} & : P^3 \\
        \text{Majority will get correct (No-1 Error)} & : 1-3P^2 + 2P^3 \\
        \text{2 Error - 3 Error} & : 3P^2 - 2P^3 \\
    \end{align*}
\end{frame}

\begin{frame}{Encoding and Decoding with an Error}

\begin{quantikz}
    & \lstick{$\alpha\ket{0} + \beta\ket{1}$} & \qw & \ctrl{1} & \qw & \ctrl{2}& \qw& \qw\\
    & \lstick{$\ket{0}$} & \qw & \targ{} & \qw & \qw & \qw& \qw \\
    & \lstick{$\ket{0}$} & \qw & \qw & \qw & \targ{}& \qw& \qw \\
\end{quantikz}

\[ \alpha\ket{10} + \beta\ket{11} \rightarrow \alpha\ket{100} + \beta\ket{111} \]
This is how we do an encoding from less qubits to more qubits.

\end{frame}

\begin{frame}{shor's three qubit example}

    \begin{quantikz}
        & \lstick{$\alpha\ket{0} + \beta\ket{1}$} & \qw & \ctrl{1} & \qw & \ctrl{2} & \qw & \gate[wires=3]{Error} & \ctrl{3}  & \qw & \qw & \qw & \qw & \qw\\
        & \lstick{$|0\rangle$} & \qw & \targ{} & \qw & \qw & \qw & \qw & \qw & \ctrl{2} & \ctrl{3} & \qw & \qw & \qw \\
        & \lstick{$|0\rangle$} & \qw & \qw & \qw & \targ{} & \qw & \qw & \qw & \qw & \qw & \ctrl{2} & \qw & \qw \\
        & \lstick{$|0\rangle$} & \qw & \qw & \qw & \qw & \qw & \qw & \targ{} & \targ{} & \qw & \qw & \qw & \meter{} \\
        & \lstick{$|0\rangle$} & \qw & \qw & \qw & \qw & \qw & \qw & \qw & \qw & \targ{} & \targ{} & \qw & \meter{} \\
    \end{quantikz}

\end{frame}


\begin{frame}{Explanation of Shor's Three-Qubit Quantum Error Correction}

\textbf{Objective:} Protect the quantum state \(\alpha\ket{0} + \beta\ket{1}\) from errors by encoding it in a three-qubit system.

\begin{itemize}
    \item \textbf{Encoding:} The initial state is spread across three qubits using CNOT gates. This creates a form of redundancy.
    
    \item \textbf{Error:} Represents any potential error that could occur on the qubits during computation. Could be a bit flip, phase flip, or both.
    
    \item \textbf{Correction:} 
    \begin{itemize}
        \item If an error occurs on a qubit, it will make the state of that qubit differ from the other two.
        \item By using two auxiliary qubits (the bottom two lines), we can determine which qubit (if any) has an error.
        \item The auxiliary qubits, after interacting with the main qubits through CNOT gates, are measured.
        \item Based on the measurement outcome, we can infer where the error occurred and correct it.
    \end{itemize}
\end{itemize}


\end{frame}


\begin{frame}{Phase flip code example}
    
\begin{quantikz}
\lstick{\ket{\psi}} & \ctrl{1} & \ctrl{2} & \gate{H} & \qw & \gate[wires=3]{E_{\text{phase}}} & \qw & \gate{H} & \ctrl{2} & \ctrl{1} & \qw & \rstick{\ket{\psi}} \qw \\
\lstick{\ket{0}} & \targ{} & \qw & \gate{H} & \qw & & \qw & \gate{H} & \qw & \targ{} & \qw & \rstick{\ket{0}} \qw \\
\lstick{\ket{0}} & \qw & \targ{} & \gate{H} & \qw & & \qw & \gate{H} & \targ{} & \qw & \qw & \rstick{\ket{0}} \qw \\
\end{quantikz}
\end{frame}


\begin{frame}{Explanation of Shor's Three-Qubit Quantum Error Correction}

\textbf{Objective:} Protect the quantum state \(\alpha\ket{0} + \beta\ket{1}\) from errors by encoding it in a three-qubit system.

\begin{itemize}
    \item \textbf{Encoding:} The initial state is spread across three qubits using CNOT gates and Hadamard gates to create an entangled state. This redundancy allows for the detection and correction of phase-flip errors.
    
    \item \textbf{Error Channel (E\textsubscript{phase}):} Represents the environment that may introduce phase flip errors, modeled by the Pauli-Z operator. The state of the system could be:
    \begin{itemize}
        \item No error: \(\alpha\ket{000} + \beta\ket{111}\)
        \item 1st qubit flipped: \(\alpha\ket{100} + \beta\ket{011}\)
    \end{itemize}
    
    \item \textbf{Error Detection and Correction:} 
    \begin{itemize}
        \item Hadamard gates are applied again to all qubits to facilitate error detection.
        \item A series of CNOT gates followed by measurements determines if an error occurred.
        \item If an error is detected, a corrective operation is applied to restore the original state.
    \end{itemize}
    
    \item \textbf{Hadamard Gate on Basis States:} Recall the transformation properties of the Hadamard gate:
    \begin{itemize}
        \item \(H \ket{0} = \frac{1}{\sqrt{2}}(\ket{0} + \ket{1})\) which is equivalent to \(\ket{+}\)
        \item \(H \ket{1} = \frac{1}{\sqrt{2}}(\ket{0} - \ket{1})\) which is equivalent to \(\ket{-}\)
    \end{itemize}
\end{itemize}

\end{frame}






\begin{frame}{Encoding and Decoding with an Error}

    \begin{align*}
    E_0 & : I \otimes I \otimes I & \text{No Bit Flip} \\
    E_1 & : \underline{X} \otimes I \otimes I & \text{Bit flip on 1} \\
    E_2 & : I \otimes \underline{X} \otimes I & \text{Bit flip on 2} \\
    E_3 & : I \otimes I \otimes \underline{X} & \text{Bit flip on 3} \\
    \end{align*}
    X means an error in the bit.

\end{frame}



\begin{frame}{Modified 9-qubit Quantum Circuit}
    \centering
    \scalebox{0.5}{
        \begin{quantikz}
            & \lstick{$\ket{\phi}$} & \gate[wires=9]{Enc} & \qw & \gate[wires=3]{H} & \gate[wires=3]{Enc} & \gate[wires=3]{R} & \gate[wires=3]{H} & \gate[wires=9]{R} & \qw \\
            & \lstick{} & \qw & \qw &  &  & \qw &  & \qw & \qw \\
            & \lstick{} & \qw & \qw &  &  & \qw &  & \qw & \qw \\
            & \lstick{} & \qw & \qw & \gate[wires=3]{H} & \gate[wires=3]{Enc} & \gate[wires=3]{R} & \gate[wires=3]{H} &  & \qw \\
            & \lstick{} & \qw & \qw &  &  & \qw &  & \qw & \qw \\
            & \lstick{} & \qw & \qw &  &  & \qw &  & \qw & \qw \\
            & \lstick{} & \qw & \qw & \gate[wires=3]{H} & \gate[wires=3]{Enc} & \gate[wires=3]{R} & \gate[wires=3]{H} &  & \qw \\
            & \lstick{} & \qw & \qw &  &  & \qw &  & \qw & \qw \\
            & \lstick{} & \qw & \qw &  &  & \qw &  & \qw & \qw \\
        \end{quantikz}
    }
    \begin{itemize}
        \item Outer Layer: Corrects phase flips using an encoding (Enc) followed by a Hadamard flip (H).
        \item Inner Layer: Addresses bit flips through an encoding (Enc), error correction (R), and Hadamard flips (H). This nested structure ensures robust error rectification for both common quantum errors.
    \end{itemize}
\end{frame}








\subsection*{Magnus and Kaden}

\begin{frame}{Arbitrary CNOT gate}{As a unitary matrix}
\begin{itemize}
    \item Background: Two-Qubit Control "U" Gate
    \item The 2-qubit CU gate is a conditional gate that applies a unitary operation U to the target qubit only if the control qubit is in the state \(|1\rangle\).
    \only<2>{\item We can express the state of the Qubits as the product of the amplitudes on \(|0\rangle\) and \(|1\rangle\)}
\end{itemize}
\begin{center}
    \begin{quantikz}
        \lstick{\textcolor{red}{\text{Control}}} & \ctrl{1} & \qw \\
        \lstick{\text{Target}} & \gate{U} & \qw
    \end{quantikz}
\end{center}
\visible<2-4>{%
\[
CU = \textcolor{red}{\underset{\text{Control Bit}}{|0\rangle\langle0|}} \otimes \visible<3->{\underset{\text{Nothing}}{I}} + \textcolor{red}{\underset{\text{Control bit}}{|1\rangle\langle1|}} \otimes \visible<3->{\underset{\text{Apply "U" gate}}{U}}
\]
}
\visible<4>{%
\begin{itemize}
    \item The order of the tensor products is reflective of which bit is the control and which bit is the target.
\end{itemize}
}
\end{frame}

\begin{frame}{Arbitrary Control Gate}{CU from control bit i to target j}
\begin{itemize}
\only<1-3>{%
\item Both these equations just present the steps to algebraically order the identity matrix \(I\), basis projection \(|0\rangle\langle0|\) or \(|1\rangle\langle1|\), and the arbitrary control matrix \(U\).
}
\vspace{0.5cm}
\only<2-3>{%
\item if (i \textless\ j)
\[
    CU_{i, j} = \left[I\otimes\right]^{(i-1)} |0\rangle\langle0| \left[\otimes I\right]^{(n-i)} + \left[I\otimes\right]^{(i-1)} |1\rangle\langle1| \left[\otimes I\otimes\right]^{(j-i)-1} U \left[\otimes I\right]^{(n-j)}
\]
}
\only<3>{%
\item if (i \textgreater\ j)
\[
    CU_{i, j} = \left[I\otimes\right]^{(i-1)} |0\rangle\langle0| \left[\otimes I\right]^{(n-i)} + \left[I\otimes\right]^{(j-1)} U \left[\otimes I \otimes\right]^{(i-j)-1} |1\rangle\langle1| \left[\otimes I\right]^{(n-i)}
\]
}
\end{itemize}
\end{frame}




\begin{frame}{Example: \(CX_{3,1}\)}
\begin{itemize}
    \only<1-4>{%
    \item In the case of \(CX_{3,1}\), \(U\) is the \(X\) gate, and the control is on the third qubit. We try to represent this operation algebraically as:
\[
CX = \underset{\text{Qubi/t 1}}{|0\rangle\langle0|} \otimes \underset{\text{Qubit 2}}{I} \otimes \underset{\text{Qubit 3}}{I} + \underset{\text{Qubit 1}}{|1\rangle\langle1|} \otimes \underset{\text{Qubit 2}}{I} \otimes \underset{\text{Qubit 3}}{X}
\]
}
    \only<2>{%
    However, this algebraic representation corresponds to a control on the first qubit, and target on the third qubit.

    \begin{center}
    \begin{quantikz}
        \lstick{Control} & \ctrl{2} & \qw \\
        \lstick{Auxiliary} & \qw & \qw \\
        \lstick{Target} & \targ{} & \qw
    \end{quantikz}
    \end{center}
    }

    \only<3-4>{\item We must reorder the tensor products to reflect a control on the 3rd qubit and target on the 1st qubit.}

    \only<4>{%
\[
CX_{3,1} = \textcolor{red}{\underset{\text{Target}}{I} \otimes \underset{\text{Nothing}}{I} \otimes \underset{\text{Control}}{|0\rangle\langle0|} + \underset{\text{Target}}{X} \otimes \underset{\text{Nothing}}{I} \otimes \underset{\text{Control}}{|1\rangle\langle1|}}
\]
}
\end{itemize}
    
\end{frame}
\begin{frame}{Computed Matrix Representation}
\vspace{-.8cm}
\begin{center}
    \[
    CX_{3,1} = \textcolor{red}{\underset{\text{Target}}{I} \otimes \underset{\text{Nothing}}{I} \otimes \underset{\text{Control}}{|0\rangle\langle0|} + \underset{\text{Target}}{X} \otimes \underset{\text{Nothing}}{I} \otimes \underset{\text{Control}}{|1\rangle\langle1|}}
    \]
    \[
    \begin{pmatrix}
    1 & 0 & 0 & 0 & 0 & 0 & 0 & 0 \\
    0 & 0 & 0 & 0 & 0 & \textcolor{red}{1} & 0 & 0 \\
    0 & 0 & 1 & 0 & 0 & 0 & 0 & 0 \\
    0 & 0 & 0 & 0 & 0 & 0 & 0 & \textcolor{red}{1} \\
    0 & 0 & 0 & 0 & 1 & 0 & 0 & 0 \\
    0 & \textcolor{red}{1} & 0 & 0 & 0 & 0 & 0 & 0 \\
    0 & 0 & 0 & 0 & 0 & 0 & 1 & 0 \\
    0 & 0 & 0 & \textcolor{red}{1} & 0 & 0 & 0 & 0
    \end{pmatrix}
    \]
\end{center}
    
\end{frame}

\begin{frame}{Apply to Three-Qubit State}
    \begin{itemize}
    \only<1-2>{%
        \item Consider a three-qubit system with qubits X, Y, and Z, each in a superposition state:
    \[
        \begin{aligned}
        X &: a_x \ket{0} + b_x \ket{1}, \\
        Y &: a_y \ket{0} + b_y \ket{1}, \\
        Z &: a_z \ket{0} + b_z \ket{1}.
        \end{aligned}
    \]
    }
    \only<2>{%
        \begin{align*}
        \ket{\psi_{\text{initial}}} &= (a_x\ket{0} + b_x\ket{1}) \otimes (a_y\ket{0} + b_y\ket{1}) \otimes (a_z\ket{0} + b_z\ket{1}) \\
        &= a_x a_y a_z \ket{000} + a_x a_y b_z \ket{001} \\
        &\quad + a_x b_y a_z \ket{010} + a_x b_y b_z \ket{011} \\
        &\quad + b_x a_y a_z \ket{100} + b_x a_y b_z \ket{101} \\
        &\quad + b_x b_y a_z \ket{110} + b_x b_y b_z \ket{111}
        \end{align*}
    }
    \only<3->{%
    \item After applying \(CX_{3,1}\) to \(\ket{\psi_{\text{initial}}}\), the final state \(\ket{\psi_{\text{final}}}\) becomes:

        \[
    \begin{aligned}
    \ket{\psi_{\text{final}}} &= a_x a_y a_z \ket{000} + \textcolor{red}{b_x} a_y b_z \ket{001} + a_x b_y a_z \ket{010} + \textcolor{red}{b_x} b_y b_z \ket{011} \\
    &+ b_x a_y a_z \ket{100} + \textcolor{red}{a_x} a_y b_z \ket{101} + b_x b_y a_z \ket{110} + \textcolor{red}{a_x} b_y b_z \ket{111}.
    \end{aligned}
    \]

    \begin{align*}
    \ket{\psi_{\text{initial}}} &= a_x a_y a_z \ket{000} + \textcolor{red}{a_x} a_y b_z \ket{001} \quad + a_x b_y a_z \ket{010} + \textcolor{red}{a_x} b_y b_z \ket{011} \\
    & \quad + b_x a_y a_z \ket{100} + \textcolor{red}{b_x} a_y b_z \ket{101} \quad + b_x b_y a_z \ket{110} + \textcolor{red}{b_x} b_y b_z \ket{111}
    \end{align*}

    \visible<4>{
    \item The \(CX_{3,1}\) operation flips the first qubit when the third qubit is in the state \(\ket{1}\). This flip is manifested as a change in the amplitude from \(a\) to \(b\) or vice versa. 
    }
    }
    \end{itemize}

\end{frame}