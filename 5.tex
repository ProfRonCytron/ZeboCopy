\SetTitle{5}{Single qubit systems}{How do we characterize a single qubit?}{singlequbit}

\section{Overiew}
\begin{frame}{Qubits and beyond}{How many outcomes can we see?}
\begin{itemize}
    \item A quantum computing system contains elements, such as electrons, ions, or photons, whose quantum behavior can be influenced to achieve computation.
    \item When measured, each quantum element will be in one of a mutually orthogonal set of $d$~states.
    \item For example, a single photon that goes through beam splitters as shown here will be measured in one of three places.
    \item The abstraction representing the state of such an element is generally a \emph{qudit}.
    \item The measurement is measured to be in 
    \item In a quantum systems, the possible measurements for a single The outcomes of a quantum measurement form a mutually orthogonal set of quantum states.
    \item If there are $d>2$ possible outcomes, we represent a quantum state by a $d$-valued 
    \href{https://en.wiktionary.org/wiki/qudit}{qudit}.  For example, an electron that has $d$ possible excited states 
    \item If $d=3$ the qudit is sometimes called
    a \href{https://en.wikipedia.org/wiki/Qutrit}{qutrit}.
    \item More commonly, in quantum computing we consider quantum states with only two outcomes.  
    
\end{itemize}
\end{frame}
\begin{frame}{test}{test}
\begin{TIKZP}
    \LightSource{}
    \Shift{1}{1}{\Mirror{}}
    \Shift{1}{0}{\BeamSplitter[rotate=45]{}}
\end{TIKZP}

\end{frame}
\begin{frame}{test}{test}
\begin{TIKZP}
    \Shift{-2}{-0.5}{%
    \LightSource{}}
    \BeamSplitter[rotate=-45]{}
    \Shift{2}{0}{
    \BeamSplitter[rotate=-45]}
\end{TIKZP}
    
\end{frame}
