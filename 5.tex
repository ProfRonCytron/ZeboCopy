\SetTitle{5}{Single qubit systems}{How do we characterize quantum systems?}{singlequbit}

\section{Overiew}

\begin{frame}{Orthogonality}{A set of mutually exclusive outcomes}
\begin{itemize}
    \item A quantum computing system contains elements, such as electrons, ions, or photons, whose quantum behavior can be influenced to achieve computation.
    \item When measured, each quantum element will be in one of a mutually orthogonal set of $d$~states.
    
\end{itemize}
\OnlyRemark{2}{In a physical system, the outcomes will be physically exclusive, so that only one of the~$d$ outcomes can ever occur.

Mathematically, we will model that behavior using orthogonal basis vectors.}
\end{frame}

\section{Quantum games}

\begin{frame}{Quantum game components}{From \href{http://play.quantumgame.io/}{Play Quantum games}}

\TwoUnequalColumns{0.65\textwidth}{0.35\textwidth}{%

\begin{itemize}

    \item<1-> This is an ideal light source that emits exactly one photon at a time.  We are interested in the path(s) a single photon can take.
    \item<2-> This mirror reflects a photon at a \Degrees{90} angle.
    \item<3-> With equal probability, \alert{yet utter unpredictability}, a beam splitter takes a photon and either
    \begin{itemize} 
       \item allows the photon to continue undisturbed, or
       \item reflects the photon like a mirror.
    \end{itemize}
\end{itemize}

}{%
\Vskip{-3em}\begin{center}
\visible<1->{
\begin{TIKZP}
    \LightSource{}
\end{TIKZP}}

\visible<2->{
\begin{TIKZP}
    \draw[->,thick,color=red] (0,0.5) -- ++(0.5,0) -- ++(0,-0.5);
    \RotateAroundCenter{-45}{\Mirror{}}
\end{TIKZP}}

\visible<3->{
\begin{TIKZP}
    \draw[->,thick,color=red] (0,0.5) -- ++(0.5,0) -- ++(0,-0.5);
    \draw[->,thick,color=red] (0.5,0.5) -- (1.0,0.5);
    \RotateAroundCenter{-45}{\BeamSplitter{}}
\end{TIKZP}}
\end{center}
}
\OnlyRemark{4}{%
The beam splitter places a photon into a \emph{superposition} of two paths.  In pictures that show possible paths, remember that there is only \emph{one} photon depicted.
}

\end{frame}

\begin{frame}{A photon in superposition}{There are two possible paths.}
\TwoColumns{%
\begin{itemize}
    \item<1-> The light source emits a single photon.
    \item<2,4-> \textcolor{orange}{In the path shown here, the beam splitter does not reflect the photon.}
    \item<3,4-> \textcolor{NavyBlue}{In the path shown here, the photon is reflected by the beam splitter and again by the mirror.}
\end{itemize}
}{%
\begin{center}
\begin{TIKZP}[scale=0.75]
    \LightSource{}
    \Shift{4}{0}{\RotateAroundCenter{-45}{\BeamSplitter}}
    \Shift{4}{-2}{\RotateAroundCenter{-45}{\Mirror}}
    \Shift{6}{0}{\Measurement[color=orange]{}}
    \Shift{6}{-2}{\Measurement[color=NavyBlue]{}}
    \visible<2,4->{\draw[thick,color=orange] (4.5,0.5) -- (6,0.5);}
    \visible<3->{\draw[thick,color=NavyBlue] (4.5,0.5) -- ++(0,-2) -- ++(1.5,0);}
    
    \draw[thick,color=red] (1,0.5) -- ++(3.5,0);

\end{TIKZP}
\end{center}

}
\OnlyRemark{4}{%
These outcomes are mutually exclusive because there is just one photon.  It can only be measured in one of the two places.   Until it is measured, it is in a \emph{superposition} of the two possible paths.
}
\end{frame}

\begin{frame}{Naming the two possible paths}{We use the \emph{ket} notation defined previously.}
\begin{itemize}
    \item<1-> With two possible paths, we have a binary system that resembles a 1-bit classical system.
    \item<2-> However, a single bit is either one way or the other, and we cannot express superpositions.
    
\end{itemize}
\TwoColumns{%
\Vskip{-2em}\begin{itemize}
    \item<3-> Let \textcolor{orange}{$\QZero{} = \PZero{}$} be \textcolor{orange}{one path}.
    \item<3-> Let \textcolor{NavyBlue}{$\QOne{} = \POne{}$} be \textcolor{NavyBlue}{the other}.
    \item<4-> We can verify these are orthogonal:
    \[
        \braket{0}{1} = \braket{1}{0} =  0
    \]
\end{itemize}
}{%
\visible<5->{%
A superposition of paths is then
\[
\alpha\textcolor{orange}{\ket{0}} + \beta\textcolor{NavyBlue}{\ket{1}} = \SQB{\alpha}{\beta}
\]
where $\alpha$ and $\beta$ characterize the \emph{presence} of \textcolor{orange}{\ket{0}} and \textcolor{NavyBlue}{\ket{1}}, respectively.  This is related to the probability of outcome, but there is more to say.
}
}
\end{frame}

\begin{frame}{Three orthogonal states}{This photon is measured in one of three places.}
\TwoColumns{%
\begin{itemize}
    \item<1-> Here we see two beam splitters.  The photon arrives a the first one and with equal probability it
    \begin{itemize}
        \item<2-> passes through or
        \item<3-> reflects downward.
    \end{itemize}
    \item<4-> At the second beam splitter, it either 
    \begin{itemize}
        \item<5-> passes through, or
        \item<6-> reflects downward.
    \end{itemize}
\end{itemize}
}{%
\begin{center}
\begin{TIKZP}[scale=0.75]
    \LightSource{}
    \Shift{2}{0}{\RotateAroundCenter{-45}{\BeamSplitter{}}}
    \Shift{4}{0}{\RotateAroundCenter{-45}{\BeamSplitter}}
    \Shift{4}{-2}{\RotateAroundCenter{-45}{\Mirror}}
    \Shift{2}{-4}{\RotateAroundCenter{-45}{\Mirror}}
    \Shift{6}{0}{\Measurement[color=orange]{}}
    \Shift{6}{-2}{\Measurement[color=NavyBlue]{}}
    \Shift{6}{-4}{\Measurement[color=OliveGreen]{}}
    \visible<2->{\draw[thick,color=orange] (2.5,0.5) -- (4.5,0.5);}
    \visible<5->{\draw[thick,color=orange] (4.5,0.5) -- (6,0.5);}
    \visible<6->{\draw[thick,color=NavyBlue] (4.5,0.5) -- ++(0,-2) -- ++(1.5,0);}
    \visible<3->{\draw[thick,color=OliveGreen] (2.5,0.5) -- ++(0,-4) -- ++ (3.5,0);}
    \draw[thick,color=red] (1,0.5) -- ++(1.5,0);

\end{TIKZP}
\end{center}
\only<3>{\textcolor{OliveGreen}{There is thus a 50\% chance of hitting the bottom measuring device.}
}
\only<5>{\textcolor{orange}{There is a 25\% chance of hitting the top measuring device.}}
\only<6>{\textcolor{NavyBlue}{There is a 25\% chance of hitting the middle measuring device.}}
}
    
\end{frame}

\begin{frame}{Naming the three possible paths}{We can add one more basis vector.}
\TwoColumns{%

    \visible<1,4->{\textcolor{orange}{$\QZero{} = \begin{pmatrix}1\\ 0\\ 0\end{pmatrix}$}}
    \visible<2,4->{\textcolor{NavyBlue}{$\QOne{} = \begin{pmatrix}0 \\ 1\\ 0\end{pmatrix}$}}
    \visible<3,4->{\textcolor{OliveGreen}{$\ket{2} = \begin{pmatrix}0 \\ 0\\ 1\end{pmatrix}$}}
    
\begin{itemize}
    \item<4-> We can verify $\Forall{i\not=j}{\braket{i}{j}=0}$ so these are mutually orthogonal basis vectors.
    \item<5> A superposition is then
    \[
    \alpha\textcolor{orange}{\ket{0}} +
    \beta\textcolor{NavyBlue}{\ket{1}} +
    \gamma\textcolor{OliveGreen}{\ket{2}} = \begin{pmatrix} \alpha \\ \beta \\ \gamma \end{pmatrix}
    \]
\end{itemize}

}{%
\begin{center}
\begin{TIKZP}[scale=0.75]
    \LightSource{}
    \Shift{2}{0}{\RotateAroundCenter{-45}{\BeamSplitter{}}}
    \Shift{4}{0}{\RotateAroundCenter{-45}{\BeamSplitter}}
    \Shift{4}{-2}{\RotateAroundCenter{-45}{\Mirror}}
    \Shift{2}{-4}{\RotateAroundCenter{-45}{\Mirror}}
    \Shift{6}{0}{\Measurement[color=orange]{}}
    \Shift{6}{-2}{\Measurement[color=NavyBlue]{}}
    \Shift{6}{-4}{\Measurement[color=OliveGreen]{}}
    \visible<1->{\draw[thick,color=orange] (2.5,0.5) -- (4.5,0.5);}
    \visible<1->{\draw[thick,color=orange] (4.5,0.5) -- (6,0.5);}
    \visible<2->{\draw[thick,color=NavyBlue] (4.5,0.5) -- ++(0,-2) -- ++(1.5,0);}
    \visible<3->{\draw[thick,color=OliveGreen] (2.5,0.5) -- ++(0,-4) -- ++ (3.5,0);}
    \draw[thick,color=red] (1,0.5) -- ++(1.5,0);

\end{TIKZP}
\end{center}
}
\end{frame}

\section{Quantum systems}

\begin{frame}{Modeling quantum systems}{We shall typically use qubits.}
\begin{itemize}
    \item<1-> Generally, a quantum element could be in a superposition of $d$ mutually orthogonal measurement outcomes.
    \item<2-> If there are $d>2$ such outcomes, we represent a quantum state by a $d$-valued 
    \href{https://en.wiktionary.org/wiki/qudit}{qudit}. 
    \visible<3->{If $d=3$ the qudit is sometimes called
    a \href{https://en.wikipedia.org/wiki/Qutrit}{qutrit}.}
    \item<4-> Quantum computing focuses on \emph{qubits}, each having two possible measurement outcomes.
    \item<5-> In the \href{https://en.wikipedia.org/wiki/Qubit\#Qubit_states}{standard basis}, those are always $\QZero{}=\PZero{}$ and $\QOne{} = \POne{}$.
    \item<6-> As with classical computing bits, larger quantum systems are implemented with multiple (2-state) qubits. An n-qubit system is computationally equivalent to a $2^{n}$-valued qudit system. A proof follows from equivalence of basis vectors.
\end{itemize}
\end{frame}

\begin{frame}{A single-qubit system revisited}{Measurement outcomes}
\TwoUnequalColumns{0.65\textwidth}{0.35\textwidth}{%
\begin{itemize}
    \item<1-> Let \ket{0} and \ket{1} represent horizontal and vertical polarization, respectively.  
    \item<2-> A photon that passes through a horizontal filter is in state \ket{0}.
    \item<3-> And a photon that passes through a vertical filter is in state \ket{1}.
\end{itemize}

}{%
\begin{center}
\begin{TIKZP}
\visible<1-4>{
\draw[->,ultra thick] (0,0) -- (1,0) node[right] {\ \ket{0}};
\draw[->,ultra thick] (0,0) -- (0,1) node[above] {\ket{1}};
}
\visible<2>{\begin{scope}[draw=NavyBlue,rotate=90]\PFilter{-1}{-1.2}{1}{1.2}{20}\end{scope}}
\visible<3>{\begin{scope}[draw=OliveGreen]\PFilter{-1}{-1.2}{1}{1.2}{20}\end{scope}}
\end{TIKZP}
\end{center}
}
\OnlyRemark{4}{Passing through a filter constitutes a \emph{measurement}, so that the photon will be observed to be \ket{0} or \ket{1} due to the filter.  In one case the photon passes through;  in the other case, its energy is absorbed by the filter.}
\end{frame}

\begin{frame}{Preparation of state}{Polarizing filters create a known starting state.}
\TwoUnequalColumns{0.65\textwidth}{0.35\textwidth}{%
\begin{itemize}
    \item<1> If we begin with a source of unpolarized light.
    \item<2-> Then a horizontal filter is a source of photons that are all in state \ket{0}.
    \item<3-> Similarly, a vertical filter acts a source of photons in state \ket{1}.
\end{itemize}

}{%
\begin{center}
\begin{TIKZP}
\visible<1>{\RadiantArrows{22.5}{->,color=BrickRed,thick}}
\visible<2>{%
\foreach \x in {-1, -0.75,...,1} {
\draw[<->,thick,color=BrickRed] (-1,\x) -- (1,\x);
}
}
\visible<3>{%
\foreach \x in {-1, -0.75,...,1} {
\draw[<->,thick,color=BrickRed] (\x,-1) -- (\x,1);
}
}
\visible<2-3>{
\draw[->,ultra thick] (0,0) -- (1,0) node[right] {\ \ket{0}};
\draw[->,ultra thick] (0,0) -- (0,1) node[above] {\ket{1}};
}
\visible<2>{\begin{scope}[draw=NavyBlue,rotate=90]\PFilter{-1}{-1.2}{1}{1.2}{20}\end{scope}}
\visible<3>{\begin{scope}[draw=OliveGreen]\PFilter{-1}{-1.2}{1}{1.2}{20}\end{scope}}
\end{TIKZP}
\end{center}
}
\OnlyRemark{4}{We say in such cases that a photon has been \emph{prepared} in state~\ket{0} or~\ket{1}, depending on the filter.}
\end{frame}

\begin{frame}{What about an arbitrarily polarized photon?}{How do we represent its state?}
\Vskip{-4em}\TwoUnequalColumns{0.5\textwidth}{0.5\textwidth}{%
\begin{itemize}
    \item<1-> If the photon has been prepared at angle $\theta$ with the horizontal axis
    \item<2-> then its state is
    \[ \cos{\theta}\ket{0} + \sin{\theta}\ket{1} \]
    \item<3-> The measurement outcomes and their likelihoods are:
    \begin{description}
        \item[\ket{0}] with probability $\cos^{2}\theta$
        \item[\ket{1}] with probability $\sin^{2}\theta$
    \end{description}
\end{itemize}
}{%
\begin{center}
\begin{TIKZP}
\visible<1->{\draw[->,color=BrickRed] (0,0) -- (70:1);
\draw[color=BrickRed] (0.2,0) node[above right] {$\theta$};
}
\visible<1->{
\draw[->, thick] (0,0) -- (1,0) node[right] {\ \ket{0}};
\draw[->, thick] (0,0) -- (0,1) node[above] {\ket{1}};
}
\end{TIKZP}
\begin{itemize}
    \item<4-> This matches photons prepared when $\theta=\Degrees{0}$ or $\theta=\Degrees{90}$.
    \item<5-> With $\cos^{2}\theta+\sin^{2}\theta=1$, every photon is measured as \ket{0} or \ket{1}.
    \item<6-> When $\theta=\Degrees{45}$, $\cos^{2}\theta = \sin^{2}\theta = \frac{1}{2
    }$
\end{itemize}
\end{center}
}
\OnlyRemark{7}{Unpolarized light behaves as if $\theta=\Degrees{45}$. The photons measure half \ket{0} and half \ket{1}.}
\end{frame}

\begin{frame}{Wave amplitudes and probabilities}{One is the square of the other.}
\TwoColumns{%
\begin{itemize}
    \item<1-> Recall the state of a photon prepared to be polarized with angle~$\theta$:
    \[ \cos\theta\ket{0} + \sin\theta\ket{1} \]
    \item<2-> More generally, a single qubit is specified as
    \[ \alpha\ket{0} + \beta\ket{1} = \SQB{\alpha}{\beta} \]
    subject to $\Prob{\alpha} + \Prob{\beta} = 1$
\end{itemize}
}{%
\begin{itemize}
    \item<1-> The wave amplitudes can be positive or negative.  Here they are real-valued but in general they are complex-valued.
    \item<3-> For real-valued coefficients, $\Prob{a} = a^{2}$
    \item<3-> For complex-valued coefficients
    $\Prob{a} = \Conj{a}a$
\end{itemize}
}
    
\end{frame}

\def\MZ{%
\LightSource{}
\Shift{2}{0}{\RotateAroundCenter{-45}{\BeamSplitter{}}}
\Shift{2}{-2}{\RotateAroundCenter{-45}{\Mirror{}}}
\Shift{6}{0}{\RotateAroundCenter{-45}{\Mirror{}}}
\Shift{6}{-2}{\RotateAroundCenter{-45}{\BeamSplitter{}}}
\Shift{8}{-2}{\Measurement[color=orange]{}}
\Shift{6}{-4}{\RotateAroundCenter{-90}{\Measurement[color=NavyBlue]{}}}
}
\begin{frame}{Why complex-valued coefficients?}{They represent phase.}
\begin{itemize}
    \item Some sources defer the use of complex values for wave amplitudes, but they are essential for quantum computing.
    \item We can demonstrate their usefulness with a simple photon diagram like the ones we have seen before.
\end{itemize}

\begin{center}
\begin{TIKZP}[scale=0.7]
\MZ{}
\end{TIKZP}
\end{center}
\end{frame}

\begin{frame}{hold}
\begin{TIKZP}

\visible<5>{\begin{scope}[draw=Sepia,rotate=-20]\PFilter{-1}{-1.2}{1}{1.2}{20}\end{scope}
\draw[->,thick] (0,0) -- (1,0);
\draw[->,thick] (0,0) -- (70:1);
\draw (0.2,0) node[above right] {$\theta$};
}

\end{TIKZP}
\end{frame}



