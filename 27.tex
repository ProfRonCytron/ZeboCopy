\SetTitle{27}{Introduction to Shor's Algorithm}{The non-quantum part}{27}

\begin{frame}{Overview}{What will we study here?}

\begin{itemize}
    \item Shor's algorithm solves a problem of great interest to the cryptographic community.
    \item Given a large number that is the product of two large primes, what are those factors?
    \item The hardness of this \emph{factoring problem} is not known.
    \begin{itemize}
        \item We do not know if this problem is \href{https://en.wikipedia.org/wiki/NP-completeness}{NP-Complete}
        \item It is believed not to be NP-Complete.
    \end{itemize}
    \item However, to factor $N$, the best algorithms currently known must try all factors up to and including $\sqrt{N}$.
    \item The speedup of Shor's algorithm using quantum computing is exponential compared with the best-known classical approach.
\end{itemize}
    
\end{frame}

\begin{frame}{Use of factoring in cryptography}{One-way functions}

\TwoColumns{%
Easy in one direction
\begin{itemize}
    \item<2-> $2\times 3$
    \item<4-> $97\times 59$
    \item<6-> $1746860020068409\times 188748146801$
\end{itemize}
}{%
Hard (as far as we know) in the other
\begin{itemize}
    \item<1-> 6
    \item<3-> 5723
    \item<5-> 329716591508669867994509609
\end{itemize}
}
\BigSkip{}%
\visible<7->{Thanks to arbitrary precision packages, we can conveniently multiply large numbers.  Languages like python seamlessly use such packages on demand.}
\begin{itemize}
    \item<8-> The \href{https://en.wikipedia.org/wiki/RSA\_(cryptosystem)}{RSA  cryptosystem} makes use of the apparent hardness of factoring.
    \item<9-> RSA is named for
    \begin{itemize}
        \item \href{https://en.wikipedia.org/wiki/Ron\_Rivest}{Ron Rivest}
        \item \href{https://en.wikipedia.org/wiki/Adi\_Shamir}{Adi Shamir}
        \item \href{https://en.wikipedia.org/wiki/Leonard\_Adleman}{Leonard Adleman}
    \end{itemize}
    \item<10-> \href{https://en.wikipedia.org/wiki/Pretty_Good_Privacy}{PGP} and other systems also can rely on RSA.
\end{itemize}
\end{frame}

\begin{frame}{Shor's algorithm}{Overview}

\begin{itemize}
    \item Shor turns the problem of factoring into the problem of \alert{finding the period} of a periodic function.
    \item Finding the period is a task for a quantum computer, on which the solution can be achieved exponentially faster than on a classical computer.
    \begin{itemize}
    \item The speedup is due to Quantum Fourier Analysis, the technique for finding the period.
    \end{itemize}
    \item Once the period is found, Shor's algorithm applies classical mathematical reasoning to obtain the factors.
    \item The entire approach hinges on a \emph{guess} at a base \ColorFour{$b$}.
\end{itemize}
    
\end{frame}
\begin{frame}{Overview using an example}{Factoring 217 by finding the period of an apparently unrelated function}
\TwoColumns{%
\Vskip{-4em}\begin{center}
{\scriptsize $g(x) = \ColorFour{5}^{x}\mbox{ mod $217$}$}\\[-0.5em]
\begin{Pixture}[width=0.9\textwidth]{27}{base5mod217.png}
\end{Pixture}\end{center}
}{%
\Vskip{-3em}\only<1-7>{\begin{itemize}
    \item We seek the smallest $x'>0$ for which $g(x')=1$.
    \item For any such function, $g(0)=1$.
    \item Finding $x'$ is the same problem as \alert{finding the period} of the function.  
    \item In this example, the period is $6$, and so we find~$\ColorFour{5}^{6}=1\SMod{217}$.
\end{itemize}}%
\only<8->{%
\begin{align*}
    \SGCD{\ColorOne{124}}{217} &= \hbox to 1em{\hss 31} \\
    \SGCD{\ColorTwo{126}}{217} &= \hbox to 1em{\hss 7}
\end{align*}
\visible<9->{This discovers $31$ and $7$ as factors of $217$.}\visible<10->{  It turns out that the base, \ColorFour{$5$}, is just a guess.  Let's try this same example next but with a base of~$11$.}
}
}
\Vskip{-4em}\TwoColumns{%
\begin{align*}
   \visible<2->{\ColorFour{5}^{6} &= 1\SMod{217} \\}
    \visible<3->{\ColorFour{5}^{6}-1 &= 0\SMod{217}\\}
    \visible<4->{{\left(\ColorFour{5}^{3}\right)}^{2}-1 &= 0\SMod{217}\\}
\end{align*}
}{%
\begin{align*}
    \visible<5->{{125}^{2} -1 &= 0\SMod{217}\\}
    \visible<6->{\ColorOne{(125 -1)} \times \ColorTwo{(125+1)}&= 0\SMod{217}\\}
    \visible<7->{\ColorOne{124} \times \ColorTwo{126}&= 0\SMod{217}}
\end{align*}
}
    
\end{frame}


\begin{frame}{Factoring $N=217$ using a base $b=11$}{Throughout, \emph{abort} requires starting over with a new (guessed) value for the base}
\Vskip{-4em}\TwoUnequalColumns{0.5\textwidth}{0.5\textwidth}{%
\begin{itemize}
    \item<1-> If \ColorFour{$b$} divides $N$, return \ColorFour{$b$} as a factor.
    \item<2-> \alert{Find the period} \ColorFive{$r$} of $\ColorFour{b}^{x}\SMod{N}$.
    \item<3-> If the period is not \emph{even}, abort.
    \item<4-> Write $\ColorFour{b}^{\ColorFive{r}}$ as $\left(\ColorFour{b}^{\ColorFive{r}/2}\right)^{2}$
    \item<5-> If $\ColorFour{b}^{\ColorFive{r}/2}=\pm 1\SMod{N}$, abort
    \item<6-> Factor $\left(\ColorFour{b}^{\ColorFive{r}/2}\right)^{2}-1$ into $\ColorOne{\left(\ColorFour{b}^{\ColorFive{r}/2}-1\right)}\times \ColorTwo{\left(\ColorFour{b}^{\ColorFive{r}/2}+1\right)}$
    \item<7-> Factors are
    \begin{itemize}
        \item $\SGCD{\ColorOne{\ColorFour{b}^{\ColorFive{r}/2}-1}}{N}$
        \item $\SGCD{\ColorTwo{\ColorFour{b}^{\ColorFive{r}/2}+1}}{N}$
    \end{itemize}
\end{itemize}
}{%
\begin{itemize}
    \item<1-> \ColorFour{$11$} does not divide $217$
    \item<2-> The period of $\ColorFour{11}^{x}\SMod{217}$ is~\ColorFive{$30$}.
    \item<3-> \ColorFive{$30$} is even, so we are good.
    \item<4-> $\ColorFour{11}^{\ColorFive{30}} = \left(\ColorFour{11}^{\ColorFive{15}}\right)^{2}$
    \item<5-> $\ColorFour{11}^{\ColorFive{15}}=92\SMod{217}$, so we are good.
    \item<6-> $\ColorFour{11}^{\ColorFive{30}}-1 =\ColorOne{\left(\ColorFour{11}^{15}-1\right)}\times\ColorTwo{\left(\ColorFour{11}^{15}+1\right)}$
    \item<7-> Factors are
    \begin{itemize}
        \item $\SGCD{\ColorOne{4177248169415650}}{217}=\hbox to 1em{\hss 7}$
        \item $\SGCD{\ColorTwo{4177248169415652}}{217}=\hbox to 1em{\hss 31}$
    \end{itemize}
\end{itemize}
}
    
\end{frame}