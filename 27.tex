\SetTitle{27}{Introduction to Shor's Algorithm}{The non-quantum part}{27}

\begin{frame}{Overview}{What will we study here?}

\begin{itemize}
    \item Shor's algorithm solves a problem of great interest to the cryptographic community.
    \item Given a large number that is the product of two large primes, what are those factors?
    \item The hardness of this \emph{factoring problem} is not known.
    \begin{itemize}
        \item We do not know if this problem is \href{https://en.wikipedia.org/wiki/NP-completeness}{NP-Complete}
        \item It is believed not to be NP-Complete.
    \end{itemize}
    \item However, to factor $N$, the best algorithms currently known must try all factors up to and including $\sqrt{N}$.
    \item The speedup of Shor's algorithm using quantum computing is exponential compared with the best-known classical approach.
\end{itemize}
    
\end{frame}

\begin{frame}{Use of factoring in cryptography}{One-way functions}

\TwoColumns{%
Easy in one direction
\begin{itemize}
    \item $2\times 3$
    \item $97\times 59$
    \item $1746860020068409\times 188748146801$
\end{itemize}
}{%
Hard (as far as we know) in the other
\begin{itemize}
    \item 6
    \item 5723
    \item 329716591508669867994509609
\end{itemize}
}
\BigSkip{}%
Thanks to arbitrary precision packages, we can conveniently multiply large numbers.  Languages like python seamlessly use such packages on demand.
\end{frame}
{
\def\SMod#1{\mbox{\small\ \ColorThree{(mod #1)}}}
\begin{frame}{Secret sauce}{Finding the period of a periodic function}
\TwoColumns{%
\Vskip{-4em}\begin{center}
{\scriptsize $g(x) = 5^{x}\mbox{ mod $217$}$}\\[-0.5em]
\begin{Pixture}[width=0.9\textwidth]{27}{base5mod217.png}
\end{Pixture}\end{center}
}{%
\Vskip{-3em}\only<1-7>{\begin{itemize}
    \item We are interested in finding the smallest $x'>0$ for which $g(x')=1$.
    \item For any such function, $g(0)=1$, so finding $x'$ is the same problem as finding the period of the function.  In this example, the period is $6$, and so we find~$g(6)=1$.
\end{itemize}}%
\only<8->{%
\begin{align*}
    GCD(\ColorOne{124},217) &= 31 \\
    GCD(\ColorTwo{126},217) &= 7
\end{align*}
This discovers $31$ and $7$ as factors of $217$.
}
}
\Vskip{-4em}\TwoColumns{%
\begin{align*}
   \visible<2->{5^{6} &= 1\SMod{217} \\}
    \visible<3->{5^{6}-1 &= 0\SMod{217}\\}
    \visible<4->{{5^{3}}^{2}-1 &= 0\SMod{217}\\}
\end{align*}
}{%
\begin{align*}
    \visible<5->{{125}^{2} -1 &= 0\SMod{217}\\}
    \visible<6->{\ColorOne{(125 -1)} \times \ColorTwo{(125+1)}&= 0\SMod{217}\\}
    \visible<7->{\ColorOne{124} \times \ColorTwo{126}&= 0\SMod{217}}
\end{align*}
}
    
\end{frame}
}