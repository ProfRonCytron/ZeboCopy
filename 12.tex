\SetTitle{12}{Games with quantum advantage}{These rely on entanglement}{12}

\section*{Overview}

\begin{frame}{Overview}{What will we study here?}
\end{frame}

\section*{CHSH}

{%
\def\Red{Red}\def\Yellow{Goldenrod}\def\Green{green}\def\Orange{orange}%
\def\RD#1{\textcolor{\Red}{#1}}
\def\YL#1{\textcolor{\Yellow}{#1}}
\def\GN#1{\textcolor{\Green}{#1}}
\def\OR#1{\textcolor{\Orange}{#1}}
\begin{frame}{Introduction to the \href{https://en.wikipedia.org/wiki/CHSH_game}{CHSH} game}{This game is due to John Clauser, Michael Horne, Abner Shimony, and Richard Holt}
    \begin{itemize}[<+->]
        \item Alice and Bob are initially proximate, so that they can strategize.\footnote{They could make some EPR pairs, but that comes later!}
        \item They then separate by a large distance.
        \item A referee randomly and uniformly sends one of two words to Alice, and one of two other words to Bob.  It is expedient to use colors for the words:
        \begin{description}
            \item[Alice] will hear \RD{red} or \YL{yellow}.
            \item[Bob] will hear \GN{green} or \OR{orange}.
        \end{description}
    \item Alice and Bob each respond with one bit ($0$ or $1$).  To win$\ldots$
    \begin{itemize}
        \item If Alice and Bob hear \RD{red} and \GN{green}, respectively, their bits must differ.
        \item Otherwise, their bits must agree.
    \end{itemize}
    \item Alice and Bob are too far apart to communicate before their answer is needed.
    \item What is the best they can do?
    \end{itemize}
\end{frame}%

\begin{frame}{Classical strategies}{How well can they do?}

\begin{itemize}[<+->]
    \item Based on the information given to Alice, she has only four possible strategies:
    \begin{itemize}
        \item Always respond $0$
        \item Always respond $1$
        \item Use the colors:  \RD{$0$}, \YL{$1$}
        \item Or oppositely:  \YL{$0$}, \RD{$1$}
    \end{itemize}
    \item Bob has the same four strategies, so between them they have~16 strategies.
    \item The simplest strategy: they each respond always with the same bit, say $0$.
    \item They win $\frac{3}{4}$ of the games they play, which turns out to be the best they can do classically.
\end{itemize}
    
\end{frame}

\begin{frame}{Analysis of a given Strategy}{We enumerate the possibilities}
\Vskip{-3em}\TwoColumns{%
\begin{itemize}
    \item<1-> The both-$0$ strategy wins with probability $0.75$.
    \item<2-> Of course, so does the both-$1$ strategy
    \item<3-> Using referee information doesn't improve the results.
\end{itemize}
}{%
\only<1,4>{%
\begin{center}
    \begin{tabular}{cccccc}
      &  \multicolumn{2}{c}{Hears} & \multicolumn{2}{c}{Responds} & Win?\\
       & Alice & Bob & Alice & Bob \\
       1 & \RD{\SCirc{}} & \GN{\SCirc{}} & $0$ & $0$ & \DisAgree{} \\
       2 & \RD{\SCirc{}} & \OR{\SCirc{}} & $0$ & $0$ & \Agree{} \\
        3 & \YL{\SCirc{}} & \GN{\SCirc{}} & $0$ & $0$ & \Agree{} \\
       4 & \YL{\SCirc{}} & \OR{\SCirc{}} & $0$ & $0$ & \Agree{} \\
    \end{tabular}
\end{center}}%
\only<2>{%
\begin{center}
    \begin{tabular}{cccccc}
      &  \multicolumn{2}{c}{Hears} & \multicolumn{2}{c}{Responds} & Win?\\
       & Alice & Bob & Alice & Bob \\
       1 & \RD{\SCirc{}} & \GN{\SCirc{}} & $1$ & $1$ & \DisAgree{} \\
       2 & \RD{\SCirc{}} & \OR{\SCirc{}} & $1$ & $1$ & \Agree{} \\
        3 & \YL{\SCirc{}} & \GN{\SCirc{}} & $1$ & $1$ & \Agree{} \\
       4 & \YL{\SCirc{}} & \OR{\SCirc{}} & $1$ & $1$ & \Agree{} \\
    \end{tabular}
\end{center}}%
\only<3>{%
\begin{center}
    \begin{tabular}{cccccc}
      &  \multicolumn{2}{c}{Hears} & \multicolumn{2}{c}{Responds} & Win?\\
       & Alice & Bob & Alice & Bob \\
       1 & \RD{\SCirc{}} & \GN{\SCirc{}} & \RD{$0$} & \GN{$1$} & \Agree{} \\
       2 & \RD{\SCirc{}} & \OR{\SCirc{}} & \RD{$0$} & \OR{$0$} & \Agree{} \\
        3 & \YL{\SCirc{}} & \GN{\SCirc{}} & \YL{$1$} & \GN{$1$} & \Agree{} \\
       4 & \YL{\SCirc{}} & \OR{\SCirc{}} & \YL{$1$} & \OR{$0$} & \DisAgree{} \\
    \end{tabular}
\end{center}}%
\only<5->{%
\begin{center}
   \begin{tabular}{cccccc}
      &  \multicolumn{2}{c}{Hears} & \multicolumn{2}{c}{Responds} & Win?\\
       & Alice & Bob & Alice & Bob \\
       1 & \RD{\SCirc{}} & \GN{\SCirc{}} & $1$ & $1$ & \DisAgree{} \\
       2 & \RD{\SCirc{}} & \OR{\SCirc{}} & $0$ & $1$ & \DisAgree{} \\
        3 & \YL{\SCirc{}} & \GN{\SCirc{}} & $0$ & $1$ & \DisAgree{} \\
       4 & \YL{\SCirc{}} & \OR{\SCirc{}} & $0$ & $1$ & \DisAgree{} \\
    \end{tabular}
\end{center}}%
}
\BigSkip{}
\begin{itemize}
        \item<4-> Using any strategy that is fixed over the game, the best they can do is win with probability~$0.75$, as with the both-$0$ strategy.
    \item<5-> Any random strategy or mix of strategies is a distribution of fixed strategies, which wins at best with probability~$0.75$, and in the worst case they might never win.
\end{itemize}
\end{frame}

\begin{frame}{Discussion}{Can quantum help?}

\begin{itemize}[<+->]
    \item The best Alice and Bob can do classically is win with probability~$0.75$.
    \item In Ekert's key-distribution algorithm, we observed that when two measurement bases are separated by $\pi/8$ radians, the probability of coincidental measurement is $\cos^{2}(\pi/8)\approx 0.85$.
    \item We can use this property to bias agreement in the three cases where such agreement is a win for Alice and Bob.  Note that while they will likely agree, the choice of whether that agreement is on~$0$ or~$1$ is (uniformly) random.
    \item We can also arrange for them to likely disagree in the one case where that is a win for them.
    \item In the slides that follow, we color a participant by the color supplied to her or him by the referee. So \RD{Alice} heard \RD{red}, but \YL{Alice} heard \YL{yellow}.  Similarly, \GN{Bob} heard \GN{green} and \OR{Bob} heard \OR{orange}.
\end{itemize}
    
\end{frame}

\begin{frame}{Quantum approach to winning CHSH more often}{Uses entanglement and proximity of bases}

\Vskip{-3em}\TwoUnequalColumns{0.48\textwidth}{0.52\textwidth}{%
\only<1-4>{%
\begin{itemize}
    \item<1-> Alice and Bob each have one qubit of an EPR pair in state $\frac{\ket{00}+\ket{11}}{\sqrt{2}}$.
    \item<1-> The computational basis, as usual.
    \item<2-> \OR{Bob} measures his qubit in that basis.
    \item<3-> \YL{Alice}, not knowing whether she faces \OR{Bob} or \GN{Bob}, always measures measures~$\pi/8$ radians counterclockwise from~\ket{0}.
    \item<4-> \RD{Alice}'s basis is rotated $\pi/8$ radians clockwise from~\ket{0}.
\end{itemize}}%
\only<5-6>{%
\begin{itemize}
    \item<5-> Consider \GN{Bob}'s other possibility, whose measurement we situate $\pi/4$ radians counterclockwise from \ket{0}.
    \item<5-> \YL{Alice} is $\pi/8$ radians away, and so they agree with the usual probability.
    \item<6-> But \RD{Alice} is $3\pi/8$ radians away.
\end{itemize}}%
\only<7->{%
Putting this all together:
\begin{itemize}
    \item<7-> Alice and Bob begin with an EPR pair and then separate.
    \item<8-> They each measure their qubit in the basis of the color they hear.
    \item<9-> They win by the requisite agreement or disagreement approximately~85\% of the time.  \end{itemize}
}
}{%}
\Vskip{-4em}\begin{center}
\begin{TIKZP}[scale=1.7]
\visible<1>{\draw[->,thick] (0,0) -- (1,0) node[right] {\ \ket{0}};}
\visible<1,7->{\draw[->,thick] (0,0) -- (0,1) node[above] {\ket{1}};}
\visible<2-4,7->{\draw[->,thick,\Orange] (0,0) -- (0:1) node[right] {Bob};}
\visible<2>{\draw[->,thick,dotted,\Orange] (0,0) -- (90:1) node[above] { };}
\visible<4,6,7->{\draw[->,thick,\Red] (0,0) -- (-22.5:1) node[right] {Alice};}
\visible<3-5,7->{\draw[->,thick,\Yellow] (0,0) -- (22.5:1) node[right] {Alice};}
\visible<3>{\draw[->] (1,0) arc (0:22.5:1.0) node[below right] {{\small $\pi/8$ radians}};}
\visible<5->{\draw[->,thick,\Green] (0,0) -- (45:1) node[above right] {Bob};}
\visible<6>{\draw (1,0)  node[above right] {{\small $3\pi/8$ radians}} arc(0:-22.5:1);
\draw[->] (1,0) arc(0:45:1);}

\end{TIKZP}
\end{center}
\only<2>{%
We show only the \ket{0}~axis of a basis.  The \ket{1}~axis is $\pi/2$~radians counterclockwise from~\ket{0}.
\BigSkip{}
If \OR{Bob} measures his qubit first, he is equally likely to measure~$0$ or~$1$.
}%
\only<3>{%
The probability that \only<3>{\YL{Alice}} and \OR{Bob} measure the same value, both~$0$ or both~$1$, is~$\cos^{2}(\pi/2)\approx 0.85$.
\MedSkip{}
If \YL{Alice} measures first, \YL{she} is more likely to see~$0$ than~$1$, but with the change in state, \OR{Bob} is still about 85\% likely to match \YL{her} measurement}%
\only<4>{%
The same is true here: \RD{Alice} and \OR{Bob} measure the same result with probability~$0.85$, no matter who measures first.
}%
\only<5>{%
If \GN{Bob} measures first, he will certainly see~$0$.  But then \YL{Alice} is about $85\%$ likely to see~$0$ after that.  If \YL{Alice} measures first, she will likely measure~$0$.  With either ensuing change of state, \GN{Bob} is $85\%$ likely to match \YL{Alice}'s measurement.
}
\only<6>{%
The probability of \emph{disagreement} between \RD{Alice} and \GN{Bob} is approximately~$0.85$, no matter who measures first. In this case they win by disagreeing.
}%
}%
\only<10->{%
\BigSkip{}
The quantum approach wins $\sim 85\%$ of the time, while classically we win at best $75\%$ of the time. \alert{Quantum behavior provides a clear and measurable advantage over classical logic.}

}
\end{frame}
}

\section*{Magic square}
{
\def\MPC#1{\fbox{\hbox to 5ex{\hss\mbox{#1}\vrule width 0pt height 3ex depth 2ex\hss}}}
\def\XP#1{\hbox to 3ex{\hss#1\hss 1}}%
\def\PP{%
\XP{$+$}}%
\def\MM{%
\XP{$-$}}%
\def\MPSquare#1#2#3#4#5#6#7#8#9{%
{%
    \setlength{\tabcolsep}{0pt}%
    \adjustbox{valign=t}{\begin{tabular}{rrccc}
   \multicolumn{1}{c}{ } & \multicolumn{1}{c}{ } & \multicolumn{3}{c}{$\downarrow\ $Bob$\ \downarrow$} \\
    & & 1 & 2 & 3 \\
        $\rightarrow\ $ & \ 1\ \ & \MPC{#1} & \MPC{#2} & \MPC{#3} \\
        \mbox{Alice }& \ 2\ \ & \MPC{#4} & \MPC{#5} & \MPC{#6} \\
        $\rightarrow\ $&\ 3\ \ & \MPC{#7} & \MPC{#8} & \MPC{#9} \\
    \end{tabular}}}}
\def\VS#1#2{\visible<#1>{#2}}
\begin{frame}{Introduction to the \href{https://en.wikipedia.org/wiki/Quantum_pseudo-telepathy\#The_Mermin\%E2\%80\%93Peres_magic_square_game}{Mermin--Peres magic square}}{This game is due to \href{https://en.wikipedia.org/wiki/N._David_Mermin}{David Mermin} and \href{https://en.wikipedia.org/wiki/Asher_Peres}{Asher Peres}}

\begin{itemize}[<+->]
    \item With CHSH, a quantum approach is probabilistically better than any classical approach but does not guarantee to win the game.
    \item The Mermin--Peres magic square is a puzzle that Alice and Bob can win at best $\frac{8}{9}$ of the time, but a quantum approach on a reliable quantum computer can always win.
    \item Alice and Bob share two EPR pairs, each initially in the state $\frac{\ket{00}+\ket{11}}{\sqrt{2}}$.
    \item Variations of the game we consider also go by the same name, and they are equivalent in difficulty and in the approach to a quantum-based solution.
    \item We follow the game as specified in the
    \href{https://en.wikipedia.org/wiki/Wiki}{Wikipedia} \href{https://en.wikipedia.org/wiki/Quantum_pseudo-telepathy\#The_Mermin\%E2\%80\%93Peres_magic_square_game}{article}.
\end{itemize}
    
\end{frame}

\begin{frame}{How is the game played?}
\Vskip{-3em}\TwoUnequalColumns{0.5\textwidth}{0.5\textwidth}{%
\only<1-6>{%
\begin{itemize}
    \item<1-> After enjoying some time together, Alice and Bob separate and cannot communicate with each other.
    \item<2-> A referee chooses a \textcolor{\RCone}{row number for Alice (2)} and a \textcolor{\RCtwo}{column number for Bob (1)}.
    \item<3-> Alice must respond with an entry for each cell of \textcolor{\RCone}{her designated row}, choosing~\PP{} or~\MM{}.
    \item<6-> The product of \textcolor{\RCone}{her entries} must be~\PP{}.
\end{itemize}}
\only<7->{%
\begin{itemize}
    \item<7-> Bob must do the same for \textcolor{\RCtwo}{his designated column}.
    \item<8-> Where the row and column \textcolor{Purple}{intersect}, Alice and Bob must agree on the value, or they lose.
    \item<10-> The product of \textcolor{\RCtwo}{his entries} must be~\MM{}.
\end{itemize}
}
}{%
\begin{center}
\MPSquare{\VS{7-}{\textcolor{\RCtwo}{\PP}}}{ }{ }{\only<3-7>{\VS{3-}{\textcolor{\RCone}{\PP}}}\only<8->{\VS{3-}{\textcolor{Purple}{\PP}}}}{\VS{4-}{\textcolor{\RCone}{\MM}}}{\VS{5-}{\textcolor{\RCone}{\MM}}}{\VS{9-}{\textcolor{\RCtwo}{\MM}}}{ }{ }
\end{center}
}%
\end{frame}

\begin{frame}{Best classical solution}{They can agree to respond as shown on this slide, winning $8$ of $9$ games on average}

\Vskip{-3em}\TwoUnequalColumns{0.6\textwidth}{0.4\textwidth}{%
\only<1-6>{%
\begin{itemize}[<+->]
    \item Alice and Bob can agree before separating to respond as shown in the square here.
    \item We can verify that for the first two rows and columns, they always agree on the common value and the products are correct.
    \item For row or column three, Alice and Bob would have to complete their response in a way that creates the correct product.
    \begin{description}
      \item[Alice] must respond \MM{} so her row's product is \PP{}.
      \item[Bob] must respond \PP{} so his column's product is \MM{}.
    \end{description}
\end{itemize}}%
\only<7->{%
\begin{itemize}
    \item<7-> While other static solutions are possible, each would have at least one square where Alice and Bob cannot agree classically on an entry.
    \item<8-> As with previous games, we will use measurements of entangled qubits to influence choices made by \textit{incommunicado} Alice and Bob.
    \item<9-> On a perfect quantum computer, they can always win.
\end{itemize}
}
}{%
\Vskip{-3em}\begin{center}
\MPSquare{\PP{}}{\PP{}}{\PP{}}{\PP{}}{\MM{}}{\MM{}}{\MM{}}{\PP{}}{\only<1-3,6-8>{\alert{\textbf{?}}}\only<4>{\MM{}}\only<5>{\PP{}}}
\end{center}
\visible<6->{%
\alert{They only lose when \emph{both} the third row and column are designated because they can't agree.}}
}
    
\end{frame}

\begin{frame}{Proof of no static solution}{We cannot place values in each cell that always work}

\Vskip{-3em}\TwoUnequalColumns{0.6\textwidth}{0.4\textwidth}{%
\Vskip{-3em}\begin{itemize}
    \item<1-> Each row's product is~\PP{} so the three rows' products is also~\PP{}.
    \item<2-> Each column's product is~\MM{} so the three columns' products is also~\MM{}.
    \item<3-> This is a contradiction.
\end{itemize}
\begin{align*}
    \visible<1>{a b c = d e f = g h k &= \PP{} \\}
    \visible<1,3>{a b c \times d e f \times g h k & = \PP{} \\}
    \visible<2>{a d g = b e h = c f k &= \MM{} \\}
    \visible<2->{a d g \times b e h \times c f k & = \MM{} \\}
    \visible<3->{abcdefghk & \neq abcdefghk}
\end{align*}

}{%
\Vskip{-3em}\begin{center}
\MPSquare{$a$}{$b$}{$c$}{$d$}{$e$}{$f$}{$g$}{$h$}{$k$}
\end{center}
}
    
\end{frame}

\begin{frame}{Variations}{They are equivalent}
\Vskip{-5em}\TwoUnequalColumns{0.53\textwidth}{0.47\textwidth}{%
\begin{itemize}[<+->]
    \item In the version we consider, we require the product of each row's entries to be \PP{} and the product of each column's entries to be \MM{}.
    \item In another version, the product of each row's and each column's entries is \PP{}, \emph{except} the product of the third column's entries is \MM{}.
    \item A solution to the first form can be reduced from a solution to the second form by always negating the values as shown.
    \item The proof of no static solution is the same.
\end{itemize}
}{%
\Vskip{-1em}\begin{center}
\MPSquare{ }{ }{ }{ }{ }{ }{\visible<3->{\alert{$-$}}}{\visible<3->{\alert{$-$}}}{ }
\end{center}
\visible<4->{%
This causes columns~1 and~2 to have product~\MM{}.  The third row still has a positive product because the two negations cancel.
}
}
\end{frame}


\begin{frame}{Quantum-based solution}{We use two EPR pairs}

\begin{itemize}[<+->]
    \item Alice and Bob begin with two EPR pairs, each in the state $\frac{\ket{00}+\ket{11}}{\sqrt{2}}$.
    \item When they separate, they take one half of each pair with them.
    \item Depending on the row (for Alice) and the column (for Bob), they will carry out measurements based on a table, in a basis related to their particular row or column.
    \item We will prove that they
    \begin{itemize}
        \item agree on a value at the cell intersecting the row and column, and
        \item obtain the necessary row and column products.
    \end{itemize}
\end{itemize}
    
\end{frame}

}
