\SetTitle{12}{Games with quantum advantage}{These rely on entanglement}{12}

\section*{Overview}

\begin{frame}{Overview}{What will we study here?}
\end{frame}

\section*{CHSH}

{%
\def\Red{Red}\def\Yellow{Goldenrod}\def\Green{green}\def\Orange{orange}%
\def\RD#1{\textcolor{\Red}{#1}}
\def\YL#1{\textcolor{\Yellow}{#1}}
\def\GN#1{\textcolor{\Green}{#1}}
\def\OR#1{\textcolor{\Orange}{#1}}
\begin{frame}{Introduction}{The CHSH game}
    \begin{itemize}[<+->]
        \item Alice and Bob are initially proximate, so that they can strategize.\footnote{They could make some EPR pairs, but that comes later!}
        \item They then separate by a large distance.
        \item A referee randomly and uniformly sends one of two words to Alice, and one of two other words to Bob.  It is expedient to use colors for the words:
        \begin{description}
            \item[Alice] will hear \RD{red} or \YL{yellow}.
            \item[Bob] will hear \GN{green} or \OR{orange}.
        \end{description}
    \item Alice and Bob each respond with one bit ($0$ or $1$).  To win$\ldots$
    \begin{itemize}
        \item If Alice and Bob hear \RD{red} and \GN{green}, respectively, their bits must differ.
        \item Otherwise, their bits must agree.
    \end{itemize}
    \item Alice and Bob are too far apart to communicate before their answer is needed.
    \item What is the best they can do?
    \end{itemize}
\end{frame}%

\begin{frame}{Classical strategies}{How well can they do?}

\begin{itemize}[<+->]
    \item Based on the information given to Alice, she has only four possible strategies:
    \begin{itemize}
        \item Always respond $0$
        \item Always respond $1$
        \item Use the colors:  \RD{$0$}, \YL{$1$}
        \item Or oppositely:  \YL{$0$}, \RD{$1$}
    \end{itemize}
    \item Bob has the same four strategies, so between them they have~16 strategies.
    \item The simplest strategy: they each respond always with the same bit, say $0$.
    \item They win $\frac{3}{4}$ of the games they play, which turns out to be the best they can do classically.
\end{itemize}
    
\end{frame}

\begin{frame}{Analysis of a given Strategy}{We enumerate the possibilities}.
\Vskip{-3em}\TwoColumns{%
\begin{itemize}
    \item<1-> The both-$0$ strategy wins with probability $0.75$y.
    \item<2-> Of course, so does the both-$1$ strategy
    \item<3-> Using referee information doesn't improve the results.
\end{itemize}
}{%
\only<1,4>{%
\begin{center}
    \begin{tabular}{cccccc}
      &  \multicolumn{2}{c}{Hears} & \multicolumn{2}{c}{Responds} & Win?\\
       & Alice & Bob & Alice & Bob \\
       1 & \RD{\SCirc{}} & \GN{\SCirc{}} & $0$ & $0$ & \DisAgree{} \\
       2 & \RD{\SCirc{}} & \OR{\SCirc{}} & $0$ & $0$ & \Agree{} \\
        3 & \YL{\SCirc{}} & \GN{\SCirc{}} & $0$ & $0$ & \Agree{} \\
       4 & \YL{\SCirc{}} & \OR{\SCirc{}} & $0$ & $0$ & \Agree{} \\
    \end{tabular}
\end{center}}%
\only<2>{%
\begin{center}
    \begin{tabular}{cccccc}
      &  \multicolumn{2}{c}{Hears} & \multicolumn{2}{c}{Responds} & Win?\\
       & Alice & Bob & Alice & Bob \\
       1 & \RD{\SCirc{}} & \GN{\SCirc{}} & $1$ & $1$ & \DisAgree{} \\
       2 & \RD{\SCirc{}} & \OR{\SCirc{}} & $1$ & $1$ & \Agree{} \\
        3 & \YL{\SCirc{}} & \GN{\SCirc{}} & $1$ & $1$ & \Agree{} \\
       4 & \YL{\SCirc{}} & \OR{\SCirc{}} & $1$ & $1$ & \Agree{} \\
    \end{tabular}
\end{center}}%
\only<3>{%
\begin{center}
    \begin{tabular}{cccccc}
      &  \multicolumn{2}{c}{Hears} & \multicolumn{2}{c}{Responds} & Win?\\
       & Alice & Bob & Alice & Bob \\
       1 & \RD{\SCirc{}} & \GN{\SCirc{}} & \RD{$0$} & \GN{$1$} & \Agree{} \\
       2 & \RD{\SCirc{}} & \OR{\SCirc{}} & \RD{$0$} & \OR{$0$} & \Agree{} \\
        3 & \YL{\SCirc{}} & \GN{\SCirc{}} & \YL{$1$} & \GN{$1$} & \Agree{} \\
       4 & \YL{\SCirc{}} & \OR{\SCirc{}} & \YL{$1$} & \OR{$0$} & \DisAgree{} \\
    \end{tabular}
\end{center}}%
\only<5->{%
\begin{center}
   \begin{tabular}{cccccc}
      &  \multicolumn{2}{c}{Hears} & \multicolumn{2}{c}{Responds} & Win?\\
       & Alice & Bob & Alice & Bob \\
       1 & \RD{\SCirc{}} & \GN{\SCirc{}} & $1$ & $1$ & \DisAgree{} \\
       2 & \RD{\SCirc{}} & \OR{\SCirc{}} & $0$ & $1$ & \DisAgree{} \\
        3 & \YL{\SCirc{}} & \GN{\SCirc{}} & $0$ & $1$ & \DisAgree{} \\
       4 & \YL{\SCirc{}} & \OR{\SCirc{}} & $0$ & $1$ & \DisAgree{} \\
    \end{tabular}
\end{center}}%
}
\BigSkip{}
\begin{itemize}
        \item<4-> Using any strategy that is fixed over the game, the best they can do is win with probability~$0.75$, as with the both-$0$ strateg
    \item<5-> Any random strategy or mix of strategies is a distribution of fixed strategies, which wins at best with probability~$0.75$, and in the worst case they might never win.
\end{itemize}
\end{frame}

\begin{frame}{Analysis of classical strategies}{They win at best with probability $0.75$.}
\begin{center}
    \begin{tabular}{c|cccc}
          & \multicolumn{4}{c}{Alice} \\
        Bob & $0$ & $1$ & \RD{$0$} & \YL{0} \\
        $0$ &  \\
        $1$ & \\
        \GN{$0$} & \\
        \OR{$0$} &
    \end{tabular}
\end{center}
\end{frame}
}

\section*{Magic square}

\begin{frame}{Introduction}{The Mermin--Peres magic square, exhibiting quantum telepathy}
    
\end{frame}