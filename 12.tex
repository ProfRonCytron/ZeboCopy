\SetTitle{12}{The CHSH game}{A game demonstrating quantum advantage}{12}

\section*{Overview}

\begin{frame}{Overview}{What will we study here?}
\begin{itemize}[<+->]
  \item We examine a game of interest to students of quantum
computing.
  \item We can characterize how often the game can be won
using only classical computation.
   \item We can then show that using quantum computation, the players can
win more often than is classically possible.
   \item These games help us understand how quantum computing has an
advantage over classical computation.
   \item These studies also explore further how
measurement operators work and how to perform measurements in the
computational basis.
\end{itemize}
\end{frame}

\section*{CHSH}

{%
\def\Red{Red}\def\Yellow{Goldenrod}\def\Green{green}\def\Orange{orange}%
\def\RD#1{\textcolor{\Red}{#1}}
\def\YL#1{\textcolor{\Yellow}{#1}}
\def\GN#1{\textcolor{\Green}{#1}}
\def\OR#1{\textcolor{\Orange}{#1}}
\begin{frame}{Introduction to the \href{https://en.wikipedia.org/wiki/CHSH_game}{CHSH} game}{This game is due to John Clauser, Michael Horne, Abner Shimony, and Richard Holt}
    \begin{itemize}[<+->]
        \item Alice and Bob are initially proximate, so that they can strategize.\footnote{They could make some EPR pairs, but that comes later!}
        \item They then separate by a large distance.
        \item A referee randomly and uniformly sends one of two words to Alice, and one of two other words to Bob.  It is expedient to use colors for the words:
        \begin{description}
            \item[Alice] will hear \RD{red} or \YL{yellow}.
            \item[Bob] will hear \GN{green} or \OR{orange}.
        \end{description}
    \item Alice and Bob each respond with one bit ($0$ or $1$).  To win$\ldots$
    \begin{itemize}
        \item If Alice and Bob hear \RD{red} and \GN{green}, respectively, their bits must differ.
        \item Otherwise, their bits must agree.
    \end{itemize}
    \item Alice and Bob are too far apart to communicate before their answer is needed.
    \item What is the best they can do?
    \end{itemize}
\end{frame}%

\begin{frame}{Classical strategies}{How well can they do?}

\begin{itemize}[<+->]
    \item Based on the information given to Alice, she has only four possible strategies:
    \begin{itemize}
        \item Always respond $0$
        \item Always respond $1$
        \item Use the colors:  \RD{$0$}, \YL{$1$}
        \item Or oppositely:  \YL{$0$}, \RD{$1$}
    \end{itemize}
    \item Bob has the same four strategies, so between them they have~16 strategies.
    \item The simplest strategy: they each respond always with the same bit, say $0$.
    \item They win $\frac{3}{4}$ of the games they play, which turns out to be the best they can do classically.
\end{itemize}
    
\end{frame}

\begin{frame}{Analysis of a given Strategy}{We enumerate the possibilities}
\Vskip{-3em}\TwoColumns{%
\begin{itemize}
    \item<1-> The both-$0$ strategy wins with probability $0.75$.
    \item<2-> Of course, so does the both-$1$ strategy
    \item<3-> Using referee information doesn't improve the results.
\end{itemize}
}{%
\only<1,4>{%
\begin{center}
    \begin{tabular}{cccccc}
      &  \multicolumn{2}{c}{Hears} & \multicolumn{2}{c}{Responds} & Win?\\
       & Alice & Bob & Alice & Bob \\
       1 & \RD{\SCirc{}} & \GN{\SCirc{}} & $0$ & $0$ & \DisAgree{} \\
       2 & \RD{\SCirc{}} & \OR{\SCirc{}} & $0$ & $0$ & \Agree{} \\
        3 & \YL{\SCirc{}} & \GN{\SCirc{}} & $0$ & $0$ & \Agree{} \\
       4 & \YL{\SCirc{}} & \OR{\SCirc{}} & $0$ & $0$ & \Agree{} \\
    \end{tabular}
\end{center}}%
\only<2>{%
\begin{center}
    \begin{tabular}{cccccc}
      &  \multicolumn{2}{c}{Hears} & \multicolumn{2}{c}{Responds} & Win?\\
       & Alice & Bob & Alice & Bob \\
       1 & \RD{\SCirc{}} & \GN{\SCirc{}} & $1$ & $1$ & \DisAgree{} \\
       2 & \RD{\SCirc{}} & \OR{\SCirc{}} & $1$ & $1$ & \Agree{} \\
        3 & \YL{\SCirc{}} & \GN{\SCirc{}} & $1$ & $1$ & \Agree{} \\
       4 & \YL{\SCirc{}} & \OR{\SCirc{}} & $1$ & $1$ & \Agree{} \\
    \end{tabular}
\end{center}}%
\only<3>{%
\begin{center}
    \begin{tabular}{cccccc}
      &  \multicolumn{2}{c}{Hears} & \multicolumn{2}{c}{Responds} & Win?\\
       & Alice & Bob & Alice & Bob \\
       1 & \RD{\SCirc{}} & \GN{\SCirc{}} & \RD{$0$} & \GN{$1$} & \Agree{} \\
       2 & \RD{\SCirc{}} & \OR{\SCirc{}} & \RD{$0$} & \OR{$0$} & \Agree{} \\
        3 & \YL{\SCirc{}} & \GN{\SCirc{}} & \YL{$1$} & \GN{$1$} & \Agree{} \\
       4 & \YL{\SCirc{}} & \OR{\SCirc{}} & \YL{$1$} & \OR{$0$} & \DisAgree{} \\
    \end{tabular}
\end{center}}%
\only<5->{%
\begin{center}
   \begin{tabular}{cccccc}
      &  \multicolumn{2}{c}{Hears} & \multicolumn{2}{c}{Responds} & Win?\\
       & Alice & Bob & Alice & Bob \\
       1 & \RD{\SCirc{}} & \GN{\SCirc{}} & $1$ & $1$ & \DisAgree{} \\
       2 & \RD{\SCirc{}} & \OR{\SCirc{}} & $0$ & $1$ & \DisAgree{} \\
        3 & \YL{\SCirc{}} & \GN{\SCirc{}} & $0$ & $1$ & \DisAgree{} \\
       4 & \YL{\SCirc{}} & \OR{\SCirc{}} & $0$ & $1$ & \DisAgree{} \\
    \end{tabular}
\end{center}}%
}
\BigSkip{}
\begin{itemize}
        \item<4-> Using any strategy that is fixed over the game, the best they can do is win with probability~$0.75$, as with the both-$0$ strategy.
    \item<5-> Any random strategy or mix of strategies is a distribution of fixed strategies, which wins at best with probability~$0.75$, and in the worst case they might never win.
\end{itemize}
\end{frame}

\begin{frame}{Discussion}{Can quantum help?}

\begin{itemize}[<+->]
    \item The best Alice and Bob can do classically is win with probability~$0.75$.
    \item In Ekert's key-distribution algorithm, we observed that when two measurement bases are separated by $\pi/8$ radians, the probability of coincidental measurement is $\cos^{2}(\pi/8)\approx 0.85$.
    \item We can use this property to bias agreement in the three cases where such agreement is a win for Alice and Bob.  Note that while they will likely agree, the choice of whether that agreement is on~$0$ or~$1$ is (uniformly) random.
    \item We can also arrange for them to likely disagree in the one case where that is a win for them.
    \item In the slides that follow, we color a participant by the color supplied to her or him by the referee. So (red) \RD{Alice} heard \RD{red}, but (yellow) \YL{Alice} heard \YL{yellow}.  Similarly, \GN{Bob} heard \GN{green} and \OR{Bob} heard \OR{orange}.
\end{itemize}
    
\end{frame}

\begin{frame}{Quantum approach to winning CHSH more often}{Uses entanglement and proximity of bases}

\Vskip{-3em}\TwoUnequalColumns{0.48\textwidth}{0.52\textwidth}{%
\only<1-4>{%
\begin{itemize}
    \item<1-> Alice and Bob each have one qubit of an EPR pair in state $\frac{\ket{00}+\ket{11}}{\sqrt{2}}$.
    \item<1-> The computational basis, as usual.
    \item<2-> \OR{Bob} measures his qubit in that basis.
    \item<3-> \YL{Alice}, not knowing whether she faces \OR{Bob} or \GN{Bob}, always measures measures~$\pi/8$ radians counterclockwise from~\ket{0}.
    \item<4-> \RD{Alice}'s basis is rotated $\pi/8$ radians clockwise from~\ket{0}.
\end{itemize}}%
\only<5-6>{%
\begin{itemize}
    \item<5-> Consider \GN{Bob}'s other possibility, whose measurement we situate $\pi/4$ radians counterclockwise from \ket{0}.
    \item<5-> \YL{Alice} is $\pi/8$ radians away, and so they agree with the usual probability.
    \item<6-> But \RD{Alice} is $3\pi/8$ radians away.
\end{itemize}}%
\only<7->{%
Putting this all together:
\begin{itemize}
    \item<7-> Alice and Bob begin with an EPR pair and then separate.
    \item<8-> They each measure their qubit in the basis of the color they hear.
    \item<9-> They win by the requisite agreement or disagreement approximately~85\% of the time.  \end{itemize}
}
}{%}
\Vskip{-4em}\begin{center}
\begin{TIKZP}[scale=1.7]
\visible<1>{\draw[->,thick] (0,0) -- (1,0) node[right] {\ \ket{0}};}
\visible<1,7->{\draw[->,thick] (0,0) -- (0,1) node[above] {\ket{1}};}
\visible<2-4,7->{\draw[->,thick,\Orange] (0,0) -- (0:1) node[right] {Bob};}
\visible<2>{\draw[->,thick,dotted,\Orange] (0,0) -- (90:1) node[above] { };}
\visible<4,6,7->{\draw[->,thick,\Red] (0,0) -- (-22.5:1) node[right] {Alice};}
\visible<3-5,7->{\draw[->,thick,\Yellow] (0,0) -- (22.5:1) node[right] {Alice};}
\visible<3>{\draw[->] (1,0) arc (0:22.5:1.0) node[below right] {{\small $\pi/8$ radians}};}
\visible<5->{\draw[->,thick,\Green] (0,0) -- (45:1) node[above right] {Bob};}
\visible<6>{\draw (1,0)  node[above right] {{\small $3\pi/8$ radians}} arc(0:-22.5:1);
\draw[->] (1,0) arc(0:45:1);}

\end{TIKZP}
\end{center}
\only<2>{%
We show only the \ket{0}~axis of a basis.  The \ket{1}~axis is $\pi/2$~radians counterclockwise from~\ket{0}.
\BigSkip{}
If \OR{Bob} measures his qubit first, he is equally likely to measure~$0$ or~$1$.
}%
\only<3>{%
The probability that \only<3>{\YL{Alice}} and \OR{Bob} measure the same value, both~$0$ or both~$1$, is~$\cos^{2}(\pi/2)\approx 0.85$.
\MedSkip{}
If \YL{Alice} measures first, \YL{she} is more likely to see~$0$ than~$1$, but with the change in state, \OR{Bob} is still about 85\% likely to match \YL{her} measurement}%
\only<4>{%
The same is true here: \RD{Alice} and \OR{Bob} measure the same result with probability~$0.85$, no matter who measures first.
}%
\only<5>{%
If \GN{Bob} measures first, he sees~$0$ half the time.  But then \YL{Alice} is about $85\%$ likely to see~$0$ after that.  If \YL{Alice} measures first, she might measure~$0$.  With either ensuing change of state, \GN{Bob} is $85\%$ likely to match \YL{Alice}'s measurement.
}
\only<6>{%
The probability of \emph{disagreement} between \RD{Alice} and \GN{Bob} is approximately~$0.85$, no matter who measures first. In this case they win by disagreeing.
}%
}%
\only<10->{%
\BigSkip{}
The quantum approach wins $\sim 85\%$ of the time, while classically we win at best $75\%$ of the time. \alert{Quantum behavior provides a clear and measurable advantage over classical logic.}

}
\end{frame}
}

