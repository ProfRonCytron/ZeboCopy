\SetTitle{12}{Games with quantum advantage}{These rely on entanglement}{12}

\section*{Overview}

\begin{frame}{Overview}{What will we study here?}
\begin{itemize}[<+->]
  \item We examine two games that are of interest to students of quantum
computing.
  \item For each game, we can characterize how often the game can be won
using only classical computation.
   \item We can then show that using quantum computation, the players can
win more often than is classically possible.
   \item These games help us understand how quantum computing has an
advantage over classical computation.
   \item While studying these games we will also explore further how
measurement operators work and how to perform measurements in the
computational basis.
\end{itemize}
\end{frame}

\section*{CHSH}

{%
\def\Red{Red}\def\Yellow{Goldenrod}\def\Green{green}\def\Orange{orange}%
\def\RD#1{\textcolor{\Red}{#1}}
\def\YL#1{\textcolor{\Yellow}{#1}}
\def\GN#1{\textcolor{\Green}{#1}}
\def\OR#1{\textcolor{\Orange}{#1}}
\begin{frame}{Introduction to the \href{https://en.wikipedia.org/wiki/CHSH_game}{CHSH} game}{This game is due to John Clauser, Michael Horne, Abner Shimony, and Richard Holt}
    \begin{itemize}[<+->]
        \item Alice and Bob are initially proximate, so that they can strategize.\footnote{They could make some EPR pairs, but that comes later!}
        \item They then separate by a large distance.
        \item A referee randomly and uniformly sends one of two words to Alice, and one of two other words to Bob.  It is expedient to use colors for the words:
        \begin{description}
            \item[Alice] will hear \RD{red} or \YL{yellow}.
            \item[Bob] will hear \GN{green} or \OR{orange}.
        \end{description}
    \item Alice and Bob each respond with one bit ($0$ or $1$).  To win$\ldots$
    \begin{itemize}
        \item If Alice and Bob hear \RD{red} and \GN{green}, respectively, their bits must differ.
        \item Otherwise, their bits must agree.
    \end{itemize}
    \item Alice and Bob are too far apart to communicate before their answer is needed.
    \item What is the best they can do?
    \end{itemize}
\end{frame}%

\begin{frame}{Classical strategies}{How well can they do?}

\begin{itemize}[<+->]
    \item Based on the information given to Alice, she has only four possible strategies:
    \begin{itemize}
        \item Always respond $0$
        \item Always respond $1$
        \item Use the colors:  \RD{$0$}, \YL{$1$}
        \item Or oppositely:  \YL{$0$}, \RD{$1$}
    \end{itemize}
    \item Bob has the same four strategies, so between them they have~16 strategies.
    \item The simplest strategy: they each respond always with the same bit, say $0$.
    \item They win $\frac{3}{4}$ of the games they play, which turns out to be the best they can do classically.
\end{itemize}
    
\end{frame}

\begin{frame}{Analysis of a given Strategy}{We enumerate the possibilities}
\Vskip{-3em}\TwoColumns{%
\begin{itemize}
    \item<1-> The both-$0$ strategy wins with probability $0.75$.
    \item<2-> Of course, so does the both-$1$ strategy
    \item<3-> Using referee information doesn't improve the results.
\end{itemize}
}{%
\only<1,4>{%
\begin{center}
    \begin{tabular}{cccccc}
      &  \multicolumn{2}{c}{Hears} & \multicolumn{2}{c}{Responds} & Win?\\
       & Alice & Bob & Alice & Bob \\
       1 & \RD{\SCirc{}} & \GN{\SCirc{}} & $0$ & $0$ & \DisAgree{} \\
       2 & \RD{\SCirc{}} & \OR{\SCirc{}} & $0$ & $0$ & \Agree{} \\
        3 & \YL{\SCirc{}} & \GN{\SCirc{}} & $0$ & $0$ & \Agree{} \\
       4 & \YL{\SCirc{}} & \OR{\SCirc{}} & $0$ & $0$ & \Agree{} \\
    \end{tabular}
\end{center}}%
\only<2>{%
\begin{center}
    \begin{tabular}{cccccc}
      &  \multicolumn{2}{c}{Hears} & \multicolumn{2}{c}{Responds} & Win?\\
       & Alice & Bob & Alice & Bob \\
       1 & \RD{\SCirc{}} & \GN{\SCirc{}} & $1$ & $1$ & \DisAgree{} \\
       2 & \RD{\SCirc{}} & \OR{\SCirc{}} & $1$ & $1$ & \Agree{} \\
        3 & \YL{\SCirc{}} & \GN{\SCirc{}} & $1$ & $1$ & \Agree{} \\
       4 & \YL{\SCirc{}} & \OR{\SCirc{}} & $1$ & $1$ & \Agree{} \\
    \end{tabular}
\end{center}}%
\only<3>{%
\begin{center}
    \begin{tabular}{cccccc}
      &  \multicolumn{2}{c}{Hears} & \multicolumn{2}{c}{Responds} & Win?\\
       & Alice & Bob & Alice & Bob \\
       1 & \RD{\SCirc{}} & \GN{\SCirc{}} & \RD{$0$} & \GN{$1$} & \Agree{} \\
       2 & \RD{\SCirc{}} & \OR{\SCirc{}} & \RD{$0$} & \OR{$0$} & \Agree{} \\
        3 & \YL{\SCirc{}} & \GN{\SCirc{}} & \YL{$1$} & \GN{$1$} & \Agree{} \\
       4 & \YL{\SCirc{}} & \OR{\SCirc{}} & \YL{$1$} & \OR{$0$} & \DisAgree{} \\
    \end{tabular}
\end{center}}%
\only<5->{%
\begin{center}
   \begin{tabular}{cccccc}
      &  \multicolumn{2}{c}{Hears} & \multicolumn{2}{c}{Responds} & Win?\\
       & Alice & Bob & Alice & Bob \\
       1 & \RD{\SCirc{}} & \GN{\SCirc{}} & $1$ & $1$ & \DisAgree{} \\
       2 & \RD{\SCirc{}} & \OR{\SCirc{}} & $0$ & $1$ & \DisAgree{} \\
        3 & \YL{\SCirc{}} & \GN{\SCirc{}} & $0$ & $1$ & \DisAgree{} \\
       4 & \YL{\SCirc{}} & \OR{\SCirc{}} & $0$ & $1$ & \DisAgree{} \\
    \end{tabular}
\end{center}}%
}
\BigSkip{}
\begin{itemize}
        \item<4-> Using any strategy that is fixed over the game, the best they can do is win with probability~$0.75$, as with the both-$0$ strategy.
    \item<5-> Any random strategy or mix of strategies is a distribution of fixed strategies, which wins at best with probability~$0.75$, and in the worst case they might never win.
\end{itemize}
\end{frame}

\begin{frame}{Discussion}{Can quantum help?}

\begin{itemize}[<+->]
    \item The best Alice and Bob can do classically is win with probability~$0.75$.
    \item In Ekert's key-distribution algorithm, we observed that when two measurement bases are separated by $\pi/8$ radians, the probability of coincidental measurement is $\cos^{2}(\pi/8)\approx 0.85$.
    \item We can use this property to bias agreement in the three cases where such agreement is a win for Alice and Bob.  Note that while they will likely agree, the choice of whether that agreement is on~$0$ or~$1$ is (uniformly) random.
    \item We can also arrange for them to likely disagree in the one case where that is a win for them.
    \item In the slides that follow, we color a participant by the color supplied to her or him by the referee. So \RD{Alice} heard \RD{red}, but \YL{Alice} heard \YL{yellow}.  Similarly, \GN{Bob} heard \GN{green} and \OR{Bob} heard \OR{orange}.
\end{itemize}
    
\end{frame}

\begin{frame}{Quantum approach to winning CHSH more often}{Uses entanglement and proximity of bases}

\Vskip{-3em}\TwoUnequalColumns{0.48\textwidth}{0.52\textwidth}{%
\only<1-4>{%
\begin{itemize}
    \item<1-> Alice and Bob each have one qubit of an EPR pair in state $\frac{\ket{00}+\ket{11}}{\sqrt{2}}$.
    \item<1-> The computational basis, as usual.
    \item<2-> \OR{Bob} measures his qubit in that basis.
    \item<3-> \YL{Alice}, not knowing whether she faces \OR{Bob} or \GN{Bob}, always measures measures~$\pi/8$ radians counterclockwise from~\ket{0}.
    \item<4-> \RD{Alice}'s basis is rotated $\pi/8$ radians clockwise from~\ket{0}.
\end{itemize}}%
\only<5-6>{%
\begin{itemize}
    \item<5-> Consider \GN{Bob}'s other possibility, whose measurement we situate $\pi/4$ radians counterclockwise from \ket{0}.
    \item<5-> \YL{Alice} is $\pi/8$ radians away, and so they agree with the usual probability.
    \item<6-> But \RD{Alice} is $3\pi/8$ radians away.
\end{itemize}}%
\only<7->{%
Putting this all together:
\begin{itemize}
    \item<7-> Alice and Bob begin with an EPR pair and then separate.
    \item<8-> They each measure their qubit in the basis of the color they hear.
    \item<9-> They win by the requisite agreement or disagreement approximately~85\% of the time.  \end{itemize}
}
}{%}
\Vskip{-4em}\begin{center}
\begin{TIKZP}[scale=1.7]
\visible<1>{\draw[->,thick] (0,0) -- (1,0) node[right] {\ \ket{0}};}
\visible<1,7->{\draw[->,thick] (0,0) -- (0,1) node[above] {\ket{1}};}
\visible<2-4,7->{\draw[->,thick,\Orange] (0,0) -- (0:1) node[right] {Bob};}
\visible<2>{\draw[->,thick,dotted,\Orange] (0,0) -- (90:1) node[above] { };}
\visible<4,6,7->{\draw[->,thick,\Red] (0,0) -- (-22.5:1) node[right] {Alice};}
\visible<3-5,7->{\draw[->,thick,\Yellow] (0,0) -- (22.5:1) node[right] {Alice};}
\visible<3>{\draw[->] (1,0) arc (0:22.5:1.0) node[below right] {{\small $\pi/8$ radians}};}
\visible<5->{\draw[->,thick,\Green] (0,0) -- (45:1) node[above right] {Bob};}
\visible<6>{\draw (1,0)  node[above right] {{\small $3\pi/8$ radians}} arc(0:-22.5:1);
\draw[->] (1,0) arc(0:45:1);}

\end{TIKZP}
\end{center}
\only<2>{%
We show only the \ket{0}~axis of a basis.  The \ket{1}~axis is $\pi/2$~radians counterclockwise from~\ket{0}.
\BigSkip{}
If \OR{Bob} measures his qubit first, he is equally likely to measure~$0$ or~$1$.
}%
\only<3>{%
The probability that \only<3>{\YL{Alice}} and \OR{Bob} measure the same value, both~$0$ or both~$1$, is~$\cos^{2}(\pi/2)\approx 0.85$.
\MedSkip{}
If \YL{Alice} measures first, \YL{she} is more likely to see~$0$ than~$1$, but with the change in state, \OR{Bob} is still about 85\% likely to match \YL{her} measurement}%
\only<4>{%
The same is true here: \RD{Alice} and \OR{Bob} measure the same result with probability~$0.85$, no matter who measures first.
}%
\only<5>{%
If \GN{Bob} measures first, he will certainly see~$0$.  But then \YL{Alice} is about $85\%$ likely to see~$0$ after that.  If \YL{Alice} measures first, she will likely measure~$0$.  With either ensuing change of state, \GN{Bob} is $85\%$ likely to match \YL{Alice}'s measurement.
}
\only<6>{%
The probability of \emph{disagreement} between \RD{Alice} and \GN{Bob} is approximately~$0.85$, no matter who measures first. In this case they win by disagreeing.
}%
}%
\only<10->{%
\BigSkip{}
The quantum approach wins $\sim 85\%$ of the time, while classically we win at best $75\%$ of the time. \alert{Quantum behavior provides a clear and measurable advantage over classical logic.}

}
\end{frame}
}

\section*{Magic square}



\begin{frame}{Introduction to the \href{https://en.wikipedia.org/wiki/Quantum_pseudo-telepathy\#The_Mermin\%E2\%80\%93Peres_magic_square_game}{Mermin--Peres magic square}}{This game is due to \href{https://en.wikipedia.org/wiki/N._David_Mermin}{David Mermin} and \href{https://en.wikipedia.org/wiki/Asher_Peres}{Asher Peres}}

\begin{itemize}[<+->]
    \item With CHSH, a quantum approach is probabilistically better than any classical approach but does not guarantee to win the game.
    \item The Mermin--Peres magic square is a puzzle that Alice and Bob can win at best $\frac{8}{9}$ of the time, but a quantum approach on a reliable quantum computer can always win.
    \item Alice and Bob share two EPR pairs, each initially in the state $\frac{\ket{00}+\ket{11}}{\sqrt{2}}$.
    \item Variations of the game we consider also go by the same name, and they are equivalent in difficulty and in the approach to a quantum-based solution.
    \item We follow the game as specified in the Wikipedia \href{https://en.wikipedia.org/wiki/Quantum_pseudo-telepathy\#The_Mermin\%E2\%80\%93Peres_magic_square_game}{article}.
\end{itemize}
    
\end{frame}

\begin{frame}{How is the game played?}
\Vskip{-3em}\TwoUnequalColumns{0.5\textwidth}{0.5\textwidth}{%
\begin{MPSquareEnv}
\only<1-6>{%
\begin{itemize}
    \item<1-> After enjoying some time together, Alice and Bob separate and cannot communicate with each other.
    \item<2-> A referee chooses a \textcolor{\RCone}{row number for Alice (2)} and a \textcolor{\RCtwo}{column number for Bob (1)}.
    \item<3-> Alice must respond with an entry for each cell of \textcolor{\RCone}{her designated row}, choosing~\PP{} or~\MM{}.
    \item<6-> The product of \textcolor{\RCone}{her entries} must be~\PP{}.
\end{itemize}}
\only<7->{%
\begin{itemize}
    \item<7-> Bob must do the same for \textcolor{\RCtwo}{his designated column}.
    \item<8-> Where the row and column \textcolor{Purple}{intersect}, Alice and Bob must agree on the value, or they lose.
    \item<10-> The product of \textcolor{\RCtwo}{his entries} must be~\MM{}.
\end{itemize}
}
\end{MPSquareEnv}
}{%
\begin{center}
\begin{MPSquareEnv}
\def\VS####1####2{\visible<####1>{####2}}
\MPSquare{\VS{7-}{\textcolor{\RCtwo}{\PP}}}{ }{ }{\only<3-7>{\VS{3-}{\textcolor{\RCone}{\PP}}}\only<8->{\VS{3-}{\textcolor{Purple}{\PP}}}}{\VS{4-}{\textcolor{\RCone}{\MM}}}{\VS{5-}{\textcolor{\RCone}{\MM}}}{\VS{9-}{\textcolor{\RCtwo}{\MM}}}{ }{ }
\end{MPSquareEnv}
\end{center}
}%
\end{frame}

\begin{frame}{Best classical solution}{They can agree to respond as shown on this slide, winning $8$ of $9$ games on average}

\Vskip{-3em}\TwoUnequalColumns{0.6\textwidth}{0.4\textwidth}{%
\begin{MPSquareEnv}
\only<1-6>{%
\begin{itemize}[<+->]
    \item Before separating, Alice and Bob can agree to respond as shown here.
    \item We can verify that for the first two rows and columns, they always agree on the common value and the products are correct.
    \item For row or column three, Alice and Bob would have to complete their response in a way that creates the correct product.
    \begin{description}
      \item[\ColorOne{Alice}] must respond \ColorOne{\MM{}} so that her row's product is \PP{}.
      \item[\ColorTwo{Bob}] must respond \ColorTwo{\PP{}} so that his column's product is \MM{}.
    \end{description}
\end{itemize}}%
\only<7->{%
\begin{itemize}
    \item<7-> While other static solutions are possible, each would have at least one square where Alice and Bob cannot agree classically on an entry.
    \item<8-> As with previous games, we will use measurements of entangled qubits to influence choices made by \textit{incommunicado} Alice and Bob.
    \item<9-> With quantum computing, they can always win.
    \item<10-> But first let's rule out the possibility of any always-win classical solution.
\end{itemize}
}
\end{MPSquareEnv}
}{%
\Vskip{-3em}\begin{center}
\begin{MPSquareEnv}
\MPSquare{\PP{}}{\PP{}}{\PP{}}{\PP{}}{\MM{}}{\MM{}}{\MM{}}{\PP{}}{\only<1-3,6-8>{\alert{\textbf{?}}}\only<4>{\ColorOne{\MM{}}}\only<5>{\ColorTwo{\PP{}}}}
\end{MPSquareEnv}
\end{center}
\visible<6->{%
\alert{They lose only when \emph{both} the third row and column are designated.  Is there a way to win always?}}
}
    
\end{frame}

\begin{frame}{Proof of no static solution}{We cannot place values in each cell that always work}

\Vskip{-3em}\TwoUnequalColumns{0.6\textwidth}{0.4\textwidth}{%
\begin{MPSquareEnv}
\Vskip{-3em}\begin{itemize}
    \item<1-> Each row's product is~\PP{} so the three rows' products is also~\PP{}.
    \item<2-> Each column's product is~\MM{} so the three columns' products is also~\MM{}.
    \item<3-> This is a contradiction.
\end{itemize}
\begin{align*}
    \visible<1>{a b c = d e f = g h k &= \PP{} \\}
    \visible<1,3>{\ColorOne{a b c \times d e f \times g h k} & = \PP{} \\}
    \visible<2>{a d g = b e h = c f k &= \MM{} \\}
    \visible<2->{\ColorTwo{a d g \times b e h \times c f k} & = \MM{} \\}
    \visible<3->{\ColorOne{abcdefghk} & \neq \ColorTwo{abcdefghk}}
\end{align*}
\visible<3->{\Vskip{-4em}\QED}
\end{MPSquareEnv}
}{%
\Vskip{-3em}\begin{center}
\begin{MPSquareEnv}
\MPSquare{$a$}{$b$}{$c$}{$d$}{$e$}{$f$}{$g$}{$h$}{$k$}
\end{MPSquareEnv}
\end{center}
}
    
\end{frame}

\begin{frame}{Variation 1}{All products are the same except one}
\Vskip{-5em}\TwoUnequalColumns{0.53\textwidth}{0.47\textwidth}{%
\begin{MPSquareEnv}
\begin{itemize}[<+->]
    \item In the version we consider, we require the product of each row's entries to be \PP{} and the product of each column's entries to be \MM{}.
    \item In another version, the product of each row's and each column's entries is \PP{}, \emph{except} the product of the third column's entries is \MM{}.
    \item A solution to the first form can be reduced from a solution to the second form by always negating the values as shown.
    \item The proof of no static solution is the same.
\end{itemize}
\end{MPSquareEnv}
}{%
\Vskip{-1em}\begin{center}
\begin{MPSquareEnv}
\MPSquare{ }{ }{ }{ }{ }{ }{\visible<3->{\alert{$-$}}}{\visible<3->{\alert{$-$}}}{ }
\end{MPSquareEnv}
\end{center}
\visible<4->{%
\begin{MPSquareEnv}%
This causes columns~1 and~2 to have product~\MM{}.  The third row still has a positive product because the two negations cancel.\end{MPSquareEnv}
}
}
\end{frame}

\begin{frame}{Variation 2}{Using sums instead of products}
\Vskip{-3em}\TwoUnequalColumns{0.6\textwidth}{0.4\textwidth}{%
\begin{MPSquareEnv}
\begin{itemize}
    \item<1-> Here is the original formulation
    \item<2-> In this variation, \ColorOne{Alice}'s rows must have even parity, and \ColorTwo{Bob}'s columns must have odd parity.
    \item<3-> The \PP{} and \MM{} of the original version are mapped to $0$ and $1$, respectively.
    \item<4-> Of course there is still no solution classically for every square.
\end{itemize}
\end{MPSquareEnv}
}{%
\Vskip{-3em}\begin{center}
\alt<1>{%
\begin{MPSquareEnv}\MPSquare{\PP{}}{\PP{}}{\PP{}}{\PP{}}{\MM{}}{\MM{}}{\MM{}}{\PP{}}{\alert{?}}\end{MPSquareEnv}}{%
\begin{MPSquareEnv}
\MPSquare{$0$}{$0$}{$0$}{$0$}{$1$}{$1$}{$1$}{$0$}{\alert{?}}
\end{MPSquareEnv}}
\end{center}
}

\end{frame}

{
\def\M#1#2{\mbox{#1{\ensuremath{\mathcal{M}_{#2}}}}}
\def\Mone{\M{\ColorThree}{1}}
\def\Mtwo{\M{\ColorFour}{2}}
\def\Mthree{\M{\ColorFive}{3}}
\begin{frame}{Successive measurements}{These are reliable if there is a common eigenbasis}
\Vskip{-4.5em}\TwoUnequalColumns{0.6\textwidth}{0.4\textwidth}{%
\only<1-8>{%
\begin{itemize}
    \item<1-> The boxes represent measurement operators applied to the two-qubit circuit.
    \item<2-> Recall that the operators are implemented by 
    \Vskip{-2em}\begin{enumerate}
        \item transformation into \PauliZ{}
        \item measurement in the \PauliZ{} basis
        \item transformation out of \PauliZ{}
    \end{enumerate}
    but to avoid clutter we show this using the measurement operators.
    \item<3-> After \Mone, the quantum system's state must be an eigenstate of \Mone.
    \item<4-> If \Mtwo{} and \Mthree{} share an eigenbasis with \Mone, then the outcome of their measurements is determined solely by \Mone.
\end{itemize}}%
\only<9-14>{%
\begin{itemize}
    \item<9-> Suppose we change the sequence of measurements, so that \Mtwo{} is first.
    \item<10-> While \Mtwo's results could be unpredictable, the resulting eigenstate \QState{} \emph{completely determines} the outcomes of \Mone{} and \Mthree.
    \item<11-> Suppose we prepare eigenstate~\QState{} by measuring in the common eigenbasis.  Then \Mtwo's outcome is also determined.
    \item<12-> Each outcome is $\pm 1$: the eigenvalue associated with \QState{} for each operator.
    \item<13-> Let the cummulative result of the three operators be the \emph{product} of their measured eigenvalues.
\end{itemize}
}%
\only<15->{%
\begin{itemize}
    \item<15-> We regard the puzzle's solution in two ways:
    \begin{itemize}
        \item<16-> A single measurement of the two qubits is performed to obtain eigenstate \alert<16>{\QState{}}, which determines the outcome of \Mone, \Mtwo, and \Mthree.  Those eigenvalues' product will yield the desired result.
        \item<17-> Three successive measurements are made, but the \alert<17>{first outcome} determines the other two outcomes.  The product of the three eigenvalues will also be as desired.
    \end{itemize}
    \item<18-> Linear algebra teaches that if each pair of operators in \Set{\Mone, \Mtwo, \Mthree} \emph{commutes}, then there is a common eigenbasis for the three operators.
\end{itemize}
}
}{%
\begin{center}
\adjustbox{scale=0.75,valign=t}{%
\begin{quantikz}[row sep=-12pt] \lstick[wires=2]{\mbox{\visible<11-14,16>{\alert<16>{\QState{}}}}}\qw & \gate[wires=2]{\alt<9-14>{\Mtwo}{\Mone}}\slice{\mbox{\visible<1-10,17->{\alert<17>{\QState{}}}}} & \qw & \gate[wires=2]{\alt<9-14>{\Mone}{\Mtwo}} & \qw & \gate[wires=2]{\Mthree} & \qw \\
\qw & \qw & \qw & \qw & \qw & \qw & \qw
\end{quantikz}}\end{center}%
\only<1-8>{%
\begin{itemize}
    \item<5-> Suppose the outcome of \Mone{} is \QState{}, an eigenstate of \Mtwo{}.
    \item<6-> Then $\Mtwo\QState{}=\pm 1\QState{}$, an eigenstate of \Mthree.
    \item<7-> Then $\Mthree\QState{}=\pm 1\QState{}$.
\end{itemize}
\SmallSkip{}\visible<8->{%
The results of \Mtwo{} and \Mthree{} are determined by \QState{}, the result of~\Mone.}
}%
\only<14>{%
\begin{MPSquareEnv}\SmallSkip{}We can choose operators such that for any eigenstate~\QState{} common to the three operators, we obtain products as follows: 
    \begin{itemize}
      \item \ColorOne{\PP{}} for \ColorOne{Alice's rows}
      \item \ColorTwo{\MM{}} for \ColorTwo{Bob's columns}
    \end{itemize}\end{MPSquareEnv}
}%
\only<18->{%
\SmallSkip{}A pair of operators \Set{A,B} commutes if
\[ \left(A\times B\right) - \left(B\times A\right) = \ZeroMatrix \]
where \ZeroMatrix{} is the all-zero matrix.}%
}
    
\end{frame}

\def\B#1{\hbox to 1.2em{\hss\ensuremath{#1}}}
\begin{frame}{Example of three commuting operators}

\Vskip{-5em}\TwoUnequalColumns{0.455\textwidth}{0.545\textwidth}{%
\only<1-10>{%
\begin{itemize}
    \item<1-> We need to find an eigenbasis common to all three operators.
    \item<2-> \texttt{MATLAB} provides this common eigenbasis for \Mone{} and \Mtwo{} \visible<3->{and their eigenvalues.}
    \item<4-> We next develop their \TensProd{\PauliX}{\PauliZ} product states  normalization.
\end{itemize}
\visible<2->{%
\Vskip{-3em}\begin{center}\small
    \begin{tabular}{cccc}
    \alert<5>{\QState{1}} & \alert<6>{\QState{2}} & \alert<7>{\QState{3}} & \alert<8>{\QState{4}} \\
     \visible<3->{$1$ & $-1$ & $-1$ & $1$ \\}
     \alert<5>{\DQB{-1}{0}{-1}{0}}
           & \alert<6>{\DQB{0}{-1}{0}{-1}}
           & \alert<7>{\DQB{-1}{0}{1}{0}}
           & \alert<8>{\DQB{0}{-1}{0}{1}} \\[2em]
        \visible<5->{   \ket{\ColorOne{+}\ColorTwo{0}}} & \visible<6->{\ket{\ColorOne{+}\ColorTwo{1}}}& \visible<7->{\ket{\ColorOne{-}\ColorTwo{0}}} &\visible<8->{\ket{\ColorOne{-}\ColorTwo{1}}}
    \end{tabular}
\end{center}}}%
\only<11->{%
\begin{center}
\adjustbox{scale=0.75,valign=t}{%
\begin{quantikz} 
\qw & \gate[wires=2]{\mbox{$\Conj{T}$}} & \qw & \meter{0\,/\,1} & \qw & \gate[wires=2]{\mbox{$T$}} & \qw \\
\qw & \qw & \qw & \meter{0\,/\,1} & \qw & \qw & \qw
\end{quantikz}}\end{center}%
\only<11-12>{%
\begin{itemize}
    \item<11-> The circuit above uses a single measurement to capture an observation in the \TensProd{\PauliX}{\PauliZ} basis.
    \item<12-> Based on the order of eigenstates in $T$, we decode the result as follows:
    \begin{center}
        \begin{tabular}{c@{$\mapsto$}c|c@{$\mapsto$}c}
        \ket{00} & \ket{+0} & \ket{01} & \ket{+1} \\
        \ket{10} & \ket{-0} & \ket{11}& \ket{-1}
         \end{tabular}
    \end{center}
\end{itemize}}%
\only<13->{%
\begin{center}
\adjustbox{scale=0.75,valign=t}{%
\begin{quantikz} 
\qw & \gate{\Hadamard} & \qw & \meter{0\,/\,1} & \qw & \gate{\Hadamard} & \qw \\
\qw & \qw & \qw & \meter{0\,/\,1} & \qw & \qw & \qw
\end{quantikz}}\end{center}%
$T$ and \Conj{T} can now be implemented separately on each qubit, as shown above, using an \Hadamard{} and \Identity{} (nothing) gate.
}
}
}{%
\only<1-4>{%
\begin{align*}
    \Mone =& \ColorThree{\TensProd{\Identity}{\PauliZ}} =& \ColorThree{\begin{pmatrix*}[r]
     \B{1}  &   \B{0}  &   \B{0}   &  \B{0} \\
     \B{0}  &  \B{-1}  &   \B{0}   &  \B{0} \\
     \B{0}  &   \B{0}  &   \B{1}   &  \B{0} \\
     \B{0}  &   \B{0}  &   \B{0}   & \B{-1}
    \end{pmatrix*}} \\
    \Mtwo =& \ColorFour{\TensProd{\PauliX}{\Identity}} =& \ColorFour{\begin{pmatrix*}[r]
     \B{0}   &  \B{0}  &   \B{1}   &  \B{0} \\
     \B{0}   &  \B{0}  &   \B{0}   &  \B{1} \\
     \B{1}   &  \B{0}  &   \B{0}   &  \B{0} \\
     \B{0}   &  \B{1}  &   \B{0}   &  \B{0}
    \end{pmatrix*}} \\
    \Mthree =& \ColorFive{\TensProd{\PauliX}{\PauliZ}} =& \ColorFive{\begin{pmatrix*}[r]
     \B{0} &     \B{0} &     \B{1} &     \B{0} \\
     \B{0} &     \B{0} &     \B{0} &    \B{-1} \\
     \B{1} &     \B{0} &     \B{0} &     \B{0} \\
     \B{0} &    \B{-1} &     \B{0} &     \B{0}
    \end{pmatrix*}}
\end{align*}}
\only<5-6>{%
\begin{align*}
\alert<5>{\DQB{-1}{0}{-1}{0}} &\equiv \RootTwo{}\DQB{1}{0}{1}{0} &=\TensProd{\ColorOne{\PPlus}}{\ColorTwo{\PZero}} \\[2em]
\visible<6->{\alert<6>{\DQB{-1}{0}{1}{0}} &\equiv \RootTwo{}\DQB{1}{0}{-1}{0} &= \TensProd{\ColorOne{\PMinus}}{\ColorTwo{\PZero}}}
\end{align*}
}%
\only<7-8>{%
\begin{align*}
\alert<7>{\DQB{0}{-1}{0}{-1}} &\equiv \RootTwo{}\DQB{0}{1}{0}{1} &=\TensProd{\ColorOne{\PPlus}}{\ColorTwo{\POne}} \\[2em]
\visible<8->{\alert<8>{\DQB{0}{-1}{0}{1}} &\equiv \RootTwo{}\DQB{0}{1}{0}{-1} &= \TensProd{\ColorOne{\PMinus}}{\ColorTwo{\POne}}}
\end{align*}
}%
\only<9->{%
\SmallSkip{}\only<9>{

This serves as an eigenbasis also for \Mthree, so we can now construct our matrices~$T$ and~\Conj{T}.}
\[
T = \RootTwo{} \alt<9>{\begin{pmatrix*}
\DQB{1}{0}{1}{0} & 
\DQB{0}{1}{0}{1} &
\DQB{1}{0}{-1}{0} &
\DQB{0}{1}{0}{-1}
\end{pmatrix*}}{%
\begin{pmatrix*}[r]
1 & 0 & 1 & 0 \\
0 & 1 & 0 & 1 \\
1 & 0 &-1 & 0 \\
0 & 1 & 0 &-1
\end{pmatrix*}
}
\]
\only<10->{%
\SmallSkip{}\only<10>{
$T$ is its own conjugate transpose, so:}
\[
\Conj{T} = T
\]}
\only<11->{%
\Vskip{-3em}\begin{itemize}
   \item<11-> Our $T$ matrix is factorable as \TensProd{\Hadamard}{\Identity}.
   \item<12-> This more clearly maps $\TensProd{\PauliZ}{\PauliZ}\leftrightarrow\TensProd{\PauliX}{\PauliZ}$ basis, sending the top qubit into and out of the \PauliX{} basis and leaving the bottom qubit undisturbed in the \PauliZ{} basis.
\end{itemize}
}
}
}
    
\end{frame}
}

\begin{frame}{Quantum-based solution}{We use two EPR pairs}

\Vskip{-4em}\begin{center}
\adjustbox{valign=t}{%
\begin{quantikz} 
\lstick{\QZero}\qw &  \gate{\Hadamard} & \ctrl{1} & \qw\rstick{\ColorOne{\mbox{$a_1$}}} \\
\lstick{\QZero}\qw & \qw & \targ{} & \qw \rstick{\ColorTwo{\mbox{$b_1$}}} 
\end{quantikz}}%
\hbox to 3ex{\hss}%
\adjustbox{valign=t}{%
\begin{quantikz} 
\lstick{\QZero}\qw &  \gate{\Hadamard} & \ctrl{1} & \qw\rstick{\ColorOne{\mbox{$a_2$}}} \\
\lstick{\QZero}\qw & \qw & \targ{} & \qw\rstick{\ColorTwo{\mbox{$b_2$}}} 
\end{quantikz}}\end{center}%

\begin{itemize}[<+->]
    \item Two EPR pairs are created, each in state \TwoSup{00}{11}.
    \item \ColorOne{Alice} receives~\ColorOne{\QState{a}=\ket{a_{1}a_{2}}} and \ColorTwo{Bob} receives~\ColorTwo{\QState{b}=\ket{b_{1}b_{2}}}
    \item Depending on the row (for Alice) and the column (for Bob), they will carry out measurements based on a table, in a basis related to their particular row or column.
    \item We will prove that they
    \begin{itemize}
        \item agree on a value at the cell intersecting the row and column, and
        \item obtain the necessary row and column products.
    \end{itemize}
\end{itemize}
    
\end{frame}

\begin{frame}{Solution}{Expressed in measurement operators}

\Vskip{-4.5em}\TwoUnequalColumns{0.6\textwidth}{0.4\textwidth}{%
\only<1-4>{%
\begin{itemize}
    \item<1-> Each square shows the measurement operators used by a player on two qubits.  
    \item<2-> The \alert<2>{\Identity{}} measurement operator has an eigenvalue of $+1$, regardless of its input.
    \item<3-> The \alert<3>{bottom left two entries} are marked so that their outcome is complimented.
    \item<4-> The result reported in a given square by a player is the eigenvalue associated with the two-qubit measurement performed by \ColorOne{her} or \ColorTwo{him}, complimented in \alert<4>{these} two squares.
\end{itemize}}
\only<5->{%
\begin{itemize}
   \item<5-> Each square's operator is already expressed as a product of two gates.
   \item<6-> A two-qubit measurement can be performed separately on each qubit, \emph{unless} the inputs are entangled.
   \item<8-> A player's qubits are not initially entangled with each other!  Each is entangled with one of the \emph{other player's} qubits.
   \item<9-> When measured separately, the overall square's result is the product of the measurements' eigenvalues\visible<10->{, \alert<10>{possibly complimented}.}
\end{itemize}
}
}{%
\begin{center}
    \begin{MPSquareEnv}
    \def\MyMarkupSquareComp####1{\alert<3,4,10>{####1}}
    \def\MyMarkupSquareI####1{\alert<2>{####1}}
    \MPSoln{}
    \end{MPSquareEnv}
\end{center}
}

\end{frame}


\begin{frame}{Proof of this solution}{Overview of three approaches}

\TwoUnequalColumns{0.6\textwidth}{0.4\textwidth}{%
\only<1-2>{Obligation:
\begin{itemize}
    \item<1-> We must show that for any row, \ColorOne{Alice}'s results multiply to~$+1$ and \ColorTwo{Bob}'s results multiply to~$-1$.
    \item<2-> We must show that for the square in common for a given round of the game, the players agree on the value there.
\end{itemize}}%
\only<3->{%
\begin{enumerate}
    \item<3-> We can show algebraically that the products in the \ColorOne{rows} are~$+1$ and the products in the \ColorTwo{columns} are~$-1$.
    \begin{itemize}
        \item This results at the end are correct.
        \item But no square-by-square account is given.
    \end{itemize}
    \item<4-> Based on a \ColorOne{row} (or \ColorTwo{column}), one two-qubit measurement is made in a prescribed basis.  That result determines what is reported in each square in that \ColorOne{row} (or \ColorTwo{column}).
    \item<5-> \ColorOne{Alice} and \ColorTwo{Bob} make and report a measurement in each square of their \ColorOne{row} or \ColorTwo{column}.
\end{enumerate}}
}{%
\begin{center}
\begin{MPSquareEnv}
    \MPSoln{}
\end{MPSquareEnv}
\end{center}
}
\end{frame}

\begin{frame}{Algebraic proof}{A sequence of measurements would culminate in the correct result}

Recall
\[ (\TensProd{A}{D}) \times (\TensProd{B}{E}) = \TensProd{(A\cdot B)}{(D\cdot E)} \]
and this extends to three product terms:
\[
(\TensProd{A}{D}) \times (\TensProd{B}{E}) \times (\TensProd{C}{F}) = \TensProd{(A\cdot B\cdot C)}{(D\cdot E\cdot F)}
\]
We consider each \ColorOne{row} and \ColorTwo{column} of the solution and show that
\begin{itemize}
    \item The product matrix of each row is \Identity, whose measurement always yields $+1$ regardless of input state.
    \item The product matrix of each column is $-\Identity$ (or \Not{\Identity}), whose measurement always yields $-1$ regardless of input state.
\end{itemize}
When operators are applied left-to-right in a circuit, their associated matrices are multipled from right-to-left.

\end{frame}

\begin{frame}{Rows}{Algebraic proof}
\Vskip{-3em}\TwoUnequalColumns{0.68\textwidth}{0.32\textwidth}{%
\begin{Reasoning}
\Reason{1}{Row 1}
\Reason{2}{Row 2}
\Reason{3}{Row 3}
\end{Reasoning}
\begin{align*}
(\ColorThree{\TensProd{\PauliZ}{\PauliZ}})\cdot (\ColorFour{\TensProd{\Identity}{\PauliZ}})\cdot (\ColorFive{\TensProd{\PauliZ}{\Identity}}) =& \TensProd{\ColorThree{\PauliZ}\ColorFour{\Identity}\ColorFive{\PauliZ}}{\ColorThree{\PauliZ}\ColorFour{\PauliZ}\ColorFive{\Identity}} \\
=& \TensProd{\Identity}{\Identity} \\
\visible<2->{(\ColorThree{\TensProd{\PauliX}{\PauliX}})\cdot (\ColorFour{\TensProd{\PauliX}{\Identity}})\cdot (\ColorFive{\TensProd{\Identity}{\PauliX}}) =& \TensProd{\ColorThree{\PauliX}\ColorFour{\PauliX}\ColorFive{\Identity}}{\ColorThree{\PauliX}\ColorFour{\Identity}\ColorFive{\PauliX}} \\
=& \TensProd{\Identity}{\Identity}} \\
\visible<3->{(\ColorThree{\TensProd{\PauliY}{\PauliY}})\cdot (\alert{\Not{\ColorFour{\TensProd{\PauliX}{\PauliZ}}}})\cdot (\alert{\Not{\ColorFive{\TensProd{\PauliZ}{\PauliX}}}}) =& \alert{- -} (\TensProd{\ColorThree{\PauliY}\ColorFour{\PauliX}\ColorFive{\PauliZ}}{\ColorThree{\PauliY}\ColorFour{\PauliZ}\ColorFive{\PauliX}}) \\
=& \TensProd{\SQBG{\relax}{-\NiceI}{0}{0}{-\NiceI}}{\SQBG{\relax}{\NiceI}{0}{0}{\NiceI}} \\
=& \TensProd{\Identity}{\Identity}}
\end{align*}
}{%
\begin{center}
\adjustbox{scale=0.7}{\begin{MPSquareEnv}
    \def\TpRt####1{\textcolor<1>{\RCthree}{####1}}%
    \def\TpCt####1{\textcolor<1>{\RCfour}{####1}}%
    \def\TpLf####1{\textcolor<1>{\RCfive}{####1}}%
    \def\MdRt####1{\textcolor<2>{\RCthree}{####1}}%
    \def\MdCt####1{\textcolor<2>{\RCfour}{####1}}%
    \def\MdLf####1{\textcolor<2>{\RCfive}{####1}}%
    \def\BtRt####1{\textcolor<3>{\RCthree}{####1}}%
    \def\BtCt####1{\textcolor<3>{\RCfour}{####1}}%
    \def\BtLf####1{\textcolor<3>{\RCfive}{####1}}%
    \MPSoln{}
\end{MPSquareEnv}}
\end{center}
}
\end{frame}

\begin{frame}{Columns}{Algebraic proof}
\Vskip{-3em}\TwoUnequalColumns{0.71\textwidth}{0.29\textwidth}{%
\begin{Reasoning}
\Reason{1}{Column 1}
\Reason{2}{Column 2}
\Reason{3}{Column 3}
\end{Reasoning}
\begin{align*}
(\alert{\Not{\ColorThree{\TensProd{\PauliZ}{\PauliX}}}})\cdot (\ColorFour{\TensProd{\Identity}{\PauliX}})\cdot (\ColorFive{\TensProd{\PauliZ}{\Identity}}) =& \alert{-} \TensProd{\ColorThree{\PauliZ}\ColorFour{\Identity}\ColorFive{\PauliZ}}{\ColorThree{\PauliX}\ColorFour{\PauliX}\ColorFive{\Identity}} \\
=& \alert{-}(\TensProd{\Identity}{\Identity}) \\
\visible<2->{(\alert{\Not{\ColorThree{\TensProd{\PauliX}{\PauliZ}}}})\cdot (\ColorFour{\TensProd{\PauliX}{\Identity}})\cdot (\ColorFive{\TensProd{\Identity}{\PauliZ}}) =& \alert{-}\TensProd{\ColorThree{\PauliX}\ColorFour{\PauliX}\ColorFive{\Identity}}{\ColorThree{\PauliZ}\ColorFour{\Identity}\ColorFive{\PauliZ}} \\
=& \alert{-}(\TensProd{\Identity}{\Identity})} \\
\visible<3->{(\ColorThree{\TensProd{\PauliY}{\PauliY}})\cdot (\ColorFour{\TensProd{\PauliX}{\PauliX}})\cdot (\ColorFive{\TensProd{\PauliZ}{\PauliZ}}) =& (\TensProd{\ColorThree{\PauliY}\ColorFour{\PauliX}\ColorFive{\PauliZ}}{\ColorThree{\PauliY}\ColorFour{\PauliX}\ColorFive{\PauliZ}}) \\
=& \TensProd{\SQBG{\relax}{-\NiceI}{0}{0}{-\NiceI}}{\SQBG{\relax}{-\NiceI}{0}{0}{-\NiceI}} \\
=& -(\TensProd{\Identity}{\Identity})}
\end{align*}
}{%
\begin{center}
\adjustbox{scale=0.6}{\begin{MPSquareEnv}
    \def\BtLf####1{\textcolor<1>{\RCthree}{####1}}%
    \def\MdLf####1{\textcolor<1>{\RCfour}{####1}}%
    \def\TpLf####1{\textcolor<1>{\RCfive}{####1}}%
    \def\BtCt####1{\textcolor<2>{\RCthree}{####1}}%
    \def\MdCt####1{\textcolor<2>{\RCfour}{####1}}%
    \def\TpCt####1{\textcolor<2>{\RCfive}{####1}}%
    \def\BtRt####1{\textcolor<3>{\RCthree}{####1}}%
    \def\MdRt####1{\textcolor<3>{\RCfour}{####1}}%
    \def\TpRt####1{\textcolor<3>{\RCfive}{####1}}%
    \MPSoln{}
\end{MPSquareEnv}}
\end{center}
}
\end{frame}

\begin{frame}{Summary}{Up to this point}
\begin{itemize}[<+->]
    \item The results we have obtained hold for any initial state.
    \item For \Alice{}, the product of measurements along any row is~$+1$.
    \item For \Bob{}, the product of measurements along any column is~$-1$.
    \item While the final results are confirmed, the proof is not instructive concerning the three measurements each player is supposed to report.
    \item This is akin to deciding not to play the game at all, because they know they should win, if only they knew how to play the game.
\end{itemize}
    
\end{frame}

\begin{frame}{Actually playing the game}{Making only one measurement}

\Vskip{-3em}\TwoUnequalColumns{0.6\textwidth}{0.4\textwidth}{%
Idea based on analysis from \href{http://philsci-archive.pitt.edu/18398/}{this paper}:
\only<1-5>{%
\begin{itemize}
    \item<1-> \Alice{} measures her two qubits \emph{once}, in a basis prescribed for her row.
    \item<2-> State \QState{} after measurement is an eigenstate for each operator in her row.
    \item<3-> However, each operator may have a \emph{different eigenvalue} associated with \QState{}.
    \item<4-> Alice reports the eigenvalue for each operator in her row, knowing their product is~$+1$.
    \item<5-> \Bob{} uses a different basis prescribed for his column, with his eigenvalues multiplying to~$-1$.
\end{itemize}}
\only<6-10>{%
\begin{itemize}
    \item<6-> \Alice{} knows the operators of her rows.
    \item<7-> She uses \texttt{MATLAB} to compute a row's operators' common eigenbasis \NamedBasis{B} \Set{\QState{0},\QState{1},\QState{2},\QState{3}}.
    \item<8-> The matrix (gate) $T$ that transforms the standard basis to \NamedBasis{B} has those states as its columns.
    \item<9-> The gate \Conj{T} transforms from \NamedBasis{B} to the standard basis.
    \item<10-> Because we perform only one measurement, we need only \Conj{T} followed by a measurement in the standard basis.
\end{itemize}}%
\only<11-13>{%
\begin{itemize}
    \item<11-> For each row, \Alice{} has a table that provides the eigenvalue for each square.
    \item<12-> The algebraic proof guarantees that those will multiply to~$+1$.
    \item<13-> The proof holds for \emph{any} initial state, but measurement in \NamedBasis{B} causes \Alice{} to report the same result as \Bob{} in their shared square.
    \end{itemize}}
    \only<14->{%
    \begin{itemize}
    \item<14-> Recall that each of \Alice's qubits is entangled with one of \Bob's qubits.  Each such pair is in the state \TwoSup{00}{11}.
    \item<15-> As a example where careless measurements cause problems, suppose
    \begin{itemize}
        \item \Alice{} has row 1 and measures her first qubit in the standard (\PauliZ) basis.
        \item \Bob{} has column 1 but measures his first qubit (thanks to Alice, now in state~\QZero{} or~\QOne{}) in the~\PauliX{} basis.
    \end{itemize}
    \item<16->\Bob{} is equally likely to see \QZero{} or \QOne{}, which could contradict \Alice's measurement.
\end{itemize}}
}{%
\begin{center}
\begin{MPSquareEnv}
    \def\TpLf####1{\textcolor<14-16>{\RCfive}{####1}}
    \MPSoln{}
\end{MPSquareEnv}
\end{center}
}
\end{frame}

\begin{frame}{The details}{How to interpret the upcoming slides}
%%
%% beamer bug is that a slide 3 is generated
%%
\only<1-2>{%
\Vskip{-3em}\TwoUnequalColumns{0.6\textwidth}{0.4\textwidth}{%
\begin{MPbasis}{Op1}{Op2}{Op3}
\def\A{\Col{\cdot}{\cdot}{\cdot}{\cdot}}
\def\B{\A}
\def\C{\A}
\def\D{\A}
\def\EigA{\Row{ }{ }{ }}
\def\EigB{\EigA}
\def\EigC{\EigA}
\def\EigD{\EigA}
\only<1>{\Bases{\relax}}
\only<2>{\Vskip{-3em}\Eigs}
    
\end{MPbasis}
\only<1>{%
\begin{itemize}
    \item The matrix $T$ is formed from the common eigenbasis of three operators in a given \ColorOne{row} or \ColorTwo{column}.
    \item We can verify that $T$ is unitary and that each column is an eigenvector for each of the three operators.
\end{itemize}}%
\only<2>{%
\begin{itemize}
    \item A table of eigenvalues is given for each of the three operators and each of the four basis states.
    \item We can verify the entries are correct.
    \item The entries for a given basis state multiply to~$+1$ for \Alice{} and to~$-1$ for \Bob. 
\end{itemize}
}
}{%
\begin{center}
\begin{MPSquareEnv}
    \MPSoln{}
\end{MPSquareEnv}
\end{center}
}
}
\end{frame}

\begin{frame}{Alice}{Row 1, using the \TensProd{\PauliZ}{\PauliZ} basis}
\TwoUnequalColumns{0.63\textwidth}{0.37\textwidth}{%
\begin{MPbasis}{\TensProd{\PauliZ}{\Identity}}{\TensProd{\Identity}{\PauliZ}}{\TensProd{\PauliZ}{\PauliZ}}
\def\A{\Col{1}{0}{0}{0}}
\def\B{\Col{0}{1}{0}{0}}
\def\C{\Col{0}{0}{1}{0}}
\def\D{\Col{0}{0}{0}{1}}
\def\EigA{\Row{\PP}{\PP}{\PP}}
\def\EigB{\Row{\PP}{\MM}{\PP}}
\def\EigC{\Row{\MM}{\PP}{\MM}}
\def\EigD{\Row{\MM}{\MM}{\PP}}
\only<1-5>{\Bases{}}

\Eigs
\end{MPbasis}%
\only<6->{%

\begin{itemize}
    \item If \Alice{} measures binary value $b$ then she uses the table, row~$b$, to report her results.
    \item For example, if \Alice{} measures \alert{\ket{10}}, then she reports the results in row \ColorFive{\QState{2}} for the three squares, left to right.  And of course, their product is $+1$.
\end{itemize}

}
}{%
\Vskip{-2em}
\begin{center}
\adjustbox{scale=0.8}{\MPCircuit}
\end{center}
\begin{center}
\adjustbox{scale=0.7}{\begin{MPSquareEnv}
    \def\TpLf####1{\ColorOne{####1}}
    \def\TpCt####1{\TpLf{####1}}
    \def\TpRt####1{\TpLf{####1}}
    \MPSoln{}
\end{MPSquareEnv}}
\end{center}}
    
\end{frame}

\begin{frame}{Alice}{Row 2, using the \TensProd{\PauliX}{\PauliX} basis}
\TwoUnequalColumns{0.63\textwidth}{0.37\textwidth}{%
\Vskip{-2em}\begin{MPbasis}{\TensProd{\Identity}{\PauliX}}{\TensProd{\PauliX}{\Identity}}{\TensProd{\PauliX}{\PauliX}}
\def\A{\Col{1}{1}{1}{1}}
\def\B{\Col{1}{-1}{1}{-1}}
\def\C{\Col{1}{1}{-1}{-1}}
\def\D{\Col{1}{-1}{-1}{1}}
\def\EigA{\Row{\PP}{\PP}{\PP}}
\def\EigB{\Row{\MM}{\PP}{\MM}}
\def\EigC{\Row{\PP}{\MM}{\MM}}
\def\EigD{\Row{\MM}{\MM}{\PP}}
\def\AltA{\ket{++}}
\def\AltB{\ket{+-}}
\def\AltC{\ket{-+}}
\def\AltD{\ket{--}}
\Bases{\frac{1}{2}}

\only<1-5>{\Eigs}
\end{MPbasis}
\only<6->{%
\begin{itemize}
    \item Note $T=\TensProd{\Hadamard}{\Hadamard}$ and $T=\Conj{T}$.
    \item This is as expected to map $(\TensProd{\PauliZ}{\PauliZ})\mapsto(\TensProd{\PauliX}{\PauliX})$
\end{itemize}
}
}{%
\Vskip{-2em}
\begin{center}
\adjustbox{scale=0.8}{\MPCircuit}
\end{center}
\begin{center}\adjustbox{scale=0.7}{%
\begin{MPSquareEnv}
    \def\MdLf####1{\ColorOne{####1}}
    \def\MdCt####1{\MdLf{####1}}
    \def\MdRt####1{\MdLf{####1}}
    \MPSoln{}
\end{MPSquareEnv}}
\end{center}}
    
\end{frame}

\begin{frame}{Alice}{Row 3, using an entangled basis}
\TwoUnequalColumns{0.63\textwidth}{0.37\textwidth}{%
\begin{MPbasis}{\Not{\TensProd{\PauliZ}{\PauliX}}}{\Not{\TensProd{\PauliX}{\PauliZ}}}{\TensProd{\PauliY}{\PauliY}}
\def\A{\Col{1}{1}{1}{-1}}
\def\B{\Col{1}{1}{-1}{1}}
\def\C{\Col{1}{-1}{1}{1}}
\def\D{\Col{-1}{1}{1}{1}}
\def\EigA{\Row{\MM}{\MM}{\PP}}
\def\EigB{\Row{\MM}{\PP}{\MM}}
\def\EigC{\Row{\PP}{\MM}{\MM}}
\def\EigD{\Row{\PP}{\PP}{\PP}}
\Bases{\frac{1}{2}}

\only<1-5>{\Eigs}
\end{MPbasis}
\only<6->{%

The columns of $T$ can also be expressed as:
\begin{center}\begin{tabular}{cc}
\ColorThree{\QState{0}} & \TwoSup{0+}{1-} \\
\ColorFour{\QState{1}} & \TwoSupOp{\ket{0+}}{\ket{1-}}{-} \\
\ColorFive{\QState{2}} & \TwoSup{1+}{0-} \\
\ColorSix{\QState{3}} & \TwoSupOp{\ket{1+}}{\ket{0-}}{-}
\end{tabular}\end{center}
}
}{%
\Vskip{-2em}
\begin{center}
\adjustbox{scale=0.8}{\MPCircuit}
\end{center}
\begin{center}\adjustbox{scale=0.7}{%
\begin{MPSquareEnv}
    \def\BtLf####1{\ColorOne{####1}}
    \def\BtCt####1{\BtLf{####1}}
    \def\BtRt####1{\BtLf{####1}}
    \MPSoln{}
\end{MPSquareEnv}}
\end{center}}
    
\end{frame}

\begin{frame}{Bob}{Column 1, using basis \TensProd{\PauliZ}{\PauliX}}
\TwoUnequalColumns{0.63\textwidth}{0.37\textwidth}{%
\begin{MPbasis}{\TensProd{\PauliZ}{\Identity}}{\TensProd{\Identity}{\PauliX}}{\Not{\TensProd{\PauliZ}{\PauliX}}}
\def\A{\Col{1}{1}{0}{0}}
\def\B{\Col{1}{-1}{0}{0}}
\def\C{\Col{0}{0}{1}{1}}
\def\D{\Col{0}{0}{1}{-1}}
\def\EigA{\Row{\PP}{\PP}{\MM}}
\def\EigB{\Row{\PP}{\MM}{\PP}}
\def\EigC{\Row{\MM}{\PP}{\PP}}
\def\EigD{\Row{\MM}{\MM}{\MM}}
\def\AltA{\ket{0+}}
\def\AltB{\ket{0-}}
\def\AltC{\ket{1+}}
\def\AltD{\ket{1-}}
\Bases{\RootTwo{}}

\only<1-5>{\Eigs}
\end{MPbasis}
\only<6->{%

\begin{itemize}
    \item Note $T=\TensProd{\Identity}{\Hadamard}$ and $T=\Conj{T}$.
    \item This is as expected to map $(\TensProd{\PauliZ}{\PauliZ})\mapsto(\TensProd{\PauliZ}{\PauliX})$
\end{itemize}
}
}{%
\Vskip{-2em}
\begin{center}
\adjustbox{scale=0.8}{\MPCircuit}
\end{center}
\begin{center}\adjustbox{scale=0.7}{%
\begin{MPSquareEnv}
    \def\TpLf####1{\ColorTwo{####1}}
    \def\MdLf####1{\TpLf{####1}}
    \def\BtLf####1{\TpLf{####1}}
    \MPSoln{}
\end{MPSquareEnv}}
\end{center}}
    
\end{frame}

\begin{frame}{Bob}{Column 2, using basis \TensProd{\PauliZ}{\PauliX}}
\TwoUnequalColumns{0.63\textwidth}{0.37\textwidth}{%
\begin{MPbasis}{\TensProd{\PauliZ}{\Identity}}{\TensProd{\Identity}{\PauliX}}{\Not{\TensProd{\PauliZ}{\PauliX}}}
\def\A{\Col{1}{0}{1}{0}}
\def\B{\Col{0}{1}{0}{1}}
\def\C{\Col{1}{0}{-1}{0}}
\def\D{\Col{0}{1}{0}{-1}}
\def\EigA{\Row{\PP}{\PP}{\MM}}
\def\EigB{\Row{\MM}{\PP}{\PP}}
\def\EigC{\Row{\PP}{\MM}{\PP}}
\def\EigD{\Row{\MM}{\MM}{\MM}}
\def\AltA{\ket{+0}}
\def\AltB{\ket{+1}}
\def\AltC{\ket{-0}}
\def\AltD{\ket{-1}}
\Bases{\RootTwo{}}

\only<1-5>{\Eigs}
\end{MPbasis}
\only<6->{%

\begin{itemize}
    \item Note $T=\TensProd{\Hadamard}{\Identity}$ and $T=\Conj{T}$.
    \item This is as expected to map $(\TensProd{\PauliZ}{\PauliZ})\mapsto(\TensProd{\PauliX}{\PauliZ})$
\end{itemize}
}
}{%
\Vskip{-2em}
\begin{center}
\adjustbox{scale=0.8}{\MPCircuit}
\end{center}
\begin{center}\adjustbox{scale=0.7}{%
\begin{MPSquareEnv}
    \def\TpCt####1{\ColorTwo{####1}}
    \def\MdCt####1{\TpCt{####1}}
    \def\BtCt####1{\TpCt{####1}}
    \MPSoln{}
\end{MPSquareEnv}}
\end{center}}
    
\end{frame}

\begin{frame}{Bob}{Column 3, using an entangled basis}
\TwoUnequalColumns{0.63\textwidth}{0.37\textwidth}{%
\begin{MPbasis}{\TensProd{\PauliZ}{\PauliX}}{\TensProd{\PauliX}{\PauliZ}}{\TensProd{\PauliY}{\PauliY}}
\def\A{\Col{1}{0}{0}{1}}
\def\B{\Col{1}{0}{0}{-1}}
\def\C{\Col{0}{1}{1}{0}}
\def\D{\Col{0}{1}{-1}{0}}
\def\EigA{\Row{\PP}{\PP}{\MM}}
\def\EigB{\Row{\PP}{\MM}{\PP}}
\def\EigC{\Row{\MM}{\PP}{\PP}}
\def\EigD{\Row{\MM}{\MM}{\MM}}
\Bases{\RootTwo{}}

\only<1-5>{\Eigs}
\end{MPbasis}
\only<6->{%

The columns of $T$ can also be expressed as:
\begin{center}\begin{tabular}{cc}
\ColorThree{\QState{0}} & \TwoSup{00}{11} \\
\ColorFour{\QState{1}} & \TwoSupOp{\ket{00}}{\ket{11}}{-} \\
\ColorFive{\QState{2}} & \TwoSup{01}{10} \\
\ColorSix{\QState{3}} & \TwoSupOp{\ket{01}}{\ket{10}}{-}
\end{tabular}\end{center}
}
}{%
\Vskip{-2em}
\begin{center}
\adjustbox{scale=0.8}{\MPCircuit}
\end{center}
\begin{center}\adjustbox{scale=0.7}{%
\begin{MPSquareEnv}
    \def\TpRt####1{\ColorTwo{####1}}
    \def\MdRt####1{\TpRt{####1}}
    \def\BtRt####1{\TpRt{####1}}
    \MPSoln{}
\end{MPSquareEnv}}
\end{center}}
    
\end{frame}
