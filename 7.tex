\SetTitle{7}{The Elitzur--Vaidman bomb}{Our first quantum algorithm}{evbomb}

\begin{frame}{Overview}

\begin{itemize}
    \item The \href{https://en.wikipedia.org/wiki/Elitzur-Vaidman_bomb_tester}{Elitzur--Vaidman} bomb is a \href{https://en.wikipedia.org/wiki/Thought_experiment}{Gedankenexperiment} in which a photon measured in a certain way triggers an explosion.
    \item We can devise a quantum algorithm that is arbitrarily likely to detect the presence of such a bomb without setting it off.
    \item This algorithm captures the flow of most quantum algorithms:
    \begin{itemize}
        \item Preparation of a known state
        \item Transformation of state by quantum gates
        \item Iteration by passes or unrolling of a circuit
        \item Measurement to obtain the result
        \item Probabilistic analysis for correctness
    \end{itemize}
    \item The algorithm uses an interferometer setup like the one we have studied.
\end{itemize}
    
\end{frame}

\begin{frame}{The classical EV bomb}{Based on our previous interferometer}

\TwoUnequalColumns{0.5\hsize}{0.5\hsize}{%
\begin{itemize}
    \only<1>{%
    \item Recall that the wave of a photon reflected twice in this diagram is shifted by $\pi$ radians.
    \item The resulting interference cancels any chance of measuring the photon at the
    \textcolor{NavyBlue}{blue} device.}
    \only<2>{%
    \item If we obstruct the bottom path as shown here, the interference cannot occur.  There are thus three possibilities.
    \begin{itemize}
        \item The photon is measured at the obstruction, which will occur $1/2$ the time.
        \item The photon is measured after passing through the first beam splitter.  From there it reaches either the \textcolor{orange}{orange} device or the \textcolor{NavyBlue}{blue} device.
    \end{itemize}
    }
    \only<3-5>{%
    \item Suppose the obstruction is a photon-activated bomb.  
    \item<4-> If the photon is measured by the bomb, then it explodes, and the world as we know it ceases to exist.}
    \only<5->{
    \item With the bomb present:
    \begin{center}
    \begin{tabular}{lc}
       What & How often? \\\hline
       Explosion & 50\% \\
       \textcolor{orange}{Orange device} & 25\% \\
       \textcolor{NavyBlue}{Blue device} & 25\%
     \end{tabular}
     \end{center}
     \item<6-> With the bomb absent:
    \begin{center}
    \begin{tabular}{lc}
       What & How often? \\\hline
       Explosion & 0\% \\
       \textcolor{orange}{Orange device} & 100\% \\
       \textcolor{NavyBlue}{Blue device} & 0\%
     \end{tabular}
     \end{center}
    }
\end{itemize}
}{%}
\Vskip{-2em}\begin{TIKZP}[scale=0.7]
\MZ{}
\draw[color=red] (1,0.5) -- (2.5,0.5);
\draw[color=purple] (2.5,0.5) -- ++(4,0) -- ++(0,-2);
\only<1>{\draw[color=OliveGreen] (2.5,0.5) -- ++(0,-2) -- ++(4,0);}
\only<2->{\draw[color=OliveGreen] (2.5,0.5) -- ++(0,-2) -- ++(2,0);}
\only<4>{\draw[ultra thick,color=OliveGreen] (2.5,0.5) -- ++(0,-2) -- ++(2,0);}
\draw[->,color=orange] (6.5,-1.5) -- ++(1.5,0);
\draw<2->[->,color=NavyBlue] (6.5,-1.5) -- ++(0,-1.5);
\visible<2-3,5->{\Shift{4}{-2}{\EVBomb{}}}
\visible<3->{\draw[white] (4.5,-1.5) node {{\tiny\bf Bomb!}};}
\visible<4>{\Shift{4}{-2}{\EVBoom{}}}
\visible<6>{%
    \fill[gray] (2,-2) rectangle ++(5,3);
    \draw[white] (4.5,-0.5) node {????};
}
\end{TIKZP}%
\MedSkip{}
\only<2>{Even if the photon isn't measured at the obstruction, its presence prevents interference.}
\only<5>{Recall that without the obstruction, all photons are measured only at the \textcolor{orange}{orange device}.}
\only<6>{%
This is a bomb-detection device.  If we ever see a photon at the \textcolor{NavyBlue}{blue sensor} we know the bomb is there, and we didn't explode.  Detection at the \textcolor{orange}{orange sensor} tells us nothing. Oh, and half the time we die if the bomb is there.
}
}%
\end{frame}


