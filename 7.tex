\SetTitle{7}{The Elitzur--Vaidman bomb}{Our first quantum algorithm}{evbomb}

\begin{frame}{Overview}

\begin{itemize}
    \item The \href{https://en.wikipedia.org/wiki/Elitzur-Vaidman_bomb_tester}{Elitzur--Vaidman} bomb is a \href{https://en.wikipedia.org/wiki/Thought_experiment}{Gedankenexperiment} in which a photon measured in a certain way triggers an explosion.
    \item We can devise a quantum algorithm that is arbitrarily likely to detect the presence of such a bomb without setting it off.
    \item This algorithm captures the flow of most quantum algorithms:
    \begin{itemize}
        \item Preparation of a known state
        \item Transformation of state by quantum gates
        \item Iteration by passes or unrolling of a circuit
        \item Measurement to obtain the result
        \item Probabilistic analysis for correctness
    \end{itemize}
    \item The algorithm uses an interferometer setup like the one we have studied.
\end{itemize}
    
\end{frame}

\begin{frame}{The classical EV bomb}{Based on our previous interferometer}

\Vskip{-3em}\TwoUnequalColumns{0.5\hsize}{0.5\hsize}{%
\Vskip{-3em}\begin{itemize}
    \only<1>{%
    \item Recall that the wave of a photon reflected twice in this diagram is shifted by $\pi$ radians.
    \item The resulting interference cancels any chance of measuring the photon at the
    \textcolor{NavyBlue}{blue} device.}
    \only<2>{%
    \item If we obstruct the bottom path as shown here, the interference cannot occur.  There are thus three possibilities.
    \begin{itemize}
        \item The photon is measured at the obstruction, which will occur $1/2$ the time.
        \item The photon is measured after passing through the first beam splitter.  From there it reaches either the \textcolor{orange}{orange} device or the \textcolor{NavyBlue}{blue} device.
    \end{itemize}
    }
    \only<3-5>{%
    \item Suppose the obstruction is a photon-activated bomb.  
    \item<4-> If the photon is measured by the bomb, then it explodes, and the world as we know it ceases to exist.}
    \only<5->{
    \item With the bomb present:
    \begin{center}
    \begin{tabular}{lc}
       What & How often? \\\hline
       Explosion & 50\% \\
       \textcolor{orange}{Orange device} & 25\% \\
       \textcolor{NavyBlue}{Blue device} & 25\%
     \end{tabular}
     \end{center}
     \item<6-> With the bomb absent:
    \begin{center}
    \begin{tabular}{lc}
       What & How often? \\\hline
       Explosion & 0\% \\
       \textcolor{orange}{Orange device} & 100\% \\
       \textcolor{NavyBlue}{Blue device} & 0\%
     \end{tabular}
     \end{center}
     \item<6-> This is a bomb detection device!
    }
\end{itemize}
}{%}
\begin{TIKZP}[scale=0.7]
\MZ{}
\draw[color=red] (1,0.5) -- (2.5,0.5);
\draw[color=purple] (2.5,0.5) -- ++(4,0) -- ++(0,-2);
\only<1>{\draw[color=OliveGreen] (2.5,0.5) -- ++(0,-2) -- ++(4,0);}
\only<2->{\draw[color=OliveGreen] (2.5,0.5) -- ++(0,-2) -- ++(2,0);}
\only<4>{\draw[ultra thick,color=OliveGreen] (2.5,0.5) -- ++(0,-2) -- ++(2,0);}
\draw[->,color=orange] (6.5,-1.5) -- ++(1.5,0);
\draw<2->[->,color=NavyBlue] (6.5,-1.5) -- ++(0,-1.5);
\visible<2-3,5->{\Shift{4}{-2}{\EVBomb{}}}
\visible<3->{\draw[white] (4.5,-1.5) node {{\tiny\bf Bomb!}};}
\visible<4>{\Shift{4}{-2}{\EVBoom{}}}
\visible<6->{%
    \fill[gray] (2,-2) rectangle ++(5,3);
    \draw[white] (4.5,-0.5) node {\bf Bomb present????};
}
\end{TIKZP}%
\MedSkip{}
\only<2>{Even if the photon isn't measured at the obstruction, its presence prevents interference.}
\only<5>{Recall that without the obstruction, all photons are measured only at the \textcolor{orange}{orange device}.}
\only<6->{%
If we ever see a photon at the \textcolor{NavyBlue}{blue sensor} we know the bomb is there, and we live to tell the tale.}
\only<7->{Detection at the \textcolor{orange}{orange sensor} tells us nothing.} \only<8->{\alert{Oh, and half the time we die if the bomb is there.}}
}%
\end{frame}

\begin{frame}{How good is this approach?}{Some analysis}

\begin{itemize}
    \item If the bomb is there, we are successful 25\% of the time in finding that it is there without blowing up.
    \item If the bomb is not there, we do not gain anything  useful information because all photons are measured by the \textcolor{orange}{orange sensor}.
    \item After many tries, we can claim with some probability that there is no bomb present.  The probability that we measure $n$ photons at the \textcolor{orange}{orange device} with a bomb present is $\left(\frac{1}{4}\right)^{n}$.
    \item Our belief that there is no bomb thus grows exponentially in the number of photons that continue to be observed at the \textcolor{orange}{orange sensor}.
\end{itemize}
    
\end{frame}

\begin{frame}{Troubling questions}{Quantum is strange}
\end{frame}

\begin{frame}{A more careful approach}{An algorithm with arbitrarily high likelihood of detecting the bomb with no explosion}
\TwoUnequalColumns{0.45\textwidth}{0.55\textwidth}{%
\begin{itemize}
    \item<1-> The bomb is attached to a polarizing filter.
    \item<2-> If the photon goes through the filter, there is no energy transferred to our bomb apparatus, and so there is no explosion.
    \item<3-> If the photon is stopped by the filter, then the energy transferred from the photon to the filter causes an explosion.
\end{itemize}
}{%
\begin{TIKZP}[scale=0.8]
%%
%% Bounding box
\draw[white] (0,1) rectangle ++(6,-4);
%%
\visible<2->{\LightSource{}}
\draw<2>[color=red] (1,0.5) -- ++(5,0);
\draw<3->[color=red] (1,0.5) -- ++(3,0);
%\draw[color=purple] (2.5,0.5) -- ++(4,0) -- ++(0,-2);
\Shift{4}{-2}{\EVBomb{}}
\draw[white] (4.5,-1.5) node {{\tiny\bf Bomb!}};
\Shift{4}{0}{\RotateAroundCenter{90}{\PFilter{0}{0}{1}{1}{10}}}
\draw[color=brown] (4.5,0) -- ++(0,-1);
\draw<3>[ultra thick,color=brown] (4.5,0) -- ++(0,-1);
\visible<3>{\Shift{4}{-2}{\EVBoom{}}}
\end{TIKZP}

\only<4>{%
If we send unpolarized light, then we expect half the photons to trigger an explosion.
}
}

\end{frame}
