\documentclass[12pt]{article}
\usepackage{amssymb,mathrsfs, amsmath,amsfonts}
\usepackage{mathtools}
\usepackage{graphicx}
\usepackage{enumitem}
\usepackage{braket}

\title{Q2 Grading Guide / Rubric}
\author{CSE 468}
\date{\today}

\begin{document}

\maketitle

\noindent \textbf{Autograded}: Tasks 1, 2, 3, 4A (85 points) 

\noindent \textbf{Manual}: 4B (15 points)

\begin{enumerate}[font=\bfseries]
    \item (Auto, 10)
    \begin{itemize}
        \item 10 - correct state is returned and gates are not added to or modified in the circuit
    \end{itemize}
    \item (Auto, 30) $2\times15$
    \begin{itemize}
        \item 1 - measurements are made
        \item 14 - correct state is achieved without using initialization
    \end{itemize}
    \item (Auto, 30) $2\times15$
    The rules checker must pass for a solution to receive any credit for this question.
    \begin{itemize}
        \item 1 - solution passes rules checker
        \item 4 - consistent measurements are observed for each gate; points divided among tests according to how many possibilities the oracle can be
        \item 10 - consistent measurements are observed for each gate and they correspond to unique basis states
    \end{itemize}
    \item (Auto/Manual, 30) $2\times15$
    \begin{itemize}
        \item 3 - solution works for $
        0\rangle$
        \item 3 - solution works for $
        1\rangle$
        \item 9 - solution works for complex $
        \Psi\rangle$
        \item 5 - (manual) attempt is made with correct U gate parameters
        \item 10 - (manual) solution is compatible with provided code and histogram shows desired ratio
    \end{itemize}
\end{enumerate}


\end{document}