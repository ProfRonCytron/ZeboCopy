\documentclass[12pt]{article}
\usepackage{amssymb,mathrsfs, amsmath,amsfonts}
\usepackage{mathtools}
\usepackage{graphicx}
\usepackage{enumitem}
\usepackage{braket}
\setcounter{secnumdepth}{0}


\usepackage{color}  
\usepackage{hyperref}
\hypersetup{
    colorlinks=true,
    linktoc=all,
    linkcolor=blue,
}
\setlength{\parindent}{0pt} 

\title{Qiskit Environment Setup Guide}
\author{CSE 468}
\date{\today}

\begin{document}

\maketitle
\tableofcontents
\newpage
\section{Python \& Miniconda}

Run \texttt{python --version} to check whether you have Python 3.7.x installed and \texttt{conda --version} to check whether conda is installed. If both requirements are satisfied, you can skip to the next section. Otherwise, use the links below to install them. Choose to ``Add to Path" where prompted. \newline

\textbf{Windows (64 bit)}:
\href{https://repo.anaconda.com/miniconda/Miniconda3-py37_4.9.2-Windows-x86_64.exe}{https://repo.anaconda.com/miniconda/Miniconda3-py37\_4.9.2-Windows-x86\_64.exe}\newline

\textbf{Windows (32 bit)}: \href{https://repo.anaconda.com/miniconda/Miniconda3-py37_4.9.2-Windows-x86.exe}{https://repo.anaconda.com/miniconda/Miniconda3-py37\_4.9.2-Windows-x86.exe}\newline

\textbf{MacOS}:
\href{https://repo.anaconda.com/miniconda/Miniconda3-py37_4.9.2-MacOSX-x86_64.pkg}{https://repo.anaconda.com/miniconda/Miniconda3-py37\_4.9.2-MacOSX-x86\_64.pkg}\newline

\textbf{Linux}:
Linux users should find the link specific to their distro \href{https://repo.anaconda.com/miniconda/}{here}. The name should include ``py37\_4.9.2-Linux". \newline

Run \texttt{python --version} and \texttt{conda --version} again to ensure the installation was successful.


\section{Conda environment}
Using conda, we can create a Python environment which manages the necessary Python packages for this class. Here is a \href{https://docs.conda.io/projects/conda/en/latest/user-guide/getting-started.html}{useful guide} for getting started with conda. To set up the environment:
\begin{enumerate}
    \item Download the \href{FIXME}{cse468\_env\_setup.yml} file from Canvas.
    \item Run \texttt{conda env create -f cse468\_env\_setup.yml}
    \item Run \texttt{conda activate cse468}
\end{enumerate}

\textbf{\underline{Each time you work on a notebook assignment}}, you should run \\ \texttt{conda activate cse468}. For Windows, this should be done in ``Anaconda Prompt", an app installed by your conda distribution. Optionally, run \texttt{conda info --envs} and you should see ``cse468" listed with an asterisk next to it, indicating that it's the active environment. Proceed to your assignment by running \texttt{jupyter notebook}.

%% DW 1/11/2022: Pretty sure we don't require these anymore with the new version of otter but keeping these around just in case

% \section{TeX}
% Some diagrams in your assignments will require TeX to render properly. Follow directions \href{https://nbconvert.readthedocs.io/en/latest/install.html#installing-tex}{at this link} to download TeX for your specific OS. These packages are large so this process may take some time. Run \texttt{tex --version} to ensure the installation was successful. 

% \section{pandoc}
% pandoc will be used to generate PDFs for your final submissions.
% OS specific downloads are available \href{https://pandoc.org/installing.html}{from this link}. Run \texttt{pandoc --version} to ensure the installation was successful.

\section{Troubleshooting}
Sometimes the installation process seems to work but something is still broken. If no useful errors occurred while installing the software, there are a few things to try:
\begin{enumerate}
\item Close and then reopen your terminal. Try running the command again and see if the output is different.
\item Conda seems to be particularly finicky on Windows. If it isn't working in CMD or Powershell, try  ``Anaconda Prompt".
\item Add the program's root directory to your PATH manually.

On MacOS, run \texttt{sudo vim /etc/paths}, add a line with the root directory, save the file, and restart the terminal.

For Windows, search for ``Environment Variables". Under ``System Variables", find the PATH environment variable, select it, and click ``Edit" to add to the user variables. 

\item Google problems and try to work through them with your peers.
\item Come to office hours.

\end{enumerate}

If you are able to successfully troubleshoot an issue, please let the professor or a TA know the problem and solution so we can improve this guide for future semesters. Thanks!
\end{document}