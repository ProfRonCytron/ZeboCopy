%To-do
\begin{document}

Approximate time to complete exercises: 5-10 minutes.

\begin{enumerate}[font=\bfseries]
    \item Is $f(x) = x^3$ a reversible function?\newline
    \textbf{Yes.} $f(x)$ has the same number of inputs and ouputs.
    \item Is taking the maximum of some set a reversible function?\newline
    \textbf{No.} A set may contain many elements while the max function outputs a single number.
    \item Is SHA256 a reversible function? Read about SHA256 here: https://en.wikipedia.org/wiki/SHA-2\newline
    \textbf{No.} SHA256 uses a ``one-way compression function" which takes two inputs and returns a single output.
    \item What quadrant of the complex unit circle is $e^{i\frac{3\pi}{4}}$ in?\newline
    Recall that we can think of theta as the distance we traverse around the complex unit circle starting from $(1,0)$. By tracing this out on a unit circle, it can be seen that $e^{i\frac{3\pi}{4}}$ is in the \textbf{quadrant 2}.\newline
    
    Alternatively, we can use $e^{i\frac{3\pi}{4}} = \cos{\frac{3\pi}{4}}+i\sin{\frac{3\pi}{4}} = -\frac{1}{\sqrt{2}}+i\frac{1}{\sqrt{2}}$.
    The real component, $-\frac{1}{\sqrt{2}}$, tells us we're on the left side of the unit circle. The imaginary coefficient, $\frac{1}{\sqrt{2}}$, puts us on the top half. The intersection of these pieces of information yields quadrant 2.
    
    
\end{enumerate}



\end{document}