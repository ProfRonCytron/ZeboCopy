\documentclass[12pt]{article}
\usepackage{exercises/handwritten/handout}

\usepackage{amssymb,mathrsfs, amsmath,amsfonts}
\usepackage{mathtools}
\usepackage{graphicx}
\usepackage{enumitem}
\usepackage{braket}
\graphicspath{ {./ps110-assets/} }
\input{quantummacros.tex}
%%
%%

\newcommand{\NameBlank}{\mbox{\hskip 4pt\vrule width 2.5in depth 2pt}\vrule width 0pt height 2.0em}
\newcommand{\BlankLine}{\mbox{\hskip 4pt\vrule width 5.5in depth 2pt}\vrule width 0pt height 2.0em}
%%
%% Leave at least #1 space, default to what is below
%%


\begin{document}

\assignment{Problem Set 230}

\begin{quote}
    As for all written assignments, you must upload your completed work to GradeScope.

    You are welcome to work collaboratively on solving these problems, but you each must write or type out your own solution, and upload that to GradeScope.

    Work submitted improperly will receive no credit.
\end{quote}

\begin{quote}\bf
You are welcome to use Matlab to help solve these problems or to check your solutions.  Where you are asked to show your work, you can paste the Matlab output.
\end{quote}

\begin{enumerate}[font=\bfseries]
    \item\Points{25} Consider Deutsch's problem (Slide Deck~210), where $f(x)$ takes in only a single bit.
    \begin{enumerate}[label=\theenumi.\arabic*]
        \item Remembering that global phases can be ignored, what is the state of the \emph{bottom} qubit on output, which originally went unmeasured?
        \begin{itemize}
            \item In the balanced case  \Ans{$\ket{-}$}
            \item In the constant case
            \Ans{$\ket{-}$}
        \end{itemize}
        \item If that bottom qubit is measured in the computational basis, what are the possible outcome(s)? \Ans{\QZero{} or \QOne{}}
        \item If that bottom qubit is first passed through a Hadamard gate and then measured in the computational basis, what are the possible outcome(s)? \Ans{Only \QOne{}}
    \end{enumerate}
    \item\Points{25} Recall the phase-kickback effect covered in Slide Deck 220.  Recall that the oracle box accepts $x$ (of $n$ qubits) and $y$ (a single qubit), and produces on its output:
    \begin{itemize}
        \item $x$ copied from left to right taking the top $n$ qubit, and
        \item $\Xor{y}{f(x)}$ using the bottom qubit on which $y$ was presented.
    \end{itemize}
    \begin{enumerate}[label=\theenumi.\arabic*]
        \item Below, sketch the quantum circuit below that realizes an oracle with the function $f(x)$ as defined in this table:
        \begin{center}
            \begin{tabular}{c|c}
                 $\ket{x}$&$f(\ket{x})$  \\\hline
                 $\ket{00}$ & 0 \\
                 $\ket{01}$ & 1 \\
                 $\ket{10}$ & 1 \\
                 $\ket{11}$ & 1 \\
            \end{tabular}
        \end{center}
        Remember that the bottom qubit must be \Xor{y}{f(x)} and not just $f(x)$.
        The fewer gates you use while having a correct answer, the more credit you will receive.
        
        
        \Ans{\begin{center}
      \adjustbox{valign=t}{\begin{quantikz}
\qw&  \gate{\PauliX}& \ctrl{1} & \gate{\PauliX}& \qw \\
\qw &   \gate{\PauliX}    &  \ctrl{1} & \gate{\PauliX} & \qw  \\
\qw &   \gate{\PauliX}    & \targ{} & \qw & \qw \ 
\end{quantikz}}%
\end{center}
The above circuit prepares the output to be \QOne{}, for the 3 of the 4 cases.  But when the first top two qubits are both \QZero{}, the \NamedGate{CCNOT} gate is triggered to flip the bottom bit.  The \PauliX{} gates after the \NamedGate{CCNOT} restore (uncompute) the action on those qubits.\MedSkip{}Although I asked for the smallest circuit let's give credit for any correct answer.}
        \item Under the phase-kickback situation, where the qubits $x$ are in a uniform superposition of all possible values, and $\ket{-}$ is presented on the $y$ qubit, what is the output state of the entire circuit (including $y$)?  The simpler your correct answer, the more credit you will receive.
        \Ans{%
        \[
        \TensProd{\frac{\ket{00} - \ket{01} - \ket{10} -\ket{11}}{2}}{\ket{-}}
        \]
    }
    \end{enumerate}
     \item\Points{30} Consider the Deutsch--Jozsa problem, covered in Slide Deck 230. In that problem, we were \emph{promised} that functions were either constant or balanced. We will now investigate what happens when the promise is broken.  For an $n$-bit problem, recall the amplitude on the $\ket{0^{\star n}}$ term is:
     \[
       \frac{1}{2^n}\sum_{x\in \{0,1\}^{n}} -1^{(f(x))}
     \]
     So let's define
         \[
       k = \left|\frac{1}{2^n}\sum_{x\in \{0,1\}^{n}} -1^{(f(x))}\right|
     \]
     We then obtain:
     \begin{description}
         \item[k=0] Then $f(x)$ is balanced
         \item[k=1] Then $f(x)$ is constant
     \end{description}
     What about other values of $k$?
    \begin{enumerate}[label=\theenumi.\arabic*]
        \item Let's start with an example for $n = 2$ qubits:
        \begin{center}
            \begin{tabular}{c|c}
                 $\ket{x}$&$f(\ket{x})$  \\\hline
                 $\ket{00}$ & 0 \\
                 $\ket{01}$ & 1 \\
                 $\ket{10}$ & 1 \\
                 $\ket{11}$ & 1 \\
            \end{tabular}
        \end{center}
        What is $k$ for this example? \Ans{$\frac{1}{2}$, because of the absolute value}
        \item At state~$\psi_{3}$, what amplitude is present on
        \begin{itemize}
            \item $\ket{00}$ 
            \Ans{
            \[
            \frac{(1\times 1) + (-1\times 1) + (-1\times 1) +  (-1\times 1)}{4} = -\frac{1}{2}
            \]
            }
             \item $\ket{01}$
                         \Ans{
            \[
            \frac{(1\times 1) + (-1\times -1) + (-1\times 1) +  (-1\times -1)}{4} = \frac{1}{2}
            \]
            }
              \item $\ket{10}$
              \Ans{
            \[
            \frac{(1\times 1) + (-1\times 1) + (-1\times -1) +  (-1\times -1)}{4} = \frac{1}{2}
            \]
            }
               \item $\ket{11}$
               \Ans{
            \[
            \frac{(1\times 1) + (-1\times -1) + (-1\times -1) +  (-1\times 1)}{4} = \frac{1}{2}
            \]
            }
        \end{itemize}
        \item What is the probability of seeing $\ket{00}$ on the first measurement? \Ans{$\Prob{\frac{1}{2}} = \frac{1}{4}$}
        \item What is the probability of \emph{not} seeing $\ket{00}$ on the first measurement? \Ans{$1-\frac{1}{4}=\frac{3}{4}$}
        \item Suppose on your first measurement you observe $\ket{01}$.  So you believe the function is balanced.  For this running example, how many more runs of the circuit must you make to be 90\% certain that $f(x)$ is keeping its promise?   

        Show your work.
        
        \Ans{%
        After seeing $\ket{01}$, we avoid $\ket{00}$ with probability 3/4.   The probability of $k$ trials without seeing $\ket{00}$ is thus $\left(\frac{3}{4}\right)^{k}$.  We see $\ket{00}$ sometime in $k$ trials with probability $1-\left(\frac{3}{4}\right)^{k}$. At the 9th trial, we exceed 90\% probability of having seen $\ket{00}$ and caught the cheating function.}
        

    \end{enumerate}
    \item \Points{20} Please answer the following truthfully, as the responses will be viewed anonymously:
    \begin{enumerate}[label=\theenumi.\arabic*]
        \item Qualitatively speaking, how is the course going for you so far? \LeaveSpace{}
        \item On a scale where 1 is easiest and 5 is most difficult, how are you finding the material covered in this course?\Blank{}
        \item On average how much time are you spending on this course per week?\Blank{}
        \item Is that more or less than you spend on your other courses?\Blank{}
        \item What could the instructor do to improve your experience in this course and your understanding of the material?
        \LeaveSpace{}
    \end{enumerate}

\end{enumerate}
\end{document}