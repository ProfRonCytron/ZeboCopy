\documentclass[12pt]{article}
\usepackage{amssymb,mathrsfs, amsmath,amsfonts}
\usepackage{mathtools}
\usepackage{graphicx}
\usepackage{enumitem}
\usepackage{braket}
\graphicspath{ {./ps1-assets/}{./exercises/handwritten/ps1-assets/} }

\title{Problem Set 1 Solutions}
\author{CSE 468}
\date{\today}

\begin{document}

\maketitle

\noindent \textbf{Note:} Many of these problems use https://lab.quantumflytrap.com/lab.

\begin{enumerate}[font=\bfseries]
    \item \begin{enumerate}
        \item \[\includegraphics[scale=0.6]{easyFilter-sol}\]
        \item Use a sequence of k filters with the initial filter rotated 45 degrees relative to the photon source and then rotate each additional filter by another 45 degrees.
        \item \begin{pmatrix}
                0 & 0 \\
                0 & 1
                \end{pmatrix}
        \item $\sin^2{\theta}$
        \item \begin{pmatrix}
                \cos^2{\theta} & \cos{\theta}\sin{\theta} \\
                \sin{\theta}\cos{\theta} & \sin^2{\theta}
                \end{pmatrix}
        \item Begin with a photon oriented completely in the vertical direction so we have $\begin{pmatrix} 0 \\ 1 \end{pmatrix}$. We then apply a filter oriented 45 degrees ($\frac{\pi}{4}$ radians) from the x-axis. This is represented by the following operation:
        \[\begin{pmatrix}
                \cos^2{(\frac{\pi}{4})} & \cos{(\frac{\pi}{4})}\sin{(\frac{\pi}{4})} \\
                \sin{(\frac{\pi}{4})}\cos{(\frac{\pi}{4})} & \sin^2{(\frac{\pi}{4})}
                \end{pmatrix}
                \begin{pmatrix} 0 \\ 1 \end{pmatrix}
                = 
                \begin{pmatrix}
                \frac{1}{2} & \frac{1}{2} \\
                \frac{1}{2} & \frac{1}{2}
                \end{pmatrix}
                \begin{pmatrix} 0 \\ 1 \end{pmatrix}
                =
                \begin{pmatrix} \frac{1}{2} \\ \frac{1}{2} \end{pmatrix}
                \]
            Note the norm of this vector is $\frac{1}{2}$ (half of the light is absorbed by the filter). We next apply a vertically oriented filter and obtain
            \[\begin{pmatrix}
                0 & 0 \\
                0 & 1
                \end{pmatrix}
                \begin{pmatrix}
                \frac{1}{2} \\ \frac{1}{2}
                \end{pmatrix}
                =
                \begin{pmatrix} 0 \\ \frac{1}{2} \end{pmatrix}
                \]
            Continue in a similar manner for the last 2 filters.
            \item No. Applying a vertical filter followed by a horizontal filter absorbs all the light. Applying a vertical filter, then a filter oriented 45 degrees from the origin, and then a horizontal filter will let $\frac{1}{8}$ of the original light through. Note this question depends on what polarization we assume the photon source admits. 
    \end{enumerate}
    \item \begin{enumerate}
        \item \[\includegraphics[scale=0.6]{beamSplit-sol}\]
        \item The maximum probability of a photon arriving at receiver A or receiver B is 50\% because we must use a polarizing filter to rotate the photons by 45 degrees to effectively use the polarizing beam splitter. This first filter allows 50\% of the original photons through, so the maximum percentage possible is 50\%.
        \item REDO THIS QUESTION TO MAKE IT EASIER
        \item See above.
    \end{enumerate}
    \item \begin{enumerate}
        \item \[\includegraphics[scale=0.6]{sugarSplit-sol}\]
        \item The maximum possible probability of a photon arriving at receiver A or receiver B is now 100\% because we can use the sugar solution to rotate photons by 45 degrees without losing any photons like we had to do when only using polarizing filters.
        \item Not sure I know how to do this one. Remove?
        \item See above.
    \end{enumerate}
    \item 
    \[\includegraphics[scale=0.6]{bohr}\]
    The key idea is to implement Bohr’s experiment and then add a polarizing filter and a receiver at the end of Bohr’s system. Then when you run a photon through this system you can check if we see the photon arrive in the receiver. In the above example we use a vertically oriented polarizing filter so if we see a photon in the receiver, we know the answer to Bohr’s problem is state 1. If we do not see the photon in the receiver, we know it hit the filter and so the answer is state 0.
    \item \begin{enumerate}
        \item We should expect to see a perfect 50/50 split. When I ran this experiment I observed a 51/49 split.
        \item The results do not match because quantum systems are random and noisy. We expect a perfect 50/50 split in the limit.
    \end{enumerate}
    \item \[\cos({10^{\circ}}) \approx .97 \]
            \[(.97)^x = 0.5\]
            \[x = 22.75\]
            Should round down to 22 filters but also accept 23 filters.
\end{enumerate}



\end{document}
