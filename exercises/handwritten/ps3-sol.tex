\documentclass[12pt]{article}
\usepackage{amssymb,mathrsfs, amsmath,amsfonts}
\usepackage{mathtools}
\usepackage{graphicx}
\usepackage{enumitem}
\usepackage{braket}

\title{Problem Set 3 Solutions}
\author{CSE 468}
\date{May 2021}

\begin{document}

\maketitle

\begin{enumerate}[font=\bfseries]
    \item Measurement
    \item \[ \frac{1}{\sqrt{2}}\begin{pmatrix}
        1 & -1 \\
        1 & 1
        \end{pmatrix}
        \]
    \item \[ \frac{1}{\sqrt{2}}\begin{pmatrix}
        0 & i & 0 & -i \\
        i & 0 & -i & 0 \\
        0 & -i & 0 & -i \\
        -i & 0 & -i & 0 \\
        \end{pmatrix}
        \]
    \item We need to consider tensor products when our state is entangled.
    \item Proof by contradiction: Let $\ket{\psi_e}$ be some entangled state. Suppose there exists some set of single qubit unitary operations that when applied to $\ket{\psi_e}$ results in some non-entangled state $\ket{\psi_n}$. Call this set of operations $A \otimes B$. We can then write
    \[\ket{\psi_n} = A(c_0\ket{0}+c_1\ket{1}) \otimes B(c_2\ket{0}+c_3\ket{1})\]
    since $\ket{\psi_n}$ is a non-entangled state. We could then apply $A^{-1} \otimes B^{-1}$ to $\ket{\psi_n}$ to obtain $\ket{\psi_e}$ as so
    \[(A^{-1} \otimes B^{-1})\ket{\psi_n} = A^{-1}A(c_0\ket{0}+c_1\ket{1}) \otimes B^{-1}B(c_2\ket{0}+c_3\ket{1})\]
    \[= (c_0\ket{0}+c_1\ket{1}) \otimes (c_2\ket{0}+c_3\ket{1}) = \ket{\psi_e}\]
    We observe above that $\ket{\psi_e}$ can be written as the tensor product of two states which contradicts our earlier assumption that $\ket{\psi_e}$ was an entangled state. (Revisit this?)
    \item \[\ket{\psi} = \frac{1}{\sqrt{2}}\ket{00} - \frac{i}{\sqrt{2}}\ket{01} = \ket{0} \otimes (\frac{1}{\sqrt{2}}\ket{0} - \frac{i}{\sqrt{2}}\ket{1})\]
    \item This is an entangled state. Proof: Note:
    \[\frac{1}{\sqrt{2}}\ket{01} + \frac{1}{\sqrt{2}}\ket{10} = 
            \frac{1}{\sqrt{2}}
            \begin{pmatrix}
            0 \\ 1 \\ 1 \\ 0
            \end{pmatrix}
            \]
        Suppose this is not an entangled state then we have:
        \[\begin{pmatrix}
        a \\b
        \end{pmatrix}
        \otimes
        \begin{pmatrix}
        c \\d
        \end{pmatrix}
        =
        \begin{pmatrix}
        0 \\ 1 \\ 1 \\ 0
        \end{pmatrix}
        \]
        Which implies $ac = 0, ad = 1, bc = 1, bd = 0$ which further implies $abcd = 1 \neq 0 = abcd$ which gives the desired contradiction.
        
    \item \begin{enumerate}
            \item \[\ket{10} = \begin{pmatrix} 0 \\ 0 \\ 1 \\0          \end{pmatrix}\]
            \item \[\ket{0+} = \begin{pmatrix} \frac{1}{\sqrt{2}} \\ \frac{1}{\sqrt{2}} \\ 0 \\0          \end{pmatrix}\]
            \item \[\ket{+-} = \begin{pmatrix} \frac{1}{2} \\ -\frac{1}{2} \\ \frac{1}{2} \\ -\frac{1}{2}         \end{pmatrix}\]
        \end{enumerate}
    \item Many solutions to this problem. One solution is provided below.
        \begin{enumerate}
            \item Pauli X: $\theta=\pi,\phi=0,\lambda=\pi$
            \item Pauli Y: $\theta=\pi,\phi=\pi/2,\lambda=\pi/2$
            \item Pauli Z: $\theta=0,\phi=\pi,\lambda=0$
            \item Hadamard: $\theta=\pi/2,\phi=0,\lambda=\pi$
        \end{enumerate}
    \item 
        \[CY = \begin{pmatrix} 
        1 & 0 & 0 & 0 \\
        0 & 1 & 0 & 0 \\
        0 & 0 & 0 & -i \\
        0 & 0 & i & 0
        \end{pmatrix} \]
        Yes we could use this gate to induce entanglement similarly to the standard entanglement circuit.
    \item Want to show:
    \[\frac{1}{\sqrt{2}}(\ket{10} - \ket{01}) = \frac{1}{\sqrt{2}}(\ket{+-} - \ket{-+})\]
    Consider:
    \[\ket{+-} = \begin{pmatrix}
                \frac{1}{2} \\ \frac{-1}{2} \\ \frac{1}{2} \\ \frac{-1}{2} 
                \end{pmatrix}
                        \]
    and 
    \[\ket{-+} = \begin{pmatrix}
                \frac{1}{2} \\ \frac{1}{2} \\ \frac{-1}{2} \\ \frac{-1}{2} 
                \end{pmatrix}
                        \]
    Therefore:
    \[\frac{1}{\sqrt{2}}(\ket{+-} - \ket{-+}) = \frac{1}{\sqrt{2}} (\begin{pmatrix}
                \frac{1}{2} \\ \frac{-1}{2} \\ \frac{1}{2} \\ \frac{-1}{2} 
                \end{pmatrix} - \begin{pmatrix}
                \frac{1}{2} \\ \frac{1}{2} \\ \frac{-1}{2} \\ \frac{-1}{2} 
                \end{pmatrix} )  = \frac{1}{\sqrt{2}}
                \begin{pmatrix}
                0 \\ -1 \\ 1 \\ 0 
                \end{pmatrix} = \frac{1}{\sqrt{2}}(\ket{10} - \ket{01})
                        \]
\end{enumerate}



\end{document}