\documentclass[12pt]{article}
\usepackage{exercises/handwritten/handout}

\usepackage{amssymb,mathrsfs, amsmath,amsfonts}
\usepackage{mathtools}
\usepackage{graphicx}
\usepackage{enumitem}
\usepackage{braket}
\input{quantummacros.tex}
%%
%%
\newcommand{\Blank}{\mbox{\hskip 4pt\vrule width 1in depth 2pt}\vrule width 0pt height 2.0em}
\newcommand{\NameBlank}{\mbox{\hskip 4pt\vrule width 2.5in depth 2pt}\vrule width 0pt height 2.0em}
\newcommand{\BlankLine}{\mbox{\hskip 4pt\vrule width 5.5in depth 2pt}\vrule width 0pt height 2.0em}
%%
%% Leave at least #1 space, default to what is below
%%
\def\DefaultSpace{1in}
\newcommand{\LeaveSpace}[1][\DefaultSpace]{%
\vskip #1 plus 1fil\relax\hbox to 0pt{\hss} %
}

\begin{document}

\assignment{Problem Set 060}

\begin{quote}
    As for all written assignments, you must upload your completed work to GradeScope.

    You are welcome to work collaboratively on solving these problems, but you each must write or type out your own solution, and upload that to GradeScope.

    Work submitted improperly will receive no credit.
\end{quote}

\begin{quote}\bf
Feel free to use Matlab as you like to help solve these problems.  Where you are asked to show your work, show the Matlab output.
\end{quote}

\begin{enumerate}[font=\bfseries]
\item With whom did you collaborate on this assignment?
    \LeaveSpace{}
    \item (20 points) Let $\ket{\psi} = \alpha\ket{0} + \beta\ket{1}$. Throughout this assignment, he {\bf boldface} matrices are the Pauli matrices.
    \begin{enumerate}[label=\theenumi.\arabic*]
        \item What is the state of $\ket{\psi_0} = \mathbf{X}\mathbf{Y}\mathbf{Z}\ket{\psi}$? \Blank{}
        \item Show how you arrived at your answer. \LeaveSpace[0.7in]
        \item Describe each of the above operations as rotations on the Bloch sphere in the order they would occur.\LeaveSpace{}
        \item What is the state of $\ket{\psi_1} = \mathbf{Z}\mathbf{Y}\mathbf{X}\ket{\psi}$? \Blank{}
        \item Show how you arrived at your answer. \LeaveSpace[0.6in]
        \item Does $\ket{\psi_0} = \ket{\psi_1}$? \Blank{}.  Explain:\LeaveSpace{}
        \item Is the probability of observing $\ket{0}$ when we measure $\ket{\psi_0}$ equal to the probability of observing $\ket{0}$ when we measure $\ket{\psi_1}$?\Blank{}
        \item Why or why not? \LeaveSpace{}
    \end{enumerate}
    \item (8 points) 
    \begin{enumerate}[label=\theenumi.\arabic*]
    \item Provide a unitary matrix $U$ below such that when we apply the operation $\mathbf{Y}U\ket{0}$, this results in a $\frac{3}{4}$ chance of measuring $\ket{1}$. \LeaveSpace{}
    \item Show your work for finding $U$. \LeaveSpace[2.0in]
    \end{enumerate}
    \item (4 points) Consider $\ket{\psi} = \mathbf{X}^a\mathbf{Y}^b\mathbf{Z}^c\ket{0}$.
    \begin{enumerate}[label=\theenumi.\arabic*]
        \item If $a,c$ are even and $b$ is odd, what is the probability of observing $\ket{0}$ when we measure $\ket{\psi}$? \Blank{}
        \item If $a,b,c$ are all even what is the probability of observing $\ket{1}$ when we measure $\ket{\psi}$? \Blank{}
    \end{enumerate}
    \item \Points{10}
    \begin{enumerate}[label=\theenumi.\arabic*] \item \Points{2} If we begin at $\ket{0}$ on the Bloch sphere and can use any number of the Pauli gates (\PauliX{}, \PauliY{}, \PauliZ{}), to which states can we travel on the Bloch sphere? \LeaveSpace{}
    \item \Points{3} Explain your answer. 
    \LeaveSpace{}
    
    \item (2 points) If we now also also allow the use of any number of Hadamard gates (\Hadamard{}), to which states can we travel on the Bloch sphere? \LeaveSpace{}
    \item \Points{3} Explain your answer. 
    \LeaveSpace{}
    \end{enumerate}
    \item \Points{15} Recall two points on the Bloch sphere are antipodal if they determine a line that passes through the center of the sphere. Recall that in polar coordinates, a quantum state $\ket{\psi}$ can be specified on the Bloch sphere using two coordinates, $\theta$ and $\phi$:
    \[\ket{\psi} = \cos(\frac{\theta}{2})\ket{0}
    + \ExpPhase{\phi}\sin(\frac{\theta}{2})\ket{1}\]
    \begin{enumerate}[label=\theenumi.\arabic*]
        \item \Points{2} All points on the Bloch sphere denote unique (under rotation by the $\mathbf{Z}$ gate) quantum states except which two points? \Blank{} and\Blank{}
        \item \Points{1} The number of quantum states on the Bloch sphere is \emph{countably} infinite or \emph{uncountably} infinite? \Blank{}
        \item \Points{12} Given a state $\ket{\psi}$ as defined above, it's \emph{antipodal} state is given by:
        \[ \ket{\psi'} = \cos({\frac{\pi - \theta}{2})\ket{0}+\ExpPhase{(\phi + \pi)}\sin({\frac{\pi - \theta}{2}})\ket{1}}
        \]
        \[
        = \cos({\frac{\pi - \theta}{2})\ket{0}-\ExpPhase{\phi}\sin({\frac{\pi - \theta}{2}})\ket{1}}
        \]
        
        Prove that $\ket{\psi}$ and $\ket{\psi'}$ are orthogonal by showing their inner product is 0. \textbf{Remember to conjugate appropriately.} \LeaveSpace[2.0in]
    \end{enumerate}
    \newpage
    % \item (5 points) Your friend, Albert Einstein, has challenged you to a quantum game. The game is played as follows: First, Einstein gives you some state $\ket{\psi}$ and tells you where it is located on the Bloch sphere by telling you $\theta$ and $\phi$. You then can apply any quantum gates you wish to $\ket{\psi}$. Before you make a measurement you tell Einstein what measurement you expect to observe. If you are right Einstein gives you \$100. If you are wrong Einstein zaps you with his many-worlds invention. Do you agree to play Einstein's game? Why or why not? \LeaveSpace{}
    \item \Points{10} Consider the quantum states
        \[\ket{+y} = \frac{1}{\sqrt{2}}\ket{0} + \frac{i}{\sqrt{2}}\ket{1}\]
        and
        \[\ket{-y} = \frac{1}{\sqrt{2}}\ket{0} - \frac{i}{\sqrt{2}}\ket{1}\]
        What quantum gate (unitary matrix) $U$ produces those states given the input qubits $\ket{0}$ and $\ket{1}$, respectively?
        \begin{enumerate}
            \item \Points{2} Define $U$. \LeaveSpace[0.75in]
            \item \Points{2} Prove $U$ is unitary. \LeaveSpace[2.5in]
            \item \Points{2} Below provide the matrix for $U^{-1}$. \LeaveSpace{}
            \item \Points{3} Show that $\ket{+y}$ and $\ket{-y}$ are eigenstates of the Pauli $\mathbf{Y}$ gate. \LeaveSpace{}
            \item \Points{1} More generally, what do the columns of such a 2 × 2 matrix (such as $U$) represent with respect to their inputs in the standard basis ($\ket{0}$ and $\ket{1}$)? \LeaveSpace{}
        \end{enumerate}

\end{enumerate}

\end{document}