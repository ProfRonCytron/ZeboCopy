\documentclass[12pt]{article}
\usepackage{amssymb,mathrsfs, amsmath,amsfonts}
\usepackage{mathtools}
\usepackage{graphicx}
\usepackage{enumitem}
\usepackage{braket}

\title{Problem Set 0 Solutions}
\author{CSE 468}
\date{\today}

\begin{document}

\maketitle

\begin{enumerate}[font=\bfseries]
    \item \begin{enumerate}
        \item Not reversible. We cannot recover the input from the output.
        \item Reversible. We can uniquely recover the input from output.
        \item Reversible. Given the sign of the input we now know if the input was positive or negative and thus can recover input from output.
        \item Not reversible.
    \end{enumerate}
    % \item No. A set may contain many elements while the max function outputs a single number. Thus, it is impossible to determine the whole set given just the maximum value of the set, meaning max is not a reversible function.
    % \item No. SHA-256 uses a ``one-way compression function" which takes two inputs and returns a single output.
    \item Recall that we can think of theta as the distance we traverse around the complex unit circle starting from $(1,0)$. By tracing this out on a unit circle, it can be seen that $e^{i\frac{3\pi}{4}}$ is in quadrant 2.
    Alternatively, we can use $e^{i\frac{3\pi}{4}} = \cos{\frac{3\pi}{4}}+i\sin{\frac{3\pi}{4}} = -\frac{1}{\sqrt{2}}+i\frac{1}{\sqrt{2}}$.
    The real component, $-\frac{1}{\sqrt{2}}$, tells us we're on the left side of the unit circle. The imaginary coefficient, $\frac{1}{\sqrt{2}}$, puts us in the top half. The intersection of these pieces of information yields quadrant 2.
    \item \begin{itemize}
        \item $\braket{a|b} =$ 10 - 3$i$
        \item $\ket{b}\bra{a} = \begin{pmatrix} 10 & -2i \\ 15 & -3i \end{pmatrix}$
        \item $\braket{b|b} = 13$
        \item $|\ket{b}|^2 = 13$
    \end{itemize}
\end{enumerate}



\end{document}
