\documentclass[12pt]{article}
\usepackage{exercises/handwritten/handout}

\usepackage{amssymb,mathrsfs, amsmath,amsfonts}
\usepackage{mathtools}
\usepackage{graphicx}
\usepackage{enumitem}
\usepackage{braket}
\usepackage{hyperref}
\input{quantummacros.tex}
%%
%%
\newcommand{\Blank}{\mbox{\hskip 4pt\vrule width 1in depth 2pt}\vrule width 0pt height 2.0em}
\newcommand{\NameBlank}{\mbox{\hskip 4pt\vrule width 2.5in depth 2pt}\vrule width 0pt height 2.0em}
%%
%% Leave at least #1 space, default to what is below
%%
\def\DefaultSpace{1in}
\newcommand{\LeaveSpace}[1][\DefaultSpace]{%
\vskip #1 plus 1fil\relax\hbox to 0pt{\hss} %
}

\begin{document}

\assignment{Problem Set 010}

\begin{quote}
    As for all written assignments, you must upload your completed work to GradeScope.

    You are welcome to work collaboratively on solving these problems, but you each must write or type out your own solution, and upload that to GradeScope.

    Work submitted improperly will receive no credit.
\end{quote}

\def\Ans#1{{\color{blue}\mbox{Answer: }#1}}
\begin{enumerate}[font=\bfseries]
    \item For this assignment, I worked with the following people:

    \Ans{Varies}
    \LeaveSpace{}
    \item \Points{15} In which quadrant of the complex unit circle do you find $e^{i\frac{3\pi}{4}}$? \Ans{II}
    \item \Points{25} Let
    \[ \ket{a} = \SQB{3}{1} \mbox{ and } \ket{b} = \SQB{1}{4} \]
    \begin{enumerate}[label=\theenumi.\arabic*]
        \item $\bra{a}=\Ans{\CQB{3}{1}}$
        \item $\braket{a|b}=\Ans{7}$
        \item $\ket{a}\kern-3pt\bra{b}=\Ans{\SQBG{\relax}{3}{12}{1}{4}}$
        \item $\ket{b}\kern-3pt\bra{a}=\Ans{\SQBG{\relax}{3}{1}{12}{4}}$
        \item $\Prob{\ket{b}}=\Ans{17}$
    \end{enumerate}


    \item \Points{20} Supply the conjugate transpose expessions below:
    \begin{enumerate}[label=\theenumi.\arabic*]
       \item $\Conj{\SQB{3}{i}}=\Ans{\CQB{3}{-\NiceI}}$ 
       \item $\Conj{\SQB{3}{7}}=\Ans{\CQB{3}{7}}$  
       \item $\Conj{\CQB{2i}{64}}=\Ans{\SQB{-2\NiceI}{64}}$
       \item $\Conj{\SQBG{\relax}{2}{i}{3}{i}}=\Ans{\SQBG{\relax}{2}{3}{-\NiceI}{-\NiceI}}$ 
    \end{enumerate}

    \item \Points{20} Suppose $\ket{a} = \begin{pmatrix}5 \\ i \end{pmatrix}$ and $\ket{b} = \begin{pmatrix}2 \\ 3 \end{pmatrix}$.
    \begin{enumerate}[label=\theenumi.\arabic*]
        \item $\braket{a|b} = \Ans{10-3\NiceI}$ 
        \item $\ket{b}\bra{a} = \Ans{\SQBG{\relax}{10}{-2\NiceI}{15}{-3\NiceI}}$
        \item $\braket{b|b} = \Ans{13}$ 
        \item $|\ket{b}|^2 = \Ans{13}$ 
    \end{enumerate}
        \item \Points{20} Recall we say a computation (or function) is \emph{reversible} if it is always possible to uniquely recover the input, given the output. Assume $x \in \mathbb{R}$.
    \begin{enumerate}[label=\theenumi.\arabic*]
        \item If $f : x \mapsto x^2$, is $f$ a reversible function? \Ans{No, we lose sign information}
        \item If $f : x \mapsto x^3$, is $f$ a reversible function? \Ans{Yes, the cube root will compute the inverse}
        \item If $f : x \mapsto (x^2,\textrm{sign}(x))$, is $f$ a reversible function? \Ans{Yes}
        \item If $f : x \mapsto 0$, is $f$ a reversible function? \Ans{No, we cannot recover $x$}
    \end{enumerate}
 
    
\end{enumerate}



\end{document}
