\documentclass[12pt]{article}
\usepackage{exercises/handwritten/handout}

\usepackage{amssymb,mathrsfs, amsmath,amsfonts}
\usepackage{mathtools}
\usepackage{graphicx}
\usepackage{enumitem}
\usepackage{braket}
\input{quantummacros.tex}
%%
%%

%%
%% Leave at least #1 space, default to what is below
%%


\begin{document}

\assignment{Problem Set 335}

\begin{itemize}
    \item Organize yourselves in groups of up to 4 people.
    \item The starting notebook is on Canvas, called \texttt{ErrCorr.ipynb}, in Module 33 for Errors and Error Detection.
    \item For this assignment, you will deploy the 3-qubit error correction scheme taught in class, which will correct a single \PauliX{} error on one of three qubits.
    \item This assignment is worth 100 points but you get those points just by completing the work.
\end{itemize}

\begin{enumerate}[font=\bfseries]

\item Organize yourselves in groups up to four people.  Your people are (names and student IDs please):
\begin{itemize}
    \item \Blank[4in]{}
    \item \Blank[4in]{}
    \item \Blank[4in]{}
    \item \Blank[4in]{}
\end{itemize}
\item Describe the code you had to insert to complete this assignment.
\LeaveSpace[3in]{}
\item What were the results in emulation?  Were these expected?
\LeaveSpace[4in]{}
\item Describe the results you obtained on real quantum hardware.  Be specific about what you saw, how many errors were corrected, what is the expected improvement based on this approach.
\LeaveSpace[4in]{}

\item If you had more time, what next steps would you take?
\end{enumerate}

\end{document}