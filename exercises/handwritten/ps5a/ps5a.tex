\documentclass[12pt]{article}
\usepackage{amssymb,mathrsfs, amsmath,amsfonts}
\usepackage{mathtools}
\usepackage{graphicx}
\usepackage{enumitem}
\usepackage{braket}
\title{Problem Set 5a}
\author{CSE 468}
\date{\today}
%%
%%
\newcommand{\Blank}[1][1in]{\mbox{\hskip 4pt\vrule width #1 depth 2pt}\vrule width 0pt height 2.0em}
\newcommand{\NameBlank}{\mbox{\hskip 4pt\vrule width 2.5in depth 2pt}\vrule width 0pt height 2.0em}
\newcommand{\BlankLine}{\mbox{\hskip 4pt\vrule width 5.5in depth 2pt}\vrule width 0pt height 2.0em}
%%
%% Leave at least #1 space, default to what is below
%%
\def\DefaultSpace{1in}
\newcommand{\LeaveSpace}[1][\DefaultSpace]{%
\vskip #1 plus 1fil\relax\hbox to 0pt{\hss} %
}

\begin{document}
\maketitle

\noindent Name:\NameBlank{} \newline
\noindent Student ID:\NameBlank{} \newline

\textbf{Note:} You may discuss these problems with other students, but you must write your own solutions. If you run out of space for any question, please use the additional page and clearly indicate which question you are answering.

\bigskip

The problems below recall the Mermin--Peres square \emph{as taught in class} (slide decks 13 and 14).

Throughout this homework
\begin{itemize}
    \item Alice is assigned row~3.
    \item Bob is assigned column~1.  
    \item Alice happens to measure first.
    \end{itemize}
\begin{enumerate}[font=\bfseries]
    \item Referring to slide deck 14, slide 13, suppose Alice is assigned row 3, she measures first, and the state of her two qubits collapses to \[ \psi_{1}=\frac{\ket{0+}-\ket{1-}}{\sqrt{2}} \]
    What is the state of Bob's qubits? \Blank[2in]
    \clearpage
    \item In that state, suppose Bob is assigned column~1.  What two outcome(s) might he see for his \textbf{Z} (first qubit) measurement?
    
    \Blank[2in]
    \item And what outcome(s) might he see for his \textbf{X} (second qubit) measurement?
    
    \Blank[2in]
    
    \item Are all combinations of outcomes possible?  \Blank{}.  If not below say which are possible
    
    \Blank[2in]{}
    \item Referring to Slide 14, what two possible states ($\psi_{?}$ or $\psi_{?}$) could Bob see when he measures after Alice?
    
    \Blank{} or \Blank{}
    \item Assuming Bob measures $\ket{0}$ for his first quit and $\ket{+}$ for his second qubit, what results ($\pm 1$) does Bob report for his
    \begin{itemize}
        \item Top square \Blank{}
        \item Middle square \Blank{}
        \item Bottom square \Blank{}
    \end{itemize}
    \item What value must Alice have reported for the first square in her row in this scenario? \Blank{}
    \clearpage
    \item What information on Slide 14 assures us that Bob will report the same result for the square they have in common (bottom left square) when he measures second?
    
\end{enumerate}


\end{document}