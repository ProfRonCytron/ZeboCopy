\documentclass[12pt]{article}
\usepackage{amssymb,mathrsfs, amsmath,amsfonts}
\usepackage{mathtools}
\usepackage{graphicx}
\usepackage{enumitem}
\usepackage{braket}

\title{Problem Set 2 Solutions}
\author{CSE 468}
\date{May 2021}

\begin{document}

\maketitle

\noindent \textbf{Note:} Pauli matrix type questions

\begin{enumerate}[font=\bfseries]
    \item 
    \begin{enumerate}
        \item \[ \mathbf{X}\mathbf{Y}\mathbf{Z}\ket{\psi} = 
                \begin{pmatrix} 
                0 & 1 \\
                1 & 0
                \end{pmatrix} 
                \begin{pmatrix} 
                0 & -i \\
                i & 0
                \end{pmatrix} 
                 \begin{pmatrix} 
                1 & 0 \\
                0 & -1
                \end{pmatrix} 
                 \begin{pmatrix} 
                \alpha \\ \beta
                \end{pmatrix} 
                =
                i
                \begin{pmatrix} 
                \alpha \\ \beta
                \end{pmatrix} 
                \]
        \item Rotation of $\pi$ radians about the z axis, rotation of $\pi$ radians about the y axis, and a rotation of $\pi$ radians about the x axis
        \item \[ \mathbf{X}\mathbf{Y}\mathbf{Z}\ket{\psi} = 
                \begin{pmatrix} 
                1 & 0 \\
                0 & -1
                \end{pmatrix} 
                \begin{pmatrix} 
                0 & -i \\
                i & 0
                \end{pmatrix} 
                 \begin{pmatrix} 
                0 & 1 \\
                1 & 0
                \end{pmatrix} 
                 \begin{pmatrix} 
                \alpha \\ \beta
                \end{pmatrix} 
                =
                -i
                \begin{pmatrix} 
                \alpha \\ \beta
                \end{pmatrix} 
                \]
        \item No.
        \item Yes.
    \end{enumerate}
    
    \item \[R_y(\theta) = \begin{pmatrix} \cos{\theta/2} & -\sin{\theta/2} \\
        \sin{\theta/2} & \cos{\theta/2}
        \end{pmatrix}\]
        \[\cos^2{\theta} = \frac{1}{4}, \theta = \frac{\pi}{3}\]
        \[\begin{pmatrix} 
        \frac{\sqrt{3}}{2} & -\frac{1}{2} \\
        \frac{1}{2} & \frac{\sqrt{3}}{2}
            \end{pmatrix}
        \]
    \item Recall Pauli matrices are involutory.
        \begin{enumerate}
            \item If $a,c$ are even then $\mathbf{X}^a\mathbf{Y}^b\mathbf{Z}^c\ket{0}$ becomes $\mathbf{Y}\ket{0}$ and so the probability of measuring $\ket{0}$ is 0. 
            \item Whole equation acts as a single identity gate and so the probability of measuring $\ket{1}$ in the $\mathbf{X}$ basis is 100\%.
        \end{enumerate}
    \item 2 unique locations
    \item 6 unique locations
    \item To-do
    \item Agree to play. Recall antipodal points on the Bloch sphere determine a basis. Apply the appropriate quantum gate to rotate into this antipodal basis and then you will know with 100\% confidence the state you will observe.
    \item 
        \begin{enumerate}
            \item \[U = \frac{1}{\sqrt{2}}\begin{pmatrix}
                        1 & 1 \\
                        i & -i
                        \end{pmatrix}\]
            \item Show $UU^* = U^*U = \mathbf{I}$. 
                \[UU^* = \frac{1}{\sqrt{2}}\begin{pmatrix}
                        1 & 1 \\
                        i & -i
                        \end{pmatrix}
                        \frac{1}{\sqrt{2}}\begin{pmatrix}
                        1 & -i \\
                        1 & i
                        \end{pmatrix}
                        =
                        \begin{pmatrix}
                        1 & 0 \\
                        0 & 1
                        \end{pmatrix}
                        = \mathbf{I}
                        \]
                \[U^*U = \frac{1}{\sqrt{2}}\begin{pmatrix}
                        1 & -i \\
                        1 & i
                        \end{pmatrix}
                        \frac{1}{\sqrt{2}}\begin{pmatrix}
                        1 & 1 \\
                        i & -i
                        \end{pmatrix}
                        =
                        \begin{pmatrix}
                        1 & 0 \\
                        0 & 1
                        \end{pmatrix}
                        = \mathbf{I}
                        \]
                
            \item \[U = \ket{0}\bra{+y} + \ket{1}\bra{-y}\]
            \item \[
                    \mathbf{Y}\ket{+y} = \begin{pmatrix}
                                        0 & -i \\
                                        i & 0
                                        \end{pmatrix}
                                        \frac{1}{\sqrt{2}}
                                        \begin{pmatrix}
                                        1 \\ i
                                        \end{pmatrix}
                                        =
                                        \frac{1}{\sqrt{2}}
                                        \begin{pmatrix}
                                        1 \\ i
                                        \end{pmatrix}
                                        =
                                        \ket{+y}
                    \]
                    \[
                    \mathbf{Y}\ket{-y} = \begin{pmatrix}
                                        0 & -i \\
                                        i & 0
                                        \end{pmatrix}
                                        \frac{1}{\sqrt{2}}
                                        \begin{pmatrix}
                                        1 \\ -i
                                        \end{pmatrix}
                                        =
                                        \frac{1}{\sqrt{2}}
                                        \begin{pmatrix}
                                        -1 \\ i
                                        \end{pmatrix}
                                        =
                                        -\frac{1}{\sqrt{2}}
                                        \begin{pmatrix}
                                        1 \\ -i
                                        \end{pmatrix}
                                        =
                                        -\ket{-y}
                    \]
            \item Left column represents transformation of $\ket{0}$. Right column represents transformation of $\ket{1}$.
        \end{enumerate}
\end{enumerate}



\end{document}