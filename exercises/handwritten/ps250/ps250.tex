\documentclass[12pt]{article}
\usepackage{exercises/handwritten/handout}

\usepackage{amssymb,mathrsfs, amsmath,amsfonts}
\usepackage{mathtools}
\usepackage{graphicx}
\usepackage{enumitem}
\usepackage{braket}
\input{quantummacros.tex}
%%
%%

%%
%% Leave at least #1 space, default to what is below
%%


\begin{document}

\assignment{Problem Set 250}


\begin{enumerate}[font=\bfseries]

\item \Points{0} (negative points if you don't fill this out!)
\begin{itemize}
    \item Your name?\Blank[20em]{}
    \item Your student ID?\Blank[12em]{}
    \item With whom did you collaborate?
    \LeaveSpace{}]
\end{itemize}

\item (20 points) Referencing the material beginning on slide 7 page 51 in 250.pdf, consider an instance of Simon with $n=10$ qubits.  

Using the formula there, how many queries must we issue classically to try to find $x$ and $y$ such that $f(x)=f(y)$ with each of the following probabilities?
\begin{description}
    \item[10\%] \Blank{}
    \item[50\%] \Blank{}
    \item[90\%] \Blank{}
    \item[99\%] \Blank{}
    \item[99.9\%] \Blank{}
\end{description}
\item\Points{20} Suppose for an instance of Simon's problem we are incredibly lucky and discover in two queries that \texttt{101101} and \texttt{000111} are mapped to the same value by the oracle.  \begin{enumerate}[label=\theenumi.\arabic*]
    \item What is the secret $s$?\Blank[1.5in]{}
    \item How many other inputs will map to the same value as do \texttt{101101} and \texttt{000111}?\Blank{}
\end{enumerate}

\item \Points{15} Consider a 3-qubit system in state $\ket{\psi}$.  Suppose \[ \textbf{H}\left(\ket{\psi}\right)=\frac{\ket{001}+\ket{100}}{\sqrt{2}}\] 

What is $\ket{\psi}$? 
Hint: Hadamard is its own inverse.
\LeaveSpace{}

    \item \Points{15} Consider the oracle portion of a circuit below for an $8$-bit instance of the Bernstein--Vazirani problem.  Recall the oracle computes $y=x\oplus s$ for a secret bit vector~$s$.
 \begin{center}   
    \includegraphics[scale=0.45]{ps250-assets/bv.png}
    \end{center}
\begin{enumerate}[label=\theenumi.\arabic*]
\item What is the secret $s$ here? \Blank[2in]{}
    \item If this oracle is used in the Deutsch--Jozsa algorithm, what possible amplitude(s) can be measured on $\ket{00000000}$?
    
    \Blank[4in]{}
    \item  What other computational basis vector(s), if any, will have non-zero amplitude for Deutsch--Jozsa if the above oracle is used?  
    
    \Blank[4in]{}
\end{enumerate}
\item\Points{15} The circuit below shows two three-qubit inputs $x$ and $y$ and output $s$.
\begin{itemize}
    \item Modify the circuit so that $s_i$ is \QOne{} if and only if $x_{i}=y_{i}$.  
    \item If you must apply circuitry to $x$ or $y$ that changes their values, be sure they are restored by the right side of the circuit.
    \item You need only concern yourself with values for $x$ and $y$ that are in the standard \PauliZ{} basis (each qubit is either \QZero{} or \QOne{}).
    \item Assume the $s$ qubits are each initialized to \QZero{}.
    \item For gates, you are allowed only the
    Pauli gates (including \Hadamard{}), the \NamedGate{CNOT} gate, and the \NamedGate{CCNOT} gates.
\end{itemize}
\BigSkip{}
\begin{center}
\Vskip{-3em}\adjustbox{valign=t, width=0.6\textwidth}{\begin{quantikz}
\lstick{$x_1$} & \qw & \qw  & \qw & \qw& \qw & \qw  & \qw & \qw\\ 
\lstick{$x_2$} & \qw & \qw  & \qw & \qw& \qw & \qw  & \qw & \qw\\ 
\lstick{$x_3$} & \qw & \qw  & \qw & \qw& \qw & \qw  & \qw & \qw\\ 
\lstick{$y_1$} & \qw & \qw  & \qw & \qw& \qw & \qw  & \qw & \qw\\ 
\lstick{$y_2$} & \qw & \qw  & \qw & \qw& \qw & \qw  & \qw & \qw\\ 
\lstick{$y_3$} & \qw & \qw  & \qw & \qw& \qw & \qw  & \qw & \qw\\
\lstick{$s_1$} & \qw & \qw  & \qw & \qw& \qw & \qw  & \qw & \qw\\ 
\lstick{$s_2$} & \qw & \qw  & \qw & \qw& \qw & \qw  & \qw & \qw\\ 
\lstick{$s_3$} & \qw & \qw  & \qw & \qw& \qw & \qw  & \qw & \qw\\
\end{quantikz}}
\end{center}

\item\Points{15} Now consider the following circuit.
\begin{itemize}
    \item The $s$ qubits are inputs and each is either \QZero{} or \QOne{}.
    \item The $a$ qubits are ancillas and each is initialized to \QZero{}.  Use them as needed.
    \item Arrange for $t$ to be \QOne{} if and only if each $s_{i}$ bit is also \QOne{}.  
    \item For gates, you are (again) allowed only the
    Pauli gates (including \Hadamard{}), the \NamedGate{CNOT} gate, and the \NamedGate{CCNOT} gates.
\end{itemize}
\begin{center}
\Vskip{-3em}\adjustbox{valign=t, width=0.6\textwidth}{\begin{quantikz}
\lstick{$s_1$} & \qw & \qw  & \qw & \qw& \qw & \qw  & \qw & \qw\\ 
\lstick{$s_2$} & \qw & \qw  & \qw & \qw& \qw & \qw  & \qw & \qw\\ 
\lstick{$s_3$} & \qw & \qw  & \qw & \qw& \qw & \qw  & \qw & \qw\\
\lstick{$a_1$} & \qw & \qw  & \qw & \qw& \qw & \qw  & \qw & \qw\\ 
\lstick{$a_2$} & \qw & \qw  & \qw & \qw& \qw & \qw  & \qw & \qw\\ 
\lstick{$a_3$} & \qw & \qw  & \qw & \qw& \qw & \qw  & \qw & \qw\\
\lstick{$t$} & \qw & \qw  & \qw & \qw& \qw & \qw  & \qw & \qw\\
\end{quantikz}}
\end{center}


\end{enumerate}

\end{document}