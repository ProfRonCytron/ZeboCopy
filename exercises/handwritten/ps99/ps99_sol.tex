\documentclass[12pt]{article}
\usepackage{amssymb,mathrsfs, amsmath,amsfonts}
\usepackage{mathtools}
\usepackage{graphicx}
\usepackage{enumitem}
\usepackage{braket}

\title{Problem Set 99 Solutions}
\author{CSE 468}
\date{\today}

\begin{document}

\maketitle

\begin{enumerate}[font=\bfseries]
    \item Placeholder
    \item The key idea is to implement Bohr’s experiment and then add a polarizing filter and a receiver at the end of Bohr’s system. Then when you run a photon through this system you can check if we see the photon arrive in the receiver. In the above example we use a vertically oriented polarizing filter so if we see a photon in the receiver, we know the answer to Bohr’s problem is state 1. If we do not see the photon in the receiver, we know it hit the filter and so the answer is state 0.
\end{enumerate}



\end{document}