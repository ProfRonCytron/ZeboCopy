\documentclass[12pt]{article}
\usepackage{amssymb,mathrsfs, amsmath,amsfonts}
\usepackage{mathtools}
\usepackage{graphicx}
\usepackage{enumitem}
\usepackage{braket}


\title{Problem Set 99}
\author{CSE 468}
\date{May 2021}

\begin{document}

\maketitle



\begin{enumerate}[font=\bfseries]

    \item Note: Questions not currently used in any of the problem sets. Some are too hard, some are not relevant enough to course, etc. No solutions given at this time.
    
    \item Consider the controlled U gate. Or look at this reference: \[https://en.wikipedia.org/wiki/Quantum_logic_gate#Controlled_gates\]
    \[\begin{pmatrix}
    1 & 0 & 0 & 0 \\
    0 & e^{i\gamma}\cos{\frac{\theta}{2}} & 0 &  -e^{i(\gamma+\lambda)}\sin{\frac{\theta}{2}}\\
    0 & 0 & 1 & 0 \\
    0 & e^{i(\gamma+\phi)}\sin{\frac{\theta}{2}} & 0 &  e^{i(\gamma+\phi+\lambda)}\cos{\frac{\theta}{2}}\\
    \end{pmatrix}
    \]
    For what values of $\gamma,\phi,\lambda,$ and $\theta$ can we use the controlled U gate to cause entanglement? Answer: https://quantumcomputing.stackexchange.com/questions/11445/does-controlled-u-gate-entangle-qubits?rq=1
    \item Consider $\ket{\psi} = \mathbf{CZ}(\ket{q_0q_1})$ where $\mathbf{CZ}$ is the same controlled Z gate as above. Give values for $q_0$ and $q_1$ that cause $\ket{\psi}$ to be entangled.
    \item What is the relationship between the factorability of a 2 qubit gate and that same gate's ability to cause entanglement? (I'm not sure)
    \item Bell states span the 2 qubit space (even for unentangled 2 qubit systems)
    \item Fun negative probabilities question
    \item From PS1:
    \begin{itemize}
        \item Give the matrix that     describes the polarizing beam splitter. 
        \item Using your result from (c) describe the state of your system at each step.
        \item Give the matrix that describes the sugar solution.
        \item Using your result from (c) describe the state of your system at each step.
        \item Your good friend, Niels Bohr, needs your help with a quantum experiment. Bohr has set up a complicated quantum system to make some computation. The problem is Bohr doesn’t know how to figure out what the actual output of his system is. Your task is, given Bohr’s system, describe how to make a measurement of either 0 or 1. For this question, a photon from the source we say begins in state 0 and is 0 degree relative to the source. A photon rotated 90 degrees from the source we say is in state 1. Assume Bohr’s result is either completely 0 or completely 1. 
    \end{itemize}
    \item In E91 protocol what happens when A and B's bases are not $pi/8$ apart, ie how does this affect the probabilities of agreement/disagreement. This also has implications for CHSH
    \item Fun mixed state questions also in relation to the Bloch sphere
\end{enumerate}



\end{document}