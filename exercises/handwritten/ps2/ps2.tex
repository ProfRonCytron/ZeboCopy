\documentclass[12pt]{article}
\usepackage{amssymb,mathrsfs, amsmath,amsfonts}
\usepackage{mathtools}
\usepackage{graphicx}
\usepackage{enumitem}
\usepackage{braket}
%%
%%
\newcommand{\Blank}{\mbox{\hskip 4pt\vrule width 1in depth 2pt}\vrule width 0pt height 2.0em}
\newcommand{\NameBlank}{\mbox{\hskip 4pt\vrule width 2.5in depth 2pt}\vrule width 0pt height 2.0em}
\newcommand{\BlankLine}{\mbox{\hskip 4pt\vrule width 5.5in depth 2pt}\vrule width 0pt height 2.0em}
%%
%% Leave at least #1 space, default to what is below
%%
\def\DefaultSpace{1in}
\newcommand{\LeaveSpace}[1][\DefaultSpace]{%
\vskip #1 plus 1fil\relax\hbox to 0pt{\hss} %
}

\title{Problem Set 2}
\author{CSE 468}
\date{\today}

\begin{document}

\maketitle

\noindent Name:\NameBlank{} \newline
\noindent Student ID:\NameBlank{} \newline
\textbf{Note:} You may discuss these problems with other students, but you must write your own solutions. If you run out of space for any question, please use the additional page and clearly indicate which question you are answering.
\begin{enumerate}[font=\bfseries]
    \item (20 points) Given $\ket{\psi} = \alpha\ket{0} + \beta\ket{1}$.
    \begin{enumerate}
        \item What is the state of $\ket{\psi_0} = \mathbf{X}\mathbf{Y}\mathbf{Z}\ket{\psi}$? Give the statevector and show how you arrived at your answer. \LeaveSpace[0.4in]
        \item Describe each of the above operations as rotations on the Bloch sphere in the order they would occur.\LeaveSpace{}
        \item What is the state of $\ket{\psi_1} = \mathbf{Z}\mathbf{Y}\mathbf{X}\ket{\psi}$? Give the statevector and show how you arrived at your answer. \LeaveSpace[0.4in]
        \item Does $\ket{\psi_0} = \ket{\psi_1}$? \Blank{}
        \item Is the probability of measuring $\ket{0}$ when we measure $\ket{\psi_0}$ equal to the probability of measuring $\ket{0}$ when we measure $\ket{\psi_1}$? Why or why not?\LeaveSpace{}
    \end{enumerate}
    \item (8 points) Give a unitary matrix $U$ that when we apply the operation $\mathbf{Y}U\ket{0}$ results in a $\frac{3}{4}$ chance of measuring $\ket{1}$. Show how you found $U$. \LeaveSpace[2.0in]
    \item (4 points) Consider $\ket{\psi} = \mathbf{X}^a\mathbf{Y}^b\mathbf{Z}^c\ket{0}$.
    \begin{enumerate}
        \item If $a,c$ are even and $b$ is odd, what is the probability of observing $\ket{0}$ when we measure $\ket{\psi}$? \Blank{}
        \item If $a,b,c$ are all even what is the probability of observing $\ket{1}$ when we measure $\ket{\psi}$? \Blank{}
    \end{enumerate}
    \item (5 points) If we begin at $\ket{0}$ on the Bloch sphere and can use any of the Pauli gates, how many unique (under rotation by the $\mathbf{Z}$ gate) locations can we travel to on the Bloch sphere? Explain your answer. \LeaveSpace{}
    \item (5 points) If we now also allow the use of the Hadamard gate, how many unique (under rotation by the $\mathbf{Z}$ gate) locations can we travel to on the Bloch sphere? Explain your answer. \LeaveSpace{}
    \item (20 points) Recall two points on the Bloch sphere are antipodal if they determine a line that passes through the center of the sphere. Recall that in polar coordinates, a quantum state $\ket{\psi}$ can be specified on the Bloch sphere using two coordinates, $\theta$ and $\phi$:
    \[\ket{\psi} = \cos(\frac{\theta}{2})\ket{0}
    + e^{i\phi}\sin(\frac{\theta}{2})\ket{1}\]
    \begin{enumerate}
        \item All points on the Bloch sphere denote unique (under rotation by the $\mathbf{Z}$ gate) quantum states except which two points? \Blank{} and\Blank{}
        \item The number of quantum states on the Bloch sphere is \emph{countably} infinite or \emph{uncountably} infinite? \Blank{}
        \item Given a state $\ket{\psi}$ defined as above, define its antipodal state $\ket{\psi'}$. \LeaveSpace{}
        \item Prove that $\ket{\psi}$ and $\ket{\psi'}$ are orthogonal by showing their inner product is 0. \LeaveSpace[2.0in]
    \end{enumerate}
    \item (5 points) Your friend, Albert Einstein, has challenged you to a quantum game. The game is played as follows: First, Einstein gives you some state $\ket{\psi}$ and tells you where it is located on the Bloch sphere by telling you $\theta$ and $\phi$. You then can apply any quantum gates you wish to $\ket{\psi}$. Before you make a measurement you tell Einstein what measurement you expect to observe. If you are right Einstein gives you \$100. If you are wrong Einstein zaps you with his many-worlds invention. Do you agree to play Einstein's game? Why or why not? \LeaveSpace{}
    \item (25 points) Consider the quantum states
        \[\ket{+y} = \frac{1}{\sqrt{2}}\ket{0} + \frac{i}{\sqrt{2}}\ket{1}\]
        and
        \[\ket{-y} = \frac{1}{\sqrt{2}}\ket{0} - \frac{i}{\sqrt{2}}\ket{1}\]
        What quantum gate (unitary matrix) $U$ produces those states given the input qubits $\ket{0}$ and $\ket{1}$, respectively?
        \begin{enumerate}
            \item Define $U$. \LeaveSpace[0.75in]
            \item Prove $U$ is unitary. \LeaveSpace[2.5in]
            \item Write $U$ as the outer product of $\ket{0}$, $\ket{1}$, $\ket{+y}$, and $\ket{-y}$. \LeaveSpace{}
            \item Show that $\ket{+y}$ and $\ket{-y}$ are eigenstates of the Pauli $\mathbf{Y}$ gate. \LeaveSpace{}
            \item More generally, what do the columns of such a 2 × 2 matrix (such as $U$) represent with respect to their inputs in the standard basis ($\ket{0}$ and $\ket{1}$)? \LeaveSpace{}
        \end{enumerate}
    \item (Bonus, up to 3 points) Write one interesting question related to the content of this homework, and indicate the correct answer. The question can be multiple-choice or free-response.  Interesting questions get credit here;  sufficiently good questions might appear on an exam.
\end{enumerate}
\newpage
\noindent \BlankLine{}
\BlankLine{}
\BlankLine{}
\BlankLine{}
\BlankLine{}
\BlankLine{}
\BlankLine{}
\BlankLine{}
\BlankLine{}
\BlankLine{}
\BlankLine{}
\BlankLine{}
\BlankLine{}
\BlankLine{}
\BlankLine{}
\BlankLine{}
\BlankLine{}
\BlankLine{}
\BlankLine{}
\BlankLine{}


\end{document}