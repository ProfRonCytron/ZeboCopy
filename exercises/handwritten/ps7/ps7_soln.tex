\documentclass[12pt]{article}
\usepackage{amssymb,mathrsfs, amsmath,amsfonts}
\usepackage{mathtools}
\usepackage{graphicx}
\usepackage{enumitem}
\usepackage{xurl}
\usepackage{braket}
\graphicspath{ {./ps6-assets/}{./exercises/handwritten/ps6/ps6-assets/} }

\title{Problem Set 7 Solutions}
\author{CSE 468}
\date{\today}

\begin{document}

\maketitle

\begin{enumerate}[font=\bfseries]
    \item Sheet here \url{https://docs.google.com/spreadsheets/d/1UB4THmSzfpqjHKYwGgE97xBkTdwS2FHdMxpnPe4qXqY/edit?usp=sharing} 

    \begin{center}
    \begin{tabular}{cc}
      10\% &  11  \\
       50\% & 27   \\
         90\% & 49   \\
          99\% & 69   \\
           99.9\% & 85   \\
    \end{tabular}
    \end{center}
    \item \ 
    \begin{enumerate}
        
     \item It's the xor of the two values, so \texttt{101010}
     \item None others, those are the only two
    \end{enumerate}
    \item The answer is \[\frac{\mathbf{H}(\ket{001}) + \mathbf{H}(\ket{100})}{\sqrt{2}}\] which I believe is
    \[ \frac{1}{2}\left(\ket{000}+\ket{010}-\ket{101}-\ket{111}\right)\]
    \item \ 
    \begin{enumerate}
    \item The secret is \texttt{01100110}
    \item Because this creates a balanced function, the amplitude on the all-zero basis vector will be $0$.
    \item All the amplitude will be on the secret's  basis vector $\ket{01100110}$
    \end{enumerate}
\end{enumerate}



\end{document}