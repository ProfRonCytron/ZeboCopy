\documentclass[12pt]{article}
\usepackage{amssymb,mathrsfs, amsmath,amsfonts}
\usepackage{mathtools}
\usepackage{graphicx}
\usepackage{enumitem}
\usepackage{braket}

\title{Problem Set 6 Solutions}
\author{CSE 468}
\date{May 2021}

\begin{document}

\maketitle

\begin{enumerate}[font=\bfseries]
    \item Placeholder
    \item \begin{enumerate}
        \item $\ket{-}$. Suppose $f(0) = f(1) = 0$
        \[\ket{\psi_2} = \ket{+} \otimes (\ket{0} - \ket{1}) = \ket{+-}\]
        Similarly obtained for $f(0) = f(1) = 1$. Don't forget to ignore global phase.
        Now suppose $f(0) = 0, f(1) = 1$.
        \[\ket{\psi_2} = \ket{-} \otimes (\ket{0} - \ket{1}) = \ket{--}\]
        Similarly obtained for $f(0) = 1, f(1) = 0$. Don't forget to ignore global phase.
        \item 50\% time see $\ket{0}$ and 50\% time see $\ket{1}$
        \item $\ket{1}$
    \end{enumerate}
    \item I first work through the analysis in Lecture 22 with $y=\ket{+}$.
    \[U_f = U_f(\ket{x}\otimes\ket{+}) = U_f(\ket{x}\otimes\frac{1}{\sqrt{2}}(\ket{0}+\ket{1}))\]
    \[=U_f(\frac{1}{\sqrt{2}}(\ket{x0}+\ket{x1})) = 
    \frac{1}{\sqrt{2}}(\ket{x(0\oplus f(x))}+ \ket{x(1\oplus f(x))})\]
    \[= \frac{1}{\sqrt{2}}(\ket{xf(x)}+\ket{x\overline{f(x)}})\]
    If we consider $f(x) = 0$ and $f(x) = 1$ (Deutsch problem) the outputs are identical and so we cannot distinguish between the two cases. More generally we do not have the form $U_f(\ket{x+}) = (-1)^{f(x)} \ket{x} \otimes \ket{+}$. And so the function result is not "kicked" onto the basis state which makes the phase kickback trick "useless" with $y = \ket{+}$ as the input. add more here?
\end{enumerate}



\end{document}