\documentclass[12pt]{article}
\usepackage{amssymb,mathrsfs, amsmath,amsfonts}
\usepackage{mathtools}
\usepackage{graphicx}
\usepackage{enumitem}
\usepackage{braket}
\usepackage{bbm}


\title{Problem Set 6}
\author{CSE 468}
\date{\today}
%%
%%
\newcommand{\Blank}{\mbox{\hskip 4pt\vrule width 1in depth 2pt}\vrule width 0pt height 2.0em}
\newcommand{\NameBlank}{\mbox{\hskip 4pt\vrule width 2.5in depth 2pt}\vrule width 0pt height 2.0em}
%%
%% Leave at least #1 space, default to what is below
%%
\def\DefaultSpace{1in}
\newcommand{\LeaveSpace}[1][\DefaultSpace]{%
\vskip #1 plus 1fil\relax\hbox to 0pt{\hss} %
}

\begin{document}

\maketitle

\noindent Name:\NameBlank{} \newline
\noindent Student ID:\NameBlank{} \newline
\textbf{Note:} You may discuss these problems with other students, but you must write your own solutions.

\begin{enumerate}[font=\bfseries]

    \item Note: Phase kickback, oracle, Deutsch questions, Deutsch Josza questions
    \item Consider the standard Deutsch problem (Lecture 21).
    \begin{enumerate}
        \item Remembering that global phases can be ignored, what is the state of the \emph{bottom} qubit, which originally went unmeasured? Be sure to analyze both the balanced and constant case.
        \item What would measure this qubit directly yield?
        \item If the bottom qubit were run through a Hadamard gate, what would measuring it yield?
    \end{enumerate}
    \item Consider the phase kickback trick studied in Lecture 22. Suppose the input for $y = \ket{+}$. Explain why the phase kickback trick no longer works by analysing the circuit in terms of the Deutsch problem as well as more generally.
    \item Consider the Deutsch-Josza problem. In this problem, we considered functions that were constant or balanced. We will now investigate how the D-J problem handles a function that isn't balanced or constant. Define a $k$-balanced function to be a function such that $|\sum_{w\in\{0,1\}^n} \mathbbm{1}{[f(w) = 0]} - \sum_{w\in\{0,1\}^n} \mathbbm{1}{[f(w) = 1]}| = k$ (|num of inputs that map to 0 - num of inputs that map to 1| = k). Should probably divide by some function of n.
    \begin{enumerate}
        \item Let's start with an example. Suppose $n = 2$ and $f(\ket{00}) = 0, f(\ket{01}) = 1, f(\ket{10}) = 1, f(\ket{11}) = 1$. Is $f$ $k$-balanced? If so, what is $k$?
        \item Find $\ket{\psi_2}$ of the D-J problem using this $f$.
        \item Find $\ket{\psi_3}$ of the D-J problem using this $f$.
        \item Conclusions?
        \item Move to general case?
        \item How many queries classically would it take to identify if $f$ is constant or $k$-balanced? (balanced or $k$-balanced?)
    \end{enumerate}
    \item (Bonus, up to 3 points) Write one interesting question related to the content of this homework, and indicate the correct answer. The question can be multiple-choice or free-response.  Interesting questions get credit here;  sufficiently good questions might appear on an exam.
\end{enumerate}



\end{document}