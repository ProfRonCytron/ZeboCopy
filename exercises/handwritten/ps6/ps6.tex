\documentclass[12pt]{article}
\usepackage{amssymb,mathrsfs, amsmath,amsfonts}
\usepackage{mathtools}
\usepackage{graphicx}
\usepackage{enumitem}
\usepackage{braket}
\usepackage{bbm}


\title{Problem Set 6}
\author{CSE 468}
\date{\today}
%%
%%
\newcommand{\Blank}{\mbox{\hskip 4pt\vrule width 1in depth 2pt}\vrule width 0pt height 2.0em}
\newcommand{\NameBlank}{\mbox{\hskip 4pt\vrule width 2.5in depth 2pt}\vrule width 0pt height 2.0em}
\newcommand{\BlankLine}{\mbox{\hskip 4pt\vrule width 5.5in depth 2pt}\vrule width 0pt height 2.0em}
%%
%% Leave at least #1 space, default to what is below
%%
\def\DefaultSpace{1in}
\newcommand{\LeaveSpace}[1][\DefaultSpace]{%
\vskip #1 plus 1fil\relax\hbox to 0pt{\hss} %
}


\begin{document}

\maketitle

\noindent Name:\NameBlank{} \newline
\noindent Student ID:\NameBlank{} \newline
\textbf{Note:} You may discuss these problems with other students, but you must write your own solutions. If you run out of space for any question, please use the additional page and clearly indicate which question you are answering.

\begin{enumerate}[font=\bfseries]
%% Note: Phase kickback, oracle, Deutsch questions, Deutsch Josza questions
    \item (4 points) Consider Deutsch's problem (Lecture~21), where $f(x)$ takes in a single bit.
    \begin{enumerate}
        \item Remembering that global phases can be ignored, what is the state of the \emph{bottom} qubit, which originally went unmeasured?
        \begin{itemize}
            \item In the balanced case \Blank{}
            \item In the constant case
            \Blank{}
        \end{itemize}
        \item If that bottom qubit is measured in the computational basis, what are the possible outcome(s)? \Blank{}
        \item If that bottom qubit is first passed through a Hadamard gate and then measured in the computational basis, what are the possible outcome(s)?\Blank{}
    \end{enumerate}
    \newpage
     \item (8 points) Consider the Deutsch--Jozsa problem. In that problem, we were \emph{promised} that functions were either constant or balanced. We will now investigate what happens when the promise is broken.  For an $n$-bit problem, recall the amplitude on the $\ket{0^{\star n}}$ term is:
     \[
       \frac{1}{2^n}\sum_{x\in \{0,1\}^{n}} -1^{(f(x))}
     \]
     So let's define
         \[
       k = \left|\frac{1}{2^n}\sum_{x\in \{0,1\}^{n}} -1^{(f(x))}\right|
     \]
     We then obtain:
     \begin{description}
         \item[k=0] Then $f(x)$ is balanced
         \item[k=1] Then $f(x)$ is constant
     \end{description}
     What about other values of $k$?
    \begin{enumerate}
        \item Let's start with an example for $n = 2$ qubits:
        \begin{center}
            \begin{tabular}{c|c}
                 $\ket{x}$&$f(\ket{x})$  \\\hline
                 $\ket{00}$ & 0 \\
                 $\ket{01}$ & 1 \\
                 $\ket{10}$ & 1 \\
                 $\ket{11}$ & 1 \\
            \end{tabular}
        \end{center}
        What is $k$ for this example?\Blank{}
        \item At state~$\psi_{3}$, what amplitude is present on
        \begin{itemize}
            \item $\ket{00}$\Blank{}
             \item $\ket{01}$\Blank{}
              \item $\ket{10}$\Blank{}
               \item $\ket{11}$\Blank{}
        \end{itemize}
        \item What is the probability of seeing $\ket{00}$ on the first measurement?\Blank{}
        \item What is the probability of \emph{not} seeing $\ket{00}$ on the first measurement?\Blank{}
        \item Suppose on your first measurement you observe $\ket{01}$.  So you believe the function is balanced.  For this running example, how many more runs of the circuit must you make to be 90\% certain that $f(x)$ is keeping its promise? \Blank{}
        
        Show your work:
        \LeaveSpace{}
    \end{enumerate}

\end{enumerate}
\newpage
\noindent \BlankLine{}
\BlankLine{}
\BlankLine{}
\BlankLine{}
\BlankLine{}
\BlankLine{}
\BlankLine{}
\BlankLine{}
\BlankLine{}
\BlankLine{}
\BlankLine{}
\BlankLine{}
\BlankLine{}
\BlankLine{}
\BlankLine{}
\BlankLine{}
\BlankLine{}
\BlankLine{}
\BlankLine{}
\BlankLine{}


\end{document}