\documentclass[12pt]{article}
\usepackage{exercises/handwritten/handout}

\usepackage{amssymb,mathrsfs, amsmath,amsfonts}
\usepackage{mathtools}
\usepackage{graphicx}
\usepackage{enumitem}
\usepackage{braket}
\usepackage{hyperref}
\graphicspath{ {./ps040-assets/} }
\input{quantummacros.tex}
%%
%%
\newcommand{\Blank}{\mbox{\hskip 4pt\vrule width 1in depth 2pt}\vrule width 0pt height 2.0em}
\newcommand{\NameBlank}{\mbox{\hskip 4pt\vrule width 2.5in depth 2pt}\vrule width 0pt height 2.0em}
\newcommand{\BlankLine}{\mbox{\hskip 4pt\vrule width 5.5in depth 2pt}\vrule width 0pt height 2.0em}
%%
%% Leave at least #1 space, default to what is below
%%
\def\DefaultSpace{1in}
\newcommand{\LeaveSpace}[1][\DefaultSpace]{%
\vskip #1 plus 1fil\relax\hbox to 0pt{\hss} %
}


\begin{document}

\assignment{Problem Set 040}

\begin{quote}
    As for all written assignments, you must upload your completed work to GradeScope.

    You are welcome to work collaboratively on solving these problems, but you each must write or type out your own solution, and upload that to GradeScope.

    Work submitted improperly will receive no credit.
\end{quote}


\noindent \textbf{Note:} Many of these problems use this link: \href{https://lab.quantumflytrap.com/lab}{https://lab.quantumflytrap.com/lab}. To learn more about any of the elements, place the element in your workspace and right-click on it.

\begin{enumerate}[font=\bfseries]
    \item With whom did you collaborate on this assignment?
    \LeaveSpace{} \Ans{Varies}
    \item (25 points) Create the following setup in your own workspace using the \href{https://lab.quantumflytrap.com/lab}{quantum flytrap site}.
    \[\includegraphics[scale=0.5]{easyFilter}\]
    \begin{enumerate}[label=\theenumi.\arabic*]
        \item Using only polarizing filters (the second element under the Polarization section) create a configuration such that the detector only detects 6$\%$ of the initial photons. What is the minimum number of polarizing filters required to achieve this? \Ans{4, each 45 degrees away from the next}
        \clearpage{}
        \item Describe a general strategy using only polarizing filters that causes the detector to detect only $n\%$ of the initial photons. You may assume $n \leq 100$ and $n = \frac{100}{2^k}$ with $k\in\mathbb{Z}^+$.\Ans{Use a sequence of $k$ filters with the initial filter rotated 45 degrees relative to the photon source and then rotate each additional filter by another 45 degrees.}
        \item We can use statevectors to represent the polarization of the photons. Let 
        \begin{itemize}
          \item $\ket{0} =\SQB{1}{0}$ represent photons completely oriented according to the horizontal axis
          \item $\ket{1} = \SQB{0}{1}$ represent photons completely oriented according to the vertical axis
        \end{itemize}
        If a photon is polarized with angle $\theta$ off the horizontal axis, then its state can be described as
        \[ \ket{\psi} = \cos{\theta}\ket{0} + \sin{\theta}\ket{1}\] 
        In terms of $\theta$, what is the probability that that photon will be measured as vertical?\Ans{$\sin^{2}{\theta}$}
       \item Consider the single-qubit gate
       \[ \PauliX = \XMatrix{} \]
       If the photon in state $\ket{\psi}$ passes through an \PauliX{} filter, then in terms
       of $\theta$, what is the probability it will be measured vertically?\Ans{$\cos^{2}{\theta}$}
        \item In terms of \QZero{} and \QOne{} defined above, the state of an \emph{unpolarized} photon can be described as:
        \[ \Ans{\RootTwo{}}\QZero{} + \Ans{\RootTwo{}}\QOne{} \]
    \end{enumerate}
    \item\label{qu:pbs} (25 points) Create the following setup:
    \[\includegraphics[scale=0.8]{beamSplit}\]
    Consider the polarizing beam splitter (the first element under the Polarization section). Be sure to right-click it to learn more about its behavior.
    
    Using only polarizing beam splitters and polarizing filters, create a configuration such that the probability of measuring a photon at Receiver A and the probability of measuring a photon at Receiver B are both greater than zero. Provide a sketch of your configuration.
    
    \Ans{Varies, but the idea is to change the source from vertically polarized (its natural state) into a superposition of vertical and horizontal, by placing a polarizing filter at 45 degrees prior to the polarizing beam splitter.}
    \[\includegraphics[scale=0.6]{polarizingbeamsplittersol}\]

    \item (25 points) Recreate the initial configuration of Question~\ref{qu:pbs}. Consider the sugar solution (the last element under the Polarization section).
    
     Using only sugar solutions and polarizing beam splitters create a configuration such that the probability of measuring a photon at Receiver A and the probability of measuring a photon at Receiver B are both greater than zero. Please sketch your configuration. \Ans{Varies, but the sugar solution will place the photons in a 45 degree polarization so the polarizing splitter will send some each way.}
        \[\includegraphics[scale=0.6]{sugarsol}\]

    \item (25 points) Consider the setup below.
    \[\includegraphics[scale=0.6]{rotate10}\]
    This website only allows us to rotate polarizing filters in increments of 45 degrees. Let's imagine that we are able to rotate polarizing filters only in increments of 10 degrees.

    With no two adjacent filters in the same orientation, what is the maximum number of
    polarizing filters could we place between the photon source and receiver and still expect to see at least 50\% of the initial photons in the receiver? \Blank{}
    
    Show your work for full credit.

   \Ans{We just need to figure out how much light passes through a filter of 10 degree offset.
    \[\cos^2({10^{\circ}}) \approx .97 \]
            \[(.97)^x = 0.5\]
            \[x = 22.75\]
            Should round down to 22 filters but also accept 22, 23, or 24 filters, because the first filter could be vertical and do nothing.}
   
    
\end{enumerate}
\end{document}
