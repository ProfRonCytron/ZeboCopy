\documentclass[12pt]{article}
\usepackage{exercises/handwritten/handout}

\usepackage{amssymb,mathrsfs, amsmath,amsfonts}
\usepackage{mathtools}
\usepackage{graphicx}
\usepackage{enumitem}
\usepackage{braket}
\usepackage{hyperref}
\graphicspath{ {./ps040-assets/} }
\input{quantummacros.tex}
%%
%%
\newcommand{\Blank}{\mbox{\hskip 4pt\vrule width 1in depth 2pt}\vrule width 0pt height 2.0em}
\newcommand{\NameBlank}{\mbox{\hskip 4pt\vrule width 2.5in depth 2pt}\vrule width 0pt height 2.0em}
\newcommand{\BlankLine}{\mbox{\hskip 4pt\vrule width 5.5in depth 2pt}\vrule width 0pt height 2.0em}
%%
%% Leave at least #1 space, default to what is below
%%
\def\DefaultSpace{1in}
\newcommand{\LeaveSpace}[1][\DefaultSpace]{%
\vskip #1 plus 1fil\relax\hbox to 0pt{\hss} %
}


\begin{document}

\assignment{Problem Set 040}

\begin{quote}
    As for all written assignments, you must upload your completed work to GradeScope.

    You are welcome to work collaboratively on solving these problems, but you each must write or type out your own solution, and upload that to GradeScope.

    Work submitted improperly will receive no credit.
\end{quote}


\noindent \textbf{Note:} Many of these problems use this link: \href{https://lab.quantumflytrap.com/lab}{https://lab.quantumflytrap.com/lab}. To learn more about any of the elements, place the element in your workspace and right-click on it.

\begin{enumerate}[font=\bfseries]
    \item With whom did you collaborate on this assignment?
    \LeaveSpace{}
    \item (20 points) Create the following setup in your own workspace using the \href{https://lab.quantumflytrap.com/lab}{quantum flytrap site}.
    \[\includegraphics[scale=0.5]{easyFilter}\]
    \begin{enumerate}[label=\theenumi.\arabic*]
        \item Using only polarizing filters (the second element under the Polarization section) create a configuration such that the detector only detects 6$\%$ of the initial photons. What is the minimum number of polarizing filters required to achieve this? \Blank{}
        \item Describe a general strategy using only polarizing filters that causes the detector to detect only $n\%$ of the initial photons. You may assume $n \leq 100$ and $n = \frac{100}{2^k}$ with $k\in\mathbb{Z}^+$.\LeaveSpace[2.0in]
        \item There is only one path in this configuration so we can use statevectors to represent the polarization of the photons. Let $\ket{0} = \begin{pmatrix}1 \\ 0 \end{pmatrix}$ represent photons completely oriented in the horizontal direction. Similarly, let $\ket{1} = \begin{pmatrix}0 \\ 1 \end{pmatrix}$ represent photons completely oriented in the vertical direction. Note we can characterize any state of these bases by a single parameter $\theta$ as follows: $\ket{\psi} = \cos{\theta}\ket{0} + \sin{\theta}\ket{1}$. Assume the photon source outputs photons completely oriented in the vertical direction. Suppose the matrix $M$ represents a polarizing filter oriented in the vertical direction. Answer the following questions.
        \begin{itemize}
            \item What does $M\ket{0} = $ \Blank{}
            \item What does $M\ket{1} = $ \Blank{} Give the matrix for $M = $\Blank{}
        \end{itemize}
        \item Given some state $\ket{\psi}$, what is the probability, in terms of $\theta$, that the photon will pass through a vertically oriented filter?\Blank{}
        \item Does applying a vertical filter, then a horizontal filter, and then a filter oriented 45 degrees from the origin produce the same result as applying a vertical filter, then a filter oriented 45 degrees from the origin, and then a horizontal filter? Why or why not?\LeaveSpace{}
    \end{enumerate}
    \item (6 points) Create the following setup:
    \[\includegraphics[scale=0.8]{beamSplit}\]
    Consider the polarizing beam splitter (the first element under the Polarization section). Be sure to right-click it to learn more about its behavior.
    \begin{enumerate}[label=\theenumi.\arabic*]
        \item Using only polarizing beam splitters and polarizing filters, create a configuration such that the probability of measuring a photon at Receiver A and the probability of measuring a photon at Receiver B are both greater than zero. Please sketch your configuration.\LeaveSpace[1.5in]
        \item Using only polarizing beam splitters and polarizing filters, can we ensure that 100\% of photons arrive at either Receiver A or Receiver B, given that the probability of measuring a photon at each must be greater than zero \Blank{}
    \end{enumerate}
    \item (9 points) Recreate the initial configuration of Question 2. Consider the sugar solution (the last element under the Polarization section).
    \begin{enumerate}[label=\theenumi.\arabic*]
        \item Using only sugar solutions and polarizing beam splitters create a configuration such that the probability of measuring a photon at Receiver A and the probability of measuring a photon at Receiver B are both greater than zero. Please sketch your configuration. \LeaveSpace[1.5in]
        \item Using only sugar solutions and polarizing beam splitters, can we ensure that 100\% of photons arrive at either Receiver A or Receiver B, given that the probability of measuring a photon at each must be greater than zero \Blank{} Briefly explain your reasoning. \LeaveSpace{}
    \end{enumerate}

    \item (3 points) Consider the setup below.
    \[\includegraphics[scale=0.6]{rotate10}\]
    This website only allows us to rotate polarizing filters in increments of 45 degrees. Let's imagine that we are able to rotate polarizing filters in increments of 10 degrees.

  Below we  require that consecutive filters have different rotations, as repeating a filter in the same rotation has no effect.
  
   How many polarizing filters could we place between the photon source and receiver and still expect to see at least 50\% of the initial photons in the receiver? \Blank{} 
   
   
    
\end{enumerate}
\end{document}
