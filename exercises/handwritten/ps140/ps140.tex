\documentclass[12pt]{article}
\usepackage{exercises/handwritten/handout}

\usepackage{amssymb,mathrsfs, amsmath,amsfonts}
\usepackage{mathtools}
\usepackage{graphicx}
\usepackage{enumitem}
\usepackage{braket}
\graphicspath{ {./ps110-assets/} }
\input{quantummacros.tex}
%%
%%

\newcommand{\NameBlank}{\mbox{\hskip 4pt\vrule width 2.5in depth 2pt}\vrule width 0pt height 2.0em}
\newcommand{\BlankLine}{\mbox{\hskip 4pt\vrule width 5.5in depth 2pt}\vrule width 0pt height 2.0em}
%%
%% Leave at least #1 space, default to what is below
%%
\def\DefaultSpace{1in}
\newcommand{\LeaveSpace}[1][\DefaultSpace]{%
\vskip #1 plus 1fil\relax\hbox to 0pt{\hss} %
}

\begin{document}

\assignment{Problem Set 140}

\begin{quote}
    As for all written assignments, you must upload your completed work to GradeScope.

    You are welcome to work collaboratively on solving these problems, but you each must write or type out your own solution, and upload that to GradeScope.

    Work submitted improperly will receive no credit.
\end{quote}

\begin{quote}\bf
You are welcome to use Matlab to help solve these problems or to check your solutions.  Where you are asked to show your work, you can paste the Matlab output.
\end{quote}

The problems below recall the Mermin--Peres square \emph{as taught in class} (slide decks~130 and~140).

Throughout this homework:
\begin{itemize}
    \item Alice is assigned row~3.
    \item Bob is assigned column~1.  
    \item Alice happens to measure first.
    \end{itemize}
\begin{enumerate}[font=\bfseries]
\item \Points{0} (negative points if you don't fill this out!)
\begin{itemize}
    \item Your name?
    \item Your student ID?
    \item With whom did you collaborate?
    \LeaveSpace{}]
\end{itemize}
    \item \Points{10} Referring to slide deck 140, slide 13, where Alice is assigned row 3, suppose she measures first, and the state of her two qubits collapses to \[ \psi_{1}=\frac{\ket{0+}-\ket{1-}}{\sqrt{2}} \]
    What is the state of Bob's qubits? \Blank[2in]
    
    \item \Points{10} In that state, suppose Bob is assigned column~1.  What two outcome(s) might he see for his \textbf{Z} (first qubit) measurement?
    
    \Blank[2in]
    \item \Points{10} And what outcome(s) might he see for his \textbf{X} (second qubit) measurement?
    
    \Blank[2in]
    
    \item \Points{10} Are all combinations of outcomes possible?  \Blank{}.  If not, say which are possible below
    
    \Blank[2in]{}
    \item \Points{10} Referring to Slide 14, what two possible states ($\psi_{?}$ or $\psi_{?}$) could Bob see when he measures after Alice?
    
    \Blank{} or \Blank{}
    \item \Points{30} Assuming Bob measures $\ket{0}$ for his first qubit and $\ket{+}$ for his second qubit, what results ($\pm 1$) does Bob report for his
    \begin{itemize}
        \item Top square \Blank{}
        \item Middle square \Blank{}
        \item Bottom square \Blank{}
    \end{itemize}
    \item \Points{10} What value must Alice have reported for the first square in her row in this scenario? \Blank{}
    \item \Points{10} What information on Slide 14 assures us that Bob will report the same result for the square they have in common (bottom left square) when he measures second?
    
\end{enumerate}

\end{document}