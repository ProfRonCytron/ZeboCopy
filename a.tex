\begin{frame}{No cloning theorem}

\TwoUnequalColumns{0.35\textwidth}{0.65\textwidth}{%
\Vskip{-2.2em}\alert<3>{For $n\geq 2$}
\adjustbox{scale=0.9,valign=t}{%
\begin{quantikz}
& \lstick{\alert<2>{\alert<2>{\QState{a}}}}\slice{\alert<8>{\QState{0}}} &  \gate[wires=6]{U}\slice{\alert<9>{\QState{1}}} &  \qw\rstick{\alert<5>{\QState{a}}}\\
\lstick[wires=5]{\alert<3>{$n-1$}} & \lstick{\alert<3>{\QZero{}}} &   & \qw\rstick{\alert<3,6>{\QState{a}}}  \\
& \lstick{\alert<4>{\QZero{}}} &   & \qw\rstick[wires=4]{\alert<7>{$f(\QState{a})$}}  \\
& \lstick{\alert<4>{$\bullet$}} &   & \qw  \\
& \lstick{\alert<4>{$\bullet$}} &   & \qw  \\
& \lstick{\alert<4>{\QZero{}}} &   & \qw 
\end{quantikz}%
}
\only<8-12>{%
\SmallSkip{}
\visible<8->{ $\alert<8>{\QState{0}} =  \ket{\QName{a}\,0^{n-1}}$}
\visible<9-12>{%
     $\alert<9>{\QState{1}}=\TensProd{\QState{a}}{\TensProd{\QState{a}}{f(\QState{a})}}$}
}%
\only<22->{%
Thus, if any $U$ can clone \ket{0} and \ket{1} then it cannot clone \ket{+}.
\QED{}
}
}{%
\only<1-4>{%
\Vskip{-3em}\begin{itemize}
    \item<1-> We seek to prove that \emph{no} circuit can accomplish cloning.
    We therefore formulate a parameterized proof using $n$, the number of inputs (and outputs) to a hypothetical gate $U$.
    \item<2-> The top input is \QState{a}, the state we wish to clone.
    \item<3-> We need at least one extra qubit to receive the putative clone.
    \item<4-> The rest of the qubits are ancillary qubits, or \href{https://en.wikipedia.org/wiki/Ancilla_bit}{ancillas}.   These are offered to $U$ as working storage, with no restrictions on their use.  Providing no information, they can be initialized to \ket{0}.
\end{itemize}}%
\only<5-7>{%
\Vskip{-3em}\begin{itemize}
    \item<5-> For output, the top qubit remains \QState{a}.
    \item<6-> The next qubit is the clone, so it is also in state \QState{a}.
    \item<7-> The rest of the qubits are in a state computed by $U$ as some arbitrary function of its input \QState{a}. 
\end{itemize}
}%
\only<10-17>{%
\Vskip{-2em}If $U$ can do the following
\begin{itemize}
    \item \textcolor<15>{OrangeRed}{$U(\TensProd{\ket{0}}{\ket{0^{n-1}}})= \TensProd{\ket{00}}{f(\ket{0})}$}
    \item \textcolor<16>{NavyBlue}{$U(\TensProd{\ket{1}}{\ket{0^{n-1}}})= \TensProd{\ket{11}}{f(\ket{1})}$}
\end{itemize}
\only<10>
then when presented with $\QState{a}=\ket{+}=\frac{\ket{0}+\ket{1}}{\sqrt{2}}$
\Vskip{-2em}\begin{align*}
\visible<12->{U(\ket{+})=& \RootTwo{}U(\TensProd{(\ket{0}+\ket{1})}{\ket{0^{n-1}}})} \\
\visible<13->{\sqrt{2}\,U(\ket{+})= & U(\TensProd{\ket{0}}{\ket{0^{n-1}}}+\TensProd{\ket{1}}{\ket{0^{n-1}}})} \\
\visible<14->{ = & 
\textcolor<15>{OrangeRed}{U(\TensProd{\ket{0}}{\ket{0^{n-1}}})} + \textcolor<16>{NavyBlue}{U(\TensProd{\ket{1}}{\ket{0^{n-1}}})}} \\
\visible<15->{ = &
\textcolor<15>{OrangeRed}{\TensProd{\ket{00}}{f(\ket{0})}} +
\visible<16->{\textcolor<16>{NavyBlue}{\TensProd{\ket{11}}{f(\ket{1})}
}}} \\
\visible<17->{%
U(\ket{+}) = & 
\frac{\left(\TensProd{\ket{00}}{f(\ket{0}})\right) 
+ \left(\TensProd{\ket{11}}{f(\ket{1}}\right)}{\sqrt{2}}
}
\end{align*}
}
}%
\only<18->{%
\Vskip{-3em}\[
U(\ket{+}) = 
\frac{\left(\TensProd{\alert<19>{\ket{00}}}{f(\ket{0}})\right) 
+ \left(\TensProd{\alert<19>{\ket{11}}}{f(\ket{1}}\right)}{\sqrt{2}}
\]
\begin{itemize}
    \item<19-> If we measure the top two qubits, we see \ket{00} and \ket{11} with equal probability.
    \item<20-> The state of the top two qubits is thus $\RootTwo{}\DQB{1}{0}{0}{1}$.
    \item<21-> But this is not $\ket{++}=\RootTwo{}\DQB{1}{1}{1}{1}$
    
\end{itemize}
}
}

    
\end{frame}