\SetTitle{4}{Physics in support of quantum effects}{So you become a believer}{physics}

\section{Superposition}

\subsection{Polarized light}
\begin{frame}{Polarized light}{Experiments you can do!}
\begin{itemize}
    \item The experiments described here can be carried out using any polarizing filter.
    \item Examples include
    \begin{description}
        \item[sunglasses] These are polarized to cut down on reflected light from surfaces such as water.  That reflected light is typically polarized in horizontally, so these sunglasses admit only vertically polarized light.
        \item[computer screen]  If you are watching these slides on an \href{https://en.wikipedia.org/wiki/Liquid-crystal_display}{LCD computer screen}, then the light reaching your eyeballs is (most likely linearly) polarized.
    \end{description}
    \item You can thus do these experiments with your computer screen and a pair of sunglasses, if both are linearly polarized.
\end{itemize}
\end{frame}

\begin{frame}{A bad theory for polarization}{Based on the classical world}
\TwoColumns{%
\Vskip{-3em}\begin{itemize}
    \item<1-> Light is a collection of waves, each due to one \href{https://en.wikipedia.org/wiki/Photon}{photon}.
    \item<2-> Unpolarized light might then be many photons, each with an arbitrary angle of polarization.
    \item<3-> What happens the photons strike a filter, as shown here?
    \item<4-> Only those waves polarized vertically, or nearly so, would pass through. 
\end{itemize}
}{%
\only<2->{%
\Vskip{-3em}\begin{center}
\begin{TIKZP}
\RadiantArrows{22.5}{->,color=BrickRed,thick}
\end{TIKZP}
\end{center}
}%
\only<3->{%
\begin{center}
\begin{TIKZP}[overlay]
\PFilter{-1}{0.0}{1}{2.5}{20}
\end{TIKZP}
\end{center}
}%
\only<4->{%
\begin{center}
\begin{TIKZP}[scale=0.5]

  \draw[<->,thick,color=BrickRed] (0,-1) -- (0,1);

\end{TIKZP}
\end{center}
}
}
\OnlyRemark{5}{If this is true, we should see a very small fraction of the light passing through. But we see $\frac{1}{2}$ of the light passing through.  How can this be?}
\end{frame}

\begin{frame}{Polarizing filters in succession}{Observations about this experiment}
\TwoColumns{%
\begin{itemize}
    \item<1-> Start with unpolarized light
    \item<2-> Impose a \textcolor{NavyBlue}{vertically polarizing filter}.
    \item<3-> Half of the light passes through.
    \item<4-> Impose a second \textcolor{OliveGreen}{identical filter}.
    \item<5-> No light is filtered out this time.
    \item<6-> Impose \textcolor{orange}{another filter} at \Degrees{45}.
    \item<7> Half of the available light passes through, now polarized according to this \textcolor{orange}{last filter}.
\end{itemize}
}{%
\begin{center}
\begin{TIKZP}
\visible<1-2>{\RadiantArrows{22.5}{->,color=BrickRed,thick}}
\visible<2>{\begin{scope}[draw=NavyBlue]\PFilter{-1}{-1.2}{1}{1.2}{20}\end{scope}}
\visible<3-4,5,6>{%
\foreach \x in {-1, -0.75,...,1} {
\draw[<->,thick,color=BrickRed] (\x,-1) -- (\x,1);
}
}
\visible<4>{\begin{scope}[draw=OliveGreen]\PFilter{-1}{-1.2}{1}{1.2}{20}\end{scope}}

\visible<6>{\begin{scope}[draw=orange,rotate=45]\PFilter{-1}{-1.2}{1}{1.2}{20}\end{scope}}
\visible<7>{%
\begin{scope}[rotate=45]
\foreach \x in {-1, -0.5,...,1} {
\draw[<->,thick,color=BrickRed] (\x,-1) -- (\x,1);
}
\end{scope}
}
\end{TIKZP}
\end{center}
}
\end{frame}



\begin{frame}{Orthogonal filters}{They can eliminate all of the light.}
\TwoColumns{%
\begin{itemize}
    \item<1-> Start with unpolarized light
    \item<2-> Impose a \textcolor{NavyBlue}{vertically polarizing filter}.
    \item<3-> Half of the light passes through.
    \item<4-> Add a second \textcolor{OliveGreen}{identical filter}, behind the first one, rotated \Degrees{90}.
    \item<5-> No light passes through.  We should expect this.
\end{itemize}
}{%
\begin{center}
\begin{TIKZP}
\visible<1-2>{\RadiantArrows{22.5}{->,color=BrickRed,thick}}
\visible<2,4>{\begin{scope}[draw=NavyBlue]\PFilter{-1}{-1.2}{1}{1.2}{20}\end{scope}}
\visible<3-4>{%
\foreach \x in {-1, -0.75,...,1} {
\draw[<->,thick,color=BrickRed] (\x,-1) -- (\x,1);
}
}
\visible<4>{\begin{scope}[draw=OliveGreen,rotate=90]\PFilter{-1}{-1.2}{1}{1.2}{20}\end{scope}}
\end{TIKZP}
\end{center}
}
\end{frame}

\begin{frame}{How much light passes through two filters?}{Getting quantitative}
\TwoUnequalColumns{.70\textwidth}{.30\textwidth}{%
\begin{itemize}
    \item<1-> When the filters align, so the angle between them is~\Degrees{0}, there is no change in the light passed through the second filter.
    \item<2-> When the filters are orthogonal, so the angle between them is~\Degrees{90}, no light passes through the second filter.
    \item<3-> If you experiment with various angles and measure the fraction of light that enters the first filter and passes through the second, you will see that it is related to $\cos^{2}\theta$, where $\theta$ is the angle between the filters' relative orientations.\LinkArrow{https://en.wikipedia.org/wiki/Polarizer\#Malus's_law_and_other_properties}
\end{itemize}
}{%
\begin{center}
\begin{TIKZP}
\visible<1-3>{\begin{scope}[draw=NavyBlue]\PFilter{-1}{-1.2}{1}{1.2}{20}\end{scope}
\draw[->,thick] (0,0) -- (0,1) ;
}
\visible<2>{\begin{scope}[draw=OliveGreen,rotate=90]\PFilter{-1}{-1.2}{1}{1.2}{20}\end{scope}
\draw[->,thick] (0,0) -- (0,1) ;
\draw[->,thick] (0,0) -- (-1,0);
}
\visible<3>{\begin{scope}[draw=Sepia,rotate=70]\PFilter{-1}{-1.2}{1}{1.2}{20}\end{scope}
\draw[->,thick] (0,0) -- (0,1);
\draw[->,thick] (0,0) -- (160:1);
\draw (0,0) node[above left] {$\theta$};
}
\end{TIKZP}
\end{center}
}
    
\end{frame}

\begin{frame}{And now a surprise}{A filter can actually cause light to emerge.}
\TwoUnequalColumns{0.65\textwidth}{0.35\textwidth}{%
\begin{itemize}
    \item<1-> Start with unpolarized light
    \item<2-> Impose a \textcolor{NavyBlue}{vertically polarizing filter}.
    \item<3-> Half of the light passes through.
    \item<4-> Add a second \textcolor{OliveGreen}{identical filter}, behind the first one, rotated \Degrees{90}.
    \item<5-> No light passes through.  We should expect this.
    \item<6-> Impose \textcolor{orange}{another filter} at \Degrees{45} in back of the other two filters.
    \item<7> Still no light passes through.  It was all absorbed by the second filter.
\end{itemize}
}{%
\begin{center}
\begin{TIKZP}
\visible<1-2>{\RadiantArrows{22.5}{->,color=BrickRed,thick}}
\visible<2,4,6>{\begin{scope}[draw=NavyBlue]\PFilter{-1}{-1.2}{1}{1.2}{20}\end{scope}}
\visible<3-4>{%
\foreach \x in {-1, -0.75,...,1} {
\draw[<->,thick,color=BrickRed] (\x,-1) -- (\x,1);
}
}
\visible<4,6>{\begin{scope}[draw=OliveGreen,rotate=90]\PFilter{-1}{-1.2}{1}{1.2}{20}\end{scope}}

\visible<6>{\begin{scope}[draw=orange,rotate=45]\PFilter{-1}{-1.2}{1}{1.2}{20}\end{scope}}
\end{TIKZP}
\end{center}
}
\end{frame}

\begin{frame}{Double-slit experiment}
\end{frame}