\SetTitle{10}{Teleportation}{Sending a qubit state requires a phone call}{10}

\section*{Overview}
\begin{frame}{Overview}
    \begin{itemize}[<+->]
        \item We will show how to use entangled states to send an arbitrary qubit state from Alice to Bob.
        \item Amazingly, there are no wires or other communication fabric between Alice and Bob.  The state is truly teleported from one location to another.
        \item Recalling the no-cloning theorem, we should not be surprised that Alice will no longer have the state.
        \item Alice and Bob can be arbitrarily far apart, but actions involving entanglement take effect immediately.
        \item However, it turns out Alice has to communicate two (classical) bits of information to Bob for him to truly obtain Alice's state at his location.  That information cannot travel faster than the speed of light.
    \end{itemize}
\end{frame}

\section*{Partial Measurement}

\begin{frame}{Partial measurements}{What happens when some bits are measured, but not all?}
\begin{itemize}[<+->]
    \item Consider 3-qubit quantum system
    \begin{align*}
  \visible<6->{\alert{k} \times }\QState{\relax}
   = & \visible<1-5>{\alpha_{0} \ket{000}
  +  \alpha_{1} \ket{001}
  +  \alpha_{2} \ket{010}
  +  \alpha_{3} \ket{011} }\\
  +  & \alpha_{4} \ket{100}
  +  \alpha_{5} \ket{101}
 \visible<1-5>{ +  \alpha_{6} \ket{110}
  +  \alpha_{7} \ket{111}}
    \end{align*}
    \item We must have $\sum_{i=0}^{7} \Prob{\alpha_{i}} = 1$
    \item What happens if we measure the first two qubits? 
    \item We obtain \ket{00}, \ket{01}, \ket{10}, or \ket{11} for those qubits.
    \item What about the third qubit?  We must view its probability now as conditional on the outcome of the other two qubits.
    \item Suppose the first two bits measure \ket{10}, then six of the 8 eight terms must vanish, and we must renormalize the system using $\alert{k}=\Prob{\alpha_4} + \Prob{\alpha_5}$. 
\end{itemize}

    
\end{frame}

\begin{frame}{Partial measurement}{Example}

\Vskip{-4em}\begin{align*}
  \QState{\relax}
   = & \alpha_{0} \ket{000}
  +  \alpha_{1} \ket{001}
  +  \alpha_{2} \ket{010}
  +  \alpha_{3} \ket{011} \\
  +  & \alpha_{4} \ket{100}
  +  \alpha_{5} \ket{101}
  +  \alpha_{6} \ket{110}
  +  \alpha_{7} \ket{111} \\
  \Prob{\alpha_{4}} & =0.250 \\
       \Prob{\alpha_{5}} & = 0.125
    \end{align*}
\Vskip{-3em}\begin{itemize}
   \item The sum of the other amplitudes' probabilities is~$0.625$.
   \item A partial measurement of the first two qubits as \ket{10} collapses the system into the state
   \[
   \frac{\alpha_{4}\ket{100} + \alpha_{5}\ket{101}}{0.375}
   \]
   \item We subsequently measure \ket{100} with probability $\frac{250}{375}=\frac{2}{3}$ and \ket{101} with probability $\frac{1}{3}$
   \item As in the system before partial measurement, we are still twice as likely to see~\ket{100} as~\ket{101}.
\end{itemize}
    
\end{frame}


\section*{Teleportation}

\begin{frame}{Harnessing entanglement}{Achieving teleportation}
\Vskip{-3.5em}\TwoUnequalColumns{0.45\textwidth}{0.55\textwidth}{%
\only<1-2>{%
\adjustbox{valign=t}{\begin{quantikz}
\lstick{\QZero{}} &  \gate{H}& \ctrl{1}\slice{\QState{1}} & \qw\rstick{\QState{A}} \\
\lstick{\QZero{}} &   \qw  & \targ{} & \qw\rstick{\QState{B}}
\end{quantikz}}}%
\only<3>{%
\adjustbox{valign=t}{\begin{quantikz}
\lstick{\QZero{}} &  \gate{H}& \ctrl{1}\slice{\QState{1}} & \qw\rstick{\QState{A}} \\[2em]
\lstick{\QZero{}} &   \qw  & \targ{} & \qw\rstick{\QState{B}}
\end{quantikz}}}%
\only<4->{%
\adjustbox{valign=t,scale=0.95}{\begin{quantikz}
\lstick{\QZero{}} &  \gate{H}& \ctrl{1}\slice{\QState{1}} & \gate{U}\slice{\QState{2}} & \qw  \rstick{\QState{A}}\\[2em]
\lstick{\QZero{}} &   \qw    &  \targ{}  & \qw & \qw  \rstick{\QState{B}}
\end{quantikz}}}%
\only<4-10>{
\Vskip{4em}}%
\only<11->{%

\BigSkip{}
{\small\setlength{\tabcolsep}{2pt}
\begin{tabular}{r@{\mbox{ }}rlcrl}
\QState{A} & \multicolumn{5}{l}{\QState{B}} \\
\ket{0} & \UGateA{}& \ket{0} & $+$ &  \UGateB{} & \ket{1} \\
\ket{1} & $\sin(\theta/2)$ & \ket{0} & 
   $+$ &  $\ExpPhase{\lambda}\cos(\theta/2)$ & \ket{1}
\end{tabular}}
}
}{%
\only<1-3>{%
\begin{itemize}
    \item<1-> We create the Bell state 
    \[\QState{1} = \RootTwo{}\DQB{1}{0}{0}{1} \]
    \item<2-> Alice takes the top qubit and Bob takes the bottom one.
    \item<3-> They then separate, let's say by a very large distance.
\end{itemize}
}%
\only<4-6>{%
\begin{itemize}
    \item<4-> Alice effects a change to her qubit, represented here by \NamedGate{U}.
    \item<5-> Recall generally
    \[U = \UGate{} \]
\end{itemize}}%
\only<7-8>{%
\Vskip{-3em}\begin{align*}
    \QState{2} & = \left(\TensProd{\NamedGate{U}}{\NamedGate{I}}\right) \RootTwo\DQB{1}{0}{0}{1}\\
    & = \RootTwo{}\DQB{\UGateA{}}{\UGateB{}}{\UGateC{}}{\UGateD{}}
\end{align*}
}%
\only<9-12>{%
\begin{itemize}
    \item<9-> What happens to this state if Alice measures her qubit?
    \item<9-> Suppose she sees \alt<9-11>{\ket{0}}{\ket{1}}.
    \item<10-> Then the system collapses as shown.
    \item<11-> Bob's state is then
    \alt<11>{\[ \UGateA{}\ket{0} + \UGateB{}\ket{1} \]}{%
    \begin{align*}
    & = \UGateC{}\ket{0} + \UGateD{}\ket{1} \\
   & =  \ExpPhase{\phi}\left( \sin(\theta/2)\ket{0}
   + \ExpPhase{\lambda}\cos(\theta/2)\right) 
    \end{align*}
    }
\end{itemize}
}%
}
\only<6-7>{%
\small{%
\[
\TensProd{\NamedGate{U}}{\NamedGate{I}} =
\begin{pmatrix*}[r]
\UGateA{} & 0 & \UGateB{}  & 0 \\
0 & \UGateA{} & 0 & \UGateB{} \\
\UGateC{} & 0 & \UGateD{}  & 0 \\
0 & \UGateC{} & 0 & \UGateD{}
\end{pmatrix*}
\]}}%
\only<8-12>{%
{\setlength{\tabcolsep}{3pt}
\[
    \sqrt{2}\QState{2} = 
    \begin{tabular}{crlcrl} 
    \visible<8-11>{ & \UGateA{} &\ket{\alert<9>{0}0} & $+$
     &\UGateB{}&\ket{\alert<9>{0}1} \\} 
    \visible<8-9,12>{ $+$ & \UGateC{}& \ket{\alert<12>{1}0}& $+$
       & \UGateD{}& \ket{\alert<12>{1}1}}
       \end{tabular}
\]
}
}%
\only<13>{%
\begin{description}
    \item[Alice measures \ket{0}]
      Bob gets blah
     
    \item[Alex measures \ket{1}]
      Bob bets blah
\end{description}
}


\end{frame}

\begin{frame}{Quantum teleportation}{Sending a quantum state from Alice to Bob}

\TwoUnequalColumns{0.4\textwidth}{0.6\textwidth}{%
\Vskip{-2em}\adjustbox{scale=0.7,valign=t}{%
\begin{quantikz}
  \QState{x}&\qw\slice{\alert<4-5>{\QState{1}}} &  \ctrl{1}\slice{\alert<6-9>{\QState{2}}} &  \gate{H}\slice{\alert<10-12>{\QState{3}}}& \meter{} &\cw \rstick{\textcolor{Blue}{\ket{x}}}\\
\lstick[wires=2]{\stackbox{EPR\\pair}}
&\lstick{\QState{A}} & \targ{} & \qw & \meter{} & \cw \rstick{\textcolor{Blue}{\ket{y}}}\\[2.7em]
& \lstick{\QState{B}} & \qw & \qw & \qw& \qw
\end{quantikz}}
\only<1-4>{%
\Vskip{-1.5em}
\begin{align*}
\visible<1->{%
    \QState{x} &=\SQB{\alpha_{x}}{\beta_{x}} \\}
\visible<2->{%
    \ket{\QName{A}\QName{B}}&=\RootTwo{}\DQB{1}{0}{0}{1}}
\end{align*}}
\only<5-8>{%
{\small
\Vskip{-1.15em}\begin{align*}
    \alert<5>{\QState{1}} &=\RootTwo{}\QQB{\alpha_x}{0}{0}{\alpha_x}{\beta_x}{0}{0}{\beta_x}
\end{align*}}
}
\only<9-11>{%
{\small
\Vskip{-1.15em}\begin{align*}
    \alert<9>{\QState{2}} &= \RootTwo{}\QQB{\alpha_x}{0}{0}{\alpha_x}{0}{\beta_x}{\beta_x}{0}
\end{align*}}
}
\only<12-13>{%
{\small
\Vskip{-1.15em}\begin{align*}
    \alert<12>{\QState{3}} &= \frac{1}{2}\QQB{\alpha_x}{\beta_x}{\beta_x}{\alpha_x}{\alpha_x}{-\beta_x}{-\beta_x}{\alpha_x}
\end{align*}}
}
\only<14->{%
\begin{center}
\begin{tabular}{r|l}
\multicolumn{1}{c}{Alice measures} & \multicolumn{1}{c}{Bob applies} \\
  \visible<14->{\ket{xy}=\ket{00} & \only<14>{?}\only<15->{nothing}} \\
  \visible<16->{\ket{xy}=\ket{01}  & \only<16>{?}\only<17->{\PauliX{}}} \\
  \visible<18->{\ket{xy}=\ket{10} & \only<18>{?}\only<19->{\PauliZ{}} }\\
  \visible<20->{\ket{xy}=\ket{11} & \only<20>{?}\only<21->{\PauliX{} then \PauliZ{}}}
\end{tabular}
\end{center}
so that Bob has \QState{x}
}

}{%

\only<1-3>{%
    \begin{itemize}
    \item<1-> Alice wants to send $\QState{x}=\alpha_{x}\ket{0}+\beta_{x}\ket{1}$.
    \item<2-> Alice and Bob share the EPR pair \ket{\QName{A}\QName{B}}.
    \item<3-> Alice and Bob separate, perhaps at a great distance, with Alice keeping \QState{x} and \QState{A}, and Bob taking \QState{B} with him.
\end{itemize}
\only<2>{%
\MedSkip{}
Recall we create \ket{\QName{A}\QName{B}} as an EPR pair:

\adjustbox{valign=t}{\begin{quantikz}
\lstick{\QZero{}} &  \gate{H}& \ctrl{1} & \qw  \rstick{\QState{A}}\\
\lstick{\QZero{}} &   \qw    &  \targ{}  & \qw  \rstick{\QState{B}}
\end{quantikz}}

}}
\only<4-5>{%
\Vskip{-5em}\begin{align*}
    \QState{1} &= \TensProd{\SQB{\alpha_x}{\beta_x}}{\RootTwo{\DQB{1}{0}{0}{1}}} \\
    &= \RootTwo{}\QQB{\alpha_x}{0}{0}{\alpha_x}{\beta_x}{0}{0}{\beta_x}
\end{align*}
}
\only<6-7>{%
\Vskip{-4em}\begin{align*}
\TensProd{\NamedGate{CNOT}}{\Identity} & = 
\TensProd{\CNOTMatrix}{\IMatrix} \\
\visible<7>{%
&= \begin{pmatrix*}[r]
1 & 0 & 0 & 0 & 0 & 0 & 0 & 0 \\
0 & 1 & 0 & 0 & 0 & 0 & 0 & 0 \\
0 & 0 & 1 & 0 & 0 & 0 & 0 & 0 \\
0 & 0 & 0 & 1 & 0 & 0 & 0 & 0 \\
0 & 0 & 0 & 0 & 0 & 0 & 1 & 0 \\
0 & 0 & 0 & 0 & 0 & 0 & 0 & 1 \\
0 & 0 & 0 & 0 & 1 & 0 & 0 & 0 \\
0 & 0 & 0 & 0 & 0 & 1 & 0 & 0
\end{pmatrix*}}
\end{align*}
}
\only<8>{%
\Vskip{-4em}{\scriptsize\begin{align*}
\TensProd{\NamedGate{CNOT}}{\Identity} & = 
 \begin{pmatrix*}[r]
1 & 0 & 0 & 0 & 0 & 0 & 0 & 0 \\
0 & 1 & 0 & 0 & 0 & 0 & 0 & 0 \\
0 & 0 & 1 & 0 & 0 & 0 & 0 & 0 \\
0 & 0 & 0 & 1 & 0 & 0 & 0 & 0 \\
0 & 0 & 0 & 0 & 0 & 0 & 1 & 0 \\
0 & 0 & 0 & 0 & 0 & 0 & 0 & 1 \\
0 & 0 & 0 & 0 & 1 & 0 & 0 & 0 \\
0 & 0 & 0 & 0 & 0 & 1 & 0 & 0
\end{pmatrix*}
\end{align*}}
}
\only<8>{%
{\scriptsize
\Vskip{-1.5em}\begin{align*}
    \QState{2} &=\left(\TensProd{\NamedGate{CNOT}}{\Identity}\right)\QState{1} = \RootTwo{}\QQB{\alpha_x}{0}{0}{\alpha_x}{0}{\beta_x}{\beta_x}{0}
\end{align*}}
}
\only<9-10>{%
{\small
\Vskip{-4em}\begin{align*}
\TensProd{\TensProd{\Hadamard}{\Identity}}{\Identity}  = 
\TensProd{\TensProd{\HMatrix}{\IMatrix}}{\IMatrix} \\[1.2em]
\visible<10>{%
= \RootTwo{}\begin{pmatrix*}[r]
1 & 0 & 0 & 0 & 1 & 0 & 0 & 0 \\
0 & 1 & 0 & 0 & 0 & 1 & 0 & 0 \\
0 & 0 & 1 & 0 & 0 & 0 & 1 & 0 \\
0 & 0 & 0 & 1 & 0 & 0 & 0 & 1 \\
1 & 0 & 0 & 0 & -1 & 0 & 0 & 0 \\
0 & 1 & 0 & 0 & 0 & -1 & 0 & 0 \\
0 & 0 & 1 & 0 & 0 & 0 & -1 & 0 \\
0 & 0 & 0 & 1 & 0 & 0 & 0 & -1
\end{pmatrix*}}
\end{align*}}
}
\only<11>{%
{\scriptsize
\Vskip{-4em}\begin{align*}
\TensProd{\TensProd{\Hadamard}{\Identity}}{\Identity}  = 
 \RootTwo{}\begin{pmatrix*}[r]
1 & 0 & 0 & 0 & 1 & 0 & 0 & 0 \\
0 & 1 & 0 & 0 & 0 & 1 & 0 & 0 \\
0 & 0 & 1 & 0 & 0 & 0 & 1 & 0 \\
0 & 0 & 0 & 1 & 0 & 0 & 0 & 1 \\
1 & 0 & 0 & 0 & -1 & 0 & 0 & 0 \\
0 & 1 & 0 & 0 & 0 & -1 & 0 & 0 \\
0 & 0 & 1 & 0 & 0 & 0 & -1 & 0 \\
0 & 0 & 0 & 1 & 0 & 0 & 0 & -1
\end{pmatrix*}
\end{align*}}
}
\only<11>{%
{\scriptsize
\Vskip{-2.5em}\begin{align*}
    \QState{3} &=\left(\TensProd{\TensProd{\Hadamard}{\Identity}}{\Identity}\right)\QState{2} = \frac{1}{2}\QQB{\alpha_x}{\beta_x}{\beta_x}{\alpha_x}{\alpha_x}{-\beta_x}{-\beta_x}{\alpha_x}
\end{align*}}
}
\only<12>{%
Recall Alice had
\[
\QState{x} = \alpha_{x}\ket{0} + \beta_{x}\ket{1}
\]
so we must have \[ \Prob{\alpha_{x}} + \Prob{\beta{x}} = 1 \]
}
\only<13-21>{%
\Vskip{-5.5em}\begin{align*}
    \QState{3} = \visible<13>{\frac{1}{2}}\ [\\ \visible<13,14-15>{+ & \alpha_{x}\ket{\alert<14-15>{00}0}  + \beta_{x}\ket{\alert<14-15>{00}1}} \\
    \visible<13,16-17>{+ & \beta_{x}\ket{\alert<16-17>{01}0} + \alpha_{x}\ket{\alert<16-17>{01}1} }\\
     \visible<13,18-19>{+ &\alpha_{x}\ket{\alert<18-19>{10}0}  -\beta_{x}\ket{\alert<18-19>{10}1} }\\ 
     \visible<13,20-21>{- & \beta_{x}\ket{\alert<20-21>{11}0} + \alpha_{x}\ket{\alert<20-21>{11}1}}\\
    ]
\end{align*}
}
\only<14-21>{
\Vskip{-4em}}
\only<14-15>{%
\begin{align*}
\QState{B} & = \alpha_{x}\ket{0} + \beta_{x}\ket{1} \\
\visible<15>{& = \QState{x}}
\end{align*}
}
\only<16-17>{%
\begin{align*}
\QState{B}  & = \beta_{x}\ket{0} + \alpha_{x}\ket{1} \\
\visible<17>{
\PauliX{}\QState{B} & = \alpha_{x}\ket{0} + \beta_{x}\ket{1} \\
 & = \QState{x}}
\end{align*}
}
\only<18-19>{%
\begin{align*}
\QState{B}  & = \alpha_{x}\ket{0} - \beta_{x}\ket{1} \\
\visible<19>{
\PauliZ{}\QState{B} & = \alpha_{x}\ket{0}+\beta_{x}\ket{1}\\
& = \QState{x}}
\end{align*}
}
\only<20-21>{%
\begin{align*}
\QState{B}  & = -\beta_{x}\ket{0} + \alpha_{x}\ket{1} \\
\visible<21>{
\PauliX{}\QState{B} & = \alpha_{x}\ket{0}-\beta_{x}\ket{1}\\
\PauliZ{}\PauliX{}\QState{B} & = \alpha_{x}\ket{0}+ \beta_{x}\ket{1}\\
& = \QState{x}}
\end{align*}
}
\only<22>{%
\begin{itemize}
    \item If Alice can tell Bob her measurements for
\ket{xy} then Bob knows what transformations, if any to apply to his qubit to obtain Alice's \QState{x}.
\item This takes classical communication from Alice to Bob.
\item Note Alice's loss of her \QState{x} when she measures \ket{x} as \ket{0} or \ket{1}.
\item Teleportation is instantaneous across any distance, but Bob needs two classical bits of information from Alice to realize \QState{x}.
\end{itemize}
}
}
\end{frame}