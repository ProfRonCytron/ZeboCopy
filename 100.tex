\SetTitle{10}{Teleportation}{Sending a qubit's state requires a phone call}{10}

\section*{Overview}
\begin{frame}{Overview}
    \begin{itemize}[<+->]
        \item We will show how to use entangled states to send an arbitrary qubit state from Alice to Bob.
        \item Amazingly, there are no wires or other communication fabric between Alice and Bob.  The state is truly teleported from one location to another.
        \item Recalling the no-cloning theorem, we should not be surprised that Alice will no longer have the state.
        \item Alice and Bob can be arbitrarily far apart, but actions involving entanglement take effect immediately.
        \item However, it turns out Alice has to communicate some (classical) bits of information to Bob for him to truly obtain Alice's state at his location.  That information cannot travel faster than the speed of light.
    \end{itemize}
\end{frame}

\begin{frame}{Some background for teleportation}{Why would you want to do that?}
\begin{itemize}[<+->]
    \item Suppose Alice has some state \QState{} that she wants to send to Bob.
    \item If Alice knows the state, she could send Bob $\theta$ and $\phi$, the two parameters of a quantum state on the Bloch sphere.  He could create the state by rotations from \QZero{}.
    \item More interesting is when Alice does not know \QState{}, such as when it is the result of a quantum algorithm.  For example, Alice may have quantum bits that represent a factor of a large number.  But such a result could be measured by Alice with outcomes that are sent classically to Bob.
    \item More interesting still would be some unknown, intermediate quantum state of an algorithm, to be sent from Alice to Bob for further processing.
    \item Finally, the theoretical idea of teleportation is interesting all on its own.
\end{itemize}
    
\end{frame}

\section*{Partial Measurement}

\begin{frame}{Partial measurements}{What happens when some bits are measured, but not all?}
\Vskip{-3em}\begin{itemize}[<+->]
    \item Consider the 3-qubit quantum system
    \begin{align*}
  \visible<6->{\alert{k} \times }\QState{\alt<6->{p}{\relax}}
   = & \visible<1-5>{\alpha_{0} \ket{\alert<3>{00}0}
  +  \alpha_{1} \ket{\alert<3>{00}1}
  +  \alpha_{2} \ket{\alert<3>{01}0}
  +  \alpha_{3} \ket{\alert<3>{01}1} }\\
  +  & \alpha_{4} \ket{\alert<3,6->{10}0}
  +  \alpha_{5} \ket{\alert<3,6->{10}1}
 \visible<1-5>{ +  \alpha_{6} \ket{\alert<3>{11}0}
  +  \alpha_{7} \ket{\alert<3>{11}1}}
    \end{align*}
    \item We must have $\sum_{i=0}^{7} \Prob{\alpha_{i}} = 1$
    \item What happens if we perform a \emph{partial measurement} of the system by measuring the \alert<3>{left two qubits}? 
    \item We obtain \ket{00}, \ket{01}, \ket{10}, or \ket{11} for those qubits.
    \item What about the third qubit? We must scale the remaining wave amplitudes appropriately.
    \item Suppose the first two bits measure \ket{\alert{10}}. All but two terms vanish to~$0$, and we must renormalize the system using $\alert{k}=\sqrt{\Prob{\alpha_4} + \Prob{\alpha_5}}$. 
\end{itemize}

    
\end{frame}

\begin{frame}{Partial measurement}{Example}
\Vskip{-4em}\begin{align*}
  \QState{\relax}
   = & \alpha_{0} \ket{000}
  +  \alpha_{1} \ket{001}
  +  \alpha_{2} \ket{010}
  +  \alpha_{3} \ket{011} \\
  +  & \alpha_{4} \ket{100}
  +  \alpha_{5} \ket{101}
  +  \alpha_{6} \ket{110}
  +  \alpha_{7} \ket{111}
    \end{align*}
\Vskip{-3em}\TwoColumns{%
\only<1-3>{%
\begin{itemize}
    \item<1-> As usual, each coefficient~$\alpha_i$ is a complex value.
    \item<2-> For this example, we assume the magnitudes of~$\alpha_4$ and~$\alpha_5$ are as shown.
    \item<3-> The sum of all other amplitudes is then as shown.
\end{itemize}}%
\only<4->{%
\begin{itemize}
    \item<4->  A partial measurement of the first two qubits as \ket{10} collapses the system into the shown state.
    \item<5-> We subsequently measure \ket{100} with probability $\frac{250}{375}=\frac{2}{3}$ and \ket{101} with probability $\frac{1}{3}$
   \item<5-> As in the system before partial measurement, we are still twice as likely to see~\ket{100} as~\ket{101}.
\end{itemize}}%
}{%

\Vskip{-3em}\begin{align*}
\visible<2->{%
\Prob{\alpha_{4}} & =0.250 \\
       \Prob{\alpha_{5}} & = 0.125 \\}
\visible<3->{%
\sum_{i\in \Set{0,1,2,3,6,7}} \Prob{\alpha_{i}} &= 0.625 \\
}
\visible<4->{%
\alert{k}\times \QState{p} &= \alpha_{4}\ket{100} + \alpha_{5}\ket{101} \\
\alert{k} &= \sqrt{0.375} \\
}
\visible<5->{%
\QState{p} &= \frac{\alpha_{4}\ket{100} +\alpha_{5}\ket{101}}{\sqrt{0.375}}
}
\end{align*}}
    
\end{frame}


\section*{Teleportation idea}

\begin{frame}{The idea}{Use an entangled pair of qubits}

\begin{itemize}[<+->]
    \item The premise is that Alice has some state \QState{A} that she wants to send to distant Bob as \QState{B}.
    \item We know from the No Cloning Theorem that Alice cannot copy an arbitrary quantum state for herself, let alone make a copy to give to Bob.
    \item What could we do to send a quantum state?
    \begin{itemize}
    \item She could physically ship her qubit to Bob via UPS.~\footnote{\visible<4->{Unitary Photon Service}}
    \item Entanglement links the fate of two quantum bits.  Maybe we can use that property to \emph{teleport} a quantum state from Alice to Bob.
    \item Because cloning is not possible, we expect that Alice will lose her qubit's state in sending it to Bob.
    \end{itemize}
    \item If entanglement alone works, then the information is sent possibly faster than the speed of light.  So there may be a catch somewhere.
\end{itemize}

\end{frame}

\section*{Almost works}

\begin{frame}{Harnessing entanglement}{Achieving teleportation, almost}
\Vskip{-3.5em}\TwoUnequalColumns{0.45\textwidth}{0.55\textwidth}{%
\only<1-2>{%
\adjustbox{valign=t,scale=0.8}{\begin{quantikz}
\lstick{\QZero{}} &  \gate{H}& \ctrl{1}\slice{\QState{1}} & \qw\rstick{\QState{A}} \\
\lstick{\QZero{}} &   \qw  & \targ{} & \qw\rstick{\QState{B}}
\end{quantikz}}}%
\only<3>{%
\adjustbox{valign=t,scale=0.8}{\begin{quantikz}
\lstick{\QZero{}} &  \gate{H}& \ctrl{1}\slice{\QState{1}} & \qw\rstick{\QState{A}} \\[2em]
\lstick{\QZero{}} &   \qw  & \targ{} & \qw\rstick{\QState{B}}
\end{quantikz}}}%
\only<4->{%
\adjustbox{valign=t,scale=0.8}{\begin{quantikz}
\lstick{\QZero{}} &  \gate{H}& \ctrl{1}\slice{\alert<1-3>{\QState{1}}} & \gate{U}\slice{\alert<7-8>{\QState{2}}} & \qw  \rstick{\QState{A}}\\[2em]
\lstick{\QZero{}} &   \qw    &  \targ{}  & \qw & \qw  \rstick{\QState{B}}
\end{quantikz}}}%
\visible<9->{%
\SmallSkip{}
{\small\setlength{\tabcolsep}{2pt}
\begin{tabular}{r@{\mbox{ }}rlcrl}
\QState{A} & \multicolumn{5}{l}{$\QState{B}$} \\
\visible<9->{\ket{0}}\visible<11->{& \UGateA{}& \ket{0} & $+$ &  \UGateB{} & \ket{1} \\}
\visible<12->{\ket{1} & $\sin(\theta/2)$ & \ket{0} & 
   $+$ &  $\ExpPhase{\lambda}\cos(\theta/2)$ & \ket{1}}
\end{tabular}}
\Vskip{2em}
}
}{%
\only<1-3>{%
\begin{itemize}
    \item<1-> We create the Bell state 
    \[\QState{1} = \RootTwo{}\DQB{1}{0}{0}{1} \]
    \item<2-> Alice takes the top qubit and Bob takes the bottom one.
    \item<3-> They then separate, let's say by a very large distance.
\end{itemize}
}%
\only<4-6>{%
\begin{itemize}
    \item<4-> Alice effects a change to her qubit, represented here by \NamedGate{U}.
    \item<5-> Recall generally
    \[\NamedGate{U} = \UGate{} \]
\end{itemize}}%
\only<7-8>{%
\Vskip{-3em}{\small\begin{align*}
    \QState{2} & = \left(\TensProd{\NamedGate{U}}{\NamedGate{I}}\right) \RootTwo\DQB{\ColorOne{1}}{0}{0}{\ColorTwo{1}}\\
    & = \RootTwo{}\DQB{\ColorOne{\UGateA{}}}{\ColorTwo{\UGateB{}}}{\ColorOne{\UGateC{}}}{\ColorTwo{\UGateD{}}}
\end{align*}}
}%
\only<9-12>{%
\Vskip{-2.5em}\begin{itemize}
    \item<9-> What happens to this state if Alice measures her qubit?
    \item<9-> Suppose she sees \alt<9-11>{\ket{0}}{\ket{1}}.
    \item<10-> Then the system collapses as shown.
    \item<11-> Bob's state is then
    \alt<11>{\[ \UGateA{}\ket{0} + \UGateB{}\ket{1} \]}{%
    \begin{align*}
    &  \UGateC{}\ket{0} + \UGateD{}\ket{1} \\
   & =  \ColorThree{\ExpPhase{\phi}}\left( \sin(\theta/2)\ket{0}
   + \ExpPhase{\lambda}\cos(\theta/2)\right) 
    \end{align*}
    }
\end{itemize}
}%
}
\only<6-7>{%
\small{%
\Vskip{-2em}\[
\TensProd{\NamedGate{U}}{\NamedGate{I}} =
\begin{pmatrix*}[r]
\ColorOne{\UGateA{}} & 0 & \UGateB{}  & \ColorTwo{0} \\
\ColorOne{0} & \UGateA{} & 0 & \ColorTwo{\UGateB{}} \\
\ColorOne{\UGateC{}} & 0 & \UGateD{}  & \ColorTwo{0} \\
\ColorOne{0} & \UGateC{} & 0 & \ColorTwo{\UGateD{}}
\end{pmatrix*}
\]}}%
\only<8-12>{%
{\setlength{\tabcolsep}{3pt}
\Vskip{-2em}\[
    \visible<8-9>{\sqrt{2}}\QState{2} = 
    \begin{tabular}{crlcrl} 
    \visible<8-11>{ & \UGateA{} &\ket{\alert<9>{0}0} & $+$
     &\UGateB{}&\ket{\alert<9>{0}1} \\} 
    \visible<8-9,12>{ $+$ & \UGateC{}& \ket{\alert<12>{1}0}& $+$
       & \UGateD{}& \ket{\alert<12>{1}1}}
       \end{tabular}
\]
}
}%


\end{frame}

\begin{frame}{Making sense of the result}{Based on Alice's collapsed state}

{\setlength{\tabcolsep}{2pt}
\begin{center}
\begin{tabular}{r@{\hbox to 4ex{\hss}}rlcrll}
\QState{A} & \multicolumn{5}{l}{$\QState{B}$} &  \\
\ket{0} & \textcolor<8>{\RCtwo}{\UGateA{}}& \ket{0} & $+$ &  \textcolor<9>{Green}{\UGateB{}} & \ket{1} & \visible<4->{=\alert{\QState{\relax}}}\\
\ket{1} & $\sin(\theta/2)$ & \ket{0} & 
   $+$ &  $\ExpPhase{\lambda}\cos(\theta/2)$ & \ket{1} & 
\end{tabular}
\end{center}}
\only<1-3>{%
\begin{itemize}
    \item<1-> If Bob measures \QState{B}, then the probability of seeing \ket{0} depends on the measured value of \QState{A}.
    \only<1>{%
    \begin{itemize}
        \item If Alice sees \ket{0}, then Bob sees \ket{0} with probability $\Prob{\cos(\theta/2)}$.
        \item Otherwise Bob sees \ket{0} with probability $\Prob{\sin(\theta/2)}$
    \end{itemize}
    }
    \item<2-> There is thus no information communicated from Alice to Bob.  The outcome of \QState{A} doesn't matter \emph{only when} $\theta=\frac{\pi}{2}$, but then Bob's outcomes are equally likely.
    \item<3-> For Bob to know the state of \QState{B}, Alice must send her measured result to Bob.  That information cannot travel faster than the speed of light.
\end{itemize}}%
\only<4-5>{%
\begin{itemize}
    \item<4-> Since Alice arranged for the effect of \NamedGate{U} to achieve \QState{A}, let's assume \WLOG{} that she intended to send Bob
    $ \alert{\QState{\relax}} = \UGateA{}\ket{0} + \UGateB{}\ket{1}$.  If Bob learns that $\QState{A}=\ket{0}$, he does nothing to his qubit as $\QState{B}=\QState{\relax}$.
    \item<5-> What must Bob do to realize \QState{\relax} for \QState{B} if Alice tells him she saw \ket{1} for \QState{A}?  We begin by ignoring phase.
\end{itemize}
}%
\only<6->{%
{\setlength{\tabcolsep}{2pt}%
\begin{center}
\begin{tabular}{rcrl@{$\ +\ $}rl}
    $\QState{B}$ & $=$ & $\sin(\theta/2)$ & \ket{0}  
   & $\ExpPhase{\lambda}\cos(\theta/2)$ &\ket{1} \\
  \visible<7->{ $\textcolor<7>{\RCone}{\NamedGate{Z}}\textcolor<7>{\RCtwo}{\NamedGate{X}}\QState{B}$ & $=$ & $\ExpPhase{\lambda}\cos(\theta/2)$  & \ket{0}  
   & $-\sin(\theta/2)$ &\ket{1} \\
   & $\equiv$ &\textcolor<8>{\RCtwo}{$\cos(\theta/2)$}  & \ket{0}  
   & \textcolor<9>{Green}{$-\textcolor<10>{purple}{\ExpNegPhase{\lambda}}\sin(\theta/2)$} &\ket{1}}
\end{tabular}
\end{center}}

\only<6-9>{%
\begin{itemize}
    \item<6-> Bob's qubit's state if $\QState{A}=\ket{1}$
    \item<7-> We need to \textcolor<7>{\RCtwo}{exchange the amplitudes} and \textcolor<7>{\RCone}{negate the second term}.
    \item<8-> Bob now has the the same probability of measuring \ket{0} for his qubit as \QState{\relax}.
    \begin{itemize}
    \item<9-> And similarly for measuring \ket{1}
    \end{itemize}
\end{itemize}}
\only<10->{%
\begin{itemize}
    \item<10-> Note however that the \textcolor<10>{purple}{phase} does not match \QState{\relax}.
    \item<11->There is no unitary gate that can conjugate that phase without knowing $\lambda$.
\end{itemize}
}
}
\end{frame}

\begin{frame}{Summary so far}{Using only an EPR pair}

\begin{itemize}[<+->]
    \item Alice and Bob begin with the state $\RootTwo{}\left(\ket{00}+\ket{11}\right)$.
    \item Alice and Bob can then separate.
    \item Alice can develop a state on her quantum bit, characterized by some unitary gate \NamedGate{U}.  We assume Alice does not know the parameters of \NamedGate{U}.
    \begin{itemize}
        \item Otherwise, she could just phone those to Bob and he could create the state himself.
    \end{itemize}
    \item Alice measures her qubit with outcome \ket{0} or \ket{1}.
    \item If Alice tells Bob the outcome, then he can apply a gate to his qubit (or not) so that the probabilities of his measuring \ket{0} and \ket{1} match Alice's before she performed her measurement.
    \item However, we cannot communicate phase information reliably using this scheme.
    \item We can truly teleport if we include another qubit in the picture.
\end{itemize}

\end{frame}

\begin{frame}{Revision notes}{To do}

\begin{itemize}
    \item In the slides that follow, at the end, I want to show the collapse after partial measurement as the tensor product of Alice's two bits and the state Bob now has.
    \item That can be done with a column vector and also by factoring out the measurement Alice saw from the state.
\end{itemize}
    
\end{frame}

\section*{Correct algorithm}

\begin{frame}{Quantum teleportation}{Sending a quantum state from Alice to Bob}

\TwoUnequalColumns{0.4\textwidth}{0.6\textwidth}{%
\Vskip{-2em}\adjustbox{scale=0.7,valign=t}{%
\begin{quantikz}
  \QState{x}&\qw\slice{\alert<4-5>{\QState{1}}} &  \ctrl{1}\slice{\alert<6-9>{\QState{2}}} &  \gate{H}\slice{\alert<10-12>{\QState{3}}}& \meter{} &\cw \rstick{\textcolor{Blue}{\ket{x}}}\\
\lstick[wires=2]{\stackbox{EPR\\pair}}
&\lstick{\QState{A}} & \targ{} & \qw & \meter{} & \cw \rstick{\textcolor{Blue}{\ket{y}}}\\[2.7em]
& \lstick{\QState{B}} & \qw & \qw & \qw& \qw
\end{quantikz}}
\only<1-4>{%
\Vskip{-1.5em}
\begin{align*}
\visible<1->{%
    \QState{x} &=\SQB{\alpha_{x}}{\beta_{x}} \\}
\visible<2->{%
    \ket{\QName{A}\QName{B}}&=\RootTwo{}\DQB{1}{0}{0}{1}}
\end{align*}}
\only<5-8>{%
{\small
\Vskip{-1.15em}\begin{align*}
    \alert<5>{\QState{1}} &=\RootTwo{}\QQB{\alpha_x}{0}{0}{\alpha_x}{\beta_x}{0}{0}{\beta_x}
\end{align*}}
}
\only<9-11>{%
{\small
\Vskip{-1.15em}\begin{align*}
    \alert<9>{\QState{2}} &= \RootTwo{}\QQB{\alpha_x}{0}{0}{\alpha_x}{0}{\beta_x}{\beta_x}{0}
\end{align*}}
}
\only<12-13>{%
{\small
\Vskip{-1.15em}\begin{align*}
    \alert<12>{\QState{3}} &= \frac{1}{2}\QQB{\alpha_x}{\beta_x}{\beta_x}{\alpha_x}{\alpha_x}{-\beta_x}{-\beta_x}{\alpha_x}
\end{align*}}
}
\only<14->{%
\begin{center}
\begin{tabular}{r|l}
\multicolumn{1}{c}{Alice measures} & \multicolumn{1}{c}{Bob applies} \\
  \visible<14->{\ket{xy}=\ket{00} & \only<14>{?}\only<15->{nothing}} \\
  \visible<16->{\ket{xy}=\ket{01}  & \only<16>{?}\only<17->{\PauliX{}}} \\
  \visible<18->{\ket{xy}=\ket{10} & \only<18>{?}\only<19->{\PauliZ{}} }\\
  \visible<20->{\ket{xy}=\ket{11} & \only<20>{?}\only<21->{\PauliX{} then \PauliZ{}}}
\end{tabular}
\end{center}
so that Bob has \QState{x}
}

}{%

\only<1-3>{%
    \begin{itemize}
    \item<1-> Alice wants to send $\QState{x}=\alpha_{x}\ket{0}+\beta_{x}\ket{1}$.
    \item<2-> Alice and Bob share the EPR pair \ket{\QName{A}\QName{B}}.
    \item<3-> Alice and Bob separate, perhaps at a great distance, with Alice keeping \QState{x} and \QState{A}, and Bob taking \QState{B} with him.
\end{itemize}
\only<2>{%
\MedSkip{}
Recall we create \ket{\QName{A}\QName{B}} as an EPR pair:

\adjustbox{valign=t}{\begin{quantikz}
\lstick{\QZero{}} &  \gate{H}& \ctrl{1} & \qw  \rstick{\QState{A}}\\
\lstick{\QZero{}} &   \qw    &  \targ{}  & \qw  \rstick{\QState{B}}
\end{quantikz}}

}}
\only<4-5>{%
\Vskip{-5em}\begin{align*}
    \QState{1} &= \TensProd{\SQB{\alpha_x}{\beta_x}}{\RootTwo{\DQB{1}{0}{0}{1}}} \\
    &= \RootTwo{}\QQB{\alpha_x}{0}{0}{\alpha_x}{\beta_x}{0}{0}{\beta_x}
\end{align*}
}
\only<6-7>{%
\Vskip{-4em}\begin{align*}
\TensProd{\NamedGate{CNOT}}{\Identity} & = 
\TensProd{\CNOTMatrix}{\IMatrix} \\
\visible<7>{%
&= \begin{pmatrix*}[r]
1 & 0 & 0 & 0 & 0 & 0 & 0 & 0 \\
0 & 1 & 0 & 0 & 0 & 0 & 0 & 0 \\
0 & 0 & 1 & 0 & 0 & 0 & 0 & 0 \\
0 & 0 & 0 & 1 & 0 & 0 & 0 & 0 \\
0 & 0 & 0 & 0 & 0 & 0 & 1 & 0 \\
0 & 0 & 0 & 0 & 0 & 0 & 0 & 1 \\
0 & 0 & 0 & 0 & 1 & 0 & 0 & 0 \\
0 & 0 & 0 & 0 & 0 & 1 & 0 & 0
\end{pmatrix*}}
\end{align*}
}
\only<8>{%
\Vskip{-4em}{\scriptsize\begin{align*}
\TensProd{\NamedGate{CNOT}}{\Identity} & = 
 \begin{pmatrix*}[r]
1 & 0 & 0 & 0 & 0 & 0 & 0 & 0 \\
0 & 1 & 0 & 0 & 0 & 0 & 0 & 0 \\
0 & 0 & 1 & 0 & 0 & 0 & 0 & 0 \\
0 & 0 & 0 & 1 & 0 & 0 & 0 & 0 \\
0 & 0 & 0 & 0 & 0 & 0 & 1 & 0 \\
0 & 0 & 0 & 0 & 0 & 0 & 0 & 1 \\
0 & 0 & 0 & 0 & 1 & 0 & 0 & 0 \\
0 & 0 & 0 & 0 & 0 & 1 & 0 & 0
\end{pmatrix*}
\end{align*}}
}
\only<8>{%
{\scriptsize
\Vskip{-1.5em}\begin{align*}
    \QState{2} &=\left(\TensProd{\NamedGate{CNOT}}{\Identity}\right)\QState{1} = \RootTwo{}\QQB{\alpha_x}{0}{0}{\alpha_x}{0}{\beta_x}{\beta_x}{0}
\end{align*}}
}
\only<9-10>{%
{\small
\Vskip{-4em}\begin{align*}
\TensProd{\TensProd{\Hadamard}{\Identity}}{\Identity}  = 
\TensProd{\TensProd{\HMatrix}{\IMatrix}}{\IMatrix} \\[1.2em]
\visible<10>{%
= \RootTwo{}\begin{pmatrix*}[r]
1 & 0 & 0 & 0 & 1 & 0 & 0 & 0 \\
0 & 1 & 0 & 0 & 0 & 1 & 0 & 0 \\
0 & 0 & 1 & 0 & 0 & 0 & 1 & 0 \\
0 & 0 & 0 & 1 & 0 & 0 & 0 & 1 \\
1 & 0 & 0 & 0 & -1 & 0 & 0 & 0 \\
0 & 1 & 0 & 0 & 0 & -1 & 0 & 0 \\
0 & 0 & 1 & 0 & 0 & 0 & -1 & 0 \\
0 & 0 & 0 & 1 & 0 & 0 & 0 & -1
\end{pmatrix*}}
\end{align*}}
}
\only<11>{%
{\scriptsize
\Vskip{-4em}\begin{align*}
\TensProd{\TensProd{\Hadamard}{\Identity}}{\Identity}  = 
 \RootTwo{}\begin{pmatrix*}[r]
1 & 0 & 0 & 0 & 1 & 0 & 0 & 0 \\
0 & 1 & 0 & 0 & 0 & 1 & 0 & 0 \\
0 & 0 & 1 & 0 & 0 & 0 & 1 & 0 \\
0 & 0 & 0 & 1 & 0 & 0 & 0 & 1 \\
1 & 0 & 0 & 0 & -1 & 0 & 0 & 0 \\
0 & 1 & 0 & 0 & 0 & -1 & 0 & 0 \\
0 & 0 & 1 & 0 & 0 & 0 & -1 & 0 \\
0 & 0 & 0 & 1 & 0 & 0 & 0 & -1
\end{pmatrix*}
\end{align*}}
}
\only<11>{%
{\scriptsize
\Vskip{-2.5em}\begin{align*}
    \QState{3} &=\left(\TensProd{\TensProd{\Hadamard}{\Identity}}{\Identity}\right)\QState{2} = \frac{1}{2}\QQB{\alpha_x}{\beta_x}{\beta_x}{\alpha_x}{\alpha_x}{-\beta_x}{-\beta_x}{\alpha_x}
\end{align*}}
}
\only<12>{%
Recall Alice had
\[
\QState{x} = \alpha_{x}\ket{0} + \beta_{x}\ket{1}
\]
so we must have \[ \Prob{\alpha_{x}} + \Prob{\beta{x}} = 1 \]
}
\only<13-21>{%
\Vskip{-5.5em}\begin{align*}
    \QState{3} = \visible<13>{\frac{1}{2}}\ [\\ \visible<13,14-15>{+ & \alpha_{x}\ket{\alert<14-15>{00}0}  + \beta_{x}\ket{\alert<14-15>{00}1}} \\
    \visible<13,16-17>{+ & \beta_{x}\ket{\alert<16-17>{01}0} + \alpha_{x}\ket{\alert<16-17>{01}1} }\\
     \visible<13,18-19>{+ &\alpha_{x}\ket{\alert<18-19>{10}0}  -\beta_{x}\ket{\alert<18-19>{10}1} }\\ 
     \visible<13,20-21>{- & \beta_{x}\ket{\alert<20-21>{11}0} + \alpha_{x}\ket{\alert<20-21>{11}1}}\\
    ]
\end{align*}
}
\only<14-21>{
\Vskip{-4em}}
\only<14-15>{%
\begin{align*}
\QState{B} & = \alpha_{x}\ket{0} + \beta_{x}\ket{1} \\
\visible<15>{& = \QState{x}}
\end{align*}
}
\only<16-17>{%
\begin{align*}
\QState{B}  & = \beta_{x}\ket{0} + \alpha_{x}\ket{1} \\
\visible<17>{
\PauliX{}\QState{B} & = \alpha_{x}\ket{0} + \beta_{x}\ket{1} \\
 & = \QState{x}}
\end{align*}
}
\only<18-19>{%
\begin{align*}
\QState{B}  & = \alpha_{x}\ket{0} - \beta_{x}\ket{1} \\
\visible<19>{
\PauliZ{}\QState{B} & = \alpha_{x}\ket{0}+\beta_{x}\ket{1}\\
& = \QState{x}}
\end{align*}
}
\only<20-21>{%
\begin{align*}
\QState{B}  & = -\beta_{x}\ket{0} + \alpha_{x}\ket{1} \\
\visible<21>{
\PauliX{}\QState{B} & = \alpha_{x}\ket{0}-\beta_{x}\ket{1}\\
\PauliZ{}\PauliX{}\QState{B} & = \alpha_{x}\ket{0}+ \beta_{x}\ket{1}\\
& = \QState{x}}
\end{align*}
}
\only<22>{%
\begin{itemize}
    \item If Alice can tell Bob her measurements for
\ket{xy} then Bob knows what transformations, if any, to apply to his qubit to obtain Alice's \QState{x}.
\item This takes classical communication from Alice to Bob.
\item Note Alice's loss of her \QState{x} when she measures \ket{x} as \ket{0} or \ket{1}.
\item Teleportation is instantaneous across any distance, but Bob needs two classical bits of information from Alice to realize \QState{x}.
\end{itemize}
}
}
\end{frame}

\section*{Demos}
\begin{frame}{Qiskit}{Implementing teleportation}
\end{frame}

\begin{frame}{Matlab}{Developing the results from the slides}
    
\end{frame}

\begin{frame}{Breaking the speed of light barrier}{Nobel prize, or bust?}
    
\end{frame}

\section*{Summary}
\begin{frame}{Summary}{Teleportation}

\begin{itemize}
    \item Alice has a qubit whose state \QState{A} she wants to send to Bob.
    \item Alice and Bob share an EPR pair, and Bob expects eventually to have \QState{A} on his qubit of the pair.
    \item With the appropriate quantum circuit, Alice measures her qubit with state \QState{A} and also measures her qubit of the EPR pair.
    \item After she sends those results to Bob, he can reconstruct the state of Alice's qubit prior to measurement.
    \item In this process, Alice loses her qubit \QState{A}.
    \item The effect on entanglement happens immediately regardless of distance.
    \item The transformations Bob must apply to his qubit depend on Alice's measurements, which cannot travel faster than the speed of light.
\end{itemize}
    
\end{frame}
