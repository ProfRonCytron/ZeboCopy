\SetTitle{22}{The phase kickback trick}{An unexpected but useful phenomenon}{22}

\begin{frame}{Overview}{What do we study here?}

\end{frame}

\begin{frame}{Recall our oracle}{With its ancillary input $y$}

\TwoUnequalColumns{0.37\textwidth}{0.63\textwidth}{%
\begin{Oracle}[scale=0.75]{$U_f$}{\alert<1>{$x$}}{\alert<2,5>{$y$}}{\alert<4,5>{$x$}}{\alert<3>{\Xor{y}{f(x)}}}
\end{Oracle}
}{%
\begin{itemize}[<+->]
    \item The input $x$ is the uniform superposition of all values.
    \item The input $y$ is \ket{-}.
    \item The bottom qubit has a constant value, regardless of $f(x)$.
    \item However, the top qubit is a superpositon of terms, each term with a sign related to the function $f$ evaluated on that term.
    \item This is \emph{phase kickback}:  The bottom qubit's input, when \ket{-}, kicks phase changes up the top qubit's output.
\end{itemize}
}

\end{frame}

\begin{frame}{Single qubit}{Phase kickback}
\TwoColumns{%
\begin{Oracle}[scale=0.75]{$U_f$}{$x$}{$y$}{$x$}{\Xor{y}{f(x)}}
\end{Oracle}
\begin{itemize}
    \item Recall
    \[ U_{f}:  \ket{x\,y}\mapsto \ket{x\,\Xor{y}{f(x)}}\]
\end{itemize}
}{%
}
    
\end{frame}