\SetTitle{22}{The phase kickback trick}{An unexpected but useful phenomenon}{22}

\begin{frame}{Overview}{What do we study here?}

\begin{itemize}[<+->]
   \item The technique we study here is a useful and common trick for analyzing function values in superposition.
   \item The effect is not something that is obvious or directly seen in the circuit.
   \item It follows from mathematical analysis of state---analysis we have performed all semester.
   \item The surprise is that the state of the bottom qubit has an interesting effect on the top qubit(s).
\end{itemize}

\end{frame}

\begin{frame}{Recall our oracle}{With its ancillary input $y$}

\TwoUnequalColumns{0.37\textwidth}{0.63\textwidth}{%
\begin{Oracle}[scale=0.75]{$U_f$}{\alert<1>{$x$}}{\alert<2,5>{$y$}}{\alert<4,5>{$x$}}{\alert<3>{\Xor{y}{f(x)}}}
\end{Oracle}
}{%
Here is what we saw:
\begin{itemize}[<+->]
    \item The input $x$ is the uniform superposition of all values.
    \item The input $y$ is \ket{-}.
    \item The bottom qubit has a constant value, regardless of $f(x)$.
    \item However, the top qubit is a superpositon of terms, each term with a sign related to the function $f$ evaluated on that term.
    \item This is \emph{phase kickback}:  The bottom qubit's input, when \ket{-}, kicks phase changes up to the top qubit's output.
\end{itemize}
}

\end{frame}

{
\def\X{\ColorOne{\ensuremath{x}}}\def\Y{\ColorTwo{\ensuremath{y}}}
\begin{frame}{Single qubit}{Here $x$ is not in a superposition but simply a basis state}
\TwoColumns{%
\begin{Oracle}[scale=0.75]{$U_f$}{$x$}{$y$}{$x$}{\Xor{y}{f(x)}}
\end{Oracle}
}{%
\only<1-7>{%
Recall
    $U_{f}:  \ket{\X\ \ \Y}\mapsto \ket{\X\ \ \Xor{\Y}{f(\X)}}$}
\only<8->{%
\Vskip{-3em}\begin{align*}
    U_f(\ket{x\,-}) &= \TwoSupOp{\ket{\X\,f(\X)}}{\ket{\X\,\Not{f(\X)}}}{-}
\end{align*}
}%
}
\only<1-7>{%
\TwoColumns{%
\begin{align*}
   \visible<1->{ U_f(\ket{x-}) &= U_f(\TensProd{\ket{x}}{\ket{-}}) \\}
    \visible<2->{&= U_f(\TensProd{\ket{x}}{\QMinus}) \\}
   \visible<3->{ &= U_{f}(\TwoSupOp{\ket{x0}}{\ket{x1}}{-}) \\}
  \visible<4->{  &= \RootTwo{}\left(U_f(\ket{\X\,\ColorTwo{0}})-U_f(\ket{\X\,\ColorTwo{1}})\right)}
\end{align*}
}{%
\begin{align*}
    \visible<5->{&= \RootTwo{}\left(\ket{\X\,(\Xor{\ColorTwo{0}}{f(\X)})}-U_f(\ket{\X\,\ColorTwo{1}})\right) \\}
    \visible<6->{&= \RootTwo\left(\ket{\X\,f(\X)} - \ket{\X\,(\Xor{\ColorTwo{1}}{f(\X)})}\right) \\}
    \visible<7->{&= \TwoSupOp{\ket{\X\,f(\X)}}{\ket{\X\,\Not{f(\X)}}}{-}}
\end{align*}
}}
\only<8-17>{%
\BigSkip{}
\TwoColumns{%
\visible<8->{Case $f(x)=0$}
\begin{align*}
   \visible<9->{U_f(\ket{x\,-}) &= \TwoSupOp{\ket{x\,0}}{\ket{x\,1}}{-} \\}
    \visible<10->{&= \TensProd{\ket{x}}{\QMinus} \\}
    \visible<11->{&= \TensProd{\ket{x}}{\ket{-}} \\}
   \visible<12->{ & = \ket{x\,-}}
\end{align*}
}{%
\visible<13->{Case $f(x)=1$}
\begin{align*}
   \visible<14->{ U_f(\ket{x\,-}) &= \TwoSupOp{\ket{x\,1}}{\ket{x\,0}}{-} \\}
    \visible<15->{&= \TensProd{\ket{x}}{\TwoSupOp{\QOne}{\QZero}{-}} \\}
   \visible<16->{ &= \TensProd{\ket{x}}{\alert{-}\QMinus} \\}
   \visible<17->{ & = \alert{-}\ket{x\,-}}
\end{align*}
}
}%
\only<18->{%
\BigSkip{}
\TwoColumns{%
Case $f(x)=0$
\[ U_f(\ket{x\,-}) = \alt<19-20>{\alert{\fbox{+1}}}{}\ket{x\,-} \]
}{%
Case $f(x)=1$
\[ U_f(\ket{x\,-}) = \alt<19-20>{\alert{\fbox{-1}}}{-}\ket{x\,-} \]
}
\BigSkip{}
\visible<19->{\[ U_f(\ket{x\,-}) \only<19>{= \alert<19-20>{\NegF{x}}\ket{x\,-}}
= \TensProd{\alert<19-20>{\NegF{x}}\ket{x}}{\ket{-}}\]}
\begin{itemize}
    \item<20-> Phase kickback associates the function result as the \emph{sign} on the state \ket{x\,-}.
    \item<21-> When $x$ is a superposition, that sign is \emph{not} a global phase.
    \item<22-> The sign can impose a phase change relative to the \QZero{} and \QOne{} terms.
\end{itemize}
}
    
\end{frame}
}

\begin{frame}{Single qubit}{In superposition}
\TwoColumns{%
\begin{Oracle}[scale=0.75]{$U_f$}{$x$}{$y$}{$x$}{\Xor{y}{f(x)}}
\end{Oracle}
}{%
\[ U_f(\ket{x\,-})
= \TensProd{\NegF{x}\ket{x}}{\ket{-}}\]
}
\BigSkip{}
\only<1-3>{%
Linearity allows the above equation to apply to a superposition:
    \begin{align*}
     \visible<1->{   \ket{x} & ={\alpha_0}\ket{0}+\alpha_{1}\ket{1} \\}
     \visible<2->{   U_f(\ket{x}) &= \alpha_{0}U_f(\ket{0}) + \alpha_{1}U_f(\ket{1}) \\}
       \visible<3->{ &= \alpha_{0} \NegF{0}\ket{0}
        +  \alpha_{1} \NegF{1}\ket{1}}
    \end{align*}
}%
\only<4-8>{%
This applies directly to Deutsch's problem ($f(x)$ constant$\longrightarrow U_f(\ket{x\,-}) = \ket{+\,-})$:
    \begin{description}
      \item[$\Forall{x}{f(x)=0}$] \visible<4->{\begin{align*}U_f(\ket{x\,-}) &= \TensProd{\TwoSupOp{\alt<4>{\alert{\NegF{0}}}{}\QZero}{\alt<4>{\alert{\NegF{1}}}{}\QOne}{+}}{\ket{-}}\only<5->{=\ket{+\,-}}\end{align*}}
      \item[$\Forall{x}{f(x)=1}$] 
      \Vskip{-3em}\begin{align*}
   \visible<6->{   U_f(\ket{x\,-}) &= \TensProd{\TwoSupOp{\alt<6>{\alert{\NegF{0}}}{-}\QZero}{\alt<6>{\alert{\NegF{1}}}{-}\QOne{}}{+}}{\ket{-}}\\}
      \visible<8->{ & \equiv \TensProd{\QPlus{}}{\ket{-}}=\ket{+\,-}}
      \end{align*}
    \end{description}
}%
\only<9-13>{%
This applies directly to Deutsch's problem ($f(x)$ balanced$\longrightarrow U_f(\ket{x\,-}) = \ket{-\,-})$:
    \begin{description}
      \item[$f(0)=0,\ f(1)=1$] \visible<9->{\begin{align*}U_f(\ket{x\,-}) &= \TensProd{\TwoSupOp{\alt<9>{\alert{\NegF{0}}}{}\QZero}{\alt<9>{\alert{\NegF{1}}}{}\QOne}{\alt<9>{+}{-}}}{\ket{-}}\only<10->{=\ket{-\,-}}\end{align*}}
      \item[$f(0)=1,\ f(1)=0$] 
      \Vskip{-3em}\begin{align*}
   \visible<11->{   U_f(\ket{x\,-}) &= \TensProd{\TwoSupOp{\alt<11>{\alert{\NegF{0}}}{-}\QZero}{\alt<11>{\alert{\NegF{1}}}{}\QOne{}}{+}}{\ket{-}}\\}
      \visible<13->{ & \equiv \TensProd{\QMinus{}}{\ket{-}}=\ket{-\,-}}
      \end{align*}
    \end{description}
}
\end{frame}

\begin{frame}{Multiple qubits}{First, their superposition}
\TwoUnequalColumns{0.55\textwidth}{0.44\textwidth}{%
\begin{itemize}
\item<1-> So far, we have studied a single qubit whose superposition is as shown.
\item<2-> If we add one more qubit, there are four states in the computational basis, and their uniform superposition is as shown.
\item<4-> It is convenient to interpret the states' encodings as binary numerals, whose numbers enumerate the possible basis states.
\item<5-> We can also view them as bit strings of length $n$.
\end{itemize}
}{%
\only<1>{%
\[ \QState{} = \QPlus{} \]
}%
\only<2-5>{%
\begin{align*}
\QState{} &= \frac{\ket{00} + \ket{01} + \ket{10} + \ket{11}}{\sqrt{4}} \\
\visible<3->{ &= \frac{\ket{0} + \ket{1} + \ket{2} +\ket{3}}{\sqrt{4}} \\}
\visible<4->{ &= \RootTwoN{n}\sum_{x=0}^{2^{n}-1} \ket{x} \\[1em]}
 \visible<5->{&= \RootTwoN{n}\SumBV{x}{n} \ket{x}}
\end{align*}
}
}

\end{frame}

\begin{frame}{Multiple qubits}{Phase kickback}
\Vskip{-4em}\begin{center}
\begin{Oracle}[scale=0.75]{$U_f$}{$x$}{$y$}{$x$}{\Xor{y}{f(x)}}
\end{Oracle}    
\end{center}
\only<1-4>{%
\Vskip{-2em}\begin{align*}
\visible<1->{U_f\left(\TensProd{\RootTwoN{n}\SumBV{x}{n}\ket{x}}{\ket{-}}\right)}\visible<2->{ &=
\RootTwoN{n}\SumBV{x}{n} U_f(\ket{x\,-}) \\}
\visible<3->{&= \RootTwoN{n} \SumBV{x}{n}\TensProd{\NegF{x}\ket{x}}{\ket{-}}}
\end{align*}
\begin{itemize}
    \item<1-> Input is the uniform superposition of all basis states for $n$ qubits, each term with positive sign, and $y=\ket{-}$.
    \item<4-> Phase kickback places a sign on each term $x$ of the summation that is determined by $f(x)$.
\end{itemize}}
\only<5->{%
\TwoColumns{%
Input $n=3$ qubits
\begin{center}
    \begin{tabular}{c|c}
         $x$ & $f(x)$ \\ \hline
  \visible<6->{       000 & 0 \\}
  \visible<7->{         001 & 1 \\}
  \visible<8->{         010 & 1 \\}
  \visible<9->{         011 & 1 \\}
  \visible<10->{         100 & 0 \\}
  \visible<11->{         101 & 1 \\}
  \visible<12->{         110 & 0\\}
  \visible<13->{         111 & 1}
    \end{tabular}
\end{center}
}{%
Output
\begin{align*}
    \RootTwoN{3} \left(\right.
   \visible<6->{ &\ket{000} }
   \visible<7-> { - \ket{001} \\}
    \visible<8-> { - &\ket{010}}
  \visible<9-> {  - \ket{011} \\}
  \visible<10-> {+ &\ket{100}}
  \visible<11-> {-\ket{101} \\}
  \visible<12-> {+ &\ket{110}}
  \visible<13-> {-\ket{111} \\}
    \left.\right)
\end{align*}
\visible<14->{%
A given $f(x)$ induces a unique signature of signs on the output basis states.}
}
}

\end{frame}

\begin{frame}{General setting:  phase kickback using a controlled-\NamedGate{U} gate}{Making a global phase local}

\TwoUnequalColumns{0.65\textwidth}{0.35\textwidth}{%
\Vskip{-3em}
\only<1-4>{%
\begin{itemize}[<+->]
    \item The top qubit is $\ket{+}=\Hadamard\QZero{}=\QPlus{}$.
    \item The bottom qubit is in state \QState{}, which is an eigenstate of the arbitrary unitary gate~\NamedGate{U}.
    \item We thus have
    \(\NamedGate{U}\QState{} = \lambda \QState{} \)
    \item This implies $\Mag{\lambda}=1$ so the result can be written:
    \[ \NamedGate{U}\QState{} = \ExpPhase{\theta}\QState{}\]
    with $\ExpPhase{\theta}$  as a \emph{global phase} on state~\QState{}.
\end{itemize}}%
\ScrollProof{5}{8}{%
\Next{\Four}{\QState{0} &= \TensProd{\ket{+}}{\QState{}}\\
&= \TensProd{\TwoSupOp{\ColorOne{\QZero}}{\ColorTwo{\QOne}}{+}}{\QState{}}\\}
\Next{\Three}{&= \TwoSupOp{\ColorOne{\QZero}\QState{}}{\ColorTwo{\QOne}\QState{}}{+} \\}
\Last{\QState{1} &= \TwoSupOp{\ColorOne{\QZero}\QState{}}{\ColorTwo{\QOne}\NamedGate{U}\QState{}}{+}}
}%
\ScrollProof{9}{12}{%
\Next{\Four}{\QState{1} &= \TwoSupOp{\ColorOne{\QZero}\QState{}}{\ColorTwo{\QOne}\NamedGate{U}\QState{}}{+} \\}
\Next{\Three}{&= \TwoSupOp{\ColorOne{\QZero}\QState{}}{\ColorTwo{\QOne}\ExpPhase{\theta}\QState{}}{+} \\}
\Next{\Two}{&= \TwoSupOp{\ColorOne{\QZero}\QState{}}{\ExpPhase{\theta}\ColorTwo{\QOne}\QState{}}{+} \\}
\Next{\One}{&= \TensProd{\TwoSupOp{\ColorOne{\QZero}}{\ExpPhase{\theta}\ColorTwo{\QOne}}{+}}{\QState{}}}
}%
\only<13->{%
\[
\QState{1} = \TensProd{\TwoSupOp{\ColorOne{\QZero}}{\ExpPhase{\theta}\ColorTwo{\QOne}}{+}}{\QState{}}
\]
\only<14-18>{%
\begin{itemize}
    \item<14-> The global phase $\ExpPhase{\theta}$ on the bottom qubit has been kicked back (up) to the top qubit.
    \item<15-> It now manifests as a local phase on \ColorTwo{\QOne} as compared with \ColorOne{\QZero}.
    \item<16-> Further processing can approximate or perhaps identify $\theta$
    \item<17-> For example, if we know that either
    \begin{description}
       \item[$\theta=0$] then $\QState{1}=\TensProd{\ket{+}}{\QState{}}$
       \item[$\theta=\pi$] then $\QState{1}=\TensProd{\ket{-}}{\QState{}}$
    \end{description}
    \item<18-> Applying \Hadamard{} to the top qubit determines $\theta$.
\end{itemize}}%
\only<19->{%
\begin{itemize}
    \item<19-> Consider $0\leq\theta\leq \pi$.
    \item<20-> The top qubit in the \PauliX{} basis can be expressed as
    \[
    \cos(\theta/2)\ket{+} + \sin(\theta/2)\ket{-}
    \]
    \item<21-> So we measure \ket{+} with probability $\cos^{2}(\theta/2)$ and \ket{-} with probability $\sin^{2}(\theta/2)$.
    \item<22-> Assuming the circuit can be reconstructed and measured an arbitrary number of times, we can estimate $\theta$ with arbitrary precision.
\end{itemize}
}
}
}{%
\begin{center}
\adjustbox{scale=1.0}{\begin{quantikz}
    \lstick{\ket{+}} & \qw\slice{\QState{0}} & \ctrl{1}\slice{\QState{1}} & \qw & \qw \\
    \lstick{\QState{}} & \qw & \gate{\NamedGate{U}} & \qw & \qw
    \end{quantikz}}
\end{center}
\visible<4->{\[ \NamedGate{U}\QState{} = \ExpPhase{\theta}\QState{}\]}
}
    
\end{frame}
