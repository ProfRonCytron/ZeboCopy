\SetTitle{2}{Reversible computations}{No energy lost or gained}{02}

\begin{frame}{Overview}
\begin{itemize}
    \item Quantum computations involve gates that are \emph{unitary} and therefore are invertible.
    \item Quantum computations are delicate and noise must be kept to a minimum.  Interactions with the environment will cause the quantum system to collapse (\href{https://en.wikipedia.org/wiki/Quantum_decoherence}{decoherence}).
    \item We study a mechanical computer first where it is clear that a nonreversible computation gains or loses energy.
    \item We can build reversible circuits for classical and quantum computations.
    
\end{itemize}
\end{frame}


\begin{frame}{The \href{https://en.wikipedia.org/wiki/Billiard-ball_computer}{billiard ball computer}}{When a single ball enters at a time}
\TwoColumns{%
Consider the mechanical \emph{and} gate shown here.
\begin{itemize}
    \item<1-> A ball entering from the top
    \only<2->{will emerge from the ``1-out'' hole.}
    \item<3-> A ball entering from the left
    \only<4->{will emerge from the ``0-out'' hole.}
    
\end{itemize}
}{%
\Vskip{-2em}\adjustbox{width=0.8\textwidth}{%
\begin{ReverseDiag}
\visible<1-2>{%
\visible<1>{\fill[\RCthree,fill opacity=0.7] (.85,1.5) circle(0.25);}
\draw<2>[->,thick,dashed] (0.85,1.25) -- ++(0,-6.25);
\visible<2>{\fill[\RCthree,fill opacity=0.7] (.85,-5.25) circle(0.25);}}
\visible<3-4>{%
\visible<3>{\fill[\RCthree,fill opacity=0.7] (-1.75,-0.85) circle(0.25);}
\draw<4>[->,thick,dashed] (-1.5,-0.85) -- ++(5.5,0);
\visible<4>{\fill[\RCthree,fill opacity=0.7] (4.25,-0.85) circle(0.25);}
}
\end{ReverseDiag}}

}
\only<5->{
\MedSkip{}No energy is put in or taken out of the system.  The ball has its own energy, maintained throughout its motion through the box.}
    
\end{frame}

\begin{frame}{The \href{https://en.wikipedia.org/wiki/Billiard-ball_computer}{billiard ball computer}}{When two balls enter simultaneously}
\TwoUnequalColumns{0.6\textwidth}{0.4\textwidth}{%
Here, two balls enter, one from the top, and one from the left
\begin{itemize}
    \item<1-> A ball entering from the top and a ball entering from the left at the same time
    \only<2->{will \alert<2>{collide here}.}
    \item<3-> They bounce internally at the same time.
    \item<4-> One ball emerges from the ``AND-output'' chute and the other emerges from the ``1-out'' chute.
    
\end{itemize}
}{%
\Vskip{-2em}\adjustbox{width=0.8\textwidth}{%
\begin{ReverseDiag}
\visible<1>{\fill[\RCthree,fill opacity=0.7] (.85,1.5) circle(0.25);}
\visible<1>{\fill[\RCthree,fill opacity=0.7] (-1.75,-0.85) circle(0.25);}
\end{ReverseDiag}}
}
\only<5->{
\SmallSkip{}Again, the energy of the box remains the same throughout.\Remark{It would take energy to \emph{stop} the ball from exiting the ``1-out'' chute.}}

    
\end{frame}

\section{Reversible gates}
\begin{frame}{Not and And}{From page 12 of \Kaye{}}
\TwoColumns{%
\begin{itemize}
    \only<1>{
    \item The \emph{not} gate is reversible and is in fact its own inverse.}
    \item<2-> Is \And{a}{b} reversible if we know $a$?  \only<3->{\alert<3>{No}.} \only<3>{If $a=0$, $b$ could be either $0$ or $1$.}
    \item<4-> We need a copy of both inputs to create a reversible \emph{and} gate.
\end{itemize}
 
}{%


\begin{center}
\only<1>{
\begin{GateBox}[scale=0.5]{2}{1}{1}
\BoxLabel{Not}
\Input{0}{$a$}
\Output{0}{\Overline{a}}
\end{GateBox}}

\only<2-3>{\begin{GateBox}{2.5}{1}{2}
\BoxLabel{Reversible?}
\Input{0}{$a$}
\Input{1}{$b$}
\Output{0}{$a$}
\Output{1}{\And{a}{b}}
\end{GateBox}}


\only<4>{\begin{GateBox}{2.5}{1}{3}
\BoxLabel{Reversible?}
\Input{0}{$a$}
\Input{1}{$b$}
\Output{0}{$a$}
\Output{1}{$b$}
\Output{2}{\And{a}{b}}
\end{GateBox}}

\only<5>{\begin{GateBox}{2.5}{1.5}{3}
\BoxLabel{\mbox{\stackbox[c]{Reversible\\ And}}}
\Input{0}{$a$}
\Input{1}{$b$}
\Input{2}{\Zero{}}
\Output{0}{$a$}
\Output{1}{$b$}
\Output{2}{\And{a}{b}}
\end{GateBox}}
\end{center}
}
\MedSkip{}
   \only<4>{
    However, this box must generate energy to produce the third output.  To conserve energy, a reversible box must have the same number of inputs as outputs.}

\only<5>{This box enables reversal of the computation and it accepts an \href{https://en.wikipedia.org/wiki/Ancilla_bit}{ancilla} bit as a third input.}

\OnlyRemark{5}{The circuit elements involved in quantum computations must be \emph{reversible} and must contain the same number of inputs as outputs.}
\end{frame}

\begin{frame}{The Toffoli / CCNOT gate}{Uses the third input to greater advantage}
\TwoColumns{%
\begin{itemize}
    \item<1-> The signals $a$ and $b$ are copied to their respective outputs.
    \item<2-> The bottom output is computed as: 
    \only<2->{
    \Vskip{-2em}\begin{description}
        \item[$c=0$] The output is \And{a}{b}.
        \item[$c=1$] The output is \Nand{a}{b}.
    \end{description}}
    \item<2-> Because it can realize Nand, this reversible gate is \emph{universal} for constructing classical circuits.
    \item<3-> The gate is known in quantum computing as the \href{https://en.wikipedia.org/wiki/Toffoli_gate}{Tofolli Gate}.
\end{itemize}
}{%
\begin{center}
\begin{GateBox}{2.5}{1.5}{3}
\only<1-2>{\BoxLabel{\mbox{f(a,b)=\And{a}{b}}}}
\only<3-4>{\BoxLabel{Tofolli Gate}}
\only<5->{\BoxLabel{\mbox{\stackbox[c]{Tofolli Gate\\ CCNOT}}}}

\Input{0}{$a$}
\Input{1}{$b$}
\Input{2}{$c$}
\Output{0}{$a$}
\Output{1}{$b$}
\only<1-3>{\Output{2}{\Xor{c}{(\And{a}{b})}}}
\only<4->{\Output{2}{\Xor{(\And{a}{b})}{c}}}
\end{GateBox}
\end{center}
\only<4->{\MedSkip{}
An alternative view of this gate:
\begin{itemize}
    \item When $a$ and $b$ are both \True{}, the incoming signal $c$ is complemented on output.
    \item<5-> It is thus also called a controlled-controlled-Not gate, or CCNOT.
\end{itemize}
}
}
    
\end{frame}

\begin{frame}{CNOT gate}{Two-bit version of CCNOT}
\Vskip{-3em}
\TwoColumns{%

\begin{itemize}
    \item We can say $c$ controls whether the bottom output is $a$ or its complement.
    \item<2-> We prefer to say $a$ controls whether the bottom output is $c$ or its complement. This is \CNOT{a}{c}.
    \item<3-> The computation is reversible because
    $\Xor{a}{(\Xor{c}{a})} = c $.  \only<3>{We can thus recover $c$.}
    \item<4-> Here you see the quantum circuit symbol for \CNOT{a}{c}:  $a$ conditionally complements $c$.
\end{itemize}
}{%
\begin{GateBox}{2}{1}{2}
\BoxLabel{CNOT}

\Input{0}{$a$}
\Input{1}{$c$}
\Output{0}{$a$}
\Output{1}{\Xor{a}{c}}
\end{GateBox}
\begin{center}
\begin{tabular}{cc||cc}
\multicolumn{2}{c}{Inputs} &
\multicolumn{2}{c}{Outputs} \\
$a$ & $c$  & $a$ & \Xor{c}{a} \\ \hline
0 & 0 & 0 & 0 \\
0 & 1 & 0 & 1 \\
1 & 0 & 1 & 1 \\
1 & 1 & 1 & 0
\end{tabular}
\end{center}

\only<4->{
\begin{center}
\begin{quantikz}
\lstick{$a$} & \ctrl{1} & \qw & \rstick{$a$}\\ 
\lstick{$c$} & \targ{}  & \qw & \rstick{\Xor{a}{c}}
\end{quantikz}
\end{center}
}
}
\end{frame}

\begin{frame}{Generally reversible computation}{From \Kaye{} page 14}
\ToDo{I need help reproducing this}
    
\end{frame}
