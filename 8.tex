\SetTitle{8}{Beyond one qubit}{Composite quantum systems}{multqubits}

\begin{frame}{Overview}

\begin{itemize}
    \item We can build larger quantum systems by using multiple qubits.
    \item The mathematics of \href{https://en.wikipedia.org/wiki/Tensor}{tensors} allows us to reason about multi-qubit systems.
    \item If $k$ qubits are separate and do not interact, they can be visualized as $k$ Bloch spheres, and they can be simulated efficiently using a classical computer.
    \item On the other hand, if some qubits are \href{https://en.wikipedia.org/wiki/Quantum_entanglement}{entangled}, then we cannot treat them independently.  Simulating them classically can take exponential time.
    \item Entanglement is a strange but powerful concept in quantum mechanics that can provide significant computational advantages.
\end{itemize}
    
\end{frame}

\begin{frame}{A system of two qubits}{We start here, and generalization will be easy.}
\begin{itemize}
    \item Consider two qubits, one in state \QState{a} and the other in state \QState{b}.
    \item We say the system of those two qubits is in state
    \[
       \QState{a}\QState{b} = \ket{\psi_{a}\psi_{b}} = \TensProd{\QState{a}}{\QState{b}}
    \]
    \item The first two expressions are interchangeable names for the system. The last expressions tells us how to compute the state:  it is the \href{https://en.wikipedia.org/wiki/Tensor_product}{tensor product} of the individual states.
\end{itemize}
\end{frame}


\begin{frame}{Tensor product of two states}{We must keep track of the combinations separately.}
\TwoColumns{%
\only<1-5>{%
\begin{itemize}
    \item<1-> Suppose
    \begin{align*}
        \QState{a} = & \alpha_{a}\ket{0} + \beta_{a}\ket{1} \\
        \QState{b} = & \alpha_{b}\ket{0} + \beta_{b}\ket{1}
    \end{align*}
    \item<2-> Measuring \QState{a} will yield \ket{0} or \ket{1}.
    \item<3-> Measuring \QState{b} will yield \ket{0} or \ket{1}.
    \item<4-> This does not capture the joint probability of outcomes.
    \item<5-> This does the job, with wave amplitudes computed as follows.
\end{itemize}}
\only<6-10>{%
We compute each entry in turn:
\begin{itemize}
    \item<7-> The wave amplitude of \ket{00}.
    \item<8-> The wave amplitude of \ket{01}.
    \item<9-> The wave amplitude of \ket{10}.
    \item<10-> The wave amplitude of \ket{11}.
\end{itemize}
}
\only<11>{%
\Vskip{-3em}\begin{align*}
& \alpha_{a}\,\alpha_{b}\DQB{1}{0}{0}{0}
+ 
\alpha_{a}\,\beta_{b}\DQB{0}{1}{0}{0} \\[1em]
+ \ &
\beta_{a}\,\alpha_{b}\DQB{0}{0}{1}{0}
+ 
\beta_{a}\,\beta_{b}\DQB{0}{0}{0}{1}
\end{align*}
}
\only<12-14>{%
\begin{itemize}
    \item<12-> State $\ket{00} = \TensProd{\ket{0}}{\ket{0}}$ is the outcome of measuring $\QState{a}=$\ket{0} and $\QState{b}=\ket{0}$.
    \item<13-> The probability of that outcome is
    \[
    \Prob{\alpha_{a}\,\alpha_{b}} = \Prob{\alpha_{a}}\,\Prob{\alpha_{b}}
    \]
    \item<14-> This is the joint probability of $\QState{a}=\ket{0}$ and $\QState{b}=\ket{0}$.
\end{itemize}
}
\only<15->{%
\begin{itemize}
    \item Each of the other coefficients is the amplitude on the basis vector for the related measurements.
    \item And the square of the magnitude of an amplitude is the joint probability of seeing the two outcomes.
\end{itemize}
}
}{%
\visible<4->{%
    \begin{align*}
    \TensProd{\QState{a}}{\QState{b}} = & \TensProd{\SQB{\alert<7,8>{\alpha_{a}}}{\alert<9,10>{\beta_{a}}}}{\SQB{\alert<7,9>{\alpha_{b}}}{\alert<8,10>{\beta_{b}}} } \\[2em]
    \only<4>{\not= & \DQB{\alpha_{a}}{\beta_{a}}{\alpha_{b}}{\beta_{b}}%
    }
    \only<5->{= & \DQB{\alert<7,12-14>{\alpha_{a}\,\alpha_{b}}}{\alert<8>{\alpha_{a}\,\beta_{b}}}{\alert<9>{\beta_{a}\,\alpha_{b}}}{\alert<10>{\alpha_{b}\,\beta_{b}}}%
    }
    \end{align*}}
}
\only<11->{%
\MedSkip{}
The resulting state is
\[
\alert<12-14>{\alpha_{a}\,\alpha_{b}}\ket{00}
+\alpha_{a}\,\beta_{b}\ket{01}
+\beta_{a}\,\alpha_{a}\ket{10}
+\beta_{a}\,\beta_{b}\ket{11}
\]
}
\end{frame}

\begin{frame}{Properties of tensor products}{There are more, but these are important for quantum computing.}
\begin{itemize}
    \item If $a \in \Co{}$ and $v$ and $w$ are quantum states, then
    \[ 
       \TensProd{(av)}{w} = \TensProd{v}{(aw)} = a\,(\TensProd{v}{w})
    \]
    \item Tensor products distribute over superpositions.  If $v_{1}$, $v_{2}$, and $w$ are quantum states, then
    \begin{align*}
       \TensProd{(v_{1}+v_{2})}{w} = & \TensProd{v_{1}}{w} + \TensProd{v_{2}}{w} \\
       \TensProd{w}{(v_{1}+v_{2})} = & \TensProd{w}{v_{1}} + \TensProd{w}{v_{2}}
    \end{align*}
\end{itemize}
\end{frame}

\begin{frame}{From states to gates}{We can use tensor arithmetic to reason about systems of gates.}
\Vskip{-3em}\TwoColumns{%
\begin{itemize}
    \item A system of $k$ qubits is characterized by the tensor product of those qubits' states.
    \item Consider for example a 3~qubit system comprised of the states
    \begin{align*}
    \QState{a} = & \PZero{} \\
    \QState{b} = & \POne{} \\
    \QState{c} = & \PHad{}
    \end{align*}
\end{itemize}
}{%
\ket{\QName{a}\QName{b}\QName{c}} is then computed as follows:
\begin{align*}
    & \TensProd{\TensProd{\PZero{}}{\POne{}}}{\PHad{}} \\[1.2em]
    = & \TensProd{\DQB{0}{1}{0}{0}}{\PHad{}}
    = \frac{1}{\sqrt{2}}\QQB{0}{0}{1}{1}{0}{0}{0}{0}
\end{align*}
}
    
\end{frame}