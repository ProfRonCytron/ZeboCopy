\SetTitle{8}{Beyond one qubit}{Composite quantum systems}{multqubits}

\begin{frame}{Overview}

\begin{itemize}
    \item We can build larger quantum systems by using multiple qubits.
    \item The mathematics of \href{https://en.wikipedia.org/wiki/Tensor}{tensors} allows us to reason about multi-qubit systems.
    \item Unless these qubits are entangled, the resulting systems gain no computational power beyond parallelism over a classical computing system.
    \item Entanglement is a strange but powerful concept in quantum mechanics that can provide significant computational advantages.
\end{itemize}
    
\end{frame}

\begin{frame}{A system of two qubits}{We start here, and generalization will be easy.}
\begin{itemize}
    \item Consider two qubits, one in state \QState{a} and the other in state \QState{b}.
    \item We say the system of the two qubits is in state
    \[
       \QState{a}\QState{b} = \ket{\psi_{a}\psi_{b}} = \TensProd{\QState{a}}{\QState{b}}
    \]
    \item The first two expressions are interchangeable names for the system. The last expressions tells us how to compute the state:  it is the \href{https://en.wikipedia.org/wiki/Tensor_product}{tensor product} of the individual states.
\end{itemize}
\end{frame}


\begin{frame}{Tensor product of two states}{We must keep track of the combinations separately.}
\TwoColumns{%
\only<1-5>{%
\begin{itemize}
    \item<1-> Suppose
    \begin{align*}
        \QState{a} = & \alpha_{a}\ket{0} + \beta_{a}\ket{1} \\
        \QState{b} = & \alpha_{b}\ket{0} + \beta_{b}\ket{1}
    \end{align*}
    \item<2-> Measuring \QState{a} will yield \ket{0} or \ket{1}.
    \item<3-> Measuring \QState{b} will yield \ket{0} or \ket{1}.
    \item<4-> This does not capture the joint probability of outcomes.
    \item<5-> This does the job, with wave amplitudes computed as follows.
\end{itemize}}
\only<6-10>{%
We compute each entry in turn:
\begin{itemize}
    \item<7-> The wave amplitude of \ket{00}.
    \item<8-> The wave amplitude of \ket{01}.
    \item<9-> The wave amplitude of \ket{10}.
    \item<10-> The wave amplitude of \ket{11}.
\end{itemize}
}
\only<11>{%
\Vskip{-3em}\begin{align*}
& \alpha_{a}\,\alpha_{b}\DQB{1}{0}{0}{0}
+ 
\alpha_{a}\,\beta_{b}\DQB{0}{1}{0}{0} \\[1em]
+ \ &
\beta_{a}\,\alpha_{b}\DQB{0}{0}{1}{0}
+ 
\beta_{a}\,\beta_{b}\DQB{0}{0}{0}{1}
\end{align*}
}
\only<12-14>{%
\begin{itemize}
    \item<12-> State $\ket{00} = \TensProd{\ket{0}}{\ket{0}}$ is the outcome of measuring $\QState{a}=$\ket{0} and $\QState{b}=\ket{0}$.
    \item<13-> The probability of that outcome is
    \[
    \Prob{\alpha_{a}\,\alpha_{b}} = \Prob{\alpha_{a}}\,\Prob{\alpha_{b}}
    \]
    \item<14-> This is the joint probability of $\QState{a}=\ket{0}$ and $\QState{b}=\ket{0}$.
\end{itemize}
}
\only<15->{%
\begin{itemize}
    \item Each of the other coefficients is the amplitude on the basis vector for the related measurements.
    \item And the square of the magnitude of an amplitude\textbf{} is the joint probability of seeing the two outcomes.
\end{itemize}
}
}{%
\visible<4->{%
    \begin{align*}
    \TensProd{\QState{a}}{\QState{b}} = & \TensProd{\SQB{\alert<7,8>{\alpha_{a}}}{\alert<9,10>{\beta_{a}}}}{\SQB{\alert<7,9>{\alpha_{b}}}{\alert<8,10>{\beta_{b}}} } \\[2em]
    \only<4>{\not= & \DQB{\alpha_{a}}{\beta_{a}}{\alpha_{b}}{\beta_{b}}%
    }
    \only<5->{= & \DQB{\alert<7,12-14>{\alpha_{a}\,\alpha_{b}}}{\alert<8>{\alpha_{a}\,\beta_{b}}}{\alert<9>{\beta_{a}\,\alpha_{b}}}{\alert<10>{\alpha_{b}\,\beta_{b}}}%
    }
    \end{align*}}
}
\only<11->{%
\MedSkip{}
The resulting state is
\[
\alert<12-14>{\alpha_{a}\,\alpha_{b}}\ket{00}
+\alpha_{a}\,\beta_{b}\ket{01}
+\beta_{a}\,\alpha_{a}\ket{10}
+\beta_{a}\,\beta_{b}\ket{11}
\]
}
\end{frame}