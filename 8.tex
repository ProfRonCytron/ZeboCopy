\SetTitle{8}{Universal quantum gates}{Several approaches, culminating in the u3 gate}{multqubits}

\begin{frame}{Universality}{Some initial thoughts}

\begin{itemize}[<+->]
    \item We captured the essence of a single qubit using two real rotational parameters $\phi$ and $\theta$.  Those parameters describe points on the unit sphere, each point corresponding to a quantum state.
    \item A quantum gate is a mapping from states to states.
    \item Because such gates must have inverses, the mapping established by such gates must be \href{https://en.wikipedia.org/wiki/Bijection}{bijective}.
    \item We can think of a gate as a permutation of states, or points on the Bloch sphere.
    \item How many parameters does it take to characterize an arbitrary single-qubit quantum gate?
\end{itemize}

    
\end{frame}

\begin{frame}{Quantum gate}{Properties}

\TwoUnequalColumns{0.4\textwidth}{0.6\textwidth}{%
\begin{center}
\SQBG{\relax}{a}{c}{b}{d}
\end{center}
}{%

\begin{itemize}[<+->]
    \item $a, b, c, d$ are complex
    \item So, apparently 8 parameters
    \item $U$ is invertible and $\Inverse{U}=\Conj{U}$
    \item The columns form an orthonormal basis
    \item The rows form an orthonormal basis
\end{itemize}}
\BigSkip{}
\visible<+->{%
These constraints will take us from 8 parameters to only 3.}
    
\end{frame}

\begin{frame}{Some theoretical results}{From \Kaye{}}
\end{frame}

\begin{frame}{Universality via Pauli gates}
    
\end{frame}

\begin{frame}{The U3 gate}{Theorem statement}
\begin{theorem}
   Up to a global phase, any $2\times 2$ unitary matrix can be expressed as
    \[
    U3(\theta, \phi, \lambda) = 
    \SQBG{\relax}{\cos(\theta/2)}{-\ExpPhase{\lambda}\sin(\theta/2)}{\ExpPhase{\phi}\sin(\theta/2)}{\ExpPhase{(\phi+\lambda)}\cos(\theta/2)}
    \]
    \end{theorem}
    For example, the Hadamard gate \HMatrix{} is $U3(\pi/2, 0, \pi)$, which yields
    \[
    \SQBG{\relax}{\cos \pi/4}{-\ExpPhase{\pi}\sin \pi/4}{\ExpPhase{0}\sin \pi/4}{\ExpPhase{\pi}\cos pi/4} = \HMatrix{}
    \]

\end{frame}

\begin{frame}{Proof}{We use the constraints}
\TwoColumns{%
$U = \SQBG{\relax}{a}{b}{c}{d}$ so
$\Conj{U} = \SQBG{\relax}{\Conj{a}}{\Conj{c}}{\Conj{b}}{\Conj{d}}$
\SmallSkip{}
The columns of $U$ are \href{https://en.wikipedia.org/wiki/Orthonormality}{orthonormal}:
\begin{align*}
    \Conj{a}a + \Conj{c}c = & \Prob{a} + \Prob{c} = 1 \\
    \Conj{b}b + \Conj{d}d = & \Prob{b} + \Prob{d} = 1 
\end{align*}
The columns of \Conj{U} are orthonormal:
\begin{align*}
    \Prob{a} + \Prob{b} = & 1 \\
    \Prob{c} + \Prob{d} = & 1
\end{align*}
}{%
\Vskip{-4em}\begin{center}
\begin{TIKZP}[scale=0.5]
\UnitComplexCircle{}
\end{TIKZP}
\end{center}
Because $\Mag{a}\leq 1$, $a$ is some point on or in the unit
complex circle.
\begin{itemize}
    \item Let $\ExpPhase{\phi_{a}}$ be a point on the circle.
    \item We can then scale it by $\cos(\theta/2)$ to obtain any point on or in the circle.
    \item We thus obtain
    \[
    a = \ExpPhase{\phi_{a}}\cos(\theta/2)
    \]
\end{itemize}
}
\end{frame}