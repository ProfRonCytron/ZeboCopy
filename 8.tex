\SetTitle{8}{Universal quantum gates}{Several approaches, culminating in the u3 gate}{multqubits}

\begin{frame}{Universality}{Some initial thoughts}

\begin{itemize}[<+->]
    \item We captured the essence of a single qubit using two real rotational parameters $\phi$ and $\theta$.  Those parameters describe points on the unit sphere, each point corresponding to a quantum state.
    \item A quantum gate is a mapping from states to states.
    \item Because such gates must have inverses, the mapping established by such gates must be \href{https://en.wikipedia.org/wiki/Bijection}{bijective}.
    \item We can think of a gate as a permutation of states, or points on the Bloch sphere.
    \item How many parameters does it take to characterize an arbitrary single-qubit quantum gate?
\end{itemize}

    
\end{frame}

\begin{frame}{Quantum gate}{Properties}

\TwoUnequalColumns{0.4\textwidth}{0.6\textwidth}{%
\begin{center}
\SQBG{\relax}{a}{c}{b}{d}
\end{center}
}{%

\begin{itemize}[<+->]
    \item $a, b, c, d$ are complex
    \item So, apparently 8 parameters
    \item $U$ is invertible and $\Inverse{U}=\Conj{U}$
    \item The columns form an orthonormal basis
    \item The rows form an orthonormal basis
\end{itemize}}
\BigSkip{}
\visible<+->{%
These constraints will take us from 8 parameters to only 3.}
    
\end{frame}

\begin{frame}{Universality via Pauli gates}
    
\end{frame}