\SetTitle{8}{Beyond one qubit}{Composite quantum systems}{multqubits}

\begin{frame}{Overview}

\begin{itemize}
    \item We can build larger quantum systems by using multiple qubits.
    \item The mathematics of \href{https://en.wikipedia.org/wiki/Tensor}{tensors} allows us to reason about multi-qubit systems.
    \item If $k$ qubits are separate and do not interact, they can be visualized as $k$ Bloch spheres, and they can be simulated efficiently using a classical computer.
    \item On the other hand, if some qubits are \href{https://en.wikipedia.org/wiki/Quantum_entanglement}{entangled}, then we cannot treat them independently.  Simulating them classically can take exponential time.
    \item Entanglement is a strange but powerful concept in quantum mechanics that can provide significant computational advantages.
\end{itemize}
    
\end{frame}

\begin{frame}{A system of two qubits}{We start here, and generalization will be easy.}
\begin{itemize}
    \item Consider two qubits, one in state \QState{a} and the other in state \QState{b}.
    \item We say the system of those two qubits is in state
    \[
       \QState{a}\QState{b} = \ket{\psi_{a}\psi_{b}} = \TensProd{\QState{a}}{\QState{b}}
    \]
    \item The first two expressions are interchangeable names for the system. The last expressions tells us how to compute the state:  it is the \href{https://en.wikipedia.org/wiki/Tensor_product}{tensor product} of the individual states.
\end{itemize}
\end{frame}


\begin{frame}{Tensor product of two states}{We must keep track of the combinations separately.}
\TwoColumns{%
\only<1-5>{%
\begin{itemize}
    \item<1-> Suppose
    \begin{align*}
        \QState{a} = & \alpha_{a}\ket{0} + \beta_{a}\ket{1} \\
        \QState{b} = & \alpha_{b}\ket{0} + \beta_{b}\ket{1}
    \end{align*}
    \item<2-> Measuring \QState{a} will yield \ket{0} or \ket{1}.
    \item<3-> Measuring \QState{b} will yield \ket{0} or \ket{1}.
    \item<4-> This does not capture the joint probability of outcomes.
    \item<5-> This does the job, with wave amplitudes computed as follows.
\end{itemize}}
\only<6-10>{%
We compute each entry in turn:
\begin{itemize}
    \item<7-> The wave amplitude of \ket{00}.
    \item<8-> The wave amplitude of \ket{01}.
    \item<9-> The wave amplitude of \ket{10}.
    \item<10-> The wave amplitude of \ket{11}.
\end{itemize}
}
\only<11>{%
\Vskip{-3em}\begin{align*}
& \alpha_{a}\,\alpha_{b}\DQB{1}{0}{0}{0}
+ 
\alpha_{a}\,\beta_{b}\DQB{0}{1}{0}{0} \\[1em]
+ \ &
\beta_{a}\,\alpha_{b}\DQB{0}{0}{1}{0}
+ 
\beta_{a}\,\beta_{b}\DQB{0}{0}{0}{1}
\end{align*}
}
\only<12-14>{%
\begin{itemize}
    \item<12-> State $\ket{00} = \TensProd{\ket{0}}{\ket{0}}$ is the outcome of measuring $\QState{a}=$\ket{0} and $\QState{b}=\ket{0}$.
    \item<13-> The probability of that outcome is
    \[
    \Prob{\alpha_{a}\,\alpha_{b}} = \Prob{\alpha_{a}}\,\Prob{\alpha_{b}}
    \]
    \item<14-> This is the joint probability of $\QState{a}=\ket{0}$ and $\QState{b}=\ket{0}$.
\end{itemize}
}
\only<15->{%
\begin{itemize}
    \item Each of the other coefficients is the amplitude on the basis vector for the related measurements.
    \item And the square of the magnitude of an amplitude is the joint probability of seeing the two outcomes.
\end{itemize}
}
}{%
\visible<4->{%
    \begin{align*}
    \TensProd{\QState{a}}{\QState{b}} = & \TensProd{\SQB{\alert<7,8>{\alpha_{a}}}{\alert<9,10>{\beta_{a}}}}{\SQB{\alert<7,9>{\alpha_{b}}}{\alert<8,10>{\beta_{b}}} } \\[2em]
    \only<4>{\not= & \DQB{\alpha_{a}}{\beta_{a}}{\alpha_{b}}{\beta_{b}}%
    }
    \only<5->{= & \DQB{\alert<7,12-14>{\alpha_{a}\,\alpha_{b}}}{\alert<8>{\alpha_{a}\,\beta_{b}}}{\alert<9>{\beta_{a}\,\alpha_{b}}}{\alert<10>{\alpha_{b}\,\beta_{b}}}%
    }
    \end{align*}}
}
\only<11->{%
\MedSkip{}
The resulting state is
\[
\alert<12-14>{\alpha_{a}\,\alpha_{b}}\ket{00}
+\alpha_{a}\,\beta_{b}\ket{01}
+\beta_{a}\,\alpha_{a}\ket{10}
+\beta_{a}\,\beta_{b}\ket{11}
\]
}
\end{frame}

\begin{frame}{Properties of tensor products}{There are more, but these are important for quantum computing.}
\begin{itemize}
    \item If $a \in \Co{}$ and $v$ and $w$ are quantum states, then
    \[ 
       \TensProd{(av)}{w} = \TensProd{v}{(aw)} = a\,(\TensProd{v}{w})
    \]
    \item Tensor products distribute over superpositions.  If $v_{1}$, $v_{2}$, and $w$ are quantum states, then
    \begin{align*}
       \TensProd{(v_{1}+v_{2})}{w} = & \TensProd{v_{1}}{w} + \TensProd{v_{2}}{w} \\
       \TensProd{w}{(v_{1}+v_{2})} = & \TensProd{w}{v_{1}} + \TensProd{w}{v_{2}}
    \end{align*}
\end{itemize}
\end{frame}


\begin{frame}{Now three qubits}{Each new qubit is incorporated by tensor product.}

\Vskip{-3em}\TwoColumns{%
\only<1-10>{%
\begin{itemize}
    \item<1-> A system of $k$ qubits is characterized by the tensor product of those qubits' states.
    \item<2-> Consider for example a 3~qubit system comprised of the states
    \begin{align*}
    \QState{a} = \PZero{} &
    \QState{b} = \POne{} \\
    \QState{c} = & \PHad{}
    \end{align*}
\end{itemize}
\only<3-10>{%

How do we compute \ket{\QName{a}\QName{b}\QName{c}}?}}%
\only<11>{%
We can describe a quantum state using a column vector, but when that is sparse, it may be more clear to write it out in terms of its basis states:
\[
\frac{\ket{010} + \ket{011}}{\sqrt{2}}
\]
Counting down a column $(0, 1, 2, \ldots)$, we include the binary encoding of each index whose entry is non-zero.
}
}{%
\only<3->{%
\begin{align*}
    & \TensProd{\TensProd{\PZero{}}{\POne{}}}{\PHad{}} \\
    = & \TensProd{\DQB{\alert<3-4>{0}}{\alert<5-6>{1}}{\alert<7-8>{0}}{\alert<9-10>{0}}}{\frac{1}{\sqrt{2}}\SQB{\alert<3,5,7,9>{1}}{\alert<4,6,8,10>{1}}}
    = \frac{1}{\sqrt{2}}\QQB{%
    \alert<3>{0}}{%
    \alert<4>{0}}{%
    \alert<5>{1}}{%
    \alert<6>{1}}{%
    \alert<7>{0}}{%
    \alert<8>{0}}{%
    \alert<9>{0}}{%
    \alert<10>{0}}
\end{align*}}
}
    
\end{frame}


\begin{frame}{From states to gates}{We can use tensor products of matrices to reason about systems of gates.}
Computing the tensor product two matrices is similar to computing the tensor product of two vectors.  It is demonstrated by the example of 
\[
\TensProd{\PauliZ{}}{\Hadamard{}} = %
\TensProd{%
   \SQBG{\relax}{%
   \alert<1-4>{1}}{%
   \alert<5-8>{0}}{%
   \alert<9-12>{0}}{%
   \alert<13-16>{-1}}}{%
   \SQBG{\frac{1}{\sqrt{2}}}{%
   \alert<1,5,9,13>{1}}{%
   \alert<2,6,10,14>{1}}{%
   \alert<3,7,11,15>{1}}{%
   \alert<4,8,12,16>{-1}}}
   = \frac{1}{\sqrt{2}}\begin{pmatrix*}[r]
   \visible<1-4,17->{ \alert<1>{1} &  \alert<2>{1} }& \visible<5-8,17->{\alert<5>{0} & \alert<6>{0} }\\
     \visible<1-4,17->{\alert<3>{1} & \alert<4>{-1} }& \visible<5-8,17->{\alert<7>{0} & \alert<8>{0} }\\
     \visible<9-12,17->{\alert<9>{0} &  \alert<10>{0}} & \visible<13-16,17->{\alert<13>{-1} & \alert<14>{-1}} \\
     \visible<9-12,17->{\alert<11>{0} &  \alert<12>{0} }& \visible<13-16,17->{\alert<15>{-1} & \alert<16>{1}}
   \end{pmatrix*}
\]
\only<18-19>{%
\begin{itemize}
\item<18-> This gate can then be applied to any two-qubit state. 
\item<19-> If that state is \TensProd{\QState{a}}{\QState{b}}, then the result will be as if \PauliZ{} and \Hadamard{} acted separately on each input.
\[
(\TensProd{\PauliZ{}}{\Hadamard{}})\,\ket{\QName{a}\QName{b}} = \TensProd{(\PauliZ{}\QState{a})}{(\Hadamard{}\QState{b})}
\]
\end{itemize}
}%
\only<20>{%
\begin{itemize}

\item Why compute the tensor product of the two gates, if the result can be computed separately?

\item There are some two-qubit states that cannot be written as any \TensProd{\QState{a}}{\QState{b}}.  Such entangled states require using the gates' tensor product.
\end{itemize}}

\end{frame}

\begin{frame}{Another example}{We can create the uniform superposition of two qubits using Hadamard gates.}
\TwoUnequalColumns{0.2\textwidth}{0.8\textwidth}{%
\adjustbox{valign=t}{\begin{quantikz}
\lstick{\QZero{}}\slice{\alert<1>{\QState{0}}} &  \gate{H}\slice{\alert<4>{\QState{1}}}  & \qw\\
\lstick{\QZero{}} &   \gate{H}    &  \qw
\end{quantikz}}
}{%
\Vskip{-3em}\begin{align*} 
\alert<1-3>{\QState{0}}= &\ket{00} 
\visible<2->{=  \TensProd{\PZero}{\PZero}}
\visible<3->{= \DQB{1}{0}{0}{0} }\\
\visible<4->{%
\alert<4->{\QState{1}} = & (\TensProd{\Hadamard{}}{\Hadamard{}})\,\DQB{1}{0}{0}{0} 
\visible<5->{= \frac{1}{\sqrt{2}}\frac{1}{\sqrt{2}}
    \begin{pmatrix*}[r]
      1 & 1 & 1 & 1 \\
      1 & -1 & 1 & -1 \\
      1 & 1 & -1 & -1 \\
      1 & -1 & -1 & 1
    \end{pmatrix*}\DQB{1}{0}{0}{0}} \\
   \visible<6->{ = & \frac{1}{2}\DQB{1}{1}{1}{1}}
   \visible<7->{= \frac{\ket{00}+\ket{01}+\ket{10}+\ket{11}}{2}
   }}
\end{align*}
}
\end{frame}

\begin{frame}{General superposition of $n$ qubits}{The idea extends to any number of qubits.}
\TwoUnequalColumns{0.25\textwidth}{0.75\textwidth}{%
\adjustbox{scale=0.9,valign=t}{\begin{quantikz}
\lstick{\QZero{}}\slice{\alert<1>{\QState{0}}} &  \gate{H}\slice{\alert<2-3>{\QState{1}}}  & \meter{\textcolor<5-6>{red}{0/1}} \\
\lstick{\QZero{}} &   \gate{H}    &  \meter{\textcolor<5-6>{red}{0/1}} \\
\lstick{$\bullet$} & \mbox{$\bullet$} & \mbox{$\bullet$}\\
\lstick{$\bullet$} & \mbox{$\bullet$} & \mbox{$\bullet$}\\
\lstick{$\bullet$} & \mbox{$\bullet$} & \mbox{$\bullet$} \\
\lstick{\QZero{}}\slice{\alert<1>{\QState{0}}} &  \gate{H}\slice{\alert<2>{\QState{1}}}  & \meter{\textcolor<5-6>{red}{0/1}}
\end{quantikz}}
}{%
\only<1-4>{%
\Vskip{-3em}\begin{align*} 
\alert<1>{\QState{0}} = & \ket{0^{n}} \\
\visible<2->{\alert<2->{\QState{1}} = & \TensSupProd{\Hadamard{}}{n}\ket{0^{n}}\\
\visible<3->{= & \frac{1}{\sqrt{2^{n}}}\sum_{\AllBits{x}{n}} \ket{x}
\visible<4->{=\frac{1}{\sqrt{2^{n}}} \sum_{j=0}^{2^{n}-1} \ket{j}}
}}
\end{align*}
\Vskip{-2em}\begin{itemize}
\item<1-> There are $n$ inputs, each \ket{0}.
\item<2-> A Hadamard gate is applied to each input.
\item<3-> The result is the uniform superposition of all possible $n$-bit vectors.  
\item<4-> Interpreted as integers base-2, the result is also the uniform superposition of $2^{n}$ values.
\end{itemize}
}%
\only<5->{%
\begin{itemize}
   \item<5->On \alert<5>{measurement} of the $n$ qubits, each outcome is equally likely to be \Zero{} or \One{}.
   \item<6-> Taken as an $n$-bit string, each of the possible $2^n$ outcomes is therefore also likely.
   \item<7-> In theory, this is a perfect random-number generator for an integer in the interval $\left[0,2^{n}\right)$.
   \begin{itemize}
       \item We know from physics that the outcome of these measurements is completely unpredictable.
       \item Uniformity of outcome depends on each \Hadamard{} gate placing its input into a perfect (unbiased) superposition of $\frac{\ket{0}+\ket{1}}{\sqrt{2}}$.
   \end{itemize}
\end{itemize}
}
}
\end{frame}

\begin{frame}{Entangling two qubits}{We use gates we have studied to create an interesting quantum state.}

\TwoUnequalColumns{0.3\textwidth}{0.7\textwidth}{%
\adjustbox{valign=t}{\begin{quantikz}
\lstick{\QZero{}}\slice{\alert<1>{\QState{0}}} &  \gate{H}\slice{\alert<2-5>{\QState{1}}} & \ctrl{1}\slice{\alert<7-8>{\QState{2}}} & \meter{\textcolor<8->{red}{0/1}}\\
\lstick{\QZero{}} &   \qw    &  \targ{}  & \meter{\textcolor<8->{red}{0/1}}
\end{quantikz}}%
\only<5>{%

We can do this because the qubits are acting independently at this point.  This state can be factored into the tensor product of two states.
}
}{%
\only<1-7>{%
\Vskip{-2.5em}\begin{itemize}
    \item<1-> As before, our initial state $\QState{0}=\ket{00}=\DQB{1}{0}{0}{0}$
    \only<2-3>{%
    \item<2-> To compute the state here, we can tensor the \Hadamard{} and \Identity{} gates, obtaining
    $\RootTwo{}\begin{pmatrix*}[r]
        1 & 0 & 1 & 0 \\
        0 & 1 & 0 & 1 \\
        1 & 0 & -1 & 0 \\
        0 & 1 & 0 & -1
    \end{pmatrix*}$
    \item<3-> Applying those gates to \QState{0} yields $\QState{1}=\RootTwo{}\DQB{1}{0}{1}{0}$}
    \only<4-5>{%
    \item It's easier to compute \QState{1} as follows:
   \begin{align*}
       \TensProd{\Hadamard{}\ket{0}}{\Identity{}\ket{0}}
     = & \TensProd{\HMatrix{}\PZero{}}{\PZero} \\
     = & \RootTwo{}\DQB{1}{0}{1}{0}
    \end{align*}}
    \only<6-7>{%
        \item<6-> $\QState{1} = \RootTwo{}\DQB{1}{0}{1}{0}$
        \item<7-> $\QState{2} = \CNOTMatrix{}\RootTwo\DQB{1}{0}{1}{0} = \RootTwo{}\DQB{1}{0}{0}{1}$
    }
    \end{itemize}}
    \only<8->{%
    \[\QState{2} =  \RootTwo{}\DQB{1}{0}{0}{1}=\frac{\ket{00}+\ket{11}}{\sqrt{2}}\]
    }
}
\only<8->{%
\begin{itemize}
\item<8-> In this state, we have a 50\% chance of seeing \ket{0} at each sensor, and a 50\% chance of seeing \ket{1} at each sensor.
\item<9-> The two measurements are absolutely correlated, \alert<10>{no matter in which order the measurements are taken}, and \alert<11>{no matter the distance between the sensors}.
\item<12->  The two outcomes are also uniformly random but completely unpredictable.
\end{itemize}
}
\end{frame}

\begin{frame}{The Bell state $\frac{\ket{00}+\ket{11}}{\sqrt{2}}$}{Our first entangled state}
\Vskip{-3.5em}\begin{itemize}
    \item<1-> The behavior of this state is an incredible consequence of quantum theory.
    \item<2-> While physics confirms quantum behavior experimentally, there is no explanation yet for \emph{why}.
    \item<3-> This state is \emph{entangled}, meaning that it cannot be expressed as the tensor product of two single-qubit states.
\end{itemize}
\only<3->{%
\SmallSkip{}
\TwoUnequalColumns{0.58\textwidth}{0.42\textwidth}{%
\visible<3->{Proof (by contradiction):}
\visible<4->{Suppose there exist
    \[
    \SQB{a}{b} \mbox{ and } \SQB{c}{d}\ |\ \RootTwo{}\TensProd{\SQB{a}{b}}{\SQB{c}{d}} = \RootTwo{}\DQB{\textcolor<6>{OrangeRed}{1}}{\textcolor<6>{NavyBlue}{0}}{\textcolor<7>{OrangeRed}{0}}{\textcolor<7>{NavyBlue}{1}}
    \]}
}{%
\visible<5->{Then}
\begin{align*}
    \visible<6->{\textcolor<6>{OrangeRed}{ac} = & 1 & \textcolor<6>{NavyBlue}{ad} =& 0 }\\
    \visible<7->{\visible<8->{\times} \textcolor<7>{NavyBlue}{bd} = & 1 & \visible<8->{\times} \textcolor<7>{OrangeRed}{bc} =& 0 }\\
    \visible<8->{---- & -- & ---- & --} \\
    \visible<9->{abcd = & 1 & abcd =& 0}
\end{align*}
\visible<10->{which is a contradiction $\square$}
}}
    
\end{frame}

\begin{frame}{How entangled are the qubits?}{Their experience with CNOT cannot be undone unless they are physically proximate again.}

\TwoUnequalColumns{0.35\textwidth}{0.65\textwidth}{%
\adjustbox{valign=t}{\begin{quantikz}
\lstick{\QZero{}}\slice{\QState{0}} &  \gate{H}\slice{\QState{1}} & \ctrl{1}\slice{\QState{2}} & \meter{0/1}\\
\lstick{\QZero{}} &   \qw    &  \targ{}  & \meter{0/1}
\end{quantikz}}%
}{%
\begin{itemize}
    \item The qubits' entanglement persists in any basis.
    \item No action taken on them separately can disentangle them.
    \item If they were brought back together and run through a \NamedGate{CNOT} gate again, then they would disentangle.  \NamedGate{CNOT} is its own inverse.
\end{itemize}
}
\MedSkip{}
Entanglement defies classical physics and logic and affords advantages for games and computations.
    
\end{frame}

\begin{frame}{Pondering entanglement}{Einstein regarded quantum theory as incomplete.}

\begin{itemize}
    \item<1-> Upon measurement, our entangled qubits collapse unpredictably to the same value, instantaneously, even if they are separated by light years.
    \item<2-> Einstein regarded this \textit{spukhafte Fernwirkung} (spooky action at a distance) as evidence that quantum theory is incomplete.
    \item<3-> For example, local hidden variables would explain how qubits manage to provide the same measurement after entanglement.
    \item<6-> We shall soon rule out local hidden variables.
    \item<7-> Perhaps there are global hidden variables, but how would we prove this?
\end{itemize}
\only<3-7>{%
\begin{center}
\begin{TIKZP}
    \draw (0,0)  -| ++(1,1) 
    node[pos=0.25,below] {$q_0$} 
    -| (0,0);
    \draw<3,5-> (1.5,0)  -| ++(1,1) 
    node[pos=0.25,below] {$q_1$} 
    -| (1.5,0);
    \draw<4> (0.5,0)  -| ++(1,1) 
    node[pos=0.25,below] {$q_1$} 
    -| (0.5,0);
    \fill<4>[red] (0.5,0) rectangle ++(0.5, 1);
    \path<4> (0.75,1) node[above] {\NamedGate{CNOT}};
    \fill<5->[red] (0,0) rectangle ++(1,1);
    \fill<5->[red] (1.5,0) rectangle ++(1,1);
\end{TIKZP}
\end{center}
}
\only<8->{%
\MedSkip{}
Gravitational attraction exists at arbitrary distances, but we seem OK with that.  Is it so strange that the fate of two qubits is influenced as theory predicts by their \NamedGate{CNOT} experience?}
    
\end{frame}