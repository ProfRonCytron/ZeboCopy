\SetTitle{7}{Universal quantum gates}{Several approaches, culminating in the qiskit U gate}{07}

\begin{frame}{Universality}{Some initial thoughts}

\begin{itemize}[<+->]
    \item We captured the essence of a single qubit using two real rotational parameters $\phi$ and $\theta$.  Those parameters describe points on the unit sphere, each point corresponding to a quantum state.
    \item A quantum gate is a mapping from states to states.
    \item Because such gates must have inverses, the mapping established by such gates must be \href{https://en.wikipedia.org/wiki/Bijection}{bijective}.
    \item We can think of a gate as a permutation of states, or points, on the Bloch sphere.
    \item How many parameters does it take to characterize an arbitrary single-qubit quantum gate?
\end{itemize}

    
\end{frame}

\begin{frame}{Quantum gate}{Properties}

\TwoUnequalColumns{0.4\textwidth}{0.6\textwidth}{%
We begin with an arbitrary unitary matrix $U$ to represent a quantum gate for a single qubit:
\[
U=\SQBG{\relax}{a}{c}{b}{d}
\]
}{%
\begin{itemize}[<+->]
    \item $a, b, c, d$ are complex
    \item So, apparently 8 parameters
    \item $U$ is invertible and $\Inverse{U}=\Conj{U}$
    \item The columns form an orthonormal basis
    \item The rows form an orthonormal basis
\end{itemize}}
\BigSkip{}
\visible<+->{%
These constraints will take us from 8 parameters to only 3.}
    
\end{frame}

\begin{frame}{The Qiskit U gate}{Theorem statement}
\begin{theorem}
   Up to a global phase, any $2\times 2$ unitary matrix can be expressed as
    \[
    U(\theta, \phi, \lambda) = 
    \UGate{}
    \] where
    $0 \leq \theta,\phi,\lambda < 2\pi$
    \end{theorem}
    For example, the Hadamard gate \HMatrix{} is $U(\pi/2, 0, \pi)$, which yields
    \[
    \SQBG{\relax}{\cos \pi/4}{-\ExpPhase{\pi}\sin \pi/4}{\ExpPhase{0}\sin \pi/4}{\ExpPhase{\pi}\cos pi/4} = \HMatrix{}
    \]

\end{frame}

\begin{frame}{Proof}{We begin with an arbitrary unitary matrix $U$ with complex entries}
\TwoColumns{%
$U = \SQBG{\relax}{a}{b}{c}{d}$ so
$\Conj{U} = \SQBG{\relax}{\Conj{a}}{\Conj{c}}{\Conj{b}}{\Conj{d}}$

\MedSkip{}
Because $U$ is unitary:
\begin{itemize}
\item<1-> Each column of $U$ (row of \Conj{U}) is \href{https://en.wikipedia.org/wiki/Orthonormality}{orthonormal}:
\begin{align*}
    \Conj{a}a + \Conj{c}c = & \Prob{a} + \Prob{c} = 1 \\
    \Conj{b}b + \Conj{d}d = & \Prob{b} + \Prob{d} = 1 
\end{align*}
\item<2-> $\Implies{\Prob{a} + \Prob{c} = 1}{\Mag{a} \leq 1}$
\end{itemize}
\visible<5>{%
\alert<5>{%
We use $\theta/2$ to accommodate $0\leq \theta < 2\pi$}}
}{%
\Vskip{-5em}\begin{center}
\begin{TIKZP}[scale=0.45]
\UnitComplexCircle{}
\end{TIKZP}
\end{center}
\visible<3->{%
Because $\Mag{a}\leq 1$, $a$ is some point on or in the unit
complex circle.
\begin{itemize}
    \item<4-> Let $\ExpPhase{\phi_{a}}$ be a point on the circle.
    \item<5-> We can then scale it by \alert<5>{$\cos(\theta/2)$} to obtain any point on or in the circle.
    \item<6-> We thus obtain
    \[
    a = \ExpPhase{\phi_{a}}\cos(\theta/2)
    \]
\end{itemize}}
}
\end{frame}

\begin{frame}{Proof (continued)}{The other entries follow from the requirement that each row and column is \href{https://en.wikipedia.org/wiki/Unit_vector}{normal}}
\[
    U=\SQBG{\relax}{a}{b}{c}{d} = 
    \SQBG{\relax}{\ExpPhase{\phi_{a}}\cos(\theta/2)}{\visible<4->{\ExpPhase{\phi_{b}}\sin(\theta/2)}}{\visible<6->{\ExpPhase{\phi_{c}}\sin(\theta/2)}}{\visible<7->{\ExpPhase{\phi_{d}}\cos(\theta/2)}}
    \]
\TwoColumns{%
\begin{itemize}
    \item<2-> Each row of $U$ is orthonormal, so 
    $\Prob{a}+\Prob{b}=1$
    \item<3-> $\Prob{b}=\sin^{2}(\theta/2)$
    \item<4-> We can let $b=\ExpPhase{\phi_{b}}\sin(\theta/2)$
    \item<5-> $\Prob{a}+\Prob{c}=1$
    \item<6-> We can let $c=\ExpPhase{\phi_{c}}\sin(\theta/2)$
    \item<7-> Same for $d$
\end{itemize}
}{%
\begin{itemize}
    \item<3-> $\Mag{\ExpPhase{\theta}}=1$ for all $\theta$
    \item<4-> Yields the correct magnitude with an arbitrary phase $\phi_{b}$
    \item<5-> $c$ and $d$ follow in the same way
    \item<7-> We now have 5 parameters instead of 8
\end{itemize}
}
\end{frame}

\begin{frame}{Proof (continued)}{We can simplify by applying orthogonality of $U$'s columns}
\Vskip{-3em}\[
    U=\SQBG{\relax}{a}{b}{c}{d} = 
    \SQBG{\relax}{\textcolor{\RCone}{\ExpPhase{\phi_{a}}\cos(\theta/2)}}{\textcolor{\RCtwo}{\ExpPhase{\phi_{b}}\sin(\theta/2)}}{\textcolor{\RCthree}{\ExpPhase{\phi_{c}}\sin(\theta/2)}}{\textcolor{\RCfour}{\ExpPhase{\phi_{d}}\cos(\theta/2)}}
    \]
    
\TwoColumns{%
\begin{itemize}
    \item<1-> The columns of $U$ are orthogonal
    \item<2-> $\CQB{\Conj{\textcolor{\RCone}{a}}}{\Conj{\textcolor{\RCthree}{c}}} \SQB{\textcolor{\RCtwo}{b}}{\textcolor{\RCfour}{d}}=\Conj{\textcolor{\RCone}{a}}\textcolor{\RCtwo}{b} + \Conj{\textcolor{\RCthree}{c}}\textcolor{\RCfour}{d}=0$
\end{itemize}
}{%
\begin{itemize}
        \item<3-> Let's plug in the matrix values
        \item<4-> Recall the conjugate of $\ExpPhase{x}$ is $\ExpNegPhase{x}$
\end{itemize}
}
\visible<4->{%
\MedSkip{}\[
\textcolor{\RCone}{\ExpNegPhase{\phi_a}\cos(\theta/2)}\,\textcolor{\RCtwo}{\ExpPhase{\phi_b}\sin(\theta/2)}
+
\textcolor{\RCthree}{\ExpNegPhase{\phi_c}\cos(\theta/2)}\,\textcolor{\RCfour}{\ExpPhase{\phi_d}\sin(\theta/2)}
= 0 \]}%
\visible<5->{\[
\cos(\theta/2)\sin(\theta/2)\left[
\ExpPhase{(\phi_{b}-\phi_{a})} +
\ExpPhase{(\phi_{d}-\phi_{c})} \right]=0
\]}%
\visible<6->{%
The value of $\theta$ must be arbitrary, so we can only get $0$ when
\[
\ExpPhase{(\phi_{b}-\phi_{a})} = - \ExpPhase{(\phi_{d}-\phi_{c})}
\]}
\end{frame}
\begin{frame}{Proof (continued)}{We can now relate the phases}

\Vskip{-3em}\[
    U=\SQBG{\relax}{a}{b}{c}{d} = 
    \SQBG{\relax}{\textcolor{\RCone}{\ExpPhase{\phi_{a}}\cos(\theta/2)}}{\textcolor{\RCtwo}{\ExpPhase{\phi_{b}}\sin(\theta/2)}}{\textcolor{\RCthree}{\ExpPhase{\phi_{c}}\sin(\theta/2)}}{\textcolor{\RCfour}{\ExpPhase{\phi_{d}}\cos(\theta/2)}}
    \]
    
\begin{align*}
\ExpPhase{(\phi_{b}-\phi_{a})} &= - \ExpPhase{(\phi_{d}-\phi_{c})} \\
 & = \ExpPhase{\pi}\,\ExpPhase{(\phi_{d}-\phi_{c})} \\
 & = \ExpPhase{(\phi_{d}-\phi_{c}+\pi)} \\
 \textcolor{\RCfour}{\phi_{d}} &= \phi_{b}+\phi_{c}-\phi_{a} -\pi \\
 &= \phi_{b}+\phi_{c}-\phi_{a} -\pi + 2\pi \\
 &= (\phi_{b}+\pi) + \phi_{c}-\phi_{a}
\end{align*}
We next eliminate $\phi_{d}$ in favor of the other phases.
\end{frame}

\begin{frame}{Proof (continued)}{Now 4 parameters, but soon only 3}
\Vskip{-3em}\[
    U=\SQBG{\relax}{a}{b}{c}{d} = 
    \SQBG{\relax}{\textcolor{\RCone}{\ExpPhase{\phi_{a}}\cos(\theta/2)}}{\textcolor{\RCtwo}{\ExpPhase{\phi_{b}}\sin(\theta/2)}}{\textcolor{\RCthree}{\ExpPhase{\phi_{c}}\sin(\theta/2)}}{\textcolor{\RCfour}{\ExpPhase{\phi_{d}}\cos(\theta/2)}}
    \]
    \[
    \phi_{d} = (\phi_{b}+\pi) + \phi_{c}-\phi_{a}
    \]
    \begin{align*}
    \visible<2->{
    U&= 
\SQBG{%
    \phantom{\ExpPhase{\phi_a}}}{%
    \textcolor{Black}{\ExpPhase{\phi_{a}}\cos(\theta/2)}}{%
    \textcolor{Black}{\ExpPhase{\phi_{b}}\sin(\theta/2)}}{%
    \textcolor{Black}{\ExpPhase{\phi_{c}}\sin(\theta/2)}}{%
    \textcolor{Black}{\ExpPhase{((\phi_{b}+\pi) + \phi_{c}-\phi_{a})}\cos(\theta/2)}} \\[1.5em]}
    \visible<3->{
    &\only<3-4>{=}\only<5->{\equiv} \SQBG{%
    \visible<3-4>{\ExpPhase{\phi_a}}}{%
    \textcolor{Black}{\cos(\theta/2)}}{%
    \textcolor{Black}{\ExpPhase{(\phi_{b}-\phi_{a})}\sin(\theta/2)}}{%
    \textcolor{Black}{\ExpPhase{(\phi_{c}-\phi_{a})}\sin(\theta/2)}}{%
    \textcolor{Black}{\ExpPhase{((\phi_{b}+\pi) + \phi_{c}-2\phi_{a})}\cos(\theta/2)}}}
    \end{align*}
\visible<4->{
The coefficient $\ExpPhase{\phi_{a}}$ is a \emph{global} phase and can therefore be \visible<5->{ignored.}}
    
\end{frame}
\begin{frame}{Conclusion of Proof}{We now have a general unitary matrix specified using only three parameters}
\def\Phii{\textcolor{\RCone}{\phi}}
\def\Lambdaa{\textcolor{\RCtwo}{\lambda}}
\[
U = \SQBG{%
    \relax}{%
    \textcolor{Black}{\cos(\theta/2)}}{%
    \ExpPhase{(\textcolor<2->{\RCtwo}{\phi_{b}-\phi_{a}-\pi}+\pi)}\sin(\theta/2)}{%
    \ExpPhase{\textcolor<2->{\RCone}{(\phi_{c}-\phi_{a})}}\sin(\theta/2)}{%
    \textcolor{Black}{\ExpPhase{((\phi_{b}+\pi) + \phi_{c}-2\phi_{a})}\cos(\theta/2)}}
    \]
\TwoColumns{%

\Vskip{-3.5em}\begin{align*}
     \visible<2->{\mbox{Let } \Phii &= \textcolor<2>{\RCone}{\phi_{c}-\phi_{a}} \\
     \mbox{Let } \Lambdaa &= \textcolor{\RCtwo}{\phi_{b}-\phi_{a}-\pi}}
\end{align*}
\Vskip{-3em}
\only<5>{%
\begin{align*}
     \visible<5->{%
     b' &= \ExpPhase{(\Lambdaa+\pi)}\sin(\theta/2) \\
     &= \ExpPhase{\pi}\ExpPhase{\Lambdaa}\sin(\theta/2) \\
     &= -\ExpPhase{\Lambdaa}\sin(\theta/2)}
\end{align*}}
\only<6->{%
\begin{align*}
    d' &= \ExpPhase{(\phi_{b}+\pi+\phi_{c}-2\phi_{a})}\cos(\theta/2) \\
       &= \ExpPhase{(\Lambdaa+\Phii+2\pi)}\cos(\theta/2) \\
       &= \ExpPhase{(\Lambdaa+\Phii)}\cos(\theta/2)
\end{align*}
}
}{%
\visible<3->{%
\[
U = \SQBG{}{%
\only<3>{a'}%
\only<4->{\cos(\theta/2)}}{%
\only<3-4>{b'}%
\only<5->{-\ExpPhase{\Lambdaa}\sin(\theta/2)}}{%
\only<3>{c'}
\only<4->{\ExpPhase{\Phii}\sin(\theta/2)}}{%
\only<3-5>{d'}%
\only<6->{\ExpPhase{(\Lambdaa+\Phii)}\cos(\theta/2)}}
\]}
\only<4-5>{
\begin{align*}
a' &= \cos(\theta/2) \\
    c' &= \ExpPhase{\Phii}\sin(\theta/2)
\end{align*}}%
\only<7->{%

We need just these three parameters to determine a quantum gate: $\theta, \Phii, \Lambdaa$. \QED{}
}
}

    
\end{frame}

\begin{frame}{Importance of this result}{The qiskit \NamedGate{U} gate}
\begin{theorem}
   Up to a global phase, any $2\times 2$ unitary matrix can be expressed as
    \[
    U(\theta, \phi, \lambda) = 
    \UGate{}
    \] where
    $0 \leq \theta,\phi,\lambda < 2\pi$
    \end{theorem}
\begin{itemize}[<+->]
   \item A qubit has a given probability of measuring \QZero{} or \QOne{}, based on its amplitude.   The \NamedGate{U} gate affects such measurements of that qubit using the parameter~$\theta$. 
   \item The qiskit environment offers the \href{https://qiskit.org/documentation/stubs/qiskit.circuit.library.UGate.html}{\NamedGate{U}} gate, which is compiled into the necessary actions on a quantum device to realize the gate.
\end{itemize}

\end{frame}

\begin{frame}{Building an actual quantum gate}{This turns out to be impossible, but we shouldn't let that stop us}
\begin{itemize}[<+->]
    \item We can specify a gate mathematically using three \href{https://en.wikipedia.org/wiki/Real_number}{real} parameters $0 \leq \theta,\phi,\lambda < 2\pi$.  So that's nice, but how do we build a gate?
    \item Suppose a device existed that could realize such a gate.  How would we communicate the three parameters to that device?
    \item The reals are \href{https://en.wikipedia.org/wiki/Uncountable_set}{uncountable} (thanks, \href{https://en.wikipedia.org/wiki/Irrational_number}{irrationals}), so there is no way to express all possible values for the parameters in any language. The number of possible values exceeds the capacity of any naming convention.
    \item The best we can hope to do is to \emph{approximate} a quantum gate.
    \item It turns out we can do that efficiently to a desired degree of accuracy.
    \item We were able to implement Boolean logic using only the \NamedGate{NAND} gate.
    \item What operations are needed to realize the approximation of a quantum gate?
\end{itemize}
\end{frame}

\begin{frame}{The programmable quantum computer}{Some background, with thanks to \href{http://www.thomaswong.net/}{Tom Wong} \LinkArrow{https://www.scottaaronson.com/qclec/16.pdf}}

\only<1-7>{%
\Vskip{-3.5em}\begin{itemize}[<+->]
    \item \href{https://en.wikipedia.org/wiki/Richard_Feynman}{Richard Feynman} gave \href{http://physics.whu.edu.cn/dfiles/wenjian/1_00_QIC_Feynman.pdf}{this famous lecture} in 1982, six years before his death, asking how quantum mechanics could be simulated on a classical computer.  That \href{http://physics.whu.edu.cn/dfiles/wenjian/1_00_QIC_Feynman.pdf}{link} includes a paper based on his talk, published in 1986.
    \item The (column-vector) representation of a quantum system grows exponentially in the number of qubits.
    \item Some of those systems are entangled, requiring a separate accounting of each of~$2^n$ possible measurements.
    \item A classical computer cannot efficiently simulate such systems.  For example, the \href{https://quantum-computing.ibm.com/lab/docs/iql/manage/simulator/}{IBM quantum simulators} currently limit programs to $32$~qubits.
    \item At $300$~qubits we exceed the number of atoms in the known universe.
    \item Feynman proposed that a programmable quantum computer could simulate quantum mechanics itself.  This would be nice for physicists.
    \item We computer scientists are interested more generally in the kinds of problems such devices can solve efficiently.
\end{itemize}}
\only<8-12>{%
\begin{itemize}
    \item<8-> A \href{https://en.wikipedia.org/wiki/Quantum_Turing_machine}{quantum Turing machine} has been defined to reason about actions a quantum computer can take in a single step.
    \item<9-> Problems that are undecidable, such as the halting problem, remain so even on a quantum computer.  No magic there.
    \item<10-> Here we are more interested in the theory supporting the construction of a programmable quantum computer whose components approximately realize quantum gates, expressed as we have studied.
    \item<11-> That theory must support development of arbitrarily large quantum system from components that accommodate a fixed number of qubits.
    \item<12-> For example, Boolean logic of arbitrary size can be constructed using only \NamedGate{NAND} gates.
\end{itemize}}
\only<13->{%
\begin{itemize}
    \item<13-> Can quantum circuits be similarly created?  For example
    \item<14-> Does entangling $n$~qubits require \ColorOne{one large gate}?
    \adjustbox{scale=0.8,valign=m}{\begin{GateBox}{2}{2}{4}
      \Input{0}{$q_{1}$}
      \Input{1}{$q_{2}$}
      \Input{2}{\RVDots{}}
      \Input{3}{$q_{n}$}
      \BoxLabel{\ColorOne{Engtangler?}}
      \Output{0}{$q_{1}$}
      \Output{1}{$q_{2}$}
      \Output{2}{\RVDots{}}
      \Output{3}{$q_{n}$}
    \end{GateBox}}
    
    \item<15-> Or can we build it from \ColorTwo{smaller components}?
    \adjustbox{scale=0.7}{\begin{quantikz}
    \lstick{$q_1$} & \gate[wires=2]{\ColorTwo{E}} & \qw & \qw \\
    \lstick{$q_2$} & & \gate[wires=2]{\ColorTwo{E}} & \qw\\
    \lstick{$q_3$} &\qw & & \qw
    \end{quantikz}}
    \item<16-> More generally, what kinds of components are needed to develop a programmable quantum device?
\end{itemize}}
\end{frame}

\begin{frame}{Some theoretical results}{From \Kaye{}}
\end{frame}

\begin{frame}{Universality via Pauli gates}{Adapted from \href{https://qiskit.org/textbook/ch-gates/proving-universality.html\#pauli}{qiskit}}


    
\end{frame}
