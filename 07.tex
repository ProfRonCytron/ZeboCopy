\SetTitle{7}{Universal quantum gates}{Several approaches, culminating in the qiskit U gate}{07}

\section*{Overview}
\begin{frame}{Overview}{What will we study here?}
\begin{itemize}
    \item We recap the postulates of quantum computing.
    \item A qubit can be described by two real parameters (recall the Bloch sphere).  Here we see that a quantum gate can be described by three such parameters.
    \item We look at how to use those three parameters, in qiskit in particular, to specify a quantum gate.
    \item We defer studying theoretical considerations of universality, but will eventually show that all quantum circuits can be realized by gate that act on only two inputs.
    \item We show that any quantum gate can be decomposed into a sum of Pauli gates, so they serve as a \emph{basis} for quantum gates.
\end{itemize}
\end{frame}

\section*{Postulates}

\begin{frame}{Postulates of quantum computing}{Brazenly borrowed from quantum mechanics}

The \href{https://en.wikipedia.org/wiki/Axiom}{postulates (axioms)} of quantum computing~(\MikeIke{}):
\only<1-3>{%
\begin{enumerate}
    \item<1-> The state space of a quantum system is a complex \href{https://en.wikipedia.org/wiki/Hilbert\_space}{Hilbert space}.  The state of the system is described by a \href{https://en.wikipedia.org/wiki/Unit\_vector}{unit} state vector in that space.
    
    Our state space is a set of qubits.
    \item <2->The evolution of a closed quantum system is described by a unitary transformation.  For us, these are the gates of a quantum computation.
    \item<3-> Quantum measurements are described by a collection of (not necessarily unitary) measurement operators.  We are mostly concerned with measurement in the computational basis.  A qubit in state $\psi=\alpha\QZero{}+\beta\QOne{}$ is measured as \QZero{} with probability $\Prob{\alpha}$ and as \QOne{} with probability $\Prob{\beta}$.  After measurement, the state of the system is the state obtained by measurement. For any measurement operator $M, M^{2}=M$.
\end{enumerate}}
\only<4->{%
\begin{enumerate}
\setcounter{enumi}{3}
   \item The state space of a composite physical system is the \href{https://en.wikipedia.org/wiki/Tensor_product}{tensor product} of the state spaces of the component physical systems. Moreover, if we have systems numbered $1$ through $n$, and system number $i$ is prepared in the state \QState{i}, then the joint state of the total system is $\QState{1}\TensOp\QState{2}\TensOp\cdots\TensOp\QState{n}$.
\end{enumerate}}
\end{frame}

\section*{Universality}
\begin{frame}{Universality}{Some initial thoughts}

\begin{itemize}[<+->]
    \item We captured the essence of a single qubit using two real rotational parameters $\phi$ and $\theta$.  Those parameters describe points on the unit sphere, each point corresponding to a quantum state.
    \item A quantum gate is a mapping from states to states.
    \item Because such gates must have inverses, the mapping established by such gates must be \href{https://en.wikipedia.org/wiki/Bijection}{bijective}.
    \item We can think of a gate as a permutation of states, or points, on the Bloch sphere.
    \item How many parameters does it take to characterize an arbitrary single-qubit quantum gate?
\end{itemize}

    
\end{frame}

\begin{frame}{Quantum gate}{Properties}

\TwoUnequalColumns{0.4\textwidth}{0.6\textwidth}{%
We begin with an arbitrary unitary matrix $U$ to represent a quantum gate for a single qubit:
\[
U=\SQBG{\relax}{a}{c}{b}{d}
\]
}{%
\begin{itemize}[<+->]
    \item $a, b, c, d$ are complex
    \item So, apparently 8 parameters
    \item $U$ is invertible and $\Inverse{U}=\Conj{U}$
    \item The columns form an orthonormal basis
    \item The rows form an orthonormal basis
\end{itemize}}
\BigSkip{}
\visible<+->{%
These constraints will take us from 8 parameters to only 3.}
    
\end{frame}

\section*{Qiskit U gate}
\begin{frame}{The Qiskit U gate}{Theorem statement}
\begin{theorem}
   Up to a global phase, any $2\times 2$ unitary matrix can be expressed as
    \[
    U(\theta, \phi, \lambda) = 
    \UGate{}
    \] where
    $0 \leq \theta,\phi,\lambda < 2\pi$
    \end{theorem}
    For example, the Hadamard gate \HMatrix{} is $U(\pi/2, 0, \pi)$, which yields
    \[
    \SQBG{\relax}{\cos \pi/4}{-\ExpPhase{\pi}\sin \pi/4}{\ExpPhase{0}\sin \pi/4}{\ExpPhase{\pi}\cos pi/4} = \HMatrix{}
    \]

\end{frame}

\begin{frame}{Proof}{We begin with an arbitrary unitary matrix $U$ with complex entries}
\TwoColumns{%
$U = \SQBG{\relax}{a}{b}{c}{d}$ so
$\Conj{U} = \SQBG{\relax}{\Conj{a}}{\Conj{c}}{\Conj{b}}{\Conj{d}}$

\MedSkip{}
Because $U$ is unitary:
\begin{itemize}
\item<1-> Each column of $U$ (row of \Conj{U}) is \href{https://en.wikipedia.org/wiki/Orthonormality}{orthonormal}:
\begin{align*}
    \Conj{a}a + \Conj{c}c = & \Prob{a} + \Prob{c} = 1 \\
    \Conj{b}b + \Conj{d}d = & \Prob{b} + \Prob{d} = 1 
\end{align*}
\item<2-> $\Implies{\Prob{a} + \Prob{c} = 1}{\Mag{a} \leq 1}$
\end{itemize}
\visible<5>{%
\alert<5>{%
We use $\theta/2$ to accommodate $0\leq \theta < 2\pi$}}
}{%
\Vskip{-5em}\begin{center}
\begin{TIKZP}[scale=0.45]
\UnitComplexCircle{}
\end{TIKZP}
\end{center}
\visible<3->{%
Because $\Mag{a}\leq 1$, $a$ is some point on or in the unit
complex circle.
\begin{itemize}
    \item<4-> Let $\ExpPhase{\phi_{a}}$ be a point on the circle.
    \item<5-> We can then scale it by \alert<5>{$\cos(\theta/2)$} to obtain any point on or in the circle.
    \item<6-> We thus obtain
    \[
    a = \ExpPhase{\phi_{a}}\cos(\theta/2)
    \]
\end{itemize}}
}
\end{frame}

\begin{frame}{Proof (continued)}{The other entries follow from the requirement that each row and column is \href{https://en.wikipedia.org/wiki/Unit_vector}{normal}}
\[
    U=\SQBG{\relax}{a}{b}{c}{d} = 
    \SQBG{\relax}{\ExpPhase{\phi_{a}}\cos(\theta/2)}{\visible<4->{\ExpPhase{\phi_{b}}\sin(\theta/2)}}{\visible<6->{\ExpPhase{\phi_{c}}\sin(\theta/2)}}{\visible<7->{\ExpPhase{\phi_{d}}\cos(\theta/2)}}
    \]
\TwoColumns{%
\begin{itemize}
    \item<2-> Each row of $U$ is orthonormal, so 
    $\Prob{a}+\Prob{b}=1$
    \item<3-> $\Prob{b}=\sin^{2}(\theta/2)$
    \item<4-> We can let $b=\ExpPhase{\phi_{b}}\sin(\theta/2)$
    \item<5-> $\Prob{a}+\Prob{c}=1$
    \item<6-> We can let $c=\ExpPhase{\phi_{c}}\sin(\theta/2)$
    \item<7-> Same for $d$
\end{itemize}
}{%
\begin{itemize}
    \item<3-> $\Mag{\ExpPhase{\theta}}=1$ for all $\theta$
    \item<4-> Yields the correct magnitude with an arbitrary phase $\phi_{b}$
    \item<5-> $c$ and $d$ follow in the same way
    \item<7-> We now have 5 parameters instead of 8
\end{itemize}
}
\end{frame}

\begin{frame}{Proof (continued)}{We can simplify by applying orthogonality of $U$'s columns}
\Vskip{-3em}\[
    U=\SQBG{\relax}{a}{b}{c}{d} = 
    \SQBG{\relax}{\textcolor{\RCone}{\ExpPhase{\phi_{a}}\cos(\theta/2)}}{\textcolor{\RCtwo}{\ExpPhase{\phi_{b}}\sin(\theta/2)}}{\textcolor{\RCthree}{\ExpPhase{\phi_{c}}\sin(\theta/2)}}{\textcolor{\RCfour}{\ExpPhase{\phi_{d}}\cos(\theta/2)}}
    \]
    
\TwoColumns{%
\begin{itemize}
    \item<1-> The columns of $U$ are orthogonal
    \item<2-> $\CQB{\Conj{\textcolor{\RCone}{a}}}{\Conj{\textcolor{\RCthree}{c}}} \SQB{\textcolor{\RCtwo}{b}}{\textcolor{\RCfour}{d}}=\Conj{\textcolor{\RCone}{a}}\textcolor{\RCtwo}{b} + \Conj{\textcolor{\RCthree}{c}}\textcolor{\RCfour}{d}=0$
\end{itemize}
}{%
\begin{itemize}
        \item<3-> Let's plug in the matrix values
        \item<4-> Recall the conjugate of $\ExpPhase{x}$ is $\ExpNegPhase{x}$
\end{itemize}
}
\visible<4->{%
\MedSkip{}\[
\textcolor{\RCone}{\ExpNegPhase{\phi_a}\cos(\theta/2)}\,\textcolor{\RCtwo}{\ExpPhase{\phi_b}\sin(\theta/2)}
+
\textcolor{\RCthree}{\ExpNegPhase{\phi_c}\cos(\theta/2)}\,\textcolor{\RCfour}{\ExpPhase{\phi_d}\sin(\theta/2)}
= 0 \]}%
\visible<5->{\[
\cos(\theta/2)\sin(\theta/2)\left[
\ExpPhase{(\phi_{b}-\phi_{a})} +
\ExpPhase{(\phi_{d}-\phi_{c})} \right]=0
\]}%
\visible<6->{%
The value of $\theta$ must be arbitrary, so we can only get $0$ when
\[
\ExpPhase{(\phi_{b}-\phi_{a})} = - \ExpPhase{(\phi_{d}-\phi_{c})}
\]}
\end{frame}
\begin{frame}{Proof (continued)}{We can now relate the phases}

\Vskip{-3em}\[
    U=\SQBG{\relax}{a}{b}{c}{d} = 
    \SQBG{\relax}{\textcolor{\RCone}{\ExpPhase{\phi_{a}}\cos(\theta/2)}}{\textcolor{\RCtwo}{\ExpPhase{\phi_{b}}\sin(\theta/2)}}{\textcolor{\RCthree}{\ExpPhase{\phi_{c}}\sin(\theta/2)}}{\textcolor{\RCfour}{\ExpPhase{\phi_{d}}\cos(\theta/2)}}
    \]
    
\begin{align*}
\ExpPhase{(\phi_{b}-\phi_{a})} &= - \ExpPhase{(\phi_{d}-\phi_{c})} \\
 & = \ExpPhase{\pi}\,\ExpPhase{(\phi_{d}-\phi_{c})} \\
 & = \ExpPhase{(\phi_{d}-\phi_{c}+\pi)} \\
 \textcolor{\RCfour}{\phi_{d}} &= \phi_{b}+\phi_{c}-\phi_{a} -\pi \\
 &= \phi_{b}+\phi_{c}-\phi_{a} -\pi + 2\pi \\
 &= (\phi_{b}+\pi) + \phi_{c}-\phi_{a}
\end{align*}
We next eliminate $\phi_{d}$ in favor of the other phases.
\end{frame}

\begin{frame}{Proof (continued)}{Now 4 parameters, but soon only 3}
\Vskip{-3em}\[
    U=\SQBG{\relax}{a}{b}{c}{d} = 
    \SQBG{\relax}{\textcolor{\RCone}{\ExpPhase{\phi_{a}}\cos(\theta/2)}}{\textcolor{\RCtwo}{\ExpPhase{\phi_{b}}\sin(\theta/2)}}{\textcolor{\RCthree}{\ExpPhase{\phi_{c}}\sin(\theta/2)}}{\textcolor{\RCfour}{\ExpPhase{\phi_{d}}\cos(\theta/2)}}
    \]
    \[
    \phi_{d} = (\phi_{b}+\pi) + \phi_{c}-\phi_{a}
    \]
    \begin{align*}
    \visible<2->{
    U&= 
\SQBG{%
    \phantom{\ExpPhase{\phi_a}}}{%
    \textcolor{Black}{\ExpPhase{\phi_{a}}\cos(\theta/2)}}{%
    \textcolor{Black}{\ExpPhase{\phi_{b}}\sin(\theta/2)}}{%
    \textcolor{Black}{\ExpPhase{\phi_{c}}\sin(\theta/2)}}{%
    \textcolor{Black}{\ExpPhase{((\phi_{b}+\pi) + \phi_{c}-\phi_{a})}\cos(\theta/2)}} \\[1.5em]}
    \visible<3->{
    &\only<3-4>{=}\only<5->{\equiv} \SQBG{%
    \visible<3-4>{\ExpPhase{\phi_a}}}{%
    \textcolor{Black}{\cos(\theta/2)}}{%
    \textcolor{Black}{\ExpPhase{(\phi_{b}-\phi_{a})}\sin(\theta/2)}}{%
    \textcolor{Black}{\ExpPhase{(\phi_{c}-\phi_{a})}\sin(\theta/2)}}{%
    \textcolor{Black}{\ExpPhase{((\phi_{b}+\pi) + \phi_{c}-2\phi_{a})}\cos(\theta/2)}}}
    \end{align*}
\visible<4->{
The coefficient $\ExpPhase{\phi_{a}}$ is a \emph{global} phase and can therefore be \visible<5->{ignored.}}
    
\end{frame}
\begin{frame}{Conclusion of Proof}{We now have a general unitary matrix specified using only three parameters}
\def\Phii{\textcolor{\RCone}{\phi}}
\def\Lambdaa{\textcolor{\RCtwo}{\lambda}}
\[
U = \SQBG{%
    \relax}{%
    \textcolor{Black}{\cos(\theta/2)}}{%
    \ExpPhase{(\textcolor<2->{\RCtwo}{\phi_{b}-\phi_{a}-\pi}+\pi)}\sin(\theta/2)}{%
    \ExpPhase{\textcolor<2->{\RCone}{(\phi_{c}-\phi_{a})}}\sin(\theta/2)}{%
    \textcolor{Black}{\ExpPhase{((\phi_{b}+\pi) + \phi_{c}-2\phi_{a})}\cos(\theta/2)}}
    \]
\TwoColumns{%

\Vskip{-3.5em}\begin{align*}
     \visible<2->{\mbox{Let } \Phii &= \textcolor<2>{\RCone}{\phi_{c}-\phi_{a}} \\
     \mbox{Let } \Lambdaa &= \textcolor{\RCtwo}{\phi_{b}-\phi_{a}-\pi}}
\end{align*}
\Vskip{-3em}
\only<5>{%
\begin{align*}
     \visible<5->{%
     b' &= \ExpPhase{(\Lambdaa+\pi)}\sin(\theta/2) \\
     &= \ExpPhase{\pi}\ExpPhase{\Lambdaa}\sin(\theta/2) \\
     &= -\ExpPhase{\Lambdaa}\sin(\theta/2)}
\end{align*}}
\only<6->{%
\begin{align*}
    d' &= \ExpPhase{(\phi_{b}+\pi+\phi_{c}-2\phi_{a})}\cos(\theta/2) \\
       &= \ExpPhase{(\Lambdaa+\Phii+2\pi)}\cos(\theta/2) \\
       &= \ExpPhase{(\Lambdaa+\Phii)}\cos(\theta/2)
\end{align*}
}
}{%
\visible<3->{%
\[
U = \SQBG{}{%
\only<3>{a'}%
\only<4->{\cos(\theta/2)}}{%
\only<3-4>{b'}%
\only<5->{-\ExpPhase{\Lambdaa}\sin(\theta/2)}}{%
\only<3>{c'}
\only<4->{\ExpPhase{\Phii}\sin(\theta/2)}}{%
\only<3-5>{d'}%
\only<6->{\ExpPhase{(\Lambdaa+\Phii)}\cos(\theta/2)}}
\]}
\only<4-5>{
\begin{align*}
a' &= \cos(\theta/2) \\
    c' &= \ExpPhase{\Phii}\sin(\theta/2)
\end{align*}}%
\only<7->{%

We need just these three parameters to determine a quantum gate: $\theta, \Phii, \Lambdaa$. \QED{}
}
}

    
\end{frame}

\begin{frame}{Importance of this result}{The qiskit \NamedGate{U} gate}
\begin{theorem}
   Up to a global phase, any $2\times 2$ unitary matrix can be expressed as
    \[
    U(\theta, \phi, \lambda) = 
    \UGate{}
    \] where
    $0 \leq \theta,\phi,\lambda < 2\pi$
    \end{theorem}
\begin{itemize}[<+->]
   \item A qubit has a given probability of measuring \QZero{} or \QOne{}, based on its amplitude.   The \NamedGate{U} gate affects such measurements of that qubit using the parameter~$\theta$. 
   \item The qiskit environment offers the \href{https://qiskit.org/documentation/stubs/qiskit.circuit.library.UGate.html}{\NamedGate{U}} gate, which is compiled into the necessary actions on a quantum device to realize the gate.
\end{itemize}

\end{frame}

\begin{frame}{Building an actual quantum gate}{This turns out to be impossible, but we shouldn't let that stop us}
\begin{itemize}[<+->]
    \item We can specify a gate mathematically using three \href{https://en.wikipedia.org/wiki/Real_number}{real} parameters $0 \leq \theta,\phi,\lambda < 2\pi$.  So that's nice, but how do we build a gate?
    \item Suppose a device existed that could realize such a gate.  How would we communicate the three parameters to that device?
    \item The reals are \href{https://en.wikipedia.org/wiki/Uncountable_set}{uncountable} (thanks, \href{https://en.wikipedia.org/wiki/Irrational_number}{irrationals}), so there is no way to express all possible values for the parameters in any language. The number of possible values exceeds the capacity of any naming convention.
    \item The best we can hope to do is to \emph{approximate} a quantum gate.
    \item It turns out we can do that efficiently to a desired degree of accuracy.
    \item We were able to implement Boolean logic using only the \NamedGate{NAND} gate.
    \item What operations are needed to realize the approximation of a quantum gate?
    \item We return to this general question after moving beyond one qubit.
\end{itemize}
\end{frame}

{
\def\UZZ{\ensuremath{\ColorOne{u_{0,0}}}}
\def\UZN{\ensuremath{\ColorTwo{u_{0,1}}}}
\def\UNZ{\ensuremath{\ColorThree{u_{1,0}}}}
\def\UNN{\ensuremath{\ColorFour{u_{1,1}}}}
\begin{frame}{Outer products}{Review}

\TwoColumns{%
\begin{itemize}
    \item $\KetBra{0}{0} = \SQB{1}{0}\times\CQB{1}{0} = \ColorOne{\SQBG{}{1}{0}{0}{0}}$
    \Vskip{1em}
    \item $\KetBra{1}{0} = \SQB{0}{1}\times\CQB{1}{0} = \ColorThree{\SQBG{}{0}{0}{1}{0}}$
\end{itemize}
}{%
\begin{itemize}
    \item $\KetBra{0}{1} = \SQB{1}{0}\times\CQB{0}{1} = \ColorTwo{\SQBG{}{0}{1}{0}{0}}$
    \Vskip{1em}
    \item $\KetBra{1}{1} = \SQB{0}{1}\times\CQB{0}{1} = \ColorFour{\SQBG{}{0}{0}{0}{1}}$
\end{itemize}
}
\BigSkip{}
\[
\SQBG{}{\UZZ}{\UZN}{\UNZ}{\UNN} = 
\UZZ\KetBra{0}{0} + \UZN\KetBra{0}{1}
+ \UNZ\KetBra{1}{0} + \UNN\KetBra{1}{1}
\]
    
\end{frame}

\section*{Pauli gate universality}

\begin{frame}{Single qubit universality via Pauli gates}{Adapted from \href{https://qiskit.org/textbook/ch-gates/proving-universality.html\#pauli}{qiskit}}

\TwoColumns{%
\Vskip{-2em}Consider a one-qubit gate:
\Vskip{-1.5em}\begin{align*}
U =& \SQBG{}{\UZZ}{\UZN}{\UNZ}{\UNN}\\
\visible<2->{=\ & \UZZ\alt<5->{\alert<5>{\frac{\Identity+\PauliZ}{2}}}{\alert<4>{\KetBra{0}{0}}} + \UZN\alt<9->{\alert<9>{\frac{\Complex{\PauliX}{\PauliY}}{2}}}{\alert<8>{\KetBra{0}{1}}} \\
 + &\UNZ\alt<11->{\alert<11>{\frac{\ComplexDiff{\PauliX}{\PauliY}}{2}}}{\alert<10>{\KetBra{1}{0}}} + \UNN\alt<7->{\alert<7>{\frac{\Identity-\PauliZ}{2}}}{\alert<6>{\KetBra{1}{1}}}}
\end{align*}%
\visible<8-11>{%
\Vskip{-1.5em}We use the following identities:
\Vskip{-2em}\begin{align*}
    \alert<10-11>{\PauliX\QZero} =& \alert<10-11>{\QOne} &
    \alert<10-11>{\PauliX\PauliZ}=& \alert<10-11>{-\NiceI\PauliY} \\
    \alert<8-9>{\PauliZ\PauliX} =& \alert<8-9>{\NiceI\PauliY}&
    \alert<8-9>{\bra{0}\PauliX} =& \alert<8-9>{\bra{1}}
\end{align*}
}
}{%
\only<1-5>{%
\begin{itemize}
    \item Recall we can express $U$ \visible<2->{as the weighted sum of outer products.}
    \item<3-> We can then express the outer products as follows:
\end{itemize}}
\only<4-5>{%
    \begin{align*}
        \alert<4>{\KetBra{0}{0}} & = \frac{1}{2}\left[\IMatrix + \ZMatrix \right]\\
        &= \alert<4-5>{\frac{\Identity+\PauliZ}{2}}
    \end{align*}
}%
\only<6-7>{%
\begin{align*}
        \alert<6>{\KetBra{1}{1}} &= \frac{1}{2}\left[\IMatrix - \ZMatrix \right]\\
        &= \alert<6-7>{\frac{\Identity-\PauliZ}{2}}
    \end{align*}
}%
\only<8-9>{%
\begin{align*}
    \alert<8>{\KetBra{0}{1}} =& \KetBra{0}{0}\PauliX \\
    = & \frac{1}{2}(\Identity+\PauliZ)\PauliX 
    = \alert<8-9>{\frac{\Complex{\PauliX}{\PauliY}}{2}}
\end{align*}%
}%
\only<10-11>{%
\begin{align*}
    \alert<10>{\KetBra{1}{0}} =& \PauliX\KetBra{0}{0}\\
    = & \PauliX\ \frac{1}{2}(\Identity+\PauliZ)
    = \alert<10-11>{\frac{\ComplexDiff{\PauliX}{\PauliY}}{2}}
\end{align*}%
}%
\only<12->{%
\begin{align*}
   \visible<12->{ = & \frac{\UZZ+\UNN}{2}&\Identity \\}
   \visible<13->{ + & \frac{\UZN+\UNZ}{2}&\PauliX \\}
   \visible<14->{+ &\frac{\UZN-\UNZ}{2}\NiceI & \PauliY \\}
   \visible<15->{+ & \frac{\UZZ-\UNN}{2} & \PauliZ}
\end{align*}
}
}
\only<16->{%

Thus, any single-qubit gate can be specified using only the three Pauli gates.}

\end{frame}}

\section*{Summary}

\begin{frame}{Summary}{What have we learned?}

\begin{itemize}
    \item A quantum gate can be specified using three real, angular parameters.
    \item The \NamedGate{U} is specified using those three parameters.   Thus, we can implement any gate we want by providing the values of those three angles.
    \item Any gate can be expressed as a weighted sum of Pauli gates.  While this doesn't lead to an implementation (as did the \NamedGate{U} gate), it is a common form for expressing gates. 
\end{itemize}
    
\end{frame}

%% stuff from temp goes here
