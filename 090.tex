\SetTitle{9}{Beyond one qubit}{Composite quantum systems}{09}

\begin{frame}{Overview}

\begin{itemize}
    \item We can build larger quantum systems by using multiple qubits.
    \item The mathematics of \href{https://en.wikipedia.org/wiki/Tensor}{tensors} allows us to reason about multi-qubit systems.
    \item If $k$ qubits are separate and do not interact, they can be visualized as $k$ Bloch spheres, and they can be simulated efficiently using a classical computer.
    \item On the other hand, if some qubits are \href{https://en.wikipedia.org/wiki/Quantum_entanglement}{entangled}, then we cannot necessarily treat them independently.  Simulating them classically can take exponential time.
    \item Entanglement is a strange but powerful concept in quantum mechanics that can provide significant computational advantages on a quantum computer.
\end{itemize}
    
\end{frame}

\section{Tensor product of states}

\begin{frame}{A system of two qubits}{We start here, and generalization will be easy}
\begin{itemize}
    \item Consider two qubits, one in state \QState{a} and the other in state \QState{b}.
    \item We say the system of those two qubits is in state
    \[
       \QState{a}\QState{b} = \ket{\psi_{a}\psi_{b}} = \TensProd{\QState{a}}{\QState{b}}
    \]
    \item The first two expressions are interchangeable names for the system. The last expressions tells us how to compute the state:  it is the \href{https://en.wikipedia.org/wiki/Tensor_product}{tensor product} of the individual states.
\end{itemize}
\end{frame}


\begin{frame}{Tensor product of two states}{We must keep track of the combinations separately}
\TwoColumns{%
\only<1-5>{%
\begin{itemize}
    \item<1-> Consider two quantum states:
    \begin{align*}
        \QState{x} = & \alpha_{x}\ket{0} + \beta_{x}\ket{1} \\
        \QState{y} = & \alpha_{y}\ket{0} + \beta_{y}\ket{1}
    \end{align*}
    \item<2-> Measuring \QState{x} will yield \ket{0} or \ket{1}.
    \item<3-> Measuring \QState{y} will yield \ket{0} or \ket{1}.
    \item<4-> This does not capture the joint probability of outcomes.
    \item<5-> This does the job, with wave amplitudes computed as follows.
\end{itemize}}
\only<6-10>{%
We compute each entry in turn:
\begin{itemize}
    \item<7-> The wave amplitude of \ket{00}.
    \item<8-> The wave amplitude of \ket{01}.
    \item<9-> The wave amplitude of \ket{10}.
    \item<10-> The wave amplitude of \ket{11}.
\end{itemize}
}
\only<11>{%
\Vskip{-3em}\begin{align*}
& \alpha_{x}\,\alpha_{y}\DQB{1}{0}{0}{0}
+ 
\alpha_{x}\,\beta_{y}\DQB{0}{1}{0}{0} \\[1em]
+ \ &
\beta_{x}\,\alpha_{y}\DQB{0}{0}{1}{0}
+ 
\beta_{x}\,\beta_{y}\DQB{0}{0}{0}{1}
\end{align*}
}
\only<12-14>{%
\begin{itemize}
    \item<12-> State $\ket{00} = \TensProd{\ket{0}}{\ket{0}}$ is the outcome of measuring $\QState{x}=$\ket{0} and $\QState{y}=\ket{0}$.
    \item<13-> The probability of that outcome is
    \[
    \Prob{\alpha_{x}\,\alpha_{y}} = \Prob{\alpha_{x}}\,\Prob{\alpha_{y}}
    \]
    \item<14-> This is the joint probability of $\QState{x}=\ket{0}$ and $\QState{y}=\ket{0}$.
\end{itemize}
}
\only<15->{%
\begin{itemize}
    \item Each of the other coefficients is the amplitude on the basis vector for the related measurements.
    \item And the square of the magnitude of an amplitude is the joint probability of seeing the two outcomes.
\end{itemize}
}
}{%
\visible<4->{%
    \begin{align*}
    \TensProd{\QState{x}}{\QState{y}} = & \TensProd{\SQB{\alert<7,8>{\alpha_{x}}}{\alert<9,10>{\beta_{x}}}}{\SQB{\alert<7,9>{\alpha_{y}}}{\alert<8,10>{\beta_{y}}} } \\[2em]
    \only<4>{\not= & \DQB{\alpha_{x}}{\beta_{x}}{\alpha_{y}}{\beta_{y}}%
    }
    \only<5->{= & \DQB{\alert<7,12-14>{\alpha_{x}\,\alpha_{y}}}{\alert<8>{\alpha_{x}\,\beta_{y}}}{\alert<9>{\beta_{x}\,\alpha_{y}}}{\alert<10>{\beta_{x}\,\beta_{y}}}%
    }
    \end{align*}}
}
\only<11->{%
\MedSkip{}
The resulting state is
\[
\alert<12-14>{\alpha_{x}\,\alpha_{y}}\ket{00}
+\alpha_{x}\,\beta_{y}\ket{01}
+\beta_{x}\,\alpha_{y}\ket{10}
+\beta_{x}\,\beta_{y}\ket{11}
\]
}
\end{frame}

\begin{frame}{Properties of tensor products}{These pertain mainly to quantum states}
\Vskip{-3.5em}\begin{itemize}[<+->]
    \item If $a \in \Co{}$ and $\ket{v}$ and $\ket{w}$ are quantum states, then
    \[ 
       \TensProd{(a\ket{v})}{\ket{w}} = \TensProd{\ket{v}}{(a\ket{w})} = a\,(\TensProd{\ket{v}}{\ket{w}}) = a\ket{vw}
    \]
    \item So
    \Vskip{-3em}\begin{align*}
        \TensProd{\RootTwo{\ket{v}}}{\RootTwo\ket{w}} &= \frac{1}{2}\TensProd{\ket{v}}{\ket{w}} = \frac{1}{2} \ket{vw}
    \end{align*}
    \item Tensor products distribute over superpositions.  If $v_{1}$, $v_{2}$, and $w$ are quantum states (dropping ket notation to avoid clutter), then
    \begin{align*}
       \TensProd{(v_{1}+v_{2})}{w} = & \TensProd{v_{1}}{w} + \TensProd{v_{2}}{w}\\
       \TensProd{w}{(v_{1}+v_{2})} = & \TensProd{w}{v_{1}} + \TensProd{w}{v_{2}} 
    \end{align*}

\end{itemize}
\end{frame}

\begin{frame}{Properties of tensor products}{These pertain only to quantum gates}

\Vskip{-3em}\begin{center}
\begin{quantikz}
\qw &  \gate{\mbox{$X$}} & \gate{\mbox{$V$}} & \qw \\
\qw &  \gate{\mbox{$Y$}} & \gate{\mbox{$W$}} & \qw
\end{quantikz}
\end{center}
If $V$, $W$, $X$, and $Y$ are quantum gates and $\bullet$ is matrix multiplication, then
\begin{align*}
    \ColorOne{(\TensProd{V}{W}) \bullet (\TensProd{X}{Y})} &= \ColorTwo{\TensProd{(V\bullet X)}{(W\bullet Y)}}
\end{align*}
\Vskip{-3em}\begin{itemize}[<+->]
    \item The circuit applies $X$ and $Y$ to the input state first, and then $V$ and $W$
    \item \ColorOne{To compute the effect of the four gates, we may first tensor $V$ with $W$ and $X$ with $Y$, and apply their product to the input state.}
    \item \ColorTwo{Or, we can multiply the matrices first, and then take the tensor product of those results}
\end{itemize}
\end{frame}


\begin{frame}{Now three qubits}{Each new qubit is incorporated by tensor product}

\Vskip{-3em}\TwoColumns{%
\only<1-10>{%
\begin{itemize}
    \item<1-> A system of $k$ qubits is characterized by the tensor product of those qubits' states.
    \item<2-> Consider for example a 3~qubit system comprised of the states
    \begin{align*}
    \QState{x} = \PZero{} &
    \QState{y} = \POne{} \\
    \QState{z} = & \PPlus{}
    \end{align*}
\end{itemize}
\only<3-10>{%

\Vskip{-2em}How do we compute \ket{\QName{x}\QName{y}\QName{z}}?}}%
\only<11>{%
We can describe a quantum state using a column vector, but when that is sparse, it may be more clear to write it out in terms of its basis states:
\[
\frac{\ket{010} + \ket{011}}{\sqrt{2}}
\]
Counting down a column $(0, 1, 2, \ldots)$, we include the binary encoding of each index whose entry is non-zero.
}
}{%
\only<3->{%
\begin{align*}
    & \TensProd{\TensProd{\PZero{}}{\POne{}}}{\PPlus{}} \\
    = & \TensProd{\DQB{\alert<3-4>{0}}{\alert<5-6>{1}}{\alert<7-8>{0}}{\alert<9-10>{0}}}{\frac{1}{\sqrt{2}}\SQB{\alert<3,5,7,9>{1}}{\alert<4,6,8,10>{1}}}
    = \frac{1}{\sqrt{2}}\QQB{%
    \alert<3>{0}}{%
    \alert<4>{0}}{%
    \alert<5>{1}}{%
    \alert<6>{1}}{%
    \alert<7>{0}}{%
    \alert<8>{0}}{%
    \alert<9>{0}}{%
    \alert<10>{0}}
\end{align*}}
}
    
\end{frame}

\section{Tensor product of gates}


\begin{frame}{From states to gates}{We can use tensor products of matrices to reason about systems of gates}
Computing the tensor product of two matrices is similar to computing the tensor product of two vectors.  It is demonstrated by the example of 
\[
\TensProd{\PauliZ{}}{\Hadamard{}} = %
\TensProd{%
   \SQBG{\relax}{%
   \alert<1-4>{1}}{%
   \alert<5-8>{0}}{%
   \alert<9-12>{0}}{%
   \alert<13-16>{-1}}}{%
   \SQBG{\frac{1}{\sqrt{2}}}{%
   \alert<1,5,9,13>{1}}{%
   \alert<2,6,10,14>{1}}{%
   \alert<3,7,11,15>{1}}{%
   \alert<4,8,12,16>{-1}}}
   = \frac{1}{\sqrt{2}}\begin{pmatrix*}[r]
   \visible<1-4,17->{ \alert<1>{1} &  \alert<2>{1} }& \visible<5-8,17->{\alert<5>{0} & \alert<6>{0} }\\
     \visible<1-4,17->{\alert<3>{1} & \alert<4>{-1} }& \visible<5-8,17->{\alert<7>{0} & \alert<8>{0} }\\
     \visible<9-12,17->{\alert<9>{0} &  \alert<10>{0}} & \visible<13-16,17->{\alert<13>{-1} & \alert<14>{-1}} \\
     \visible<9-12,17->{\alert<11>{0} &  \alert<12>{0} }& \visible<13-16,17->{\alert<15>{-1} & \alert<16>{1}}
   \end{pmatrix*}
\]
\only<18-19>{%
\begin{itemize}
\item<18-> This gate can then be applied to any two-qubit state. 
\item<19-> If that state is \TensProd{\QState{a}}{\QState{b}}, then the result will be as if \PauliZ{} and \Hadamard{} acted separately on each input, because this is a \href{https://en.wikipedia.org/wiki/Separable_state}{product, or separable state}.
\[
(\TensProd{\PauliZ{}}{\Hadamard{}})\,\ket{\QName{a}\QName{b}} = \TensProd{(\PauliZ{}\QState{a})}{(\Hadamard{}\QState{b})}
\]
\end{itemize}
}%
\only<20>{%
\begin{itemize}

\item Why compute the tensor product of the two gates if the result can be computed separately?

\item There are some two-qubit states that cannot be written as any \TensProd{\QState{a}}{\QState{b}}.  Such entangled states require using the gates' tensor product (more later).
\end{itemize}}

\end{frame}

\section{Uniform superpositions}

\begin{frame}{Another example}{We can create the uniform superposition of two qubits using Hadamard gates}
\TwoUnequalColumns{0.22\textwidth}{0.78\textwidth}{%
\adjustbox{valign=t}{\begin{quantikz}
\lstick{\QZero{}}\slice{\alert<1>{\QState{0}}} &  \gate{H}\slice{\alert<4>{\QState{1}}}  & \qw\\
\lstick{\QZero{}} &   \gate{H}    &  \qw
\end{quantikz}}
}{%
\Vskip{-3em}\begin{align*} 
\alert<1-3>{\QState{0}}= &\ket{00} 
\visible<2->{=  \TensProd{\PZero}{\PZero}}
\visible<3->{= \DQB{1}{0}{0}{0} }\\
\visible<4->{%
\alert<4->{\QState{1}} = & (\TensProd{\Hadamard{}}{\Hadamard{}})\,\DQB{1}{0}{0}{0} 
\visible<5->{= \frac{1}{\sqrt{2}}\frac{1}{\sqrt{2}}
    \begin{pmatrix*}[r]
      1 & 1 & 1 & 1 \\
      1 & -1 & 1 & -1 \\
      1 & 1 & -1 & -1 \\
      1 & -1 & -1 & 1
    \end{pmatrix*}\DQB{1}{0}{0}{0}} \\
   \visible<6->{ = & \frac{1}{2}\DQB{1}{1}{1}{1}}
   \visible<7->{= \frac{\ket{00}+\ket{01}+\ket{10}+\ket{11}}{2}
   }}
\end{align*}
}
\end{frame}

\begin{frame}{General superposition of $n$ qubits}{The idea extends to any number of qubits}
\TwoUnequalColumns{0.25\textwidth}{0.75\textwidth}{%
\adjustbox{scale=0.9,valign=t}{\begin{quantikz}
\lstick{\QZero{}}\slice{\alert<1>{\QState{0}}} &  \gate{H}\slice{\alert<2-3>{\QState{1}}}  & \meter{\textcolor<5-6>{red}{0/1}} \\
\lstick{\QZero{}} &   \gate{H}    &  \meter{\textcolor<5-6>{red}{0/1}} \\
\lstick{$\bullet$} & \mbox{$\bullet$} & \mbox{$\bullet$}\\
\lstick{$\bullet$} & \mbox{$\bullet$} & \mbox{$\bullet$}\\
\lstick{$\bullet$} & \mbox{$\bullet$} & \mbox{$\bullet$} \\
\lstick{\QZero{}}\slice{\alert<1>{\QState{0}}} &  \gate{H}\slice{\alert<2>{\QState{1}}}  & \meter{\textcolor<5-6>{red}{0/1}}
\end{quantikz}}
}{%
\only<1-4>{%
\Vskip{-3em}\begin{align*} 
\alert<1>{\QState{0}} = & \ket{0^{n}} \\
\visible<2->{\alert<2->{\QState{1}} = & \TensSupProd{\Hadamard{}}{n}\ket{0^{n}}\\
\visible<3->{= & \frac{1}{\sqrt{2^{n}}}\SumBV{x}{n} \ket{x}
\visible<4->{=\frac{1}{\sqrt{2^{n}}} \sum_{j=0}^{2^{n}-1} \ket{j}}
}}
\end{align*}
\Vskip{-2em}\begin{itemize}
\item<1-> There are $n$ inputs, each \ket{0}.
\item<2-> A Hadamard gate is applied to each input.
\item<3-> The result is the uniform superposition of all possible $n$-bit vectors.  
\item<4-> Interpreted as integers base-2, the result is also the uniform superposition of $2^{n}$ values.
\end{itemize}
}%
\only<5->{%
\begin{itemize}
   \item<5->On \alert<5>{measurement} of the $n$ qubits, each outcome is equally likely to be \Zero{} or \One{}.
   \item<6-> Taken as an $n$-bit string, each of the possible $2^n$ outcomes is therefore also equally likely.
   \item<7-> In theory, this is a perfect random-number generator for an integer in the interval $\left[0,2^{n}\right)$.
   \begin{itemize}
       \item We know from physics that the outcome of these measurements is completely unpredictable.
       \item Uniformity of outcome depends on each \Hadamard{} gate placing its input into a perfect (unbiased) superposition of $\frac{\ket{0}+\ket{1}}{\sqrt{2}}$.
   \end{itemize}
\end{itemize}
}
}
\end{frame}

\section{Entanglement}

%%
\begin{frame}{The \CNOT{x}{y} gate}{Quantum's \texttt{if then else}}
\TwoColumns{%
\adjustbox{valign=t}{\begin{quantikz}
\lstick{\ket{x}}  & \ctrl{1} & \qw \\
\lstick{\ket{y}}  & \targ{} &\qw
\end{quantikz}}

\begin{itemize}[<+->]
\item If $\ket{x}=\QOne{}$ then complement \ket{y}, else leave \ket{y} alone.
\item We first think of \NamedGate{CNOT} in terms of how it transforms each basis vectors~\ket{xy}
\item \ColorOne{$\ket{00} \mapsto \ket{00}$}
\item \ColorTwo{$\ket{01} \mapsto \ket{01}$}
\item \ColorThree{$\ket{10} \mapsto \ket{11}$}
\item \ColorFour{$\ket{11} \mapsto \ket{10}$}
\end{itemize}
}{%
\Vskip{-3em}{\small\[\CNOTMatrix{}\]}
\begin{TIKZP}[overlay]
  \draw[white] (0,0) circle(1pt);  % anchor
  \draw<3-6>[\RCone,thick] (2.35,1.1) ellipse(0.25 and 0.95);
  \draw<4-6>[\RCtwo,thick] (2.95, 1.1) ellipse (0.25 and 0.95);
  \draw<5-6>[\RCthree,thick] (3.5, 1.1) ellipse (0.25 and 0.95);
  \draw<6>[\RCfour,thick] (4.05, 1.1) ellipse (0.25 and 0.95);
\end{TIKZP}
\visible<7->{%
The above matrix implement this.

For superpositions such as \[ \NamedGate{CNOT}\left(\TwoSup{00}{10}\right)\]
use linearity to obtain \[ \TwoSup{00}{11}\]
}
}
    
\end{frame}
%%
\begin{frame}{Entangling two qubits}{We use gates we have studied to create an interesting quantum state}

\TwoUnequalColumns{0.32\textwidth}{0.68\textwidth}{%
\adjustbox{valign=t}{\begin{quantikz}
\lstick{\QZero{}}\slice{\alert<1>{\QState{0}}} &  \gate{H}\slice{\alert<2-5>{\QState{1}}} & \ctrl{1}\slice{\alert<7-8>{\QState{2}}} & \meter{\textcolor<8->{red}{0/1}}\\
\lstick{\QZero{}} &   \qw    &  \targ{}  & \meter{\textcolor<8->{red}{0/1}}
\end{quantikz}}%
\only<5>{%

We can do this because the qubits are acting independently at this point.  This state can be factored into the tensor product of two states.
}
}{%
\only<1-7>{%
\Vskip{-2.5em}\begin{itemize}
    \item<1-> As before, our initial state $\QState{0}=\ket{00}=\DQB{1}{0}{0}{0}$
    \only<2-3>{%
    \item<2-> To compute the state here, we can tensor the~\Hadamard{} and~\Identity{} gates, obtaining
    $\RootTwo{}\begin{pmatrix*}[r]
        1 & 0 & 1 & 0 \\
        0 & 1 & 0 & 1 \\
        1 & 0 & -1 & 0 \\
        0 & 1 & 0 & -1
    \end{pmatrix*}$
    \item<3-> Applying those gates to \QState{0} yields $\QState{1}=\RootTwo{}\DQB{1}{0}{1}{0}$}
    \only<4-5>{%
    \item It's easier to compute \QState{1} as follows:
   \begin{align*}
       \TensProd{\Hadamard{}\ket{0}}{\Identity{}\ket{0}}
     = & \TensProd{\HMatrix{}\PZero{}}{\PZero} \\
     = & \RootTwo{}\DQB{1}{0}{1}{0}
    \end{align*}}
    \only<6-7>{%
        \item<6-> $\QState{1} = \RootTwo{}\DQB{1}{0}{1}{0}$
        \item<7-> $\QState{2} = \CNOTMatrix{}\RootTwo\DQB{1}{0}{1}{0} = \RootTwo{}\DQB{1}{0}{0}{1}$
    }
    \end{itemize}}
    \only<8->{%
    \[\QState{2} =  \RootTwo{}\DQB{1}{0}{0}{1}=\frac{\ket{00}+\ket{11}}{\sqrt{2}}\]
    }
}
\only<8->{%
\begin{itemize}
\item<8-> In this state, we have a 50\% chance of seeing \ket{0} at each sensor, and a 50\% chance of seeing \ket{1} at each sensor.
\item<9-> The two measurements are absolutely correlated, \alert<10>{no matter in which order the measurements are taken}, and \alert<11>{no matter the distance between the sensors}.
\item<12->  The two outcomes are also uniformly random but completely unpredictable.
\end{itemize}
}
\end{frame}

\begin{frame}{The Bell state $\frac{\ket{00}+\ket{11}}{\sqrt{2}}$}{Our first entangled state}
\Vskip{-3.5em}\begin{itemize}
    \item<1-> The behavior of this state is an incredible consequence of quantum theory.
    \item<2-> While physics confirms quantum behavior experimentally, there is no explanation yet for \emph{why}.
    \item<3-> The Bell state is \emph{entangled}, meaning that it cannot be expressed as the tensor product of two single-qubit states.
\end{itemize}
\only<3->{%
\SmallSkip{}
\TwoUnequalColumns{0.58\textwidth}{0.42\textwidth}{%
\visible<3->{Proof (by contradiction):}
\visible<4->{Suppose there exist
    \[
    \SQB{a}{b} \mbox{ and } \SQB{c}{d}\ |\ \RootTwo{}\TensProd{\SQB{a}{b}}{\SQB{c}{d}} = \RootTwo{}\DQB{\textcolor<6>{\RCone}{1}}{\textcolor<6>{\RCtwo}{0}}{\textcolor<7>{\RCone}{0}}{\textcolor<7>{\RCtwo}{1}}
    \]}
}{%
\visible<5->{Then}
\begin{align*}
    \visible<6->{\textcolor<6>{\RCone}{ac} = & 1 & \textcolor<6>{\RCtwo}{ad} =& 0 }\\
    \visible<7->{\visible<8->{\times} \textcolor<7>{\RCtwo}{bd} = & 1 & \visible<8->{\times} \textcolor<7>{\RCone}{bc} =& 0 }\\
    \visible<8->{---- & -- & ---- & --} \\
    \visible<9->{abcd = & 1 & abcd =& 0}
\end{align*}
\visible<10->{which is a contradiction $\square$}
}}
    
\end{frame}



\begin{frame}{How entangled are the qubits?}{Their experience with CNOT cannot be undone unless they are physically proximate again}

\TwoUnequalColumns{0.35\textwidth}{0.65\textwidth}{%
\adjustbox{valign=t}{\begin{quantikz}
\lstick{\QZero{}}\slice{\QState{0}} &  \gate{H}\slice{\QState{1}} & \ctrl{1}\slice{\QState{2}} & \meter{0/1}\\
\lstick{\QZero{}} &   \qw    &  \targ{}  & \meter{0/1}
\end{quantikz}}%
}{%
\begin{itemize}
    \item The qubits' entanglement persists in any basis.
    \item No unitary action taken on them separately can disentangle them (see next frame).
    \item If they were brought back together and run through a \NamedGate{CNOT} gate again, then they would disentangle, because \NamedGate{CNOT} is its own inverse.
\end{itemize}
}
\MedSkip{}
Entanglement defies classical physics and logic and affords advantages for games and computations.
    
\end{frame}

\begin{frame}{Entanglement persists}{No unitary action on a single qubit can disentangle them}


\Vskip{-3.5em}\TwoUnequalColumns{0.45\textwidth}{0.55\textwidth}{%
\only<1-2>{%
\adjustbox{valign=t,scale=0.8}{\begin{quantikz}
\lstick{\QZero{}} &  \gate{H}& \ctrl{1}\slice{\QState{1}} & \qw\rstick{\QState{A}} \\
\lstick{\QZero{}} &   \qw  & \targ{} & \qw\rstick{\QState{B}}
\end{quantikz}}}%
\only<3>{%
\adjustbox{valign=t,scale=0.8}{\begin{quantikz}
\lstick{\QZero{}} &  \gate{H}& \ctrl{1}\slice{\QState{1}} & \qw\rstick{\QState{A}} \\[2em]
\lstick{\QZero{}} &   \qw  & \targ{} & \qw\rstick{\QState{B}}
\end{quantikz}}}%
\only<4->{%
\adjustbox{valign=t,scale=0.8}{\begin{quantikz}
\lstick{\QZero{}} &  \gate{H}& \ctrl{1}\slice{\alert<1-3>{\QState{1}}} & \gate{U}\slice{\alert<7-8>{\QState{2}}} & \qw  \rstick{\QState{A}}\\[2em]
\lstick{\QZero{}} &   \qw    &  \targ{}  & \qw & \qw  \rstick{\QState{B}}
\end{quantikz}}}%
}{%
\only<1-3>{%
\begin{itemize}
    \item<1-> We create the Bell state 
    \[\QState{1} = \RootTwo{}\DQB{1}{0}{0}{1} \]
    \item<2-> Alice takes the top qubit and Bob takes the bottom one.
    \item<3-> They then separate, let's say by a very large distance.
\end{itemize}
}%
\only<4-6>{%
\begin{itemize}
    \item<4-> Alice effects a change to her qubit, represented here by \NamedGate{U}.
    \item<5-> Recall generally
    \[\NamedGate{U} = \UGate{} \]
\end{itemize}}%
\only<7-8>{%
\Vskip{-3em}{\small\begin{align*}
    \QState{2} & = \left(\TensProd{\NamedGate{U}}{\NamedGate{I}}\right) \RootTwo\DQB{\ColorOne{1}}{0}{0}{\ColorTwo{1}}\\
    & = \RootTwo{}\DQB{\ColorOne{\UGateA{}}}{\ColorTwo{\UGateB{}}}{\ColorOne{\UGateC{}}}{\ColorTwo{\UGateD{}}}
\end{align*}}
}%
}
\only<6-7>{%
\small{%
\Vskip{-2em}\[
\TensProd{\NamedGate{U}}{\NamedGate{I}} =
\begin{pmatrix*}[r]
\ColorOne{\UGateA{}} & 0 & \UGateB{}  & \ColorTwo{0} \\
\ColorOne{0} & \UGateA{} & 0 & \ColorTwo{\UGateB{}} \\
\ColorOne{\UGateC{}} & 0 & \UGateD{}  & \ColorTwo{0} \\
\ColorOne{0} & \UGateC{} & 0 & \ColorTwo{\UGateD{}}
\end{pmatrix*}
\]}}%
\only<8->{%
\begin{itemize}
    \item It is an exercise to show that \QState{2} is not tensor factorable.
    \item Thus, Alice's imposition of the \NamedGate{U} gate cannot disentangle the Bell state.  The entanglement persists, but in a different basis.
\end{itemize}
}%
    
\end{frame}

\begin{frame}{Dealing with entanglement}{Circuit drawings are not always clear concerning entanglement}
    
\Vskip{-3.5em}\TwoUnequalColumns{0.2\textwidth}{0.8\textwidth}{%
\adjustbox{valign=t,scale=1.0}{\begin{quantikz}
\qw & \qw\slice{\QState{0}} &  \gate{\NamedGate{A}} \slice{\QState{1}}  & \qw  \\
\qw & \qw &  \gate{\NamedGate{B}} &\qw
\end{quantikz}}%
}{%
\Vskip{-2.5em}\begin{itemize}[<+->]
    \item Physically, the circuit can be implemented as shown.  The following operations occur separately to the qubits:
    \begin{itemize}
        \item \NamedGate{A} is applied to the top qubit
        \item \NamedGate{B} is applied to the bottom qubit
    \end{itemize}
    \visible<3->{The circuit works regardless of the entangled nature of \QState{0}.}
    \item However, to \emph{model} the circuit mathematically, it helps to know whether \QState{0} is entangled or not:
    \begin{itemize}
        \item If it is not entangled, then \NamedGate{A} and \NamedGate{B} may be applied separately to their respective qubits, as shown, to compute \QState{1}.  This is easy and efficient.
        \item But the circuit is misleading when \QState{0} might be entangled!
        Then we must compute \TensProd{\NamedGate{A}}{\NamedGate{B}}, obtaining
        \[
        \QState{1} = \left(\TensProd{\NamedGate{A}}{\NamedGate{B}}\right)\QState{0}
        \]
    \end{itemize}
\end{itemize}
}
\end{frame}

\begin{frame}{Pondering entanglement}{Einstein regarded quantum theory as incomplete}

\begin{itemize}
    \item<1-> Upon measurement, our entangled qubits collapse unpredictably to the same value, instantaneously, even if they are separated by light years.
    \item<2-> Einstein regarded this \textit{spukhafte Fernwirkung} (spooky action at a distance) as evidence that quantum theory is \emph{incomplete}.
    \item<3-> For example, local \href{https://en.wikipedia.org/wiki/Hidden-variable_theory}{hidden variables} would explain how qubits manage to provide the same measurement after entanglement.
    \item<6-> Thanks to Bell, we shall soon rule out local hidden variables.
    \item<7-> Perhaps there are \href{https://en.wikipedia.org/wiki/Principle_of_locality}{nonlocal} hidden variables, but how would we prove this?
\end{itemize}
\only<3-7>{%
\begin{center}
\begin{TIKZP}
    \draw (0,0)  -| ++(1,1) 
    node[pos=0.25,below] {$q_0$} 
    -| (0,0);
    \draw<3> (1.5,0)  -| ++(1,1) 
    node[pos=0.25,below] {$q_1$} 
    -| (1.5,0);
    \draw<4> (0.5,0)  -| ++(1,1) 
    node[pos=0.25,below] {$q_1$} 
    -| (0.5,0);
    \fill<4>[red] (0.5,0) rectangle ++(0.5, 1);
    \path<4> (0.75,1) node[above] {\NamedGate{CNOT}};
    \fill<5->[red] (0,0) rectangle ++(1,1);
    \fill<5->[red] (11.5,0) rectangle ++(1,1);
    \draw<5-> (11.5,0)  -| ++(1,1) 
    node[pos=0.25,below] {$q_1$};
    \path<5-> (6,0) node[above] {$\longleftarrow\ $light years of separation$\ \longrightarrow$};
\end{TIKZP}
\end{center}
}
\only<8->{%
\MedSkip{}
Gravitational attraction exists at arbitrary distances, but we seem OK with that.  Is it so strange that entanglement affects the fate of two qubits no matter their distance?}
    
\end{frame}

\section{EPR paradox}

\begin{frame}{EPR ``paradox'' \LinkArrow{https://en.wikipedia.org/wiki/EPR_paradox}}{More on Einstein's disbelief}
\begin{itemize}[<+->]
    \item Einstein and others held that the Bell states were paradoxical, in that they violate one or both of the dogmatic concepts stated below.
    \item Their conclusion was that quantum theory was \emph{incomplete}, meaning there must be an as-yet unreached and deeper level of understanding of how quantum mechanics works.  \href{https://en.wikipedia.org/wiki/Hidden-variable_theory}{Hidden variables} could bridge the gap.
\end{itemize}
\begin{description}
  \item[realism]  Objects have characteristics whether we measure them or not.  My pencil has a certain weight whether I weigh it or not.
  \item[locality] A measurement made in one system cannot instantaneously affect a measurement made in a different system.
\end{description}
\only<3>{%
Bell's predictions have been substantiated repeatedly by experiments, and we will see that \href{https://en.wikipedia.org/wiki/Hidden-variable_theory}{hidden variables} cannot explain quantum behavior.}
\end{frame}

\begin{frame}{Illustrating the ``paradox''}{Using a Bell state}

\Vskip{-5em}\TwoColumns{%
\begin{itemize}[<+->]
    \item Consider the state
    \[\QState{\relax} = \RootTwo{}\left(\ket{\textcolor<2->{\RCone}{0}\textcolor<3->{\RCtwo}{1}}-\ket{\textcolor<2->{\RCone}{1}\textcolor<3->{\RCtwo}{0}}\right)
    \]
    \item \textcolor{\RCone}{Alice gets the first qubit.}
    \item \textcolor{\RCtwo}{Bob gets the second qubit.}
    \item Recalling
    \begin{align*}
        \ket{+}= & \RootTwo{}\SQB{1}{1} \\
        \ket{-}= & \RootTwo{}\SQB{1}{-1}
    \end{align*}
    
\end{itemize}
}{%
\begin{itemize}
    \item<5->  We can show
    \[\QState{\relax} = \RootTwo{}\left(\ket{\textcolor<2->{\RCone}{+}\textcolor<3->{\RCtwo}{-}}-\ket{\textcolor<2->{\RCone}{-}\textcolor<3->{\RCtwo}{+}}\right)
    \]
    \item<6-> Alice can choose to measure her qubit in the computational (\PauliZ{}) basis or in the \PauliX{} basis.  Let's assume she sees \ket{0} in the former case and \ket{+} in the latter case.
    \item<7-> That choice causes Bob, light years away, to obtain state \ket{1} or \ket{-} for his qubit, but those are entirely different states.
\end{itemize}
}
\MedSkip{}
\only<8->{\alert{%
This violates belief in locality and realism.}}
    
\end{frame}

\section*{Measurement}

\begin{frame}{Measurement of a multiqubit system}{The measured system's eigenvalue is the product of the states' measured eigenvalues}
\TwoUnequalColumns{0.7\textwidth}{0.3\textwidth}{%
\only<1>{
\Vskip{-3em}\begin{itemize}
    \item As shown, each qubit is a physical object that is measured separately from the other qubits, even if the qubits are entangled.
    \item Each measured qubit collapses into an eigenstate of its measurement operator.
    \begin{itemize}
    \item In the computational basis, this is either \QZero{} or \QOne{}.  
    \item The associated eigenvalues are $+1$ and $-1$, respectively.
    \end{itemize}
    \item The system's measurement is an $n$-bit vector, consistent with any constraints imposed by the computation.
    \item The system's eigenvalue is the \emph{product} of the measured qubits' eigenvalues.  \alert{Why? $\ldots$}
\end{itemize}}%
\only<2>{%
\Vskip{-3em}\begin{itemize}
    \item Suppose the resulting state is 
    \[
    \lambda_{1} \QState{1} \TensOp{} \lambda_{2} \QState{2} \TensOp{} \cdots \TensOp{} \lambda_{n} \QState{n}
    \]
    \item This equals
    \[
    \left(\lambda_{1} \times \lambda_{2} \times \cdots \times \lambda_{n}\right)
    \ \left( \QState{1} \TensOp{}  \QState{2} \TensOp{} \cdots \TensOp{}  \QState{n}\right)
    \]
    \item Thus, the resulting eigenvalue on the system's state is the product of the individual eigenvalues.
    \item Each $\lambda_{i}$ is $\pm 1$, so the resulting eigenvalue is the count of the $-1$ values, mod 2.
\end{itemize}
}
}{%
\begin{center}
\adjustbox{scale=1.0,valign=t}{%
\begin{quantikz}
\lstick{$q_1$} &\qw & \meter{0/1}\\
\lstick{$q_2$} & \qw & \meter{0/1} \\
\lstick{\RVDots}     \\
\lstick{$q_n$} & \qw & \meter{0/1}
\end{quantikz}}

\end{center}}
\end{frame}

\section*{No-cloning theorem}

\begin{frame}{Cloning a state}{What state should we obtain?}

\begin{itemize}
    \item<1-> What would it take to clone an arbitrary quantum state 
    $\QState{a}=\SQB{\alpha}{\beta}$?
    \item<2-> This would create $\QState{b}=\SQB{\alpha}{\beta}$.
    \item<3-> The resulting system would be the (unentangled)
    state 
    \[
     \ket{\QName{a}\QName{b}} = \TensProd{\SQB{\alpha}{\beta}}{\SQB{\alpha}{\beta}} = \DQB{\alpha^{2}}{\alpha\beta}{\alpha\beta}{\beta^{2}}
    \]
    \item<4> We will show this is not generally possible:  we can clone \QState{a} only if we know~$\alpha$ and $\beta$.
\end{itemize}

\end{frame}

\begin{frame}{Why not \NamedGate{CNOT}?}{It copies basis states faithfully}

\TwoUnequalColumns{0.3\textwidth}{0.7\textwidth}{%
\adjustbox{valign=t}{\begin{quantikz}
\lstick{\QState{a}} &  \ctrl{1}\ & \qw \rstick{\QState{a}} \\
\lstick{\QZero{}} &    \targ{}  & \qw \rstick{\QState{b}}
\end{quantikz}}%
}{%
\only<1-3>{%
Recall $\NamedGate{CNOT} = \CNOTMatrix{}$.}
\only<4->{%
But then $\NamedGate{CNOT}\left(\TensProd{\SQB{\alpha}{\beta}}{\PZero{}}\right) = \DQB{\alpha}{0}{0}{\beta}\alert{\not=}\DQB{\alpha^2}{\alpha\beta}{\alpha\beta}{\beta^2}$}
}
\only<1-4>{%
    \begin{description}
    \item<2->[$\QState{a}=\ket{0}$] $\NamedGate{CNOT} \left(
    \TensProd{\ket{0}}{\ket{0}}
    = \TensProd{\PZero{}}{\PZero{}} = \DQB{1}{0}{0}{0}
    \right)
    =  \DQB{1}{0}{0}{0} = \ket{00}$
    \item<3->[$\QState{a}=\ket{1}$] $\NamedGate{CNOT} \left(
    \TensProd{\ket{1}}{\ket{0}}
    = \TensProd{\POne{}}{\PZero{}} = \DQB{0}{0}{1}{0}
    \right)
    =  \DQB{0}{0}{0}{1} = \ket{11}$
    \end{description}}
\only<5>{%
\BigSkip{}
\begin{itemize}
 \item \NamedGate{CNOT} successfully clones $\SQB{\alpha}{\beta}$ only when $\alpha=1$ or $\beta=1$.
 \item How can we be sure there is \emph{no} quantum gate that can clone \QState{a}?
 
\end{itemize}
}
    
\end{frame}

\begin{frame}{No cloning theorem}{It is impossible to construct a unitary gate that clones an arbitrary qubit}

\TwoUnequalColumns{0.35\textwidth}{0.65\textwidth}{%
\Vskip{-2.2em}\alert<3>{For $n\geq 2$}
\adjustbox{scale=0.9,valign=t}{%
\begin{quantikz}
& \lstick{\alert<2>{\alert<2>{\QState{a}}}}\slice{\alert<8>{\QState{0}}} &  \gate[wires=6]{U}\slice{\alert<9>{\QState{1}}} &  \qw\rstick{\alert<5>{\QState{a}}}\\
\lstick[wires=5]{\alert<3>{$n-1$}} & \lstick{\alert<3>{\QZero{}}} &   & \qw\rstick{\alert<3,6>{\QState{a}}}  \\
& \lstick{\alert<4>{\QZero{}}} &   & \qw\rstick[wires=4]{\alert<7>{$f(\QState{a})$}}  \\
& \lstick{\alert<4>{$\bullet$}} &   & \qw  \\
& \lstick{\alert<4>{$\bullet$}} &   & \qw  \\
& \lstick{\alert<4>{\QZero{}}} &   & \qw 
\end{quantikz}%
}
\only<8-12>{%
\SmallSkip{}
\visible<8->{ $\alert<8>{\QState{0}} =  \ket{\QName{a}\,0^{n-1}}$}
\visible<9-12>{%
     $\alert<9>{\QState{1}}=\TensProd{\QState{a}}{\TensProd{\QState{a}}{f(\QState{a})}}$}
}%
\only<22->{%
\SmallSkip{}
Thus, if any $U$ can clone \ket{0} and \ket{1} then it cannot clone \ket{+}. \QEDsym{}
}
}{%
\Vskip{-1em}
\only<1-4>{%
\Vskip{-3em}\begin{itemize}
    \item<1-> We seek to prove that \emph{no} circuit can accomplish cloning.
    We therefore formulate a parameterized proof using $n$, the number of inputs (and outputs) to a hypothetical gate $U$.
    \item<2-> The top input is \QState{a}, the state we wish to clone.
    \item<3-> We need at least one extra qubit to receive the putative clone.
    \item<4-> The rest of the qubits are ancillary qubits, or \href{https://en.wikipedia.org/wiki/Ancilla_bit}{ancillas}.   These are offered to $U$ as working storage, with no restrictions on their use.  Providing no information, they can be initialized to \ket{0}.
\end{itemize}}%
\only<5-7>{%
\Vskip{-3em}\begin{itemize}
    \item<5-> For output, the top qubit remains \QState{a}.
    \item<6-> The next qubit is the clone, so it is also in state \QState{a}.
    \item<7-> The rest of the qubits are in a state computed by $U$ as some arbitrary function of its input \QState{a}. 
\end{itemize}
}%
\only<10-17>{%
\Vskip{-2em}If $U$ can do the following
\begin{itemize}
    \item \textcolor<15>{\RCone}{$U(\TensProd{\ket{0}}{\ket{0^{n-1}}})= \TensProd{\ket{00}}{f(\ket{0})}$}
    \item \textcolor<16>{\RCtwo}{$U(\TensProd{\ket{1}}{\ket{0^{n-1}}})= \TensProd{\ket{11}}{f(\ket{1})}$}
\end{itemize}
\only<10>
then when presented with $\QState{a}=\ket{+}=\frac{\ket{0}+\ket{1}}{\sqrt{2}}$
\Vskip{-2em}\begin{align*}
\visible<12->{U(\ket{+})=& \RootTwo{}U(\TensProd{(\ket{0}+\ket{1})}{\ket{0^{n-1}}})} \\
\visible<13->{\sqrt{2}\,U(\ket{+})= & U(\TensProd{\ket{0}}{\ket{0^{n-1}}}+\TensProd{\ket{1}}{\ket{0^{n-1}}})} \\
\visible<14->{ = & 
\textcolor<15>{\RCone}{U(\TensProd{\ket{0}}{\ket{0^{n-1}}})} + \textcolor<16>{\RCtwo}{U(\TensProd{\ket{1}}{\ket{0^{n-1}}})}} \\
\visible<15->{ = &
\textcolor<15>{\RCone}{\TensProd{\ket{00}}{f(\ket{0})}} +
\visible<16->{\textcolor<16>{\RCtwo}{\TensProd{\ket{11}}{f(\ket{1})}
}}} \\
\visible<17->{%
U(\ket{+}) = & 
\frac{\left(\TensProd{\ket{00}}{f(\ket{0}})\right) 
+ \left(\TensProd{\ket{11}}{f(\ket{1}}\right)}{\sqrt{2}}
}
\end{align*}
}
}%
\only<18->{%
\Vskip{-3em}\[
U(\ket{+}) = 
\frac{\left(\TensProd{\alert<19>{\ket{00}}}{f(\ket{0}})\right) 
+ \left(\TensProd{\alert<19>{\ket{11}}}{f(\ket{1}}\right)}{\sqrt{2}}
\]
\begin{itemize}
    \item<19-> The state of the top two qubits is $\frac{\ket{00}+\ket{11}}{\sqrt{2}}$.
    \item<20-> As a column vector, that state is $\RootTwo{}\DQB{1}{0}{0}{1}$.
    \item<21-> But this is not $\ket{++}=\RootTwo{}\DQB{1}{1}{1}{1}$
    
\end{itemize}
}%
}
\end{frame}



\begin{frame}{Summary}{What have we learned?}

\begin{itemize}
  \item Reasoning mathematically about multiple qubit systems requires tensor products of input states and gates.
  \item Some states are pure, in that they can be expressed as the tensor product of single-qubit states.
  \item Others are entangled, and no  tensor-factoring of such states is possible.
  \item We cannot hope in general to clone an arbitrary, unknown quantum state.
  \item This has serious implications for fault tolerance.  Classically we can use redundancy to overcome faults, but in quantum computing we cannot duplicate an unknown state.
\end{itemize}
    
\end{frame}
