\lecture[1]{Background}{lec:background}

\title{Background}
\subtitle{What you need to know}

\date{\today}

\begin{frame}
\maketitle
\end{frame}

\section{Classical Logic}

\begin{frame}{Overview of classical logic}
\begin{itemize}
    \item You should be familiar with \href{https://en.wikipedia.org/wiki/Boolean_algebra}{Boolean algebra} but we will review it here.
    \item The two constants
    \begin{itemize}
        \item \Zero{} or \False{}
        \item \One{} or \True{}
    \end{itemize}
    \item We will review functions over those values.
    \item Classical logic use \emph{bits} that are either \True{} or \False{}.
    \item Quantum logic uses \emph{qubits} that can be in a \emph{superposition} of \Zero{} and \One{}.
\end{itemize}

    
\end{frame}

\section{Complex arithmetic}

\begin{frame}{Overview of complex arithmetic}
\begin{itemize}
    \item Complex arithmetic is especially convenient for quantum computing.
    \item We will use both forms:
    \begin{itemize}
        \item \href{https://en.wikipedia.org/wiki/Cartesian_coordinate_system}{Cartesian coordinates}
        \item \href{https://en.wikipedia.org/wiki/Polar_coordinate_system}{Polar coordinates}
    \end{itemize}
    \item Complex values allow us to express the magnitude of a wave as well as its phase.  This is helpful for understanding interference.
    \item We could defer using complex arithmetic in this course, but it is wise to embrace it from the start.
\end{itemize}
    
\end{frame}

\section{Linear algebra}

\begin{frame}{Overview of linear algebra}{Quantum systems are linear!}
\begin{itemize}
    \item While much of our world behaves nonlinearly, quantum systems can surprisingly be captured using linear algebra.
    \item Vectors and matrices can represent states and gates.  
    \begin{itemize}
        \item The state of a quantum system is most often described as a column vector, called a \href{https://en.wikipedia.org/wiki/Bra-ket_notation}{ket}.
         \item The effect of a quantum gate can be described as a square matrix, which maps in input ket to an output ket.
    \end{itemize}
    \item This is convenient but not efficient in the number of quantum bits.
    \item The composition of states and gates is computed using \href{https://en.wikipedia.org/wiki/Tensor}{tensor arithmetic}.
\end{itemize}
    
\end{frame}


