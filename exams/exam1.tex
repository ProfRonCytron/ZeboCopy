\documentclass[12pt]{article}
\usepackage[margin=0.5in]{geometry} 
\usepackage{amsmath,amsthm,amssymb,amsfonts, enumitem, fancyhdr, color, comment, graphicx, environ}
\usepackage{course}
\usepackage{cse468-Spring22}
\usepackage{program}
\usepackage{fancybox}
\usepackage{adjustbox}
\usepackage{quantikz}
\def\Gate#1{\mbox{\textbf{#1}}}
\def\X{\Gate{X}}
\def\Z{\Gate{Z}}
\def\I{\Gate{I}}
\def\H{\Gate{H}}
\def\QZero{\ket{0}}
\def\QState#1{\ensuremath{\psi_{#1}}}

\def\Obox#1{\Ovalbox{\hbox to 1ex{\vrule width 0pt height 1ex\hss #1\hss}}}
\def\TFMarked#1#2{\stackbox[l][m]{\Obox{#1}~\textbf{true}\\\Obox{#2}~\textbf{false}}}
\def\TF{\TFMarked{\relax}{\relax}}
\def\exp#1{\ensuremath{e^{#1}}}
\newcommand{\Blank}[1][1in]{\mbox{\vrule width #1 depth 2pt}\vrule width 0pt height 2.0em}
\def\BlQb{\mbox{\ensuremath{\Blank[4em]\ket{0}+\Blank[4em]\ket{1}}}}
\newcommand{\Blanket}[1][3em]{%
\mbox{\ensuremath{|\,\Blank[#1]\,\rangle}}}

\def\Tall{\vrule width 0pt height 2em depth 0.5em}

\def\SQB#1#2{%
\ensuremath{%
\begin{pmatrix*}[r] #1 \\ #2\end{pmatrix*}}}

\def\SQBB{\SQB{\Blank[2em]}{\Blank[2em]}}

\def\DQB#1#2#3#4{%
\ensuremath{%
\begin{pmatrix*}[r] #1 \\ #2 \\ #3 \\ #4\end{pmatrix*}}}
\def\DQBB{\DQB{\Blank[2em]}{\Blank[2em]}{\Blank[2em]}{\Blank[2em]}}

\begin{document}

\begin{assignment}{Exam I}{9 March 2022}{End of class}

{\small {\large \fbox{READ THIS before starting!}}
This exam is open-book, open-notes, open-Internet, but you must do this
work on your own without contact or conversations with any person.
Because this exam is given in a somewhat distributed manner, no questions will be answered, and no clarifications will be given.  State your assumptions and count on us to be fair and flexible, especially if we have been unclear.


Your work must be legible.  Work that is
difficult to read will receive no credit.  There is a blank page at the end
if you want to show extra work there.

You must sign the pledge below for your exam to count.  Any cheating will
cause the students involved to receive an F for this course. Other unpleasant
actions
may be taken.

You must fill in your identifying information correctly.  
}

\begin{center}\large
\begin{tabular}{|c|c|c|} \hline
\multicolumn{3}{|c|}{{\bf Print  clearly} the following information:}  \\ \hline
\multicolumn{3}{|l|}{Name (print clearly):\Tall{}\hbox to 3in{\hss}}  \\ \hline
\multicolumn{3}{|l|}{Student 6-digit ID (print {\it really} clearly):\Tall{}\hbox to 3in{\hss}} \\ \hline
\end{tabular}
\end{center}

{\bf Pledge:} On my honor, I have neither
given nor received any unauthorized aid on this exam.

Signed:  \Blank\Blank\Blank\Blank \\ \hbox to 5em{\hss}(Be 
sure you filled in your information in the box above!)
%
%
%
\clearpage
\begin{enumerate}
\item\Points{30} For the \textbf{true}/\textbf{false} questions below, indicate your response by marking an~\textbf{x} in the appropriate box, like this:
\TFMarked{\textbf{x}}{\relax} or \TFMarked{\relax}{\textbf{x}}.
\begin{itemize}
    \item All unitary gates have an inverse, which is that gate's conjugate transpose \TF{}
    \item On the Bloch sphere, recall that~$\theta$ is the angle off the $z$-axis and~$\phi$ is the angle off the $x$-axis.  Consider a state \[ \ket{\psi}=\alpha\ket{0}+\beta\ket{1}\]
    where $\alpha$ and $\beta$ are \emph{real} (no imaginary component).
    Locating \QState{} on the Bloch sphere, we must have $\phi=0$. \TF{}
    \item 
\end{itemize}

\item\Points{30} For each question below, fill in the blank.  Your answer must appear in the provided blank for proper credit.
\begin{itemize}
    \item The minimum number of real parameters necessary to describe an arbitrary state of a single qubit is~\Blank{}.
    \item The minimum number of real parameters needed to describe an arbitrary, single-qubit, unitary gate is~\Blank{}.
    \item On the Bloch sphere, \ket{0} is the \Blank{} pole.
    \item The states \ket{+} and \ket{-} are found on which axis of the Bloch sphere? \Blank{}
    \item Consider a state $\psi$ on the Block sphere which is an eigenstate of some unitary gate~$U$.  The maximum angular distance between~$\psi$ and~$U(\psi)$ is~\Blank{}~radians.
    \item $\X\ket{0}=\BlQb$
    \item $\Z\ket{+}=\BlQb$
    \item $\H{\ket{0}}=\BlQb$
    \item Consider this circuit:
    
    \adjustbox{valign=t}{\begin{quantikz}
\lstick{\QZero{}}&\gate{X}\slice{\QState{0}} & \ctrl{1}\slice{\QState{1}} & \gate{H}\slice{\QState{1}}  & \qw\\
\lstick{\QZero{}} &\qw & \targ{} &  \qw    &  \qw
\end{quantikz}}

Fill in the column vectors to show your analysis of the states at the indicated positions of the circuit.

$\QState{0}=\DQBB{}$  $\QState{1}=\DQBB{}$  $\QState{2}=\DQBB{}$

Scratch space for your work:

\vskip 2in
\item Express \QState{2} as the product of two 1-quibit states:

\[\QState{2} = \SQBB \otimes \SQBB \]

\end{itemize}


\clearpage\item\Points{10}
Consider the Elitzur--Vaidman bomb, but with a small change.  The
polarizing filter attached to the bomb is
oriented at $\pi/4$ radians (a $45^{\circ}$ angle).  In class and in your homework assignment,
the filter was horizontal, at $0$ radians.

\begin{itemize}
\item
A photon hitting this filter that is measured at $\pi/4$ radians causes
no explosion and passes through oriented at $\pi/4$ radians.
\item
A photon hitting this filter that is measured at $3\pi/4$ radians causes
an explosion.
\item If the filter (and bomb) are not present, the photon passes through
unchanged.
\end{itemize}
Let \ket{0} represent a photon oriented horizontally, at $0$ radians.
Let \ket{1} represent a photon oriented vertically, at $\pi/2$ radians.

For this problem you can only start with photons in the state \ket{0} or
\ket{1} but you are allowed to use any quantum gates you like to solve
the problem.

Likewise, measurements outside the box (possibly containing a filter and
bomb) can only be made in the 
standard basis of \ket{0} and
\ket{1}.

As in class, we would like to detect the absence or presence of the bomb, 
with the
probability of an explosion occurring under our control based on
the angle of rotation each iteration.
Below sketch a solution to the bomb problem based on the above configuration,
explaining carefully and clearly what is happening at each step.  Because
this is a take-home exam, you are expected to draw and explain clearly and
carefully in the space below.

\clearpage\item\Points{10} Block Sphere Orientiering:  all answers are in the standard basis.  Write each amplitude in the provided blank space, fully above the dark line.  For example, to express $\frac{i}{\sqrt{3}}$ you would write \Blank{}\hbox to 0pt{\hskip -4em\raisebox{4pt}{$i/\sqrt{3}$}\hss}.



\begin{itemize}
    \item We begin at the North pole of the Block sphere.
    \item We are in state \BlQb{}.
    \item We experience a \Gate{Y} gate.  We are now at \BlQb{}.
    \item We return to the North pole.
    \item We experience an \H{} gate.  We are now at \BlQb{}.
    \item We experience a \Z{} gate.  We are now at state \BlQb{}.
\end{itemize}


\clearpage\item\Points{10}
Consider the following entangled state of three qubits:
\[\frac{\ket{000}+\ket{111}}{\sqrt{2}}\]
In lecture we studied how to entangle that third qubit with the first two.  In this problem we consider \emph{disentangling} the third qubit from the first two, without resorting to measurement.

Recall that we entangle the third new qubit $q_{3}$ with those that came before ($q_{1}, q_{2}$) using a \Gate{CNOT} gate, controlled by either $q_{1}$ or $q_{2}$, and applied to $q_{3}$ which is initialized to~\ket{0}.  Because \Gate{CNOT} is its own inverse, we try useing the gate again to disentangle~$q_{3}$.
\begin{enumerate}
  \item\Points{3} 
  \[(\I\otimes\Gate{CNOT})(\ket{000}) = \Blanket{}\]
\item\Points{3} 
  \[(\I\otimes\Gate{CNOT})(\ket{111} = \Blanket{}\]
\item By linearity,
\[(\I\otimes\Gate{CNOT})\left(\frac{\ket{000}+\ket{111}}{\sqrt{2}}\right) = \frac{\Blank[4em]+\Blank[4em]}{\sqrt{2}} = \frac{\ket{00}+\ket{11}}{\sqrt{2}} \otimes \Blanket{}\]
\item Qubit $q_3$ is now disentangled and in its current state it measures as~$\Blanket{}$.
\item Measuring $q_3$ causes collapse of the other two qubits? \TF{}
\item delete this and ask about measurements of all three. Prior to disentanglement, if $q_3$ is the first qubit to be measured, we expect to see the following value for $q_3$'s measurement how often?
\begin{center}
\begin{tabular}{c|c}
Measures as &  What percent of the time? \\
\ket{0} & \Blank \\
\ket{1} & \Blank
\end{tabular}
\end{center}
\item After disentanglement, we can achieve that same result for $q_3$ by placing which gate in the unmarked box?~\Blank{}
\item The outcomes at this point seem identical:
\begin{itemize}
    \item Prior to entanglement, measurement of any qubit would yield either \ket{000} or \ket{111}.  The value of $q_3$ was equally likely to be \ket{0} or \ket{1}.  
    \item After entanglement, with your gate in place above, we are equally likely to measure \ket{0} and \ket{1} for $q_3$.
\end{itemize}
\item Let's see if the states are the same.  What is the state of the system at \QState{x}?  Write your answer in the space below:

\end{enumerate}

\end{enumerate}

\end{assignment}
\Bpage{}

\end{document}
