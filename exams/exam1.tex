\documentclass[12pt]{article}
\usepackage{exercises/handwritten/handout}

\usepackage{amssymb,mathrsfs, amsmath,amsfonts}
\usepackage{mathtools}
\usepackage{graphicx}
\usepackage{enumitem}
\usepackage{braket}
\usepackage{bbkey}
%\graphicspath{ {./ps110-assets/} }

\input{quantummacros.tex}

\def\TwoQBlank{%
\DQB{\Blank[8em]}{\Blank[8em]}{\Blank[8em]}{\Blank[8em]}%
}
\begin{document}
\assignment{Exam 1}

\begin{center}\bf
Due 20 March 2025, 4 PM CDT in GradeScope\\
No late extensions for the exam!
\end{center}
\begin{quote}
    This exam has been prepared so that you can write answers in provided spaces.  If you choose to write your answers into some other document, your answers must be formatted in the same manner as the answer spaces provided here. For example, if a column vector with blanks is provided here (as in Problem~\ref{prob:five}), your answer must be a column vector.

    Where you see free-form responses, such as for your name and ID below, you can supply that information however you wish.

    Failure to follow these instructions will cause you to receive no credit for your work.

    If your work is not shown, or the work shown is unclear, hard to read, or messy, you will receive no partial credit.   Where you are asked to show work, your work must be clear, neat, and easy to read to receive any credit.

    {\bf Academic integrity} You may not discuss the exam with anybody except the instructor. You are free to use any resources available to support your learning and your computations, but you may not seek answers to the questions themselves.
\end{quote}

\begin{enumerate}
    \item\Points{0} \begin{itemize}
    \item Your name? \Blank[3in]{}
    \item Your student ID? \Blank[3in]{}
\end{itemize}

\item \Points{10} To make grading easier please use the format below, or if you write your answers out on your own, please imitate the format of the response section.  The notation $\NamedGate{U}^{n}$ denotes $n$ consecutive applications of \NamedGate{U} on a single qubit, \emph{not} an $n$-way tensor product.

\textbf{Questions (do not respond here---see response section below)}
\begin{enumerate}[label=\theenumi.\arabic*]
   \item\label{tf:a} All states in the $x-z$ plane on the Bloch sphere can be described without using relative phase.
   \item\label{tf:b} Each of the Pauli gates (\PauliX,\PauliY,\PauliZ,\Hadamard) is its own inverse.
   \item\label{tf:c} The Control-\PauliZ{} gate cannot produce an entangled state, unless it is presented with an already entangled state.
   \item\label{tf:d} If the state \TwoSup{00}{01} is presented to gates \TensProd{\Hadamard}{\PauliX}, the tensor product of those gates must be formed to compute the correct resulting state.
   \item\label{tf:e} The only valid bases in terms of measurement on the Bloch sphere are the \PauliZ{} and \PauliX{} bases.
   \item\label{tf:f} What state is $\PauliX(\Hadamard^{n}(\QZero))$ if $n$ is even?
      \item\label{tf:g} What state is $\PauliX(\Hadamard^{n}(\QZero))$ if $n$ is odd?
    \item\label{tf:h} In teleportation, what gates must Bob apply to his qubit to recover Alice's state if she tells him both of her qubits measured \QOne{}?
    \item\label{tf:i} Due to the no cloning theorem, it is not possible to make a copy of a state that is either \QZero{} or \QOne{}.
    \item\label{tf:j} What is the eigenvalue associated with (eigen)state $\ket{01101010010111110000}$ in the computational basis?
\end{enumerate}

\def\BlankPrompt#1{\ref{tf:#1}~\Blank{}\hskip 2em plus 1em 
 minus 1em}
\def\TFPrompt#1{\ref{tf:#1}~\TF{}\hskip 2em plus 1em 
 minus 1em}
\textbf{Responses}

\TFPrompt{a}\TFPrompt{b}\TFPrompt{c}\TFPrompt{d}\TFPrompt{e}

\BlankPrompt{f}\BlankPrompt{g}\BlankPrompt{h}\TFPrompt{i}\BlankPrompt{j}

\item\Points{5} Consider the usual polarization basis, where \QZero{} is horizontal at 0 radians, \QOne{} is vertical at $\pi/2$ radians, so there are $\pi/2$ radians between the two basis vectors.

You are given a source of photons along with the promise that exactly one of the following is true:
\begin{itemize}
    \item All photons are polarized at $\pi/4$ radians (45 degrees), xor
    \item All photons are unpolarized.
\end{itemize}
\begin{enumerate}[label=\theenumi.\arabic*]
    \item\Points{1} Consider the case where all photons are polarized at $\pi/4$ radians.  In the basis using \QZero{} and \QOne{}, what is the state of each photon? 
    
    \Blank[5in]{}
    \item\Points{1} Consider the case where all photons are unpolarized.   In the basis using \QZero{} and \QOne{}, what is the state of each photon? 

    \Blank[5in]{}
    \item\Points{3} Assuming you can receive as many photons from the source as you wish, describe an experiment to determine which of the two possible streams of photons you are receiving.  For full credit
    \begin{itemize}
        \item Describe the setup in sufficient detail.
        \item Analyze how certain you are of your determination after $n$ photons have been sent to you. 
    \end{itemize}

    \LeaveSpace{}
\end{enumerate}
\item\Points{5} The questions below refer to the Elitzur--Vaidman bomb, specifically the \Quote{more careful approach} presented on slides 8 and following of slide deck 080.
\begin{enumerate}[label=\theenumi.\arabic*]
  \item\Points{3} What angle should be chosen for $\theta$ so that the bomb is
  \begin{itemize}
      \item certain to explode?\Blank{}~radians
      \item predicted to explode with probability~0.5?\Blank{}~radians
  \item guaranteed \emph{not} to explode?\Blank{}~radians
  \end{itemize}
  \item\Points{2} Why will the algorithm fail to work if the bomb is guaranteed never to explode?\LeaveSpace{}
\end{enumerate}

\item\label{prob:five}\Points{25} Recall from homework that a point on the Bloch sphere
\[
\ket{\psi} = \cos\left(\frac{\theta}{2}\right)\ket{0}
    + \ExpPhase{\phi}\sin\left(\frac{\theta}{2}\right)\ket{1} = \SQB{\cos(\frac{\theta}{2})}{\ExpPhase{\phi}\sin(\frac{\theta}{2})}
\]
has its antipodal point at
        \[
        \ket{\psi'}= \sin\left(\frac{\theta}{2}\right)\ket{0}-\ExpPhase{\phi}\cos\left(\frac{\theta}{2}\right)\ket{1} = \SQB{\sin(\frac{\theta}{2})}{-\ExpPhase{\phi}\cos(\frac{\theta}{2})}
        \]
\begin{enumerate}[label=\theenumi.\arabic*]
\item\Points{3} \[
\TensProd{\ket{\psi}}{\ket{\psi}} = \ket{\psi\psi} = \TwoQBlank{}
\]
\item\Points{3} \[
\TensProd{\ket{\psi'}}{\ket{\psi'}} = \ket{\psi'\psi'} = \TwoQBlank{}
\]
\item\label{prob:fourpointthree}\Points{3}
Please simplify your answer below as much as you can.

\[
\TwoSup{\psi\psi}{\psi'\psi'} = \RootTwo{}\TwoQBlank{}
\]
\item\Points{1} Is this mathematically equal to the Bell State \TwoSup{00}{11}? \Blank{}
\item\Points{3} If the states are mathematically equal, show your computation of that below.  If not, then describe how they are different.
\LeaveSpace{}
\item\Points{2} Suppose the state in Problem~\ref{prob:fourpointthree} is measured in the usual \TensProd{\PauliZ}{\PauliZ} basis.  Show the probability associated with each possible outcome below:
\begin{description}
    \item[$\ket{00}$] \Blank[3in]{}
    \item[$\ket{01}$] \Blank[3in]{}
    \item[$\ket{10}$] \Blank[3in]{}
    \item[$\ket{11}$] \Blank[3in]{}
\end{description}
\item\label{prob:matrix}\Points{3} Returning for a moment to single qubits, consider $\ket{\psi}$ and $\ket{\psi'}$ described at the beginning of this problem. In homework you showed that $\braket{\psi'|\psi}=0$, which makes them suitable as a measurement basis.

Below specify the matrix that maps
\begin{align*}
    \ket{0} &\mapsto \ket{\psi} \\
    \ket{1} &\mapsto \ket{\psi'}
\end{align*}
\[
T = \begin{pmatrix*}[r]
\Blank[10em]{} & \Blank[10em]{} \\[4em]
\Blank[10em]{} & \Blank[10em]{}
\end{pmatrix*}
\]
\item\Points{4} Recall that \Conj{T} is the conjugate transpose of matrix $T$.  To measure in the basis formed by $\psi$ and $\psi'$ we would perform the following steps in order, with reference to the matrix $T$ you defined in Problem~\ref{prob:matrix}:
\begin{itemize}
    \item Apply matrix (circle or indicate one) \hbox to 3em{\hss$T$\hss} or \hbox to 3em{\hss\Conj{T}\hss}  to a qubit's current state
    \item Measure in the computational basis
    \item Apply matrix (circle or indicate one) \hbox to 3em{\hss$T$\hss} or \hbox to 3em{\hss\Conj{T}\hss}  to the result of the measurement
\end{itemize}
\item\Points{3} Suppose Alice and Bob begin with the Bell state \TwoSup{00}{11}. Alice takes the left qubit and Bob takes the right. If Alice applied some unitary operator \NamedGate{U} to her qubit obtaining state $\QState{}$, under what conditions does Bob's state become mathematically $\QState{}$?
\LeaveSpace{}
\end{enumerate}
\item\label{prob:phasegate}\Points{25} Consider the \href{https://docs.quantum.ibm.com/api/qiskit/qiskit.circuit.library.PhaseGate}{Phase Gate}, shown here using parameter $x$ so as not to confuse it with the parameters of the universal \NamedGate{U} gate:
\[
P(x) = \SQBG{\relax}{1}{0}{0}{\ExpPhase{x}}, 0\leq x < 2\pi
\]
\begin{enumerate}[label=\theenumi.\arabic*]
  \item\Points{3} What is the \emph{simplest} (has the fewest non-zero parameters) \href{https://docs.quantum.ibm.com/api/qiskit/qiskit.circuit.library.U3Gate}{universal gate} (slide 13 in deck 070) formulation for $P(x)$?
  \[
  P(x) = \SQBG{\relax}{\cos{\left(\frac{\Blank[2em]}{2}\right)}}{-\ExpPhase{\Blank[2em]}\sin{\left(\frac{\Blank[2em]}{2}\right)}}{\ExpPhase{\Blank[2em]}\sin{\left(\frac{\Blank[2em]}{2}\right)}}{\ExpPhase{\Blank[2em]}\cos{\left(\frac{\Blank[2em]}{2}\right)}}
  \]
  Your answers above must conform to the specification of the \NamedGate{U} gate referenced above.
  \item\Points{2} Consider the state \[
  P(x)\ket{+} = P(x)\PPlus{}= \SQBG{\relax}{1}{0}{0}{\ExpPhase{x}}\PPlus{} =\RootTwo{}\SQB{1}{\ExpPhase{x}} = \TwoSupOp{1\QZero{}}{\ExpPhase{x}\QOne}{+}
  \]
  If we measure the above state in the computational (\PauliZ{}) basis, then fill in the probabilities associated with each possible outcome:
  \begin{itemize}
      \item \QZero{} \Blank{}\%
      \item \QOne{} \Blank{}\%
  \end{itemize}
  \item\label{prob:psi}\Points{4} With \Hadamard{} as our Hadamard gate, compute the state $\ket{\psi}$ below, taking note of the coefficient $\frac{1}{2}$ already provided.
  \[
  \ket{\psi} = \Hadamard(P(x)\ket{+}) = \frac{1}{2}\ \SQB{\Blank[8em]}{\Blank[8em]} = \SQB{\alpha}{\beta}
  \]
  Note that $\ket{\psi}$ is actually also a function of $x$, but the parameter is omitted to avoid clutter.  For the rest of this problem, know that state~$\ket{\psi}$ is dependent on~$x$.
  \item\Points{3} With all measurements actually performed in the computational basis, applying \Hadamard{} as above prepares the state $\ket{\psi}$ above as if it were measured in the \Blank[2em]{} basis, with the result of \begin{itemize}
      \item $\ket{\Blank[2em]}$ corresponding to $\ket{+}$ and 
      \item $\ket{\Blank[2em]}$  corresponding to $\ket{-}$.
   \end{itemize}
  \item\Points{2} For state $\ket{\psi}$ as defined in Problem~\ref{prob:psi}, we next explore the probability that \QZero{} will be observed. We must therefore compute $\Prob{\alpha}$. Complete the work below concerning $\alpha$ in $\ket{\psi}$ for your answer to part~\ref{prob:psi}.  
  \begin{itemize} 
  \item Each of your responses will be in terms of the parameter~$x$.
  \item Be sure to include the effects of the fraction $1/2$ shown outside the column vector in part~\ref{prob:psi}.
  
  
  \item You will likely need \href{https://en.wikipedia.org/wiki/Euler%27s_formula}{Euler's formula} to expand a polar value into its real and imaginary parts. 
  \end{itemize}
  
  For $\alpha$, the coefficient of $\ket{0}$ in Problem~\ref{prob:psi},
  \begin{itemize}
      \item its \emph{real} part is \Blank[20em]{}, and
      \item its \emph{imaginary} part is \Blank[20em]{}.
      \item The square of its real part is~\Blank[20em]{}, and
      \item the square of its imaginary part is~\Blank[20em]{}.
  \end{itemize}
  \item\label{prob:prob}\Points{2} The probability of observing $\ket{0}$ if we measure $\ket{\psi}$ in the computational basis is therefore:
  \[
  p(x) = \Prob{\alpha} = \Blank[25em]{}
  \]
  For full credit, ensure that your answer above is in the simplest form you can manage.
  \item\Points{2} Verify below that for all $x$, $0 \leq p(x) \leq 1$.  (If this verification fails, your answer for $p(x)$ is wrong.)
  \LeaveSpace{}
  \item\Points{3} Given 1,000 qubits, each in state~$\ket{\psi}$, how can we use the above result to estimate~$x$, the parameter to the phase gate?
  \LeaveSpace{}
  \item\Points{4} Suppose I have prepared 1,000 qubits, each in state $\ket{\psi}$ with a particular value of~$x$. After measuring those qubits, you observe:
  \begin{itemize}
      \item \QZero{} 468 times
      \item \QOne{} 532 times
  \end{itemize}
  What is the approximate value of $x$, with precision of six decimal digits after the decimal point, based on the above measurements?
  
  \Blank[15em]{} radians

  Note: you can verify your answer using the \texttt{PhaseGate} \href{https://piazza.com/class/m5mjpzl3hch1cf/post/76}{notebook} posted to piazza.
\end{enumerate}
\item \Points{15} Key Distribution
\begin{enumerate}[label=\theenumi.\arabic*]
    \item\Points{5} Consider an instance of Diffie--Hellman as taught in class (slide deck 110), where the base $g$ is 4483, and the modulus $p$ is 1746860020068409.  A calculator to help with this problem can be found \href{https://www.boxentriq.com/code-breaking/modular-exponentiation}{here}.
    \begin{itemize}
        \item Bob's secret is 468468
        \item Alice's secret is 864864
    \end{itemize}
    Their shared agreed-upon key is~\Blank[20em]{}.
    \item\Points{5} Consider the following table using the BB84 protocol taught in class (slide deck 110).

\begin{BBKey}
\begin{center}
    \begin{tabular}{c|cccccccc}
      \href{https://en.wikipedia.org/wiki/Quantum_key_distribution\#BB84_protocol:_Charles_H._Bennett_and_Gilles_Brassard_(1984)}{BB84} & 1 & 2 & 3 & 4 & 5 & 6 & 7 & 8 \\ \hline
    Alice's basis   & \STD{} &  \HDM{} & \STD{} & \HDM{} & \STD{} & \HDM{} & \STD{} & \HDM{}  \\
    Alice sends  & \BBUp{} & \BBNe{} & \BBUp{} & \BBNe{} & \BBUp{} & \BBNe{} & \BBUp{} & \BBNe{}  \\
    Bob's basis  & \STD{} & \STD{} & \STD{} & \STD{} & \HDM{} & \HDM{} & \HDM{} & \HDM{} 
    \end{tabular}
\end{center}
\end{BBKey}
Assuming Eve is not present, what shared key will Alice and Bob agree upon?

\Blank[20em]{}
\item\Points{5} In the table above, in which column(s) would Bob need to disagree with Alice to indicate that Eve might be present?  List all that are applicable.

\Blank[20em]{}

\end{enumerate}

\item\Points{15} Consider the quantum circuit shown below:
\begin{center}
\adjustbox{valign=t}{\begin{quantikz}
\lstick{\QZero{}}\slice{$\QState{0}$}&  \gate{H}\slice{$\QState{1}$} & \ctrl{1}\slice{$\QState{2}$} & \qw\slice{$\QState{3}$} & \qw \rstick{$a$}\\
\lstick{\QZero{}} &   \qw    &  \targ{} & \targ{} & \qw \rstick{$b$} \\
\lstick{\QZero{}} &   \gate{H}    & \qw & \ctrl{-1} & \qw \rstick{$c$} 
\end{quantikz}}%
\end{center}
We are familiar with the \NamedGate{CNOT} gate from class, but we have not seen that last gate in the circuit, which is an upside-down \NamedGate{CNOT}.  In terms of the diagram, the gate could be stated as follows:
\begin{itemize}
    \item If the bottom bit ($c$ in the circuit) is true, then the other bit ($b$ in the circuit) is flipped via an \PauliX{} gate. 
    \item Otherwise, the other bit is left alone.
\end{itemize}
Let's call this gate \NamedGate{UDCNOT}, and you will define this two-qubit gate based on how it should treat basis vectors using the above description:
\begin{center}
    \begin{tabular}{cc}
    $\ket{bc}$  & \NamedGate{UDCNOT}$\ket{bc}$ \\ \hline
    $\ket{00}$  & $\ket{00}$ \\
    $\ket{01}$  & $\ket{11}$ \\
    $\ket{10}$  & $\ket{10}$ \\
    $\ket{11}$  & $\ket{01}$ 
    \end{tabular}
\end{center}

\begin{enumerate}[label=\theenumi.\arabic*]
\item\Points{4} Expressed as a two-qubit unitary matrix
\[
\NamedGate{UDCNOT} = \begin{pmatrix*}[r]
\Blank[2em]{} & \Blank[2em]{} & \Blank[2em]{} & \Blank[2em]{}\\[0.5em]
\Blank[2em]{} & \Blank[2em]{} & \Blank[2em]{} & \Blank[2em]{}\\[0.5em]
\Blank[2em]{} & \Blank[2em]{} & \Blank[2em]{} & \Blank[2em]{}\\[0.5em]
\Blank[2em]{} & \Blank[2em]{} & \Blank[2em]{} & \Blank[2em]{}
\end{pmatrix*}
\]
\item\Points{3} The \NamedGate{SWAP} gate exchanges the position of two qubits and is described \href{https://docs.quantum.ibm.com/api/qiskit/qiskit.circuit.library.SwapGate}{here}. Prove that 
\[
\NamedGate{SWAP} \times \NamedGate{CNOT} \times \NamedGate{SWAP} = \NamedGate{UDCNOT}
\]
\LeaveSpace{}
\item\Points{4} Compute the states shown in the circuit diagram, filling in your answers below. Observe the common factor pulled out already, when it is present.  Fill in the blanks with ones and zeros.
\def\FS#1#2{%
\QState{#1} = #2\begin{pmatrix*}[r]
\Blank[2em]{} \\
\Blank[2em]{} \\
\Blank[2em]{} \\
\Blank[2em]{} \\
\Blank[2em]{} \\
\Blank[2em]{} \\
\Blank[2em]{} \\
\Blank[2em]{}
\end{pmatrix*}%
}
\[ \FS{0}{\relax}\ \ 
\FS{1}{\frac{1}{2}}\ \ 
\FS{2}{\frac{1}{2}}\ \ 
\FS{3}{\frac{1}{2}} \]

\item\Points{4} Below either prove $\QState{3}$ is entangled or show its decomposition.
\LeaveSpace{|}
\end{enumerate}


\end{enumerate}
\end{document}
