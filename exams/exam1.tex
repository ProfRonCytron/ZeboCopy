\documentclass[12pt]{article}
\usepackage{exercises/handwritten/handout}

\usepackage{amssymb,mathrsfs, amsmath,amsfonts}
\usepackage{mathtools}
\usepackage{graphicx}
\usepackage{enumitem}
\usepackage{braket}
%\graphicspath{ {./ps110-assets/} }

\input{quantummacros.tex}

\def\TwoQBlank{%
\DQB{\Blank}{\Blank}{\Blank}{\Blank}%
}
\begin{document}
\assignment{Exam 1}

\begin{center}\bf
Due 20 March 2025, 4 PM CDT in GradeScope\\
No late extensions on this assignment
\end{center}

\begin{enumerate}
    \item \begin{itemize}
    \item Your name? \Blank[3in]{}
    \item Your student ID? \Blank[3in]{}
\end{itemize}

\item \Points{10} Multiple choice

\item Consider the usual polarization basis, where \QZero{} is horizontal at 0 radians, \QOne{} is vertical at $\pi/2$ radians, so there are $\pi/2$ radians between the two basis vectors.

You are given a source of photons along with the promise that exactly one of the following is true:
\begin{itemize}
    \item All photons are polarized at $\pi/4$ radians (45 degrees), or
    \item All photons are unpolarized.
\end{itemize}
\begin{enumerate}[label=\theenumi.\arabic*]
    \item Consider the case where all photons are polarized at $\pi/4$ radians.  In the basis using \QZero{} and \QOne{}, what is the state of each photon? 
    
    \Blank[5in]{}
    \item Consider the case where all photons are unpolarized.   In the basis using \QZero{} and \QOne{}, what is the state of each photon? 

    \Blank[5in]{}
    \item Assuming you can receive as many photons from the source as you wish, describe an experiment to determine which of the two possible streams of photons you are receiving.  For full credit
    \begin{itemize}
        \item Describe the setup in sufficient detail.
        \item Analyze how certain you are of your determination after $n$ photons have been sent to you. 
    \end{itemize}

    \LeaveSpace[2in]{}
\end{enumerate}
\item Recall from homework that a point on the Bloch sphere
\[
\ket{\psi} = \cos(\frac{\theta}{2})\ket{0}
    + \ExpPhase{\phi}\sin(\frac{\theta}{2})\ket{1}
\]
has its antipodal point at
        \[
        \ket{\psi'}= \sin({\frac{\theta}{2})\ket{0}-\ExpPhase{\phi}\cos({\frac{\theta}{2}})\ket{1}}
        \]
\begin{enumerate}[label=\theenumi.\arabic*]
\item \[
\ket{\psi\psi} = \TwoQBlank{}
\]
\item \[
\ket{\psi'\psi'} = \TwoQBlank{}
\]
\item\label{prob:fourpointthree}
\[
\TwoSup{\psi\psi}{\psi'\psi'} = \RootTwo{}\TwoQBlank{}
\]
\item Is this mathematically equal to the Bell State \TwoSup{00}{11}? \Blank{}
\clearpage\item If the states are mathematically equal, show your computation of that below.  If not, then describe how they are different.
\LeaveSpace{}
\item Suppose the state in Problem~\ref{prob:fourpointthree} is measured in the \TensProd{\PauliZ}{\PauliZ} basis.  Describe the possible outcomes and their probabilities below:
\begin{description}
    \item[$\ket{00}$] \Blank[3in]{}
    \item[$\ket{01}$] \Blank[3in]{}
    \item[$\ket{10}$] \Blank[3in]{}
    \item[$\ket{11}$] \Blank[3in]{}
\end{description}
\item Returning for a moment to a single qubit, consider $\ket{\psi}$ and $\ket{\psi'}$ described at the beginning of this problem. In homework you showed that $\braket{\psi'|\psi}=0$, which makes them suitable as a measurement basis.

Below specify the matrix that maps
\begin{align*}
    \ket{0} &\mapsto \ket{\psi} \\
    \ket{1} &\mapsto \ket{\psi'}
\end{align*}
\[
T = \begin{pmatrix*}[r]
\Blank[10em]{} & \Blank[10em]{} \\[4em]
\Blank[10em]{} & \Blank[10em]{}
\end{pmatrix*}
\]
\item To measure in the basis formed by $\psi$ and $\psi'$ we would perform the following steps in order:
\begin{itemize}
    \item Apply \Blank{} to a qubit's current state
    \item Measure in the computational basis
    \item Apply \Blank{} to the result of the measurement
\end{itemize}
\item Suppose Alice and Bob begin with the Bell state \TwoSup{00}{11}. Alice takes the left qubit and Bob takes the right.
\end{enumerate}
\end{enumerate}
\end{document}
