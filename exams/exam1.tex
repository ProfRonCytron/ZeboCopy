\documentclass[12pt]{article}
\usepackage{exercises/handwritten/handout}

\usepackage{amssymb,mathrsfs, amsmath,amsfonts}
\usepackage{mathtools}
\usepackage{graphicx}
\usepackage{enumitem}
\usepackage{braket}
%\graphicspath{ {./ps110-assets/} }

\input{quantummacros.tex}

\def\TwoQBlank{%
\DQB{\Blank}{\Blank}{\Blank}{\Blank}%
}
\begin{document}
\assignment{Exam 1}

\begin{center}\bf
Due 20 March 2025, 4 PM CDT in GradeScope\\
No late extensions the exam!
\end{center}

\begin{enumerate}
    \item \begin{itemize}
    \item Your name? \Blank[3in]{}
    \item Your student ID? \Blank[3in]{}
\end{itemize}

\item \Points{10} Multiple choice and true/false

\item Consider the usual polarization basis, where \QZero{} is horizontal at 0 radians, \QOne{} is vertical at $\pi/2$ radians, so there are $\pi/2$ radians between the two basis vectors.

You are given a source of photons along with the promise that exactly one of the following is true:
\begin{itemize}
    \item All photons are polarized at $\pi/4$ radians (45 degrees), xor
    \item All photons are unpolarized.
\end{itemize}
\begin{enumerate}[label=\theenumi.\arabic*]
    \item Consider the case where all photons are polarized at $\pi/4$ radians.  In the basis using \QZero{} and \QOne{}, what is the state of each photon? 
    
    \Blank[5in]{}
    \item Consider the case where all photons are unpolarized.   In the basis using \QZero{} and \QOne{}, what is the state of each photon? 

    \Blank[5in]{}
    \item Assuming you can receive as many photons from the source as you wish, describe an experiment to determine which of the two possible streams of photons you are receiving.  For full credit
    \begin{itemize}
        \item Describe the setup in sufficient detail.
        \item Analyze how certain you are of your determination after $n$ photons have been sent to you. 
    \end{itemize}

    \LeaveSpace[2in]{}
\end{enumerate}
\item The questions below refer to the Elitzur--Vaidman bomb, specifically the \Quote{more careful approach} presented on slides 8 and following of slide deck 080.
\begin{enumerate}[label=\theenumi.\arabic*]
  \item What angle should be chosen for $\theta$ so that the bomb is
  \begin{itemize}
      \item certain to explode?\Blank{}
      \item predicted to explode with probability~0.5?\Blank{}
  \item guaranteed \emph{not} to explode?\Blank{}
  \end{itemize}
  \item Why will the algorithm fail to work if the bomb is guaranteed never to explode?\LeaveSpace{}
\end{enumerate}

\item Recall from homework that a point on the Bloch sphere
\[
\ket{\psi} = \cos(\frac{\theta}{2})\ket{0}
    + \ExpPhase{\phi}\sin(\frac{\theta}{2})\ket{1}
\]
has its antipodal point at
        \[
        \ket{\psi'}= \sin({\frac{\theta}{2})\ket{0}-\ExpPhase{\phi}\cos({\frac{\theta}{2}})\ket{1}}
        \]
\begin{enumerate}[label=\theenumi.\arabic*]
\item \[
\TensProd{\ket{\psi}}{\ket{\psi}} = \ket{\psi\psi} = \TwoQBlank{}
\]
\item \[
\TensProd{\ket{\psi'}}{\ket{\psi'}} = \ket{\psi'\psi'} = \TwoQBlank{}
\]
\item\label{prob:fourpointthree}
\[
\TwoSup{\psi\psi}{\psi'\psi'} = \RootTwo{}\TwoQBlank{}
\]
\item Is this mathematically equal to the Bell State \TwoSup{00}{11}? \Blank{}
\item If the states are mathematically equal, show your computation of that below.  If not, then describe how they are different.
\LeaveSpace{}
\item Suppose the state in Problem~\ref{prob:fourpointthree} is measured in the \TensProd{\PauliZ}{\PauliZ} basis.  Describe the possible outcomes and their probabilities below:
\begin{description}
    \item[$\ket{00}$] \Blank[3in]{}
    \item[$\ket{01}$] \Blank[3in]{}
    \item[$\ket{10}$] \Blank[3in]{}
    \item[$\ket{11}$] \Blank[3in]{}
\end{description}
\item\label{prob:matrix} Returning for a moment to single qubits, consider $\ket{\psi}$ and $\ket{\psi'}$ described at the beginning of this problem. In homework you showed that $\braket{\psi'|\psi}=0$, which makes them suitable as a measurement basis.

Below specify the matrix that maps
\begin{align*}
    \ket{0} &\mapsto \ket{\psi} \\
    \ket{1} &\mapsto \ket{\psi'}
\end{align*}
\[
T = \begin{pmatrix*}[r]
\Blank[10em]{} & \Blank[10em]{} \\[4em]
\Blank[10em]{} & \Blank[10em]{}
\end{pmatrix*}
\]
\item Recall that \Conj{T} is the conjugate transpose of matrix $T$.  To measure in the basis formed by $\psi$ and $\psi'$ we would perform the following steps in order, with reference to the matrix $T$ you defined in Problem~\ref{prob:matrix}:
\begin{itemize}
    \item Apply matrix (circle or indicate one) \hbox to 3em{\hss$T$\hss} or \hbox to 3em{\hss\Conj{T}\hss}  to a qubit's current state
    \item Measure in the computational basis
    \item Apply matrix (circle or indicate one) \hbox to 3em{\hss$T$\hss} or \hbox to 3em{\hss\Conj{T}\hss}  to the result of the measurement
\end{itemize}
\item Suppose Alice and Bob begin with the Bell state \TwoSup{00}{11}. Alice takes the left qubit and Bob takes the right. If Alice applied some unitary operator \NamedGate{U} to her qubit obtaining state $\QState{}$, under what conditions does Bob's state become mathematically $\QState{}$?
\LeaveSpace{}
\end{enumerate}
\item Consider the \href{https://docs.quantum.ibm.com/api/qiskit/qiskit.circuit.library.PhaseGate}{Phase Gate}, show here parameter $x$ so as not to confuse it with the parameters of the universal \NamedGate{U} gate:
\[
P(x) = \SQBG{\relax}{1}{0}{0}{\ExpPhase{x}}, 0\leq x < 2\pi
\]
\begin{enumerate}[label=\theenumi.\arabic*]
  \item What is the (simplest) universal gate formulation for $P(x)$?
  \[
  P(x) = \SQBG{\relax}{\cos{\frac{\Blank[2em]}{2}}}{-\ExpPhase{\Blank[2em]}\sin{\frac{\Blank[2em]}{2}}}{\ExpPhase{\Blank[2em]}\sin{\frac{\Blank[2em]}{2}}}{\ExpPhase{\Blank[2em]}\cos{\frac{\Blank[2em]}{2}}}
  \]
  Your answers must conform to the specification of the \NamedGate{U} gate.
  \item Consider the state \[
  P(x)\ket{+} = P(x)\PPlus{}= \SQBG{\relax}{1}{0}{0}{\ExpPhase{x}}\PPlus{} =\RootTwo{}\SQB{1}{\ExpPhase{x}} = \TwoSupOp{1\QZero{}}{\ExpPhase{x}\QOne}{+}
  \]
  If we measure the above state in the computational (\PauliZ{}) basis, then fill in the probabilities associated with each possible outcome:
  \begin{itemize}
      \item \QZero{} \Blank{}\%
      \item \QOne{} \Blank{}\%
  \end{itemize}
  \item\label{prob:psi} With \Hadamard{} as our Hadamard gate, compute the state $\ket{\psi}$ below, taking note of the coefficient $\frac{1}{2}$ already provided.
  \[
  \ket{\psi} = \Hadamard(P(x)) = \frac{1}{2}\ \SQB{\Blank[8em]}{\Blank[8em]}
  \]
  Note that $\ket{\psi}$ is actually also a function of $x$, but the parameter is omitted to avoid clutter.  For the rest of this problem, know that state~$\ket{\psi}$ is dependent on~$x$.
  \item With all measurements actually performed in the computational basis, applying \Hadamard{} as above prepares the state $\ket{\psi}$ above as if it were measured in the \Blank[2em]{} basis, with the result of \begin{itemize}
      \item $\ket{\Blank[2em]}$ corresponding to $\ket{+}$ and 
      \item $\ket{\Blank[2em]}$  corresponding to $\ket{-}$.
   \end{itemize}
  \item For state $\ket{\psi}$ as defined in Problem~\ref{prob:psi}, we next explore the probability that \QZero{} will be observed.  To answer this, complete the work below concerning the coefficient of \QZero{} in $\ket{\psi}$ for your answer to part~\ref{prob:psi}.  
  \begin{quote}
  You will likely need \href{https://en.wikipedia.org/wiki/Euler%27s_formula}{Euler's formula} to expand a polar value into its real and imaginary parts. 
  \end{quote} 
  
  For the coefficient of $\ket{0}$,
  \begin{itemize}
      \item its \emph{real} part is \Blank[10em]{}, and
      \item its \emph{imaginary} part is \Blank[10em]{}.
      \item The square of its real part is~\Blank[10em]{}, and
      \item the square of its imaginary part is~\Blank[10em]{}.
  \end{itemize}
  The probability of observing $\ket{0}$ if we measure $\psi$ in the computational basis is therefore:
  \[
  \Blank[20em]{}
  \]
  Kindly ensure that your answer above is in the simplest form you can manage.
  \item How can we use the above result to estimate~$x$, the parameter to the phase gate?
  \LeaveSpace{}
  \item Suppose I have prepared $\ket{\psi}$ with a particular value of~$x$, and after 1000 measurements of~$\ket{\psi}$, you see:
  \begin{itemize}
      \item \QZero{} 468 times
      \item \QOne{} 532 times
  \end{itemize}
  What is the approximate value of $x$, based on your measurements?
  
  \Blank{} radians
\end{enumerate}
\end{enumerate}
\end{document}
