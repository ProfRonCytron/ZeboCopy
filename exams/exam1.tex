\documentclass[12pt]{article}
\usepackage[margin=0.5in]{geometry} 
\usepackage{amsmath,amsthm,amssymb,amsfonts, enumitem, fancyhdr, color, comment, graphicx, environ}
\usepackage{course}
\usepackage{cse468-Spring24}
\usepackage{program}
\usepackage{fancybox}
\usepackage{adjustbox}
\usepackage{quantikz}
\usepackage{../bbkey}
\def\SquareOutline{%
\path (0,0) rectangle (1,1);%
}%
\def\Gate#1{\mbox{\textbf{#1}}}
\def\X{\Gate{X}}
\def\Y{\Gate{Y}}
\def\Z{\Gate{Z}}
\def\I{\Gate{I}}
\def\H{\Gate{H}}
\def\QZero{\ket{0}}
\def\QState#1{\ensuremath{\psi_{#1}}}

\def\Obox#1{\Ovalbox{\hbox to 1ex{\vrule width 0pt height 1ex\hss #1\hss}}}
\def\TFMarked#1#2{\ \stackbox[l][m]{\Obox{#1}~\textbf{true}\\\Obox{#2}~\textbf{false}}}
\def\TF{\TFMarked{\relax}{\relax}}
\def\TFH{\Obox{\relax}~\textbf{true}~~~\Obox{\relax}~\textbf{false}}
\def\exp#1{\ensuremath{e^{#1}}}
\newcommand{\Blank}[1][1in]{\mbox{\vrule width #1 depth 2pt}\vrule width 0pt height 2.0em}
\def\BlQb{\mbox{\ensuremath{\Blank[4em]\ket{0}+\Blank[4em]\ket{1}}}}
\newcommand{\Blanket}[1][3em]{%
\mbox{\ensuremath{|\,\Blank[#1]\,\rangle}}}

\def\Tall{\vrule width 0pt height 2em depth 0.5em}

\def\SQB#1#2{%
\ensuremath{%
\begin{pmatrix*}[r] #1 \\ #2\end{pmatrix*}}}

\def\SQBB{\SQB{\Blank[2em]}{\Blank[2em]}}

\def\DQB#1#2#3#4{%
\ensuremath{%
\begin{pmatrix*}[r] #1 \\ #2 \\ #3 \\ #4\end{pmatrix*}}}
\def\DQBB{\DQB{\Blank[2em]}{\Blank[2em]}{\Blank[2em]}{\Blank[2em]}}

\def\FactorProof{%
\begin{align*}
\SQB{a}{b} \otimes \SQB{c}{d} &= \DQBB{} \mbox{ (copy this from your \QState{2} answer top of page)}\\
a\cdot c &= \Blank[3em] \\
a\cdot d &= \Blank[3em] \\
b\cdot c &= \Blank[3em] \\
b\cdot d &= \Blank[3em]
\end{align*}
What is the contradiction, if any?
\LeaveSpace{2in}
}
\long\def\GenTable#1#2{%
\GenTableDiff{#1}{#2}{#1}{#2}}%
\long\def\GenTableDiff#1#2#3#4{%
\begin{center}
\begin{tabular}{ccc}
Alice & Bob & Possible Outcome? \\
\ensuremath{#1}    & \ensuremath{#3}  & \TFH \\[0.5em]
\ensuremath{#1}    & \ensuremath{#4}  & \TFH \\[0.5em]
\ensuremath{#2}    & \ensuremath{#3}  & \TFH \\[0.5em]
\ensuremath{#2}    & \ensuremath{#4}  & \TFH
\end{tabular}
\end{center}
}
\def\EmptyFour{%
\frac{1}{\Blank[2em]} \begin{pmatrix*}[r]
      \Blank[1.5em] \\
      \Blank[1.5em] \\
      \Blank[1.5em] \\
      \Blank[1.5em]
    \end{pmatrix*}
}
\def\EmptyFourByFour{%
\frac{1}{\Blank[2em]} \begin{pmatrix*}[r]
 \Blank[3em]{} & \Blank[3em]{} & \Blank[3em]{} & \Blank[3em]{} \\
 \Blank[3em]{} & \Blank[3em]{} & \Blank[3em]{} & \Blank[3em]{} \\
 \Blank[3em]{} & \Blank[3em]{} & \Blank[3em]{} & \Blank[3em]{} \\
 \Blank[3em]{} & \Blank[3em]{} & \Blank[3em]{} & \Blank[3em]{}\end{pmatrix*}}
\def\Bell{%
\BellTwo{00}{11}}
\def\BellTwo#1#2{%
\BellTwoSign{#1}{#2}{+}}
\def\BellTwoSign#1#2#3{%
\ensuremath{\frac{\ket{#1} #3 \ket{#2}}{\sqrt{2}}}}
\def\BellM{\BellTwoSign{01}{10}{-}}
\begin{document}

\begin{assignment}{Exam I}{6 March 2024}{In GradesScope by 22 March 2024, 9 AM CDT}

{\small {\large \fbox{READ THIS before starting!}}
This exam is open-book, open-notes, open-Internet, but you must do this
work on your own without contact or conversations with any person.
Because this exam is given in a distributed manner, no questions will be answered, and no clarifications will be given.  State your assumptions and count on us to be fair and flexible, especially if we have been unclear.


Your work must be legible.  Work that is
difficult to read will receive no credit.  There is a blank page at the end
if you want to show extra work there.  To assure no problems with GradeScope,
you should
print out this exam, complete it on paper, and scan it in with a full-page
scanner.  You can be more creative but if you do not properly register the
pages with GradeScope, credit will not be given.

There are ??? points available for this exam, but it will only be scored out of 100.  Extra points earned here will count toward your total exam grade, including Exam~II.

You must sign the pledge below for your exam to count.  Any cheating will
cause the students involved to receive an F for this course. Other unpleasant
actions
may be taken.

You must fill in your identifying information correctly.  You must upload this
completed exam to GradeScope properly by the due date and time.
}

\begin{center}\large
\begin{tabular}{|c|c|c|} \hline
\multicolumn{3}{|c|}{{\bf Print  clearly} the following information:}  \\ \hline
\multicolumn{3}{|l|}{Name (print clearly):\Tall{}\hbox to 2.5in{\hss}}  \\ \hline
\multicolumn{3}{|l|}{Student 6-digit ID (print {\it really} clearly):\Tall{}\hbox to 2.5in{\hss}} \\ \hline
\end{tabular}
\end{center}

{\bf Pledge:} On my honor, I have neither
given nor received any unauthorized aid on this exam.

Signed:  \Blank\Blank\Blank\Blank \\ \hbox to 5em{\hss}(Be 
sure you filled in your information in the box above!)
%
%
%
\clearpage
\begin{enumerate}
\item\Points{6} For the \textbf{true}/\textbf{false} questions below, indicate your response by marking an~\textbf{x} in the appropriate box, like this:
\TFMarked{\textbf{x}}{\relax} or \TFMarked{\relax}{\textbf{x}}.   

\textbf{Remember that $\equiv$ means equal up to a global phase.}
Boldface gates are the usual Pauli gates, and \Gate{H} is the Hadamard gate.
\begin{itemize}
    \item $\Gate{X}\ket{0}=\Gate{H}\ket{-}$~\TF{}
    \item For $k>0$, $\Gate{CNOT}^{k}\left(\frac{1}{\sqrt{2}}\left(\ket{00}+\ket{10}\right)\right)$ is an entangled state if and only if $k$ is odd.~\TF{}
    \item After a qubit has been measured, it cannot be subjected to subsequent unitary gates.~\TF{}
    \item All superpositions of two qubits are entangled states.~\TF{}
    \item $\Gate{X}\Gate{Y}\Gate{Z}\Gate{Z}\Gate{Y}\equiv\Gate{Y}\Gate{Z}$~\TF{}
     \item The state obtained after measuring the left qubit of \BellM{} is an entangled state.~\TF{}
    \item The $i$ present in the \Gate{Y} gate can be factored out as a global phase.~\TF{}
    \item The $i$ present in the \Gate{Y} gate can be factored out as a global phase for the controlled-\Gate{Y} gate that you developed in homework.~\TF{}
 
    \item At the end of teleportation, Alice tells Bob that she measured \ket{11} for her
    two qubits.  Which of the following gates will \textbf{by itself} recover Alice's original state? (circle only one gate)
    \begin{center}
    \mbox{}\\
    Gate \Gate{X}~~~~~~~Gate \Gate{Y}~~~~~~~Gate \Gate{Z}\\ \mbox{}
    \end{center}
   \item Suppose you are told that measurement of a given state \ket{\psi} is more likely to yield \ket{0} than \ket{1}.  What is the most precise statement you can make about \ket{\psi}'s location on the Bloch sphere?
    \LeaveSpace{0.5in}
\end{itemize}




\clearpage\item\label{prob:follow}\Points{20} Alice and Bob prepare to participate in one of the
many schemes studied this semester in which they begin with the 
Bell (also called an EPR) state:
\[ \ket{\psi_{1}} = \frac{\ket{00}+\ket{11}}{\sqrt{2}} = \frac{1}{\sqrt{2}} \begin{pmatrix} 1\\0\\0\\1\end{pmatrix}\]
When they separate, Alice takes the left qubit and Bob takes the right qubit.
They each then measure their qubit in a basis as described below.

\begin{enumerate}
\item Suppose they each measure $\ket{\psi_{1}}$ in the computational (\textbf{Z})
basis.
Which measurement outcomes are possible for this state?
\GenTable{0}{1}

\item
In class it was shown that
\[
\ket{\psi_{1}} = \Bell{} \longleftrightarrow \ket{\psi_{1}}= \frac{\ket{++}+\ket{--}}{\sqrt{2}}
\]
by showing the associated column vectors are mathematically equal:
\[
\ket{\psi_{1}} = \frac{1}{\sqrt{2}} \begin{pmatrix} 1\\0\\0\\1\end{pmatrix}
= \frac{1}{\sqrt{2}} \left[ \frac{1}{2}\begin{pmatrix}1\\1\\1\\1\end{pmatrix}
+  \frac{1}{2}\begin{pmatrix*}[r]1 \\-1\\-1\\1\end{pmatrix*}\right]
\]
The \texttt{Matlab} file is \texttt{bellstate2ways.txt}.

After separation, Alice and Bob each measure their qubit of $\ket{\psi_{1}}$ in the
\textbf{X} basis, each using a magical \textbf{X}-basis measuring device.  

Which measurement outcomes are possible?
\GenTable{+}{-}
\Continued{}

\item This time, due to miscommunication, Alice measures her qubit in the \textbf{X} basis
but Bob measures his in the computational basis.  
This is modeled by applying the unitary gate $\textbf{H}\otimes\textbf{I}$
to the state $\ket{\psi_{1}}=\frac{1}{\sqrt{2}}\begin{pmatrix}1\\0\\0\\1\end{pmatrix}$.
In other words, Alice applies \textbf{H} to her qubit while Bob does nothing
to his.  Fill in the matrix below:
\[ \textbf{H}\otimes\textbf{I} = \EmptyFourByFour{} \]
The state resulting from that operation is:

\[ \ket{\psi_{2}} = \left(\textbf{H}\otimes\textbf{I}\right)\ket{\psi_{1}} = \EmptyFour{} \]

Alice and Bob each measure their qubit of $\ket{\psi_{2}}$ in the computational
basis.  Which of the following measurement outcomes are possible?
\GenTable{0}{1}
After that measurement, they reapply
$\textbf{H}\otimes\textbf{I}$ to the measured outcome.

In which of the following states could the quantum system be now?
\GenTableDiff{+}{-}{0}{1}
\end{enumerate}
\clearpage\item\Points{10}
In class we saw that if Alice and Bob each take a qubit of \Bell{}, then if they each apply gate $U$ (limited to real values in class) to their respective qubits, the resulting state is still \Bell{}. This means that if they each apply the same gate $U$ to their separate but entangled qubits, they will measure the same results consistently. The \texttt{Matlab} script \texttt{bellstatepersists.txt}  from class is posted on piazza for this exam.

In this problem we look at another of Bell's states, namely
\[\ket{\psi_{1}} = \BellM{}\]
Alice takes the left qubit and Bob takes the right qubit, as usual.

Here, Alice and Bob expect to see opposite measurement results.  Specifically, basis state \ket{01} means that if Alice sees a~\Blank{} then Bob sees a \Blank{}.  Similarly, basis state \ket{10} means that if Alice sees a \Blank{} then Bob sees a~\Blank{}.

Referring to state~\ket{\psi_{1}} above, the minus sign between the two basis vectors is important for this problem, but doesn't change the expectations from measurement, because:
\begin{itemize}
    \item The amplitude on \ket{01} is \Blank[3em]{} and its magnitude squared is \Blank[3em]{}. 
    \item The amplitude on \ket{10} is \Blank[3em]{} and its magnitude squared is \Blank[3em]{}.
\item The amplitude on basis states \ket{00} and \ket{11} are each \Blank[3em]{}.
\end{itemize}

In \texttt{bellstatepersists.txt} we define the gate \[U = \begin{pmatrix*}[r] a & b \\ c & d\end{pmatrix*}\]
with $a, b, c, d$ all real-valued, to make matters
simple.  You can keep those values real for the rest of this problem as well.
\Continued{}
With \texttt{Matlab} as your (perhaps recent) faithful friend, complete the equation below to show what happens if Alice and Bob each apply $U$ (as defined above) to \ket{\psi_{1}} (be sure to represent the sign on \ket{10} properly for \ket{\psi_{1}}):
\[
\ket{\psi_{2}} = (U\otimes U)\ket{\psi_{1}} = \frac{1}{\sqrt{2}}\begin{pmatrix*}[r]
\Blank[3em]{} \\
\Blank[7em]{} \\
\Blank[7em]{} \\
\Blank[3em]{} 
\end{pmatrix*}
\]
%\Continued{}

Here is a useful fact (mention this in passing at your next party):  
\[ \mbox{If $U$ is unitary then }\left|\mbox{det}(U)\right| =  1 \]

That is, the determinant of a unitary matrix has magnitude~$1$.

Equipped, aware, and fascinated by that useful fact, prove that
\[ \ket{\psi_{2}} \equiv \BellM{}\]
In other words, Alice and Bob can apply $U$ to this other Bell state \ket{\psi_{1}} and they still have (up to a global phase) state \ket{\psi_{1}}.  

Your proof below:
\LeaveSpace{2in}
If Alice and Bob begin in state \ket{\psi_{1}}, does that entanglement follow them after any $U$ is applied to each qubit, so that their measurements consistently disagree?\Blank{}
\clearpage\item\Points{20}
We next explore the strength of commitment Alice and Bob bring to their
entanglement, as Eve enters the picture and seeks to become just as entangled
with Alice and Bob as they are with each other.

Alice and Bob share the left and right qubits respectively of $\ket{\psi_{1}}=\Bell{}$.
We have seen in class and in exam problem~\ref{prob:follow} that entanglement of those bits follows into the 
\textbf{X} basis.  If they both measure their qubits in the \textbf{X} basis,
the outcomes are just as correlated as they would be in the computational
basis.

As shown in class, Eve now joins them in such a way as to create the state
\[
\ket{\psi_{abe}} = \BellTwo{000}{111}
\]
where Alice, Bob, and Eve are assigned the left, middle, and right qubits,
respectively.

\begin{enumerate}
\item\Points{5} If $\ket{\psi_{abe}}$ is measured in the computational basis, what possible outcomes are there
for Alice, Bob, and Eve at this point?
\LeaveSpace{0.5in}
\item\Points{5} Suppose Alice and Bob prepare to measure $\ket{\psi_{abe}}$ in the \textbf{X} basis
by each applying the Hadamard gate to their qubits, but Eve does nothing to her qubit.  The names of the gates under tensor
product that model this are:
\[ \Blank{} \otimes \Blank{} \otimes \Blank{} \]
\item\Points{5} 
Applying the tensored gates to $\ket{\psi_{abe}}$ obtains the following 
state:
\[
\ket{\psi_{xxz}} = \frac{1}{\Blank[2em]}\begin{pmatrix*}
\Blank[2em] \\
\Blank[2em] \\
\Blank[2em] \\
\Blank[2em] \\
\Blank[2em] \\
\Blank[2em] \\
\Blank[2em] \\
\Blank[2em] 
\end{pmatrix*}
\]
\Continued{}
\item\Points{5} If $\ket{\psi_{xxz}}$ is measured now, just as it is, in
the computational basis, what outcomes are possible for Alice and Bob?
\GenTable{0}{1}
\item\Points{5} How does this differ from the situation when Eve was absent?
\LeaveSpace{1in}
\item\Points{5} How could you use this information to tell that Eve
was present from the beginning, in state $\ket{\psi_{abe}}$?
\LeaveSpace{1in}
\item\Points{5} Suppose that instead of doing nothing, Eve is sufficiently clever to impose
a Hadamard gate on her qubit just as Alice and Bob do that to their qubits.

Applying a Hadamard to each state of $\ket{\psi_{abe}}$ produces:
\[
\ket{\psi_{xxx}} = \frac{1}{\Blank[2em]}\begin{pmatrix*}
\Blank[2em] \\
\Blank[2em] \\
\Blank[2em] \\
\Blank[2em] \\
\Blank[2em] \\
\Blank[2em] \\
\Blank[2em] \\
\Blank[2em] 
\end{pmatrix*}
\]
\item\Points{5} What are the possible
outcomes for Alice's and Bob's measurements of $\ket{\psi_{xxx}}$ in
the computational basis?
\GenTable{0}{1}

\item\Points{5} How can Alice and Bob catch Eve's presence in
$\ket{\psi_{abe}}$ now?
\end{enumerate}




\clearpage\item\Points{15} Consider the standard circuit we have used for
entanglement:
\begin{center}   
    \adjustbox{valign=t,scale=1.5}{\begin{quantikz}
\qw &\gate{H} & \ctrl{1} & \qw \\
\qw &\qw & \targ{} &   \qw
\end{quantikz}}\end{center}
\begin{enumerate}
\item\Points{5}
Below, specify the single, two-qubit matrix that has the same effect as
the circuit above.  Please factor out a coefficient where possible, in
the space provided,  so that
the matrix entries are mostly $0$ or $1$.

{\small
\[
U = \EmptyFourByFour{} \]}
If you have given the right answer, then $U\ket{00}=\frac{\ket{00}+\ket{11}}{\sqrt{2}}$.  Complete the results of $U$ on the other basis vectors similarly, below:

\item\Points{2} $U\ket{01}=\frac{\Blank{}}{\sqrt{2}}$
\item\Points{2} $U\ket{10}=\frac{\Blank{}}{\sqrt{2}}$
\item\Points{2} $U\ket{11}=\frac{\Blank{}}{\sqrt{2}}$
\item\Points{4} If $U$ can be tensor-factored, show the resulting $2\times 2$ matrices below. Otherwise, prove that it cannot be tensor factored.
\end{enumerate}





\clearpage\item\Points{10} Bloch Sphere Orienteering:  all answers are in the standard basis.  Write each amplitude in the provided blank space, fully above the dark line.  For example, to express $\frac{i}{\sqrt{3}}$ you would write \Blank{}\hbox to 0pt{\hskip -4em\raisebox{4pt}{$i/\sqrt{3}$}\hss}.

Each response is worth 1 point.



\begin{itemize}
    \item We begin at the North pole of the Bloch sphere.
    \item We are in state \BlQb{}.
    \item We rotate about the \Z{} axis $\pi/4$ radians.  We are now at \BlQb{}.
    \item \vline height 2em width0pt We begin again at the North pole.
    \item We experience a \Y{} gate.  We are now at \BlQb{}.
    \item We then experience a \Z{} gate.  We are now at state \BlQb{}.
    \item We finally experience an \H{} gate.  We are now at state \BlQb{}.
\end{itemize}



\clearpage\item\Points{5 extra credit}  If there were a unitary gate $U$ that could conjugate the arbitrary relative phase~$\phi$ of
a quantum state, then
\[  U\begin{pmatrix*}[r] \cos{\theta} \\ e^{i\phi}\sin{\theta}\end{pmatrix*} = \begin{pmatrix*}[r] \cos{\theta} \\ e^{-i\phi}\sin{\theta}\end{pmatrix*} \]
There is such a gate when $\phi=-\pi$ and that is the \Blank{} gate.  But Arbitrary phase means we do not the value of $\phi$.  

If such a gate $U$ existed, then
\begin{itemize}
    \item $U\SQB{1}{0} = \raisebox{-1.5em}{\Blank{}}$
    \item $U\SQB{0}{1} = \raisebox{-1.5em}{\Blank{}}$
\end{itemize}

Below, provide gate $U$ or prove that no such
gate $U$ can exist without knowing $\phi$.

\end{enumerate}
\end{assignment}
\Bpage{}

\end{document}
