\documentclass[12pt]{article}
\usepackage{exercises/handwritten/handout}

\usepackage{amssymb,mathrsfs, amsmath,amsfonts}
\usepackage{mathtools}
\usepackage{graphicx}
\usepackage{enumitem}
\usepackage{braket}
\usepackage{bbkey}
%\graphicspath{ {./ps110-assets/} }

\input{quantummacros.tex}

\def\TwoQBlank{%
\DQB{\Blank[8em]}{\Blank[8em]}{\Blank[8em]}{\Blank[8em]}%
}
\def\TwoU{%
\SQBG{\relax}{\Blank[4em]{}}{\Blank[4em]{}}{\Blank[4em]{}}{\Blank[4em]{}}}
\begin{document}
\assignment{Exam 2}

\begin{center}\bf
Due 4 May 2025, 11:59 PM CDT in GradeScope\\
No late extensions for the exam!
\end{center}
\begin{quote}
    This exam has been prepared so that you can write answers in provided spaces.  If you choose to write your answers into some other document, your answers must be formatted in the same manner as the answer spaces provided here. For example, if a column vector with blanks is provided here, your answer must be a column vector.

    Where you see free-form responses, such as for your name and ID below, you can supply that information however you wish.

    Failure to follow these instructions will cause you to receive no credit for your work.

    If your work is not shown, or the work shown is unclear, hard to read, or messy, you will receive no partial credit.   Where you are asked to show work, your work must be clear, neat, and easy to read to receive any credit.

    {\bf Academic integrity} You may not discuss the exam with anybody except the instructor. You are free to use any resources available to support your learning and your computations, but you may not seek answers to the questions themselves.
\end{quote}

\begin{enumerate}
    \item\Points{0} \begin{itemize}
    \item Your name? \Blank[3in]{}
    \item Your student ID? \Blank[3in]{}
\end{itemize}

\item \Points{10} To make grading easier please use the format below, or if you write your answers out on your own, please imitate the format of the response section.  The notation $\NamedGate{U}^{n}$ denotes $n$ consecutive applications of \NamedGate{U} on a single qubit, \emph{not} an $n$-way tensor product.

\textbf{Questions (do not respond here---see response section below)}
\begin{enumerate}[label=\theenumi.\arabic*]
   \item\label{tf:a} The university course evaluation for this course is worth one of the five points for
participation in this class. By turning in this exam, I acknowledge that the due
date for receiving this participation point is Monday May 5, 9 AM.  I understand I will not receive participation credit for my evaluation if it submitted after that time.
   \item\label{tf:b} Each of the Pauli gates (\PauliX,\PauliY,\PauliZ,\Hadamard) is its own inverse.
   \item\label{tf:c} The Control-\PauliZ{} gate cannot produce an entangled state, unless it is presented with an already entangled state.
   \item\label{tf:d} If the state \TwoSup{00}{01} is presented to gates \TensProd{\Hadamard}{\PauliX}, the tensor product of those gates must be formed to compute the correct resulting state.
   \item\label{tf:e} The only valid bases in terms of measurement on the Bloch sphere are the \PauliZ{} and \PauliX{} bases.
   \item\label{tf:f} What state is $\PauliX(\Hadamard^{n}(\QZero))$ if $n$ is even?
      \item\label{tf:g} What state is $\PauliX(\Hadamard^{n}(\QZero))$ if $n$ is odd?
    \item\label{tf:h} In teleportation, what gates must Bob apply to his qubit to recover Alice's state if she tells him both of her qubits measured \QOne{}?
    \item\label{tf:i} Due to the no cloning theorem, it is not possible to make a copy of a state that is either \QZero{} or \QOne{}.
    \item\label{tf:j} What is the eigenvalue associated with (eigen)state $\ket{01101010010111110000}$ in the computational basis?
\end{enumerate}

\def\BlankPrompt#1{\ref{tf:#1}~\Blank{}\hskip 2em plus 1em 
 minus 1em}
\def\TFPrompt#1{\ref{tf:#1}~\TF{}\hskip 2em plus 1em 
 minus 1em}
\textbf{Responses}

\TFPrompt{a}\TFPrompt{b}\TFPrompt{c}\TFPrompt{d}\TFPrompt{e}

\BlankPrompt{f}\BlankPrompt{g}\BlankPrompt{h}\TFPrompt{i}\BlankPrompt{j}

\item\Points{10}  We have studied the 3-qubit correction scheme that can correct a single \PauliX{} error on at most one of the 3 entangled qubits.  To achieve this, we conducted \emph{syndrome} measurements on two other qubits.

The qubits receiving the symdrome values are expected to be in state \QZero{} or \QOne{}, and the combined value of the two sydnrome qubits indicates what, if any, corrective action must be taken on the 3 entangled qubits.

In this section we explore the life of the syndrome qubits after they have been used once.

Ideally we would like to reset those syndrome qubits to $\ket{00}$ so that they could be used for another round of error correction.

\begin{enumerate}[label=\theenumi.\arabic*]
  \item Let's see if a unitary operator could do the job.
  Below, complete the matrix so that it sends $\QZero{}\mapsto\QZero{}$ and $\QOne{}\mapsto\QZero{}$.
  \[
  \TwoU{}
  \]
  \item Prove or disprove that the above matrix is unitary:
  \LeaveSpace{}
  \item Have them figure out how to do this for reals
    
\end{enumerate}

\item\Points{15} Recall the Mermin--Peres game. Recall that the following states are generated in support of Alice and Bob [ Slide deck 140, slide 2 ]:
\begin{itemize}
    \item $\ket{a_{1}b_{1}} = \TwoSup{00}{11}$
    \item $\ket{a_{2}b_{2}} = \TwoSup{00}{11}$
\end{itemize}
where Alice receives $\ket{a_{1}}$ and $\ket{a_{2}}$ and Bob receives the other two qubits.

In class we studied how the initial state of these qubits leads to guranteed success in the Mermin--Peres game, for any row or column assigned to Alice or Bob.

Now suppose that, subsequent to qubit generation but prior to playing the game, the above qubits collapse out of superposition, both times favoring $\ket{00}$.  They thus obtain  \[
\ket{a_1}=\ket{a_2}=\ket{b_1}=\ket{b_2}=\QZero{}
\]

Note well:
\begin{itemize}
\item For the rest of this problem, all parts, the qubits start in the above states, each \QZero{}.
\item In playing the Mermin--Peres game, a measurement may be performed on a qubit, which can change its state.  The game is played as usual, as taught in class.
\item However, at the beginning of the next part to this problem, the qubits are back to their \QZero{} state, as if the collapse had just happened.
\end{itemize}

\begin{enumerate}[label=\theenumi.\arabic*]
    \item What values can Alice report in the squares for row 1, from left to right?
{\def\T#1#2#3{%
$#1$ & $#2$ & $#3$ & {\TF}
}
\begin{center}
    \begin{tabular}{cccc}
    Left & Center & Right & Possible \\
    Square & Square & Square & Outcome? \\ \hline
    \T{+1}{+1}{+1} \\[1.2em]
    \T{+1}{+1}{-1} \\[1.2em]
    \T{+1}{-1}{+1} \\[1.2em]
    \T{+1}{-1}{-1} \\[1.2em]
    \T{-1}{+1}{+1} \\[1.2em]
    \T{-1}{+1}{-1} \\[1.2em]
    \T{-1}{-1}{+1} \\[1.2em]
    \T{-1}{-1}{-1}
    \end{tabular}
    \end{center}}
    \clearpage\item What values can Bob report for column 1, from top to bottom?
    {\def\T#1#2#3{%
$#1$ & $#2$ & $#3$ & {\TF}
}
\begin{center}
    \begin{tabular}{cccc}
    Top & Middle & Bottom & Possible \\
    Square & Square & Square & Outcome?\\ \hline
    \T{+1}{+1}{+1} \\[1.2em]
    \T{+1}{+1}{-1} \\[1.2em]
    \T{+1}{-1}{+1} \\[1.2em]
    \T{+1}{-1}{-1} \\[1.2em]
    \T{-1}{+1}{+1} \\[1.2em]
    \T{-1}{+1}{-1} \\[1.2em]
    \T{-1}{-1}{+1} \\[1.2em]
    \T{-1}{-1}{-1}
    \end{tabular}
    \end{center}}
    \clearpage\item With the qubits starting again at \QZero{}, for which rows and which columns are Alice and Bob both \emph{certain} to report the expected, correct product values ($+1$ for Alice's row, $-1$ for Bob's column)?  
    
    This is one of those problems where you could grind out answers, but with some thought, it is easy to answer.
   
    Answer this by choosing \textbf{true} or \textbf{false} for each combination below:
    {%
    \def\B#1{\fbox{\hbox to 3em{\vrule width 0pt height 2em\hss #1\hss}}}
    \begin{center}
        \begin{tabular}{ccc}
                 When Alice & When Bob &  Do Alice and Bob \\
                is assigned  & is assigned & report the correct product \\
                 row number & column number& certainly? \\ \hline \\
                 1 & 1 & \TF{} \\[1.3em]
                 1 & 2 & \TF{} \\[1.3em]
                 1 & 3 & \TF{} \\[1.3em]
                 2 & 1 & \TF{} \\[1.3em]
                 2 & 2 & \TF{} \\[1.3em]
                 2 & 3 & \TF{} \\[1.3em]
                 3 & 1 & \TF{} \\[1.3em]
                 3 & 2 & \TF{} \\[1.3em]
                 3 & 3 & \TF{}

        \end{tabular}
    \end{center}}
    \clearpage\item Each qubit starts again at \QZero{}.  Below is a depiction of the Mermin--Peres square with its measurement operators.  For each square, circle it if, when Alice and Bob have that square in common, they are \emph{guaranteed} to report the same value.  Again, with some thought, this is easily answered, but you can also grind through all nine squares.
    \[\includegraphics[scale=0.8]{mp}\]
        \item Each qubit starts again at \QZero{}. Below is a another depiction of the Mermin--Peres square with its measurement operators.  For each square, circle it if, when Alice and Bob have that square in common, they \emph{might} (probability greater than zero) report the same value.  Again, with some thought, this is easily answered, but you can also grind through all nine squares.
    \[\includegraphics[scale=0.8]{mp}\]
    \end{enumerate}

\item\Points{10}  In this problem you will consider solving Simon's problem using Grover's algorithm.  In this problem, the domain values for the Simon function are $n$~qubits.  You are given the following components:
\begin{itemize}
    \item A box $U_S$ that accepts $n$ qubits as its input and computes (in a quantum way using unitary gates) $f(x)$ on its $n-1$ outputs, where $f(x)$ behaves according to the properties of Simon's problem:
    \[
    f(x) = f(x\oplus s)
    \]
    where $s$ is the secret value hidden in box $U_S$.
    \item A box $C$ that accepts two $k$-qubit values, $v$ and $w$, and outputs \QOne{} if the qubits of $v$ are identical to those in $w$, and \QZero{} if they are not.  In other words $C(v,w)=1$ if and only if
    \[
    \forall\ i\ \in [0,k)\ v_{i} = w_{i}
    \]
    
    You can pick values you need for $k$.
    \begin{enumerate}[label=\theenumi.\arabic*]
        \item Your first task is to design an oracle that accepts only $n$ qubits (so just one possible input to function $f(x)$) and produces a $1$ if and only if the $n$ qubits are the secret $s$.  You can use the components described above in your oracle, any quantum components studied this semester, and as many ancillas as you need.

        Draw a \emph{clear} picture of your oracle labeling the boxes and showing how many qubits each line in and out requires.
        \LeaveSpace[4in]{}
        \item If the quantum counting algorithm were applied to your oracle, for how many inputs would it say the oracle produces $1$?\Blank{}
        \item Please take a look at slide deck 260, slide number 7.  The slide depicts an angle between states $\ket{s}$ and $\ket{r}$ as taught for Grover's algorithm (this $\ket{s}$ is different from our Simon secret $s$).  For an $n$-qubit secret, what is the angle between $\ket{s}$ and $\ket{r}$?\Blank{}~radians
        
        
    \end{enumerate}
\end{itemize}

\item\Points{5}  It's been taught in class, and widely published, that the Quantum Fourier Transformation can be implemented as the inverse of the Phase Estimation Algorithm.

Based on this, Pat created a QFT quantum circuit by placing the Phase Estimation circuit components in the reverse order.  But the circuit did not compute the correct answers.

What did Pat do wrong?
\LeaveSpace{}

\item\Points{10} In this problem you consider the Deutsch--Jozsa algorithm applied to an $n$-qubit function.  For each of the scenarios below, characterize the oracle's function 
\[ f(x): \BinAlph{n}\mapsto \Set{0,1}
\]
as precisely as possible, based on the most likely nature of the function.  Throughout, assume a robust, noiseless quantum computer: in other words, assume that these results come
from perfect emulation of a quantum computer.
\begin{itemize}
    \item In 1000 runs, each run measured $\ket{\TensSupProd{0}{n}}$.
    \[f(x) = \Blank[10em]{} \]
    \item In 1000 runs, each run measured $\ket{1\TensSupProd{0}{n-1}}$.
     \[f(x) = \Blank[10em]{} \]
     \item In 1000 runs, no measurement of $\ket{\TensSupProd{0}{n}}$ was observed.
      \[f(x) = \Blank[10em]{} \]
      \item In 1000 runs, half of them measured $\ket{\TensSupProd{0}{n}}$ and half of them did not.
       \[f(x) = \Blank[10em]{} \]
\end{itemize}

\item\Points{10}  In class we experimented with instances of the Bernstein--Vazirani problem with the goal of understanding where noise is present.  Based on the results presented in class, and the published solutions submitted by students, answer the following questions.

Throughout this problem, the following notation is used:
\begin{quote}
\begin{description}
    \item[BV] abbreviates Bernstein--Vazirani
    \item[$s$] is the secret used in a BV instance
    \item[$n$] is the number of qubits in $s$
\end{description}\end{quote}

\begin{enumerate}[label=\theenumi.\arabic*]
    \item Given a random $s$, we observed that noise increased as $n$ increased. \TF{}
    \item If we make use of qubits outside those needed for BV, we observed that the noise grew as the number of unrelated qubits increased. \TF{}
    \item The probability of measuring $s$ as $n$ increased from $1$ to $50$ dropped continuously and linearly.\TF{}
    \item The probability of measuring values other than $s$ increased with the number of \NamedGate{CNOT} gates present in an instance. \TF{}
    \item The specific contents of $s$, in terms of its \texttt{1}s and \texttt{0}s, was unrelated to the errors we observed.\TF{}
    \item\label{prob:trans} IBMs transpilation of a BV instance apparently allocates qubits on the target quantum computer to logoical qubits in a quantum circuit at random, paying no special attention to the circuit.\TF{}
    \item Elaborate on why you picked the answer you did for question~\ref{prob:trans}.
    \LeaveSpace{}
    \item In what ways is the topology of a specific IBM quantum computer related to the errors we saw in a BV instance?  For reference, read about the topological diagram or coupling map \href{https://quantum.cloud.ibm.com/docs/en/guides/qpu-information}{here}.
    \LeaveSpace{}
\end{enumerate}

\item\Points{10} The IBM Quantum Computer offers a two-qubit instruction \href{https://docs.quantum.ibm.com/api/qiskit/qiskit.circuit.library.ECRGate}{\NamedGate{ECR}} (Echoed Cross-Resonance) that is defined as follows (there are other definitions, but for the exam, use this one):
\[
\NamedGate{ECR} = \RootTwo\left({\TensProd{\Identity}{\PauliX} - \TensProd{\PauliX}{\PauliY}}\right) = \RootTwo{}\begin{pmatrix*}[r]
0 & 1 & 0 & \NiceI{} \\
1 & 0 & -\NiceI{} & 0 \\
0 & \NiceI{} & 0 & 1 \\
-\NiceI{} & 0 & 1 & 0
\end{pmatrix*}
\]
This is a primitive instruction that IBM's quantum computer can execute on two qubits to create entanglement.  Your goal here is to use this instruction to create the (phase-free) Bell state \TwoSup{00}{11}, beginning with state $\ket{0+}$.  You are urged to use \texttt{Matlab} for this problem!
\begin{itemize}
    \item Compute and complete the following:
    \[ \NamedGate{ECR} \ket{0+} = \ \Blank[4em]{}\TwoQBlank \]
    \item From this state, describe how to obtain the state \[
    \RootTwo{}\DQB{1}{0}{0}{-\NiceI}
    \]
    using named gates we have studied (for example, \Hadamard{}, \PauliX{}, \PauliZ{}, \PauliY{}), applied only to the second qubit.
    \LeaveSpace{}
    \item Finally, let's rotate away the phase on $\ket{11}$.  What gate will do this for you, applied only to the second qubit?  Fill in the contents of the gate below:
    \[ \TwoU{} \]
    \LeaveSpace{}
    \item Below show the math that uses your gate to obtain \TwoSup{00}{11}:
    \LeaveSpace{}
\end{itemize}

\end{enumerate}
\end{document}
