\documentclass[12pt]{article}
\usepackage{exercises/handwritten/handout}

\usepackage{amssymb,mathrsfs, amsmath,amsfonts}
\usepackage{mathtools}
\usepackage{graphicx}
\usepackage{enumitem}
\usepackage{braket}
\usepackage{bbkey}
%\graphicspath{ {./ps110-assets/} }

\input{quantummacros.tex}

\def\TwoQBlank{%
\DQB{\Blank[8em]}{\Blank[8em]}{\Blank[8em]}{\Blank[8em]}%
}
\def\TwoU{%
\SQBG{\relax}{\Blank[4em]{}}{\Blank[4em]{}}{\Blank[4em]{}}{\Blank[4em]{}}}
\begin{document}
\assignment{Exam 2}

\begin{center}\bf
Due 4 May 2025, 11:59 PM CDT in GradeScope\\
No late extensions for the exam!
\end{center}
\begin{quote}
    This exam has been prepared so that you can write answers in provided spaces.  If you choose to write your answers into some other document, your answers must be formatted in the same manner as the answer spaces provided here. For example, if a column vector with blanks is provided here, your answer must be a column vector.

    Where you see free-form responses, such as for your name and ID below, you can supply that information however you wish.

    Failure to follow these instructions will cause you to receive no credit for your work.

    If your work is not shown, or the work shown is unclear, hard to read, or messy, you will receive no partial credit.   Where you are asked to show work, your work must be clear, neat, and easy to read to receive any credit.

    {\bf Academic integrity} You may not discuss the exam with anybody except the instructor. You are free to use any resources available to support your learning and your computations, but you may not seek answers to the questions themselves.
\end{quote}

\begin{enumerate}
    \item\Points{0} \begin{itemize}
    \item Your name? \Blank[3in]{}
    \item Your student ID? \Blank[3in]{}
\end{itemize}

\item \Points{10} To make grading easier please use the format below, or if you write your answers out on your own, please imitate the format of the response section.  The notation $\NamedGate{U}^{n}$ denotes $n$ consecutive applications of \NamedGate{U} on a single qubit, \emph{not} an $n$-way tensor product.

\textbf{Questions (do not respond here---see response section below)}
\begin{enumerate}[label=\theenumi.\arabic*]
   \item\label{tf:a} All states in the $x-z$ plane on the Bloch sphere can be described without using relative phase.
   \item\label{tf:b} Each of the Pauli gates (\PauliX,\PauliY,\PauliZ,\Hadamard) is its own inverse.
   \item\label{tf:c} The Control-\PauliZ{} gate cannot produce an entangled state, unless it is presented with an already entangled state.
   \item\label{tf:d} If the state \TwoSup{00}{01} is presented to gates \TensProd{\Hadamard}{\PauliX}, the tensor product of those gates must be formed to compute the correct resulting state.
   \item\label{tf:e} The only valid bases in terms of measurement on the Bloch sphere are the \PauliZ{} and \PauliX{} bases.
   \item\label{tf:f} What state is $\PauliX(\Hadamard^{n}(\QZero))$ if $n$ is even?
      \item\label{tf:g} What state is $\PauliX(\Hadamard^{n}(\QZero))$ if $n$ is odd?
    \item\label{tf:h} In teleportation, what gates must Bob apply to his qubit to recover Alice's state if she tells him both of her qubits measured \QOne{}?
    \item\label{tf:i} Due to the no cloning theorem, it is not possible to make a copy of a state that is either \QZero{} or \QOne{}.
    \item\label{tf:j} What is the eigenvalue associated with (eigen)state $\ket{01101010010111110000}$ in the computational basis?
\end{enumerate}

\def\BlankPrompt#1{\ref{tf:#1}~\Blank{}\hskip 2em plus 1em 
 minus 1em}
\def\TFPrompt#1{\ref{tf:#1}~\TF{}\hskip 2em plus 1em 
 minus 1em}
\textbf{Responses}

\TFPrompt{a}\TFPrompt{b}\TFPrompt{c}\TFPrompt{d}\TFPrompt{e}

\BlankPrompt{f}\BlankPrompt{g}\BlankPrompt{h}\TFPrompt{i}\BlankPrompt{j}

\item\Points{5}  We have studied the 3-qubit correction scheme that can correct a single \PauliX{} error on at most one of the 3 entangled qubits.  To achieve this, we conducted \emph{syndrome} measurements on two other qubits.

The qubits receiving the symdrome values are expected to be in state \QZero{} or \QOne{}, and the combined value of the two sydnrome qubits indicates what, if any, corrective action must be taken on the 3 entangled qubits.

In this section we explore the life of the syndrome qubits after they have been used once.

Ideally we would like to reset those syndrome qubits to $\ket{00}$ so that they could be used for another round of error correction.

\begin{enumerate}[label=\theenumi.\arabic*]
  \item Let's see if a unitary operator could do the job.
  Below, complete the matrix so that it sends $\QZero{}\mapsto\QZero{}$ and $\QOne{}\mapsto\QZero{}$.
  \[
  \TwoU{}
  \]
  \item Prove or disprove that the above matrix is unitary:
  \LeaveSpace{}
  \item Have them figure out how to do this for reals
    
\end{enumerate}

\item Recall the Mermin--Peres game. Recall that the following states are generated in support of Alice and Bob [ Slide deck 140, slide 2 ]:
\begin{itemize}
    \item $\ket{a_{1}b_{1}} = \TwoSup{00}{11}$
    \item $\ket{a_{2}b_{2}} = \TwoSup{00}{11}$
\end{itemize}
where Alice receives $\ket{a_{1}}$ and $\ket{a_{2}}$ and Bob receives the other two qubits.

In class we studied how the initial state of these qubits leads to guranteed success in the Mermin--Peres game, for any row or column assigned to Alice or Bob.

Now suppose that subsequent to qubit generation but prior to playing the game, the above qubits collapse out of superposition, obtaining \[
\ket{a_1}=\ket{a_2}=\ket{b_1}=\ket{b_2}=\QZero{}
\]
With the qubits now in those states, answer the following questions:
\begin{enumerate}[label=\theenumi.\arabic*]
    \item For which rows and which columns will Alice and Bob report the expected, correct product values ($+1$ for Alice's column, $-1$ for Bob's row)?
   
    Answer this by placing a check mark in the squares below that reflect the row and column assigned to Alice and Bob, respectively.
    {%
    \def\B#1{\fbox{\hbox to 3em{\vrule width 0pt height 2em\hss #1\hss}}}
    \begin{center}
        \begin{tabular}{ccc}
                 Alice's assigned row & Bob's assigned column &  Correct product? \\
                 1 & 1 & \B{} \\
                 1 & 2 & \B{} \\
                 1 & 3 & \B{} \\
                 2 & 1 & \B{} \\
                 2 & 2 & \B{} \\
                 2 & 3 & \B{} \\
                 3 & 1 & \B{} \\
                 3 & 2 & \B{} \\
                 3 & 3 & \B{} \\

        \end{tabular}
    \end{center}}
    \item Below is a depiction of the Mermin--Peres square with its measurement operators.  For each square, circle it iff when Alice and Bob have that square in common, they are guaranteed to report the same value.
    \[\includegraphics[scale=0.8]{mp}\]
    \end{enumerate}


\end{enumerate}
\end{document}
