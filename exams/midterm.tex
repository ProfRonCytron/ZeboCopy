\documentclass[12pt]{article}
\usepackage[margin=0.5in]{geometry} 
\usepackage{amsmath,amsthm,amssymb,amsfonts, enumitem, fancyhdr, color, comment, graphicx, environ}
\usepackage[braket,qm]{qcircuit}
\usepackage{course}
\usepackage{cse468-Spring22}
\usepackage{program}
\usepackage{fancybox}


\def\TF{\Ovalbox{\hbox to 1ex{\vrule width 0pt height 1ex\hss}}~\textbf{true}
\def\exp#1{\ensuremath{e^{#1}}}
\newcommand{\Blank}{\mbox{\vrule width 1in depth 2pt}\vrule width 0pt height 2.0em}
\def\Tall{\vrule width 0pt height 2em depth 0.5em}
\begin{document}

\begin{assignment}{Exam I}{9 March 2022}{End of class}

{\small {\large \fbox{READ THIS before starting!}}
This exam is open-book, open-notes, open-Internet, but you must do this
work on your own without contact or conversations with other students.
Because this exam is given in a somewhat distributed manner, no questions will be answered, and no clarifications will be given.  State your assumptions and count on us to be fair and flexible, especially if we have been unclear.


Your work must be legible.  Work that is
difficult to read will receive no credit.  There is a blank page at the end
if you want to show extra work there.

You must sign the pledge below for your exam to count.  Any cheating will
cause the students involved to receive an F for this course. Other unpleasant
actions
may be taken.

You must fill in your identifying information correctly.  
}

\begin{center}\large
\begin{tabular}{|c|c|c|} \hline
\multicolumn{3}{|c|}{{\bf Print  clearly} the following information:}  \\ \hline
\multicolumn{3}{|l|}{Name (print clearly):\Tall{}\hbox to 3in{\hss}}  \\ \hline
\multicolumn{3}{|l|}{Student 6-digit ID (print {\it really} clearly):\Tall{}\hbox to 3in{\hss}} \\ \hline
\end{tabular}
\end{center}

{\bf Pledge:} On my honor, I have neither
given nor received any unauthorized aid on this exam.

Signed:  \Blank\Blank\Blank\Blank \\ \hbox to 5em{\hss}(Be 
sure you filled in your information in the box above!)
%
%
%
\clearpage
\begin{enumerate}
\item\Points{30} 
blah blah \TF{}
Recall we defined the state \ket{y} on the Bloch Sphere as follows:
\begin{itemize}
\item
$\theta=\pi/2$ 
\item  $\phi=\pi/2$.
\end{itemize}
\begin{enumerate}
  \item\Points{2} What is \ket{y} in the standard basis?
\begin{eqnarray*}
   \ket{y} & = & \Blank \ket{0} + \Blank \ket{1}
\end{eqnarray*}
  \item\Points{2}  What is \ket{y}'s antipodal state (let's call that
state $\ket{-y}$) in the standard
basis?
\begin{eqnarray*}
   \ket{-y} & = & \Blank \ket{0} + \Blank \ket{1}
\end{eqnarray*}
  \item\Points{2} Show that \ket{y} and $\ket{-y}$ are orthogonal\footnote{%
Remember to conjugate and transpose a ket to obtain its bra.}:
\LeaveSpace{1in}
  \item\Points{2} Show that \ket{y} is a unit vector:
\LeaveSpace{1in}
  \item\Points{2} The probability of measuring \ket{0} for state \ket{y} is~\Blank{}.
\end{enumerate}

\clearpage\item\Points{10}
Consider the following matrix
\[
A_{\theta}=
\begin{bmatrix}
\cos{\theta} & \sin{\theta} \\
\sin{\theta} & \cos{\theta}
\end{bmatrix}
\]
In this problem you will analyze whether $A_{\theta}$ is suitable as a unitary quantum
gate.
\begin{enumerate}
 \item\Points{1} Is each column and row of $A_\theta$ a unit vector for all
$\theta$?\Blank{}
 \item\Points{2} Show your work in support of your claim.
\LeaveSpace{2cm}
 \item\Points{1} Are the columns of $A_\theta$ orthogonal for all $\theta$?\Blank{}
 \item\Points{2} Show your work in support of your claim.
\LeaveSpace{2cm}
 \item\Points{1} Are the rows of $A_\theta$ orthogonal for all $\theta$?\Blank{}
 \item\Points{2} Show your work in support of your claim.
\LeaveSpace{2cm}
  \item\Points{1} Below explain why $A_\theta$ is or is not suitable as a
unitary quantum gate.
\LeaveSpace{2cm}
\end{enumerate}
\clearpage\item\Points{10}
Consider the Elitzur--Vaidman bomb, but with a small change.  The
polarizing filter attached to the bomb is
oriented at $\pi/4$ radians (a $45^{\circ}$ angle).  In class and in your homework assignment,
the filter was horizontal, at $0$ radians.

\begin{itemize}
\item
A photon hitting this filter that is measured at $\pi/4$ radians causes
no explosion and passes through oriented at $\pi/4$ radians.
\item
A photon hitting this filter that is measured at $3\pi/4$ radians causes
an explosion.
\item If the filter (and bomb) are not present, the photon passes through
unchanged.
\end{itemize}
Let \ket{0} represent a photon oriented horizontally, at $0$ radians.
Let \ket{1} represent a photon oriented vertically, at $\pi/2$ radians.

For this problem you can only start with photons in the state \ket{0} or
\ket{1} but you are allowed to use any quantum gates you like to solve
the problem.

Likewise, measurements outside the box (possibly containing a filter and
bomb) can only be made in the 
standard basis of \ket{0} and
\ket{1}.

As in class, we would like to detect the absence or presence of the bomb, 
with the
probability of an explosion occurring under our control based on
the angle of rotation each iteration.
Below sketch a solution to the bomb problem based on the above configuration,
explaining carefully and clearly what is happening at each step.  Because
this is a take-home exam, you are expected to draw and explain clearly and
carefully in the space below.

\clearpage\item\Points{5} Consider the following state expressed in
the standard basis:
\[ \ket{\psi} = \frac{1}{2}\left( \ket{00} + \ket{01} + \ket{10} + \ket{11} \right)
\]

\begin{enumerate}
  \item\Points{1} Can this be factored into the tensor product of two
qubits?\Blank{}
  \item\Points{4} Prove or disprove your claim below:
\LeaveSpace{2cm}
\end{enumerate}

\item\Points{5} Consider the following state expressed in the standard
basis:
\[
\ket{\psi} = 
\frac{1}{\sqrt{2}}\ket{00} +
\frac{1}{\sqrt{8}}\ket{01} +
\frac{1}{\sqrt{8}}\ket{10} +
\frac{1}{\sqrt{4}}\ket{11}
\]

\begin{enumerate}
  \item\Points{1} Can this be factored into the tensor product of two
qubits?\Blank{}
  \item\Points{4} Prove or disprove your claim below:
\LeaveSpace{2cm}
\end{enumerate}
\clearpage\item\Points{15}
Recall the trigonometric identities:
\begin{eqnarray*}
   \sin(-\theta) & = & -\sin(\theta) \\
   \cos(-\theta) & = & \ \ \,\cos(\theta)
\end{eqnarray*}
Recall the unitary gate for rotating by $\theta$ radians is:
\[
R_{\theta}=
\begin{bmatrix}
\cos{\theta} & -\sin{\theta} \\
\sin{\theta} & \cos{\theta}
\end{bmatrix}
\]
\begin{enumerate}
  \item\Points{4} Fill in the matrix below to make a unitary gate that
rotates by $-\theta$~radians, simplified using the above identities where
possible:
\[
R_{-\theta}=
\begin{bmatrix}
\Blank{} & \Blank{} \\
\Blank{} & \Blank{} 
\end{bmatrix}
\]
\item\Points{6} Below show that $R_\theta$ and $R_{-\theta}$ are inverses
of each other:
\LeaveSpace{2cm}
\item\Points{5} Consider a 2-qubit system in which one qubit is rotated by
$\theta$~radians and the other is rotated by $-\theta$~radians.
Below fill in the matrix that describes the effect of those rotations using
a single $4\times4$ quantum gate on a two-qubit system.  Show your work below and
fill in the matrix.
\LeaveSpace{4cm}
\[
A=
\begin{bmatrix}
\Blank{} & \Blank{} & \Blank{} & \Blank{}  \\
\Blank{} & \Blank{} & \Blank{} & \Blank{}  \\
\Blank{} & \Blank{} & \Blank{} & \Blank{}  \\
\Blank{} & \Blank{} & \Blank{} & \Blank{} 
\end{bmatrix}
\]
\end{enumerate}
\clearpage\item\Points{10}
Recall the CNOT gate:
\[
CNOT =
\begin{bmatrix}
1 & 0 & 0 & 0 \\
0 & 1 & 0 & 0 \\
0 & 0 & 0 & 1 \\
0 & 0 & 1 & 0
\end{bmatrix}
\]
\begin{enumerate}
  \item\Points{3} Below fill in the matrix of the gate that is the \emph{inverse} of
the CNOT gate:
\[
CNOT^{-1}=
\begin{bmatrix}
\Blank{} & \Blank{} & \Blank{} & \Blank{}  \\
\Blank{} & \Blank{} & \Blank{} & \Blank{}  \\
\Blank{} & \Blank{} & \Blank{} & \Blank{}  \\
\Blank{} & \Blank{} & \Blank{} & \Blank{} 
\end{bmatrix}
\]
 \item\Points{7} Below prove or disprove that the CNOT gate can be written as the
tensor product of two $2\times2$ single qubit gates:
\end{enumerate}

\end{enumerate}

\end{assignment}
\Bpage{}

\end{document}
