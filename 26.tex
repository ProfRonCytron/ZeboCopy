\SetTitle{26}{Grover's Algorithm}{Unstructured search in $\sqrt{N}$ time}{26}

\begin{frame}{Overview}{What do we study here?}

\begin{itemize}
    \item Some search problems have \emph{structure}, which can lead to efficient methods of solving the search problem.
    \item For example, a sorted array of integers can be subject to binary search, discovering whether an integer is present in $\Theta(\log{N})$ time for an array of size~$N$.
    \item An \emph{unstructured} search problem has no such advantage.  Classically, we may have to examine each of the $N$ items in a collection to find an object of interest.
    \item Any problem in \href{https://complexityzoo.net/Complexity_Zoo:N\#np}{\CompClass{NP}} can be posed as unstructured search problem.
\end{itemize}
    
\end{frame}

\begin{frame}{Oracle for a problem in \href{https://complexityzoo.net/Complexity_Zoo:N\#np}{\CompClass{NP}}}{We search for an $x$ such that $f(x)=1$}

\TwoColumns{%
\begin{itemize}[<+->]
    \item Input $x$ of $n$ qubits.
    \item Another input is the \href{https://en.wikipedia.org/wiki/Ancilla\_bit}{ancilla} $y$.
    \item Input $x$ is copied to output.
    \item $f(x)$ is computed in polynomial time with result
    \begin{itemize}
      \item[0] if $x$ is not a solution
      \item[1] if $x$ is a solution
    \end{itemize}
    \item Our problem is to find $x\ |\ f(x)=1$.
    \item There may be multiple solutions to this problem, but we assume there is at least one.  
\end{itemize}
}{%
\begin{center}
    \begin{GateBox}{2}{3}{5}
      \visible<1->{%
      \Input{0}{$x_{1}$}
      \Input{1}{$x_{2}$}
      \Input{2}{\RVDots}
      \Input{3}{$x_{n}$}}
      \visible<2->{\Input{4}{$y$}}
      \visible<3->{\Output{0}{$x_{1}$}
      \Output{1}{$x_{2}$}
      \Output{2}{\RVDots}
      \Output{3}{$x_{n}$}}
      \visible<4->{\Output{4}{$\Xor{y}{f(x)}$}}
      \BoxLabel{Oracle}
    \end{GateBox}
\end{center}

\visible<9->{%
Any search problem, including those that are computationally challenging, can be posed in the form of such a problem.

We next consider two examples.
}
}
    
\end{frame}


\begin{frame}{Examples}{Cryptographic and an \href{https://complexityzoo.net/Complexity_Zoo:N\#npc}{\CompClass{NPC}} application}

\TwoColumns{%
\visible<1->{
\href{https://en.wikipedia.org/wiki/RSA\_(cryptosystem)}{RSA}
\begin{itemize}
    \item Consider two large primes $p$ and $q$ whose products are $r$.
    \item Let $x$ have $n$ qubits, sufficient to encode $p$ and $q$: \[n=\Theta(\log{p}+\log{q})=\Theta(\log{pq})\]
    \item Given a suitably encoded input $x=pq$, our oracle returns $1$ if  $p\times q=r$. Otherwise, $0$ is returned.
\end{itemize}}
}{%
\visible<2->{%
\href{https://en.wikipedia.org/wiki/Travelling\_salesman\_problem}{Traveling Salesperson}
\begin{itemize}
    \item Is there a tour of a given graph $G=(V,E)$ with cost (sum of weighted edges in the tour) $B$?
    \item Let $x$ have $n$ qubits, sufficient to encode a list of elements in $V$:
    \[n = \Theta(|V|\log{|V|})\]
    \item The oracle returns $1$ if the cost of $x$ is $B$.  Otherwise, $0$ is returned.
\end{itemize}
}}
\end{frame}

{
\def\S{\ColorOne{\ket{s}}}
\def\W{\ColorThree{\ket{w}}}
\def\R{\ColorTwo{\ket{r}}}
\begin{frame}{A useful orthogonality for this problem}{For an $n$-qubit instance, from all possible solutions to the one(s) of interest}

\TwoColumns{%
\Vskip{-4em}\begin{align*}
    \S &= \TensSupProd{\left(\PPlus\right)}{n} = \RootTwoN{n}\SumBV{x}{n}\ket{x} \\
    \visible<2->{\W &= \frac{1}{\sqrt{\Mag{T}}}\sum_{t \in T} \ket{t} \\}
    \visible<6->{\R &= \frac{1}{\sqrt{2^{n}-\Mag{T}}}\sum_{t \not\in T} \ket{t}}
\end{align*}

\BigSkip{}
}{%
\Vskip{-4em}\begin{center}
\begin{TIKZP}[scale=3]
    \visible<1->{\draw[->,thick,\RCone] (0,0) -- (15:1) node[right]{\S};} \visible<2->{\draw[->,thick,\RCthree] (0,0) -- (0,1) node[above]{\W} ;}
    \visible<6->{\draw[->,thick,\RCtwo] (0,0) -- (1,0) node[right] {\R};}
\end{TIKZP}
\end{center}}
\only<1>{We begin with the superposition of all inputs $\S=\TensSupProd{\Hadamard}{n}\left(\TensSupProd{0}{n}\right)$}%
\only<2>{%
Let $T=\Set{t\ | n\ f(t)=1}$.   Measurement of \W{} obtains a solution to the problem.}

\only<4-5>{%

Unfortunately, \S{} and \W{} are not orthogonal:
\(
\braket{\ColorOne{s}}{\ColorThree{w}} = \frac{1}{\sqrt{2^{n}\Mag{T}}}\sum_{t \in T} 1\visible<5->{ \alert{\not= 0}}
\)}%
\end{frame}
}
