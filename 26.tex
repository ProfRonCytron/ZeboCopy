\SetTitle{26}{Grover's Algorithm}{Unstructured search in $\sqrt{N}$ time}{26}

\begin{frame}{Overview}{What do we study here?}

\begin{itemize}
    \item Some search problems have \emph{structure}, which can lead to efficient methods of solving the search problem.
    \item For example, a sorted array of integers can be subject to binary search, discovering whether an integer is present in $\Theta(\log{N})$ time for an array of size~$N$.
    \item An \emph{unstructured} search problem has no such advantage.  Classically, we may have to examine each of the $N$ items in a collection to find an object of interest.
    \item Any problem in \href{https://complexityzoo.net/Complexity_Zoo:N\#np}{\CompClass{NP}} can be posed as unstructured search problem.
\end{itemize}
    
\end{frame}

\begin{frame}{Oracle for a problem in \href{https://complexityzoo.net/Complexity_Zoo:N\#np}{\CompClass{NP}}}{We search for an $x$ such that $f(x)=1$}

\TwoColumns{%
\begin{itemize}[<+->]
    \item Input $x$ of $n$ qubits.
    \item Another input is the \href{https://en.wikipedia.org/wiki/Ancilla\_bit}{ancilla} $y$.
    \item Input $x$ is copied to output.
    \item $f(x)$ is computed in polynomial time with result
    \begin{itemize}
      \item[0] if $x$ is not a solution
      \item[1] if $x$ is a solution
    \end{itemize}
    \item Our problem is to find $x\ |\ f(x)=1$.
    \item There may be multiple solutions to this problem, but we assume there is at least one.  
\end{itemize}
}{%
\begin{center}
    \begin{GateBox}{2}{3}{5}
      \visible<1->{%
      \Input{0}{$x_{1}$}
      \Input{1}{$x_{2}$}
      \Input{2}{\RVDots}
      \Input{3}{$x_{n}$}}
      \visible<2->{\Input{4}{$y$}}
      \visible<3->{\Output{0}{$x_{1}$}
      \Output{1}{$x_{2}$}
      \Output{2}{\RVDots}
      \Output{3}{$x_{n}$}}
      \visible<4->{\Output{4}{$\Xor{y}{f(x)}$}}
      \BoxLabel{Oracle}
    \end{GateBox}
\end{center}

\visible<9->{%
Any search problem, including those that are computationally challenging, can be posed in the form of such a problem.

We next consider two examples.
}
}
    
\end{frame}


\begin{frame}{Examples}{Cryptographic and an \href{https://complexityzoo.net/Complexity_Zoo:N\#npc}{\CompClass{NPC}} application}

\TwoColumns{%
\visible<1->{
\href{https://en.wikipedia.org/wiki/RSA\_(cryptosystem)}{RSA}
\begin{itemize}
    \item Consider two large primes $p$ and $q$ whose products are $r$.
    \item Let $x$ have $n$ qubits, sufficient to encode $p$ and $q$: \[n=\Theta(\log{p}+\log{q})=\Theta(\log{pq})\]
    \item Given a suitably encoded input $x=pq$, our oracle returns $1$ if  $p\times q=r$. Otherwise, $0$ is returned.
\end{itemize}}
}{%
\visible<2->{%
\href{https://en.wikipedia.org/wiki/Travelling\_salesman\_problem}{Traveling Salesperson}
\begin{itemize}
    \item Is there a tour of a given graph $G=(V,E)$ with cost (sum of weighted edges in the tour) $B$?
    \item Let $x$ have $n$ qubits, sufficient to encode a list of elements in $V$:
    \[n = \Theta(|V|\log{|V|})\]
    \item The oracle returns $1$ if the cost of $x$ is $B$.  Otherwise, $0$ is returned.
\end{itemize}
}}
\end{frame}

{
\def\S{\ColorOne{\ket{s}}}
\def\W{\ColorThree{\ket{w}}}
\def\R{\ColorTwo{\ket{r}}}
\begin{frame}{A useful orthogonality for this problem}{For an $n$-qubit instance, from all possible solutions to the one(s) of interest}

\TwoColumns{%
\only<1-6>{%
\Vskip{-4em}\begin{align*}
    \S &= \TensSupProd{\left(\PPlus\right)}{n} = \RootTwoN{n}\SumBV{x}{n}\ket{x} \\
    \visible<2->{\W &= \frac{1}{\sqrt{\Mag{T}}}\sum_{t \in T} \ket{t} \\}
    \visible<5->{\R &= \frac{1}{\sqrt{2^{n}-\Mag{T}}}\sum_{t \not\in T} \ket{t}}
\end{align*}
}%
}{%
\Vskip{-4em}\begin{center}
\begin{TIKZP}[scale=3]
    \visible<1->{\draw[->,thick,\RCone] (0,0) -- (15:1) node[right]{\S};} \visible<2->{\draw[->,thick,\RCthree] (0,0) -- (0,1) node[above]{\W} ;}
    \visible<5->{\draw[->,thick,\RCtwo] (0,0) -- (1,0) node[right] {\R};}
\end{TIKZP}
\end{center}}
\only<1>{We begin with the superposition of all inputs $\S=\TensSupProd{\Hadamard}{n}\left(\TensSupProd{0}{n}\right)$}%
\only<2-3>{%
Let $T=\Set{t\ |\ f(t)=1}$.   \only<3>{Measurement of \W{} obtains a solution to the problem.}
}
\only<4-5>{%
Unfortunately, \S{} and \W{} are not orthogonal:
\(
\braket{\ColorOne{s}}{\ColorThree{w}\textbf{}} = \frac{1}{\sqrt{2^{n}\Mag{T}}}\sum_{t \in T} 1\visible<5->{ \alert{\not= 0}}
\)}%
\only<5>{%

But we can construct \R{} so that it is orthogonal to \W.}
\only<6>{%

Grover's algorithm computes vectors in the space spanned by \W{} and \R{}, starting at \S{} and moving iteratively toward \W{}.
}
\end{frame}
}



{
\def\S{\ColorOne{\bra{s}}}
\def\W{\ColorThree{\bra{w}}}
\def\R{\ColorTwo{\bra{r}}}
\begin{frame}{A two-qubit example to illustrate Grover's algorithm}{Oracle $f(x)=1$ only when $x=2$ (using bras instead of kets to save space here)}
\TwoColumns{%
\only<1-3>{%
\begin{itemize}
    \item<1-> We begin at \S, the uniform superposition of all possible inputs.
    \item<2-> Our secret \W{} captures $x=2$.
    \item<3-> Here is \R, orthogonal to \W.
\end{itemize}
}%
\only<4>{%
\Vskip{-3em}\begin{align*}
    \visible<4->{\cos{\theta} &= \braket{\ColorOne{s}}{\ColorTwo{r}} =\frac{\sqrt{3}}{2}\\
    \theta &= \frac{\pi}{6}
    }
\end{align*}}%
\only<5>{%
Expressing \S{} in this basis:
\begin{align*}
    \S &= \cos{\theta}\R + \sin{\theta}\W \\
       &= \frac{\sqrt{3}}{2}\R + \frac{1}{2}\W \\
       &= \frac{1}{2}\CDQB{1}{1}{1}{1}
\end{align*}
}%
\only<6->{%
\Vskip{-3em}\begin{itemize}
\item The first step of the algorithm reflects the current state \S{} around the horizontal axis \R{} to obtain \ColorFour{\BState{f}}.
\item This is accomplished using the familiar phase-kickback trick, which flips the sign of the component \bra{x} if $f(x)=1$.
\item<7-> The second step reflects the current state \ColorFour{\BState{f}} around \S{} to obtain \ColorFive{\BState{d}}.
\item<7-> In this example, we obtain \W{} exactly.  Typically we must repeat the two steps of the algorithm to get close to \W.
\end{itemize}
}
}{%
\Vskip{-3em}\begin{center}
\begin{TIKZP}[scale=2.5]
    \visible<1->{\draw[->,thick,\RCone] (0,0) -- (30:1) node[right]{$\S = \RootTwoN{2}\CDQB{1}{1}{1}{1}$};} 
    \visible<2-6>{\draw[->,thick,\RCthree] (0,0) -- (0,1) node[above right]{$\W=\CDQB{0}{0}{1}{0}\visible<7->{\ColorFive{=\BState{d}}}$} ;}
    \visible<7->{\draw[->,thick,\RCfive] (0,0) -- (0,1) node[above right]{$\W=\CDQB{0}{0}{1}{0}\visible<7->{\ColorFive{=\BState{d}}}$} ;}
    \visible<3->{\draw[->,thick,\RCtwo] (0,0) -- (1,0) node[right] {$\R=\RootN{3}\CDQB{1}{1}{0}{1}$};}
    \visible<4-6>{
     \draw[->,thick] (1,0) arc (0:30:1) coordinate(zz);
     \draw (0.7,0.17) node {$\theta=\frac{\pi}{6}$};
     }
     \visible<6>{
     \draw[->,thick] (1,0) arc (0:-30:1);
     \draw (0.7,-0.17) node {$-\frac{\pi}{6}$};}
     \visible<6->{
     \draw[->,thick,\RCfour] (0,0) -- (-30:1) node[right]{$\BState{f} = \RootTwoN{2}\CDQB{1}{1}{-1}{1}$};
     }
     \visible<7>{\draw[->,thick,\RCfive] (zz) arc (30:90:1);
     \draw (0.25,0.5) node {$2\times\frac{\pi}{6}$};
     }
\end{TIKZP}
\end{center}
}

\end{frame}
}
