\SetTitle{30}{Phase estimation}{What separates the superpositions in a state?}{30}

\begin{frame}{Overview}{What will we study here?}

\begin{itemize}
    \item Phase has been an elusive aspect of quantum state.
    \item As we have studied, the probability of a measurement outcomes is related to the amplitude of the coefficient present the associated basis state.
    \item We use phase during a quantum computation to establish interference patterns of interest.
    \item While we cannot directly measure phase, we can estimate a constant phase shift between basis states in a superposition.
    \item The study of this material leads directly to quantum Fourier analysis, which in turn is the basis of Shor's famous algorithm.
\end{itemize}
    
\end{frame}

\begin{frame}{A return to the complex circle}{The basics we need for phase estimation}
\TwoColumns{%
\adjustbox{valign=t, width=\textwidth}{\begin{TIKZP}
\UnitComplexCircle{}
\visible<1>{%
\TZPEast{}\TZPNorth{}\TZPSouth{}\TZPWest{}}
\visible<2-3>{%
\draw[->,thick] (1,0) arc (0:45:1) coordinate (z) ;
    \draw[thick,dotted] (0,0) -- (z) ;
    \TZText{0.5,0}{$\theta$}{above}
    \ColorOne{\TZText{z}{\ExpPhase{\theta}}{above right}}
    \visible<3>{%
    \TZPoint{1,0}{\ExpPhase{0}}{above right}
    \TZPoint{0,1}{\ExpPhase{\pi/2}}{above right}
    \TZPoint{-1,0}{\ExpPhase{\pi}}{above left}
    \TZPoint{0,-1}{\ExpPhase{3\pi/2}}{right}
    }
}%
\visible<4>{%
\draw[->,thick] (1,0) arc (0:359:1) coordinate (z) ;
    \draw[thick,dotted] (0,0) -- (z) ;
    \TZText{z}{$\ExpPhase{0}=\ExpPhase{2\pi}$}{above right}
}%
\end{TIKZP}}}{%
\begin{itemize}
    \item<1-> Unit points on each axis
    \item<2-> Traveling $\theta$ radians counterclockwise from $+1$
\end{itemize}
}
    
\end{frame}

