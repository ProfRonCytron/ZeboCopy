\SetTitle{30}{Phase estimation}{What separates the superpositions in a state?}{30}

\begin{frame}{Overview}{What will we study here?}

\begin{itemize}
    \item Phase has been an elusive aspect of quantum state.
    \item As we have studied, the probability of a measurement outcomes is related to the amplitude of the coefficient present the associated basis state.
    \item We use phase during a quantum computation to establish interference patterns of interest.
    \item While we cannot directly measure phase, we can estimate a constant phase shift between basis states in a superposition.
    \item The study of this material leads directly to quantum Fourier analysis, which in turn is the basis of Shor's famous algorithm.
\end{itemize}
    
\end{frame}

\begin{frame}{A return to the complex circle}{The basics we need for phase estimation}
\TwoUnequalColumns{0.6\textwidth}{0.4\textwidth}{%
\adjustbox{valign=t, width=\textwidth}{\begin{TIKZP}
\UnitComplexCircle{}
\visible<1>{%
\TZPEast{}\TZPNorth{}\TZPSouth{}\TZPWest{}}
\visible<2>{%
  \draw[->,thick,\RCtwo] (1,0) arc (0:45:1) coordinate (z) ;
  \draw[thick,dotted,\RCtwo] (0,0) -- (z) ;
  \TZText{0.5,0}{$\theta$}{above}
    \ColorTwo{\TZText{z}{\ExpPhase{\theta}}{above right}}
}
\visible<3>{%
   \TZPoint{1,0}{$\ExpPhase{0}=+1$}{above right}}
\visible<4>{
      \TZPoint{0,1}{$\ExpPhase{\pi/2}=i$}{above right}
      \draw[->,thick,\RCtwo] (1,0) arc (0:90:1) coordinate (z) ;
    \draw[thick,dotted] (0,0) -- (z);}
\visible<5>{%
      \TZPoint{-1,0}{$\ExpPhase{\pi}=-1$}{above left}
      \draw[->,thick,\RCtwo] (1,0) arc (0:180:1) coordinate (z) ;
    \draw[thick,dotted] (0,0) -- (z);
    }
\visible<6>{%
      \TZPoint{0,-1}{$\ExpPhase{3\pi/2}=-i$}{below right}
      \draw[->,thick,\RCtwo] (1,0) arc (0:270:1) coordinate (z) ;
    \draw[thick,dotted] (0,0) -- (z);
    }
\visible<7-8>{%
      \TZPoint{1,0}{$\ExpPhase{2\pi}\visible<8>{=}$}{above right}
       \visible<8>{\TZPoint{1,0}{$\ExpPhase{0}=+1$}{below right}}
      \draw[->,thick,\RCtwo] (1,0) arc (0:359:1) coordinate (z) ;
    \draw[thick,dotted] (0,0) -- (z);
}
\end{TIKZP}}}{%
\begin{itemize}
    \item<1-> Unit points on each axis
    \item<2-> Traveling $\theta$ radians counterclockwise from $+1$
\end{itemize}
}
    
\end{frame}

\begin{frame}{Periodic states}{Definition}
\TwoUnequalColumns{0.4\textwidth}{0.6\textwidth}{%
\adjustbox{valign=t, width=\textwidth}{\begin{TIKZP}
\UnitComplexCircle{}
\visible<3-4>{%
 \draw[->,thick,\RCone] (1,0) arc (0:40:1) coordinate (z2);
 \draw[->,thick,\RCone] (1,0) arc (0:60:1) coordinate (z3);
 \draw[->,thick,\RCone] (1,0) arc (0:80:1) coordinate (z4);
 \draw[->,thick,\RCone] (1,0) arc (0:100:1) coordinate (z5);
 \draw[thick,dotted] (0,0) -- (z2);
 \draw[thick,dotted] (0,0) -- (z3);
 \draw[thick,dotted] (0,0) -- (z4);
 \draw[thick,dotted] (0,0) -- (z5);
}
\visible<1-4>{%
  \draw[->,thick,\RCtwo] (1,0) arc (0:20:1) coordinate (z) ;
  \draw[thick,dotted,\RCtwo] (0,0) -- (z);
  \visible<1>{\TZText{0.55,0.15}{\mbox{\small $\theta$}}{right}}
  \visible<1>{
    \ColorTwo{\TZText{z}{\ExpPhase{\theta}}{above right}}}
  \visible<2->{
    \ColorTwo{\TZText{z}{\ExpPhase{w\cdot 2\pi}}{above right}}}
}
\visible<5->{%
  \draw[->,ultra thick,\RCthree] (0,0) -- (1,0);
  \visible<5>{\ColorThree{\TZPoint{1,0}{$\ExpPhase{0}=+1$}{above right}}}
  \visible<6->{%
  \draw[->,thick,\RCtwo] (1,0) arc (0:20:1) coordinate (z2);
  \ColorTwo{\TZText{z2}{\ \ExpPhase{w\cdot 2\pi\cdot 1}}{right}}
 \draw[->,thick,\RCone] (1,0) arc (0:40:1) coordinate (z3);
 \draw[->,thick,\RCone] (1,0) arc (0:60:1) coordinate (z4);
 \draw[->,thick,\RCone] (1,0) arc (0:80:1) coordinate (z5);
 \draw[thick,dotted] (0,0) -- (z2);
 \draw[thick,dotted] (0,0) -- (z3);
 \ColorOne{\TZText{z3}{\ \ExpPhase{w\cdot 2\pi\cdot 2}}{right}}
 \draw[thick,dotted] (0,0) -- (z4);
 \ColorOne{\TZText{z4}{\ \ExpPhase{w\cdot 2\pi\cdot 3}}{above right}}
 \draw[thick,dotted] (0,0) -- (z5);
 \ColorOne{\TZText{z5}{\raisebox{4pt}{\ \ExpPhase{w\cdot 2\pi\cdot 4}}}{above right}}}
}
\end{TIKZP}}
}{%
\only<1-3>{%
\begin{itemize}
    \item<1-> The basis states will be related to each other in phase by a constant shift of $\theta$ radians.
    \item<2-> We capture the extent of the rotation by $\ColorTwo{w}, 0\leq \ColorTwo{w} < 1$, which represents a fraction of a complete rotation of $2\pi$ radians.
    \item<3-> Each \ColorOne{subsequent basis vector} has amplitude shifted by $\ColorTwo{w}\cdot 2\pi$ radians.
\end{itemize}}
\only<4->{%
\begin{itemize}
    \item<4-> Global phase consists of a uniform rotation of all points on the complex circle. 
    \item<5-> We can therefore assert that the first point occurs at $+1$.
    \item<6-> Each \ColorOne{subsequent point} is $\ColorTwo{w}\cdot 2\pi$ radians counterclockwise on the circle.
\end{itemize}}
}
\end{frame}

\begin{frame}{\QPlus{} and \QMinus}{These are periodic states}
    
\end{frame}

