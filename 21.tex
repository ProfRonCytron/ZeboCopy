\SetTitle{21}{Deutsch's Problem}{Is $f(x)$ constant or balanced?}{20}

\begin{frame}{Deutsch's problem}{Formulation}

\begin{itemize}
    \item We are given an oracle $f(x): \Set{0,1} \mapsto \Set{0,1}$
    \item Which of the following characterizes $f(x)$?
    \begin{description}
        \item[constant]   \Forall{x}{f(x)=k}, where $k$ is $0$ or $1$
        \item[balanced]   For half of its inputs, $f(x)=1$ and for the other half $f(x)=0$
    \end{description}
    \item Because $x$ is a single bit, $f(x)$ must be one of the above.
\end{itemize}
    
\end{frame}

\begin{frame}{Possible functions}{There are only four.}

\TwoColumns{%
\visible<2->{\begin{center}\alert{Constant}\end{center}}
\SBitTable{A}{0}{0}{1}{0}
\SBitTable{B}{0}{1}{1}{1}
}{%
\visible<3>{\begin{center}\alert{Balanced}\end{center}}
\SBitTable{C}{0}{0}{1}{1}
\SBitTable{D}{0}{1}{1}{1}
}
    
\end{frame}

\begin{frame}{Form of the oracle}{It has to be reversible.}
\TwoUnequalColumns{0.6\textwidth}{0.4\textwidth}{%
\begin{itemize}[<+->]
    \item We consider a quantum gate $U_f$ that implements $f(x)$.
    \item It accepts $x$ and an \href{https://en.wikipedia.org/wiki/Ancilla_bit}{ancillary} input $y$.
    \item It copies $x$ onto the top output qubit.
    \item It computes \Xor{y}{f(x)} on the bottom output qubit.
    \item When $y=\ket{0}$ the bottom qubit is $f(x)$.
    \item When $y=\ket{1}$ the bottom qubit is \Not{$f(x)$}.
\end{itemize}
}{%
\BigSkip{}
\only<1>{%
\begin{Oracle}[scale=0.75]{$U_f$}{\relax}{\relax}{\relax}{\relax}
\end{Oracle}}%
\only<2>{%
\begin{Oracle}[scale=0.75]{$U_f$}{$x$}{$y$}{\relax}{\relax}
\end{Oracle}}%
\only<3>{%
\begin{Oracle}[scale=0.75]{$U_f$}{$x$}{$y$}{$x$}{\relax}
\end{Oracle}}%
\only<4>{%
\begin{Oracle}[scale=0.75]{$U_f$}{$x$}{$y$}{$x$}{\Xor{y}{f(x)}}
\end{Oracle}}%
\only<5>{%
\begin{Oracle}[scale=0.75]{$U_f$}{$x$}{\ket{0}}{$x$}{$f(x)$}
\end{Oracle}}%
\only<6>{%
\begin{Oracle}[scale=0.75]{$U_f$}{$x$}{\ket{1}}{$x$}{\Not{$f(x)$}}
\end{Oracle}}%
}    
\end{frame}

\begin{frame}{Realizing the oracle for \Quote{A}}{We design a circuit that meets the classical specification.}

\TwoColumns{%
\Vskip{-3em}\begin{center}
\begin{TTable}{A}{$x$}{$f(x)$}
\TRow{ }{0}{0}
\TRow{ }{1}{0}
\end{TTable}
\end{center}
\BigSkip{}
\visible<2->{%
\begin{center}
\begin{UTable}{$U_A$}{$x$}{$y$}{$x$}{\Xor{y}{f(x)}}
\URow{}{0}{0}{0}{0}
\URow{}{0}{1}{0}{1}
\URow{}{1}{0}{1}{0}
\URow{}{1}{1}{1}{1}
\end{UTable}\end{center}}
}{%
\begin{Oracle}[scale=0.75]{$U_A$}{$x$}{$y$}{$x$}{\Xor{y}{f(x)}}
\end{Oracle}

\BigSkip{}
\visible<3->{%
\Forall{x}{f(x)=0}, so $\Xor{y}{f(x)}=\Xor{y}{0}=y$.  
\MedSkip{}
Thus, $U_A$ is just the identity function, so our quantum circuit doesn't have to do anything to its inputs to realize $U_A$.}

\BigSkip{}
\visible<4>{%
\begin{center}
\adjustbox{valign=t}{\begin{quantikz}
\lstick{\ket{x}} &  \qw & \qw \rstick{\ket{x}} \\
\lstick{\ket{y}} &  \qw & \qw \rstick{\ket{\Xor{y}{f(x)}}}
\end{quantikz}}\end{center}
}
}
    
\end{frame}