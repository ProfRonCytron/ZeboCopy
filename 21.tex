\SetTitle{21}{Deutsch's Problem}{Is $f(x)$ contant or balanced?}{20}

\begin{frame}{Deutsch's problem}{Formilation}

\begin{itemize}
    \item We are given an oracle $f(x): \Set{0,1} \mapsto \Set{0,1}$
    \item Which of the following characterizes $f(x)$?
    \begin{description}
        \item[constant]   \Forall{x}{f(x)=k}, where $k$ is $0$ or $1$
        \item[balanced]   For half of its inputs, $f(x)=1$ and for the other half $f(x)=0$
    \end{description}
    \item Because $x$ is a single bit, $f(x)$ must be one of the above.
\end{itemize}
    
\end{frame}
{
\def\FF#1#2#3#4#5{%
\begin{center}
\begin{tabular}{c|cc}
#1 & $x$ & $f(x)$ \\
\hline
0  &  #2 & #3 \\
1  &  #4 & #5
\end{tabular}
\end{center}
}
\begin{frame}{Possible functions}{There are only four.}

\TwoColumns{%
\visible<2->{\begin{center}\alert{Constant}\end{center}}
\FF{A}{0}{0}{1}{0}
\FF{B}{0}{1}{1}{1}
}{%
\visible<3>{\begin{center}\alert{Balanced}\end{center}}
\FF{C}{0}{0}{1}{1}
\FF{D}{0}{1}{1}{1}
}
    
\end{frame}

\begin{frame}{Form of the oracle}{It has to be reversible.}
\TwoUnequalColumns{0.6\textwidth}{0.4\textwidth}{%
\begin{itemize}[<+->]
    \item We consider a quantum gate $U_f$ that implements $f(x)$.
    \item It accepts $x$ and an \href{https://en.wikipedia.org/wiki/Ancilla_bit}{ancillary} input $y$.
    \item It copies $x$ onto the top output qubit.
    \item It computes \Xor{y}{f(x)} on the bottom output qubit.
    \item When $y=\ket{0}$ the bottom qubit is $f(x)$.
    \item When $y=\ket{1}$ the bottom qubit is $\Not{f(x)}$.
\end{itemize}
}{%
\begin{TIKZP}[scale=0.7]
\draw (0,0) rectangle node[pos=0.5] {$U_f$} ++(4,2);
\visible<2->{\draw[<-] (0,0.5) node[right] {\only<1-4>{$y$}\only<5>{\ket{0}}\only<6>{\ket{1}}} -- ++(-1,0);
\draw[<-] (0,1.5) node[right] {$x$} -- ++(-1,0);}
\visible<4->{\draw[->] (4,0.5) node[left] {\only<1-4>{\Xor{y}{f(x)}}\only<5>{$f(x)$}\only<6>{\Not{$f(x)$}}} -- ++(1,0);}
\visible<3->{\draw[->] (4,1.5) node[left] {$x$} -- ++(1,0);}
\end{TIKZP}
}    
\end{frame}
}