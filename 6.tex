\SetTitle{6}{A single qubit}{Representation}{introqsys}

\section{Overview}
\begin{frame}{Overview}{What we will cover}
    
\end{frame}

\begin{frame}{State space}{Definition}
\Postulate{State Space}{The state of a quantum system is a unit vector in a complex \href{https://en.wikipedia.org/wiki/Hilbert_space}{Hilbert space}}
\begin{itemize}
    \item We need a Hilbert space, rather than the simpler \href{https://en.wikipedia.org/wiki/Euclidean_space}{Euclidean space}, because we use complex numbers to represent state.
    \item We have used two dimensions to specify a single qubit, one each for $\alpha$ and $\beta$: $\ket{\psi}=\SQB{\alpha}{\beta}$, where
    $\Prob{\alpha}+\Prob{\beta}=1$.
  
    \item The state of an $n$-qubit system is a unit column vector (ket)  of $\TwoToThe{n}$ complex numbers.
\end{itemize}
\OnlyRemark{1}{%
We next examine a more concise and visually compelling definition of a qubit.
}
\end{frame}

\begin{frame}{Coordinates for a qubit}{Cartesian and polar}
\Vskip{-5em}\TwoUnequalColumns{0.4\textwidth}{0.6\textwidth}{%
\ColHead{Cartesian}
\begin{itemize}
    \item $\ket{\psi}=\alpha\ket{0} + \beta\ket{1}$ with $\Mag{\alpha}+\Mag{\beta}=1$.
    \item Letting
    \begin{itemize}
        \item $\alpha=\Complex{a_{0}}{b_{0}}$
        \item $\beta=\Complex{a_{1}}{b_{1}}$
    \end{itemize}
    \item This requires \alert{four real parameters}
\end{itemize}
}{%
\ColHead{Polar}
\begin{itemize}
    \item Write $\alpha = r_{a}\left(\cos\theta_{a}+i\, sin\theta_{a}\right) = \Polar{r_{a}}{\theta_{a}}$
    \item Write $\beta = r_{b}\left(\cos\theta_{b}+i\, sin\theta_{b}\right) = \Polar{r_{b}}{\theta_{b}}$
    \item Then \begin{eqnarray*} \ket{\psi} & = & \Polar{r_{a}}{\theta_{a}}\ket{0} + \Polar{r_{b}}{\theta_{b}}\ket{1} \\
    & = & \visible<1>{\Exp{i \theta_{a}}}\left( \alert<4>{r_{a}}\ket{0} + \Polar{\alert<4>{r_{b}}}{
    \only<1-2>{(\theta_{b}-\theta_{a})}
    \only<3->{\alert<4>{\phi}}
    }\ket{1} \right)
    \end{eqnarray*}
    \item<2-> Discard $\Exp{i \theta_{a}}$ as a global phase
    \item<3-> Let $\phi=\theta_{b}-\theta_{a}$
    \item<4-> So we need just \alert{three real parameters}
\end{itemize}
}
\end{frame}

\begin{frame}{Can we do better?}{It takes only two real parameters to denote the state of one qubit?}
\begin{itemize}
   \item We start with 
   \[ \ket{\psi}= r_{a}\left( \ket{0} + r_{b}\Exp{\phi}\ket{1}\right) \]
   \item We have not yet accounted for the unit nature of \ket{\psi}.   
   \item We require $r_{a}^{2}+r_{b}^{2}=1$.  We can show $\Implies{x^{2}\leq 1}{-1 \leq x \leq 1}$.
   \item<2-> We therefore substitute for $r_{a}$ and $r_{b}$ as follows:
   \[
     r_{a} = \cos\frac{\theta}{2}, 
     r_{b} = \sin\frac{\theta}{2} \]
   The use of $\theta/2$ is arbitrary for now but will work out well later.
   \item<3-> We can thus characterize $\ket{\psi}$ using only \alert{two (angular) real parameters}:
   \[ \ket{\psi} = \cos{\frac{\theta}{2}}\ket{0} + \Polar{}{\phi}\sin{\frac{\theta}{2}}\ket{1}
   \]
\end{itemize}

    
\end{frame}
