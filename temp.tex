\SetTitle{99}{temp}{faster}{99}

\begin{frame}{Measurement in a basis other than the computational basis}{Overview}

\begin{itemize}[<+->]
    \item Qiskit, in emulation and on the available hardware, performs measurements only in the computational basis.
    \item Recall that \QZero{} and \QOne{} are the North and South poles of the Bloch sphere, respectively.
    \item Some computations require measuring in a \emph{different} basis.
    \item How are we to achieve that?
    \item We can rotate from our desired basis to the computational basis.
    \item We then measure in the computational basis, which causes collapse upon measurement to~\QZero{} or~\QOne{}
    \item If there is more to the computation, then we must rotate back so that we are in the corresponding eigenstate of our desired basis.
\end{itemize}
    
\end{frame}

\begin{frame}{Example}{Measurement in the \NamedGate{X} (Hadamard) basis}
\Vskip{-4em}\TwoColumns{%
\begin{itemize}
    \item<1-> Consider the observable \PauliX{} with eigenvectors \visible<2->{\alert<2>{\ket{+}}} \visible<3->{and \alert<3>{\ket{-}}.}
    \item<4-> If we had an \NamedGate{X} measurement device, then the measured qubit would collapse to~\alert<4>{\ket{+}} \visible<5->{ or~\alert<5>{\ket{-}}}
    \item<6-> The detector reports the corresponding \ColorTwo{eigenvalue}, revealing the qubit's collapsed state.
    \item<7-> \PauliX{} is both a unitary gate and a measurement operator.
\end{itemize}
}{%
\Vskip{-2em}\begin{alignat*}{5}
    \visible<2->{\XMatrix{} & \alert<2>{\PHad{}} & \mbox{ = } & \ColorTwo{+1}  & \alert<2,4>{\PHad{}}  \\}
    \visible<3->{\XMatrix{} & \alert<3>{\PHadm{}} & \mbox{ = } & \ColorTwo{-1}  & \alert<3,5>{\PHadm{}} }
\end{alignat*}
}
    
\end{frame}