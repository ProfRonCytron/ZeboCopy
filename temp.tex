\SetTitle{99}{temp}{faster}{99}

{
\def\R#1#2#3{#1 & #2 & #3}
\def\RA#1#2#3#4{\alert<#1>{#2} & \alert<#1>{#3} & \alert<#1>{#4}}
\begin{frame}{Observations using our example}{Building intuition about finding the secret~$s$}
\Vskip{-3em}\TwoUnequalColumns{0.3\textwidth}{0.7\textwidth}{%
\begin{center}
    \begin{tabular}{ccc}
    \multicolumn{3}{c}{$s=101$} \\
    $w$ & \And{w}{s} & \DotP{w}{s} \\ \hline
        \RA{2-3}{000}{000}{0} \\
        \R{001}{001}{1} \\
        \RA{2-3}{010}{000}{0} \\
        \R{011}{001}{1} \\
        \R{100}{100}{1} \\
        \RA{2-3}{101}{101}{0} \\
        \R{110}{100}{1} \\
        \RA{2-3}{111}{101}{0}
    \end{tabular}
\end{center}
}{%
\begin{itemize}
    \item<1-> The table shows \DotP{w}{s} for each basis state \ket{w}.
    \item<2-> The only \emph{observable} states \ket{w} are those for which $\DotP{w}{s}=0$ for our secret~$s$.  All other states have~$0$ amplitude and thus no probability of measurement.
    \item<3-> Each of the \alert{four} states is observed with probability $\frac{1}{4}$.
    \begin{description}
        \item[000]
        \item[010]
        \item[101]
        \item[110]
    \end{description}
\end{itemize}
}
    
\end{frame}
\def\S#1#2#3#4{#1 & 0 & \visible<3->{#2} & \visible<5->{#3} & \visible<7->{#4}}
\begin{frame}{What does each $w$ tell us about $s$?}
\TwoUnequalColumns{0.5\textwidth}{0.5\textwidth}{%
\begin{center}
    \begin{tabular}{ccccc}
                    & \multicolumn{4}{c}{Possible observation $w$} \\
                    & \alert<1-2>{000} & \alert<3-8>{010} & 111 & 101 \\
   $s$? & {\small\DotP{s}{000}} & {\small\DotP{s}{010}} & {\small\DotP{s}{111}} & {\small\DotP{s}{101}} \\ \hline%
        \invisible<9->{\S{000}{0}{0}{0}} \\
         \invisible<15-17>{\S{\alert<11-17,19->{001}}{0}{1}{1}} \\
         \invisible<8-17>{\S{\alert<4-17>{010}}{1}{1}{0}} \\
         \invisible<8-17>{\S{\alert<5-17,20->{011}}{1}{0}{1}} \\
         \invisible<15-17>{\S{\alert<13-17,21->{100}}{0}{1}{1}} \\
         \S{\alert<16>{101}}{0}{0}{0} \\
         \invisible<8-17>{\S{\alert<6-17,22->{110}}{1}{0}{1}} \\
         \invisible<8-17>{\S{\alert<7-17>{111}}{1}{1}{0}}
    \end{tabular}
\end{center}}{%
\only<1-2>{
\begin{itemize}
    \item We can always observe \TensSupProd{0}{n} because its dot product with any~$s$ is $0$.
    \item However, it does not narrow the field of possible $s$ candidates.
    \item<2-> Any putative $s$ whose inner product with the observed~$w$ is $1$ cannot be our secret.
\end{itemize}}%
\only<3-9>{%
\begin{itemize}
    \item<3-> Observing 010 eliminates
    \begin{itemize}
        \item<4-> 010
        \item<5-> 011
        \item<6-> 110 
        \item<7-> 111
    \end{itemize}
    \item<8-> Our secret $s$ could be any of the remaining values except $000$, which is never our secret.
    \item<9-> So we can eliminate it from consideration.
\end{itemize}}%
\only<10-17>{%
\begin{itemize}
    \item<10-> Observation 111 eliminates
    \begin{itemize}
        \item<11-> 001
        \item<12-> 010 (already gone)
        \item<13-> 100 
        \item<14-> 111 (already gone)
    \end{itemize}
    \item<15-> So we eliminate those.
    \item<16-> We find our secret $s=101$ as the only possibility remaining.
    \item<17-> For completeness we show observation 101 next.
\end{itemize}}
\only<18->{%
\begin{itemize}
    \item<18-> Observation 101 eliminates
    \begin{itemize}
        \item<19-> 001
        \item<20-> 011
        \item<21-> 100
        \item<22-> 110
    \end{itemize}
\end{itemize}}
}%
\end{frame}
}
\begin{frame}{Some concerns about this approach}{This small example may be deceiving}

\begin{itemize}[<+->]
    \item How large are these tables for $n$~qubits?
    \item They are $\Theta(2^n)$, which is as bad as the classical time bound.
    \item What is the count of observations $w$ such that $\DotP{w}{s}=0$?
    \item Half of the basis states, so this count is also $\Theta(2^n)$.   
    \item We better not need to see very many observations to find $s$.
    \item And how do we find $s$ without eliminating impossible solutions from a large table?
    \item We will focus on the bits of $s$, as there are only $n$ of them.
    \item We will use Gaussian elimination, whose complexity is $\Theta(n^{3})$
\end{itemize}
    
\end{frame}

\begin{frame}{Our example revisited}{We use logic to determine the bits of $s$}

    
\end{frame}

\begin{frame}{Frame Title}

\begin{center}
\begin{DotPBox}{4}{0.5}{6}
\Qbit{0}{v_1}
\Qbit{1}{v_2}
\Qbit{2}{\cdot}
\Qbit{3}{\cdot}
\Qbit{4}{\cdot}
\Qbit{5}{v_n}
\end{DotPBox} \\
\begin{DotPBox}{4}{0.5}{6}
\Qbit{0}{w_1}
\Qbit{1}{w_2}
\Qbit{2}{\cdot}
\Qbit{3}{\cdot}
\Qbit{4}{\cdot}
\Qbit{5}{w_n}
\end{DotPBox}
\end{center}
    
\end{frame}