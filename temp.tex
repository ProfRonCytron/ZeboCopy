\SetTitle{99}{temp}{faster}{99}





\begin{frame}{On the nature of a universal set of gates}{Some thoughts \LinkArrow{https://www.scottaaronson.com/qclec/16.pdf}}

\Vskip{-2em}Using \href{https://www.youtube.com/watch?v=LVIaZSYLwtE}{calculatus eliminatus} we first considers sets of gates that \emph{cannot} work:
\begin{description}
\item<1->[no superposition]  The gate set \Set{\NamedGate{CNOT}} cannot create superposition on its own.  \only<1>{It maps basis states to other basis states, but cannot introduce superposition where there was none before. \begin{center}\CNOTMatrix\end{center}}
\item<2->[no entanglement]  The set \Set{\Hadamard} creates superpositions, but not entanglement.\only<2>{  Any gate that operates only a single qubit cannot create entanglement.}
\item<3->[no phasing]  \Set{\NamedGate{CNOT},\Hadamard} cannot introduce phase.\only<3>{ These gates create superpositions and entanglement, but given real inputs they can create only real outputs. A gate must be able to create \emph{complex outputs} from real inputs to introduce phase.}
\item<4->[the \emph{stabilizer} gates] \Set{\NamedGate{CNOT},\Hadamard,\NamedGate{S}}, where $\NamedGate{S}=\SQBG{}{1}{0}{0}{\NiceI}$.  This  set includes the \href{https://en.wikipedia.org/wiki/Quantum_logic_gate\#Phase_shift_gates}{phase gate} \NamedGate{S}, so it has none of the above deficiencies, but it is not universal.  \only<4>{The \href{https://en.wikipedia.org/wiki/Gottesman\%E2\%80\%93Knill_theorem}{Gottesman--Knill theorem} proves that circuits comprised of these gates can be simulated classically in polynomial time, so they cannot capture problems for which quantum gives exponential speedup over classical computing.  We study this later.}

\end{description}

\end{frame}

\begin{frame}{Towards a universal gate set}{What are our concerns?}
\begin{description}
  \item[size] How many inputs must a quantum gate accept?
  \only<1>{%
\begin{itemize}
    \item Consider a unitary transformation on $n$ qubits, where arbitrary entanglement might be a pre- or post-condition of the transformation.
    \item Must we realize this using a gate that handles $n$~qubits?
    \item As a classical version of this problem, consider computing a function of $n$~bits that is true if and only if all the $n$~bits are true: $\mbox{And}\left(q_{1},  q_{2}, \ldots, q_{n}\right)$.
    \item We might think that this operation must take in all the bits to produce an answer, but we know we can break this down into operations involving only two bits at a time.
\end{itemize}}
  \item[accuracy] How closely can we approximate a quantum computation?
  \only<2>{%
  \begin{itemize}
      \item Quantum systems (states and gates) can be specified using irrational constants.
      \item The number of such components is thus uncountably infinite.
      \item We necessarily approximate the behavior of such systems.
      \item How well do measurements of a finite quantum system approximate measurements of its ideal system?
      \item As a classical version of this problem, consider decisions based on real arithmetic
      \item On a computer this is carried out in (finite) floating-point.  How does that approximation affect decisions?
  \end{itemize}}
\end{description}
\end{frame}
{
\def\CC{\alt<3->{$0$}{\CFCirc{\RCone}}}
\def\CD{\alt<3->{\CFCirc{\RCtwo}}{\CFCirc{\RCone}}}
\def\CE{\alt<3->{$1$}{\CFCirc{\RCone}}}
\begin{frame}{Two-level quantum systems are exact and universal}{We do not need quantum gate with more than 2 inputs and outputs}

\TwoUnequalColumns{0.63\textwidth}{0.37\textwidth}{%
\only<1-4>{%
\begin{itemize}
    \item<1-> We begin with a quantum gate for an~$n$~qubit system.
    \item<2-> Conceptually this can be viewed as a~$2^{n}\times2^{n}$ matrix~$U$.
    \item<3-> We apply transformations to $U$, each of which operates only on 2~qubits at a time, to reduce $U$ \ColorTwo{to a form} where it also operates only on $2$~qubits.
    \item<4-> The matrix is \emph{not} necessarily tensor-factorable (the \NamedGate{CNOT} gate is an example), so in what sense does it operate only on 2~qubits?
\end{itemize}}%
\only<5->{%
\begin{itemize}
    \item<5-> The matrix on the right acts as identity for every basis vector \emph{except} \ket{\TensSupProd{0}{n-1}0} and \ket{\TensSupProd{0}{n-1}1}.
    \item<6-> If we look at the first $n-1$~qubits, if they are any pattern \emph{except} \TensSupProd{0}{n-1}, then the unitary gate corresponding to the matrix does not act on any qubit.
\end{itemize}
}
}{%

\[2^{n}\begin{pmatrix*}
 \CE & \CC &  \CC & \cdots& \CC & \CC \\
 \CC & \CE &  \CC & \cdots& \CC & \CC \\
 \CC & \CC &  \CE & \cdots& \CC & \CC \\
 \RVDots & \RVDots & \RVDots  & \RVDots & \RVDots  & \RVDots\\
 \CC & \CC &  \CC & \cdots& \CD & \CD \\
 \CC & \CC & \CC & \cdots& \CD & \CD
\end{pmatrix*}\]
}
\end{frame}}
