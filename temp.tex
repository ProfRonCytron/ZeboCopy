\SetTitle{99}{temp}{faster}{99}



\begin{frame}{Our example revisited}{We use logic to determine the bits of $s$}

\begin{itemize}
    \item We consider $s=s_{1}s_{2}s_{3}$ so we can reason about the bits of $s$ separately.  Since $s$ does not necessarily denote a binary numerical value, we are free to number these left-to-right, as shown.
    \item Each observable value $w$ can provide clues about the bits of $s$, as follows:
\begin{description}
   \item[010]  Because $\DotP{010}{s_{1}s_{2}s_{3}}=0$, we can conclude $s_{2}=0$.
   \item[101]  We now require \[
   \Implies{\DotP{101}{s_{1}0s_{3}}=0}{\Xor{s_1}{s_3}=0}\]
   Thus $s_1$ and $s_3$ are either both~$0$ or both~$1$.
\end{description}
   \item Since $s=000$ is not allowed, we must have the latter case.  
   \item This yields the correct answer, $s=101$
\end{itemize}

    
\end{frame}
{
\def\L#1{\raisebox{-0.65em}{#1}}
\begin{frame}{Formalizing this logic}{Using Gaussian elimination}
\begin{center}
\begin{tabular}{rclcc}
\begin{DotPBox}{3}{1}{3}
\Qbit{0}{1}
\Qbit{1}{1}
\Qbit{2}{0}
\end{DotPBox} & & \begin{DotPBox}{1}{1}{1}\Qbit{0}{s_1}\end{DotPBox}  & & \L{0}\\
\begin{DotPBox}{3}{1}{3}
\Qbit{0}{1}
\Qbit{1}{0}
\Qbit{2}{1}
\end{DotPBox}& \L{\FCirc{0.3}} & \begin{DotPBox}{1}{1}{1}\Qbit{0}{s_2}\end{DotPBox}  &\L{=}& \L{0}\\
\begin{DotPBox}{3}{1}{3}
\Qbit{0}{1}
\Qbit{1}{1}
\Qbit{2}{0}
\end{DotPBox}& & \begin{DotPBox}{1}{1}{1}\Qbit{0}{s_3}\end{DotPBox} & & \L{0}
\end{tabular}
\end{center}
\begin{itemize}
    \item Each $w$ is a row of the above matrix, and its dot product with the column vector~$s$ must be $0$.
    \item To obtain a solution, we need $n$ linearly independent rows.
    \item Performing \href{https://en.wikipedia.org/wiki/Gaussian_elimination}{Gaussian elimination} on the system to obtain $s$ takes $\Theta(n^{3})$ time using the most common algorithm.  Multiplication and addition are \And{}{} and \Xor{}{} here.
\end{itemize}
    
\end{frame}



\begin{frame}{Larger example, $n=4$ qubits}{We don't know $s$ yet}
\TwoColumns{%
\only<1-13>{%
Suppose we see $w$ observations in the following order:
\begin{itemize}
    \item<2-> 0101
    \item<4-> 1010
    \item<6-> 0000
    \item<8-> 1111
    \item<10-> 1011
    \item<12-> 0001
\end{itemize}}%
\only<14->{%
\begin{itemize}
    \item<14-> From row~4 we obtain $s_{4}=0$.
    \item<15-> So we know its value.
    \item<16-> From row~1 we now obtain $s_{2}=0$.
    \item<18->Row~2 or ~3 can be used to show $s_{1}=s_{3}=1$
    \item<20-> We thus obtain $s=1010$.
\end{itemize}
}
}{%
\begin{center}
\begin{tabular}{rclcc}
\visible<3->{\begin{DotPBox}[scale=0.7]{3}{1}{4}
\Qbit{0}{0}
\Qbit{1}{\alert<16>{1}}
\Qbit{2}{0}
\Qbit{3}{1}
\end{DotPBox}} & & \begin{DotPBox}[scale=0.7]{1}{1}{1}\Qbit{0}{\alt<19->{1}{\alert<18>{s_1}}}\end{DotPBox}  & & \L{0}\\
\visible<5->{\begin{DotPBox}[scale=0.7]{3}{1}{4}
\Qbit{0}{1}
\Qbit{1}{0}
\Qbit{2}{1}
\Qbit{3}{0}
\end{DotPBox}}& \L{\FCirc{0.3}} & \begin{DotPBox}[scale=0.7]{1}{1}{1}\Qbit{0}{\alert<16>{\alt<17->{\alert<17>{0}}{s_2}}}\end{DotPBox}  &\L{=}& \L{0}\\
\visible<11->{\begin{DotPBox}[scale=0.7]{3}{1}{4}
\Qbit{0}{1}
\Qbit{1}{0}
\Qbit{2}{1}
\Qbit{3}{1}
\end{DotPBox}}& & \begin{DotPBox}[scale=0.7]{1}{1}{1}\Qbit{0}{\alt<19->{1}{\alert<18>{s_3}}}\end{DotPBox} & & \L{0} \\
\visible<13->{\begin{DotPBox}[scale=0.7]{3}{1}{4}
\Qbit{0}{0}
\Qbit{1}{0}
\Qbit{2}{0}
\Qbit{3}{\alert<14-15>{1}}
\end{DotPBox}}& & \begin{DotPBox}[scale=0.7]{1}{1}{1}\Qbit{0}{\alt<15->{\alert<15>{0}}{\alert<14>{s_4}}}\end{DotPBox} & & \L{0}
\end{tabular}
\end{center}
\BigSkip{}
\only<3>{%
We accept this as the first equation and set the matrix values accordingly.
}%
\only<5>{%
This is another, independent $w$ and we use it as the second equation.
}%
\only<9>{%
This value is in the span of the first two ($\Xor{0101}{1010}=11111$) so it does not contribute a linearly independent row to our developing matrix.  We do not use this observation.}%
\only<7>{%
$w=0000$ never reveals anything of $s$, so we reject this.}%
\only<11>{%
This is our third equation.
}%
\only<13>{%
Our fourth equation
}
}
\end{frame}

\def\G#1#2#3{\alert<1>{#1} & #2 & #3}
\def\B#1#2#3{\invisible<2->{#1 & #2 & #3}}
\begin{frame}{Showing the work}{Now that we know $s$}

\TwoColumns{%
\begin{center}
    \begin{tabular}{ccc}
    $w$ & \And{w}{s} & \DotP{w}{s} \\ \hline
        \G{0000}{0000}{0} \\
        \G{0001}{0000}{0} \\
        \B{0010}{0010}{1} \\
        \B{0011}{0010}{1} \\
        \G{0100}{0000}{0} \\
        \G{0101}{0000}{0} \\
        \B{0110}{0010}{1} \\
        \B{0111}{0010}{1}
    \end{tabular}
\end{center}
}{%
\begin{center}
    \begin{tabular}{ccc}
    $w$ & \And{w}{s} & \DotP{w}{s} \\ \hline
        \B{1000}{1000}{1} \\
        \B{1001}{1000}{1} \\
        \G{1010}{1010}{0} \\
        \G{1011}{1010}{0} \\
        \B{1100}{1000}{1} \\
        \B{1101}{1000}{1} \\
        \G{1110}{1010}{0} \\
        \G{1111}{1010}{0}
    \end{tabular}
\end{center}
}%
\BigSkip{}
\visible<2->{These are the values of $w$ we might observe.}


\end{frame}
}