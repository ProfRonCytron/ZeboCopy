\SetTitle{99}{temp}{faster}{99}


{
\def\TTerm#1{\TwoSupOp{\QZero}{\ExpPhase{2\pi\left(2^{#1}\W{}\right)}\QOne}{+}}
\def\N{\textcolor{purple}{\ensuremath{n}}}\def\W{\ColorTwo{\ensuremath{w}}}\def\Y{\ColorFour{y}}
\begin{frame}{Our theorem}{Needs a proof, but first a restatement}
    \Vskip{-3em}\begin{align*}
 \RootTwoN{\N}\alt<3->{\alert<3>{\SumBV{\Y}{\N}}}{\SumPH{\Y}{\N}} \ExpPhase{2\pi \W{}\Y} \ket{\Y} =& \TTerm{\N{}-1}  \\
 \TensOp{} & \TTerm{\N{}-2} \\
 \TensOp{} & \cdots \\
 \TensOp{} & \TTerm{\N{}-\N{}} 
\end{align*}
\only<1-2>{%
\begin{itemize}
    \item<1-> The use of \Y{} as the variable for the summation and as \ket{\Y} requires us to view \Y{} as a \href{https://en.wikipedia.org/wiki/Number}{number}, taking on the sequence $0, 1, \ldots,2^{\N}-1$,
    \item<2-> For the purposes of our proof it is more useful to view \Y{} as a base-2 \href{https://en.wikipedia.org/wiki/Numeral}{numeral}:  a string of \N{}~bits.
\end{itemize}}%
\only<3-4>{%
\begin{itemize}
    \item The summation in our theorem changes as shown here.
    \item The summation is now over all bit strings of length~$n$.  These must be interpreted numerically in the exponent.
\end{itemize}
}%
\only<5->{%
\begin{itemize}
    \item<5-> In truth this is more accurate, as the tensor products on the right pile up bit strings, not integers.
    \item<6-> To help the proof fit on these slides, we will define functions that generate portions of the theorem.
\end{itemize}}
\end{frame}
%%%
%%%
%%%
\def\ThSumTerm#1{%
\ensuremath{\ExpPhase{2\pi \W{} #1} \ket{#1}}}
\def\ThSum#1{%
\ensuremath{\RootTwoN{#1}\SumBV{\Y}{#1} \ThSumTerm{\Y}}}
%%%
%%%
\def\ThTop#1{%
\fbox{\adjustbox{valign=t, width=0.97\textwidth}{
$\ColorOne{\SS{#1}=\ThSum{#1}}\ \mbox{\vrule width 1.5pt height 2em depth 1.3em}\ \ColorTwo{\TT{#1}= \TTerm{#1-1}\TensOp{}\TTerm{#1-2}\TensOp{}\cdots\TensOp\TTerm{#1-#1}}$}}}
\def\SS#1{\ColorOne{\ensuremath{S(#1)}}}
\def\TT#1{\ColorTwo{\ensuremath{T(#1)}}}

\begin{frame}{Theorem}{And proof}
\Vskip{-3em}\begin{theorem}
\TwoColumns{%
\begin{align*}
    S(\N) =& \ThSum{\N}
\end{align*}
}{%
  \begin{align*}
    T(\N) = &\TTerm{\N-1}  \\
 \TensOp{} & \TTerm{\N-2} \\
 \TensOp{} & \cdots \\
 \TensOp{} & \TTerm{\N-\N} 
\end{align*}}
\begin{center}
Define $P(\N):  S(\N)=T(\N)$ \\
\Forall{\N\geq 1}{P(\N)}\end{center}
\end{theorem}
Proof by induction.  We first show $P(1)$ and then show $\Implies{P(k-1)}{P(k)}$.
    
\end{frame}

\begin{frame}{Base case}{Show $P(1): S(1)=T(1)$}
\Vskip{-4em}\begin{center}\ThTop{\N}\end{center}
\SmallSkip{}
\begin{Reasoning}
\Reason{1}{Expansion of sum}
\Reason{2}{Simplify constants in exponents}
\Reason{3}{To satisfy the tensor-factored form in the theorem}
\end{Reasoning}

\Vskip{-2em}\ScrollProof{1}{3}{%
   \Next{\Three}{  S(1) &= \ThSum{1} \\
    &=\TwoSupOp{\ThSumTerm{\ColorFour{0}}}{\ThSumTerm{\ColorFour{1}}}{+} \\}
    \Next{\Two}{           &= \TwoSupOp{\QZero}{\ExpPhase{2\pi\W}\QOne}{+} \\}
      \Next{\One}{         &= \TTerm{\ColorOne{1-1}} = T(1)}
}
\visible<3->{\Vskip{-3em}\QED{}}
\end{frame}

\def\W{\ensuremath{w}}
\def\L{\ColorOne{\ensuremath{L}}}\def\R{\ColorTwo{\ensuremath{R}}}
\begin{frame}{Induction step}{With $P(n): S(n)=T(n)$, show \Implies{P(k-1)}{P(k)}}
\Vskip{-4.3em}\begin{center}\ThTop{\N}\end{center}
\SmallSkip{}

\begin{Reasoning}
\Reason{1}{\Highlight{orange}{6.69}{0.8}{6.5}{0.75}Note \R{} is the right-hand side excluding the first term}
\Reason{2}{Using induction hypothesis $P(k-1)$}
\Reason{3}{\Highlight{orange}{3.3}{0.8}{2.8}{0.75}Tensor both sides with the \ColorThree{same term}, which completes the right-hand side}
\Reason{7}{Carry out the tensor product}
\end{Reasoning}
{\small
\Vskip{-1.5em}\begin{align*}
\mbox{Let }\L=& \ColorOne{\ThSum{k-1}} \\
 \mbox{and }    \R=& \ColorTwo{\TTerm{k-2} \TensOp{}\cdots\TensOp{}\TTerm{k-k}}
\end{align*}
\Vskip{-3em}\ScrollProof{2}{4}{%
\Next{\Three}{\R &= \L \\}
\gdef\CarryOne{\TensProd{\ColorThree{\TTerm{k-1}}}{\R}} \gdef\CarryTwo{\TensProd{\ColorThree{\TTerm{k-1}}}{\L}}
\Last{\CarryOne &= \CarryTwo}
}%
\ScrollProof{5}{8}{%
\Next{\Four}{&\CarryOne = \CarryTwo \\}
\Next{\Three}{&\Indent{2em} = \TensProd{\ColorThree{\TTerm{k-1}}}{\ColorOne{\ThSum{k-1}}}\\}
\Last{&\Indent{2em} = \RootTwoN{k}\left( \ColorThree{\QZero}\ColorOne{\SumBV{y}{k=1}\ExpPhase{2\pi wy}\ket{y}} + \ColorThree{\ExpPhase{2\pi(2^{k-1}w)}\QOne{}}\ColorOne{\SumBV{y}{k=1}\ExpPhase{2\pi wy}\ket{y}}\right)}
}}
\end{frame}
}


