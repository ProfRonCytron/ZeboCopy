\SetTitle{99}{temp}{faster}{99}

\section*{The Z basis}
\begin{frame}{Reconstructing the computational basis}{A single qubit}

\TwoUnequalColumns{0.6\textwidth}{0.4\textwidth}{%
\only<1-2>{\begin{itemize}
    \item<1-> We have a device that performs measurements, collpasing state as follows with the associated eigenvalues:
       \begin{center}
        \begin{tabular}{cc}
        Eigentate & Eigenvalue \\
        \ColorOne{\QZero} & \ColorThree{$+1$} \\
        \ColorTwo{\QOne} & \ColorFour{$-1$} 
        \end{tabular}\end{center}
    \item<2-> The matrix $T$ containing those eigenstates as columns transforms the current basis into the computational basis, which is the identity matrix, since the bases are the same.


\end{itemize}}
\only<3->{\begin{itemize}
    \item The associated \emph{measurement} operator~$M$ is formed as follows:
    \[
    M = T\times D \times \Conj{T}
    \]
    where $D$ is a diagonal matrix containing the eigenvalues in order:
    \begin{align*}
        D =& \SQBG{\relax}{\ColorThree{+1}}{0}{0}{\ColorFour{-1}} \\
        \visible<4->{M = & \ColorFive{\ZMatrix}}
    \end{align*}
\end{itemize}}
}{%
\visible<4->{\begin{align*}
    \ColorFive{\ZMatrix{}}\ColorOne{\PZero{}} &= \ColorThree{+1}\PZero{} \\
    \ColorFive{\ZMatrix{}}\ColorTwo{\POne{}} &= \ColorFour{-1}\POne{}
\end{align*}}
\only<1-4>{%
\SmallSkip{}
\visible<2->{\[T =
\IMatrix
\]}}%
\only<5->{%

$M$ is our familiar \ColorFive{Z} matrix, which we have been claiming as our measurement operator for the computational basis.
}
}
    
\end{frame}

\begin{frame}{Reconstructing the computational basis}{Two qubits}
\TwoUnequalColumns{0.6\textwidth}{0.4\textwidth}{%
\begin{itemize}
    \item<1-> If \PauliZ{} is our operator for one qubit, then we expect \TensProd{\PauliZ}{\PauliZ} to be the measurement operator for two qubits.
    \item<2-> Its eigenstates and their eigenvalues are:
    \begin{center}
        \begin{tabular}{cc}
        State & Eigenvalue \\
        \ket{00} & $+1$ \\
        \ket{01} & $-1$ \\
        \ket{10} & $-1$ \\
        \ket{11} & $+1$
         \end{tabular}
    \end{center}
    \item<3-> The matrix of its eigenvalues is again \Identity.
\end{itemize}
}{%
\[
\TensProd{\PauliZ}{\PauliZ} = 
\begin{pmatrix*}[r]
1 & 0 & 0 & 0 \\
0 & -1 & 0 & 0 \\
0 & 0 & -1 & 0 \\
0 & 0 & 0 & 1
\end{pmatrix*}
\]
\SmallSkip\visible<3->{\[T =
\begin{pmatrix*}[r]
1 & 0 & 0 & 0 \\
0 & 1 & 0 & 0 \\
0 & 0 & 1 & 0 \\
0 & 0 & 0 & 1
\end{pmatrix*}
\]}
}
\end{frame}

\begin{frame}{Generally}{For $n$ qubits}
\begin{theorem}
In the computational basis of an $n$-qubit system, the operator \TensSupProd{\PauliZ}{n} has $2^n$ eigenstates: $\ket{x}\ |\ \AllBits{x}{n}$.
\SmallSkip{}
Eigenstate $\ket{x}$ has eigenvalue $\pm 1$ that is consistent with the parity of $x$:  even-parity states have eigenvalue $+1$ and odd-parity states have eigenvalue $-1$.
\end{theorem}
    
\end{frame}

\section*{Measurement in another basis}

\begin{frame}{Measurement in a basis other than the computational basis}{Overview}

\begin{itemize}[<+->]
    \item Qiskit, in emulation and on the available hardware, performs measurements only in the computational basis.
    \item Recall that for a single qubit, \QZero{} and \QOne{} are the North and South poles of the Bloch sphere, respectively.
    \item How can we measure that qubit in a \emph{different} basis?
    \begin{enumerate}
    \item We can transform state from our desired basis to the computational basis.
    \item We then measure in the computational basis, which causes collapse upon measurement to~\QZero{} or~\QOne{}
    \item If there is more to the computation, then we must transform back so that we are in the corresponding eigenstate of our desired basis.
    \end{enumerate}
    \item In this discussion it is helpful to keep in mind the basis under consideration.  For example, we know
    \(\ket{+} = \QPlus \).  These are two names for the same state but the right-hand side is expressed in the computational basis.
\end{itemize}
    
\end{frame}

\begin{frame}{More detail}{On the three steps}

\Vskip{-3em}\begin{enumerate}
    \item<1-> Transform our state into the computational basis.  We establish a mapping between our basis vectors and the computational basis vectors.  All states then transform correctly by linearity.
    
    For example, we might transform from the \TensProd{\PauliX}{\PauliX} to \TensProd{\PauliZ}{\PauliZ} as follows:
    \begin{center}
        \begin{tabular}{c@{\ $\alt<3>{\leftarrow}{\rightarrow}$\ }c}
        Our state & \TensProd{\PauliZ}{\PauliZ} state \\
        \hline
        \ColorOne{\ket{++}} & \ColorOne{\ket{00}} \\
        \ColorTwo{\ket{+-}} & \ColorTwo{\ket{01}} \\
        \ColorThree{\ket{-+}} & \ColorThree{\ket{10}} \\
        \ColorFour{\ket{--}} & \ColorFour{\ket{11}}
        \end{tabular}
    \end{center}
    Any such mapping suffices, but the above mapping is somewhat standard.  
    \item<2-> We measure in the computational basis, using the mapping to interpret the results in our basis.  Measuring \ColorThree{\ket{10}} is akin to measuring \ColorThree{\ket{-+}} in our basis.
    \item<3-> To transform back, we use the inverse of the above mapping, creating the measurement outcome in our basis.
\end{enumerate}
    
\end{frame}

\begin{frame}{Transforming state to and from another basis}{Overview}

\begin{itemize}[<+->]
    \item For an $n$-qubit system, consider a measurement in our \emph{desired} basis.
    \item We find $2^n$ orthonormal eigenstates in our desired basis. These are the distinct, collapsed outcomes we expect from measurement in our basis.
    \item Because these will eventually map to the computational basis, we must determine that half of our eigenstates have eigenvalue~$+1$ and the other half~$-1$.
    \item We form a matrix $T$ whose columns are those eigenstates.  The matrix must be unitary:
    \begin{itemize}
        \item It is halfway there, as the columns are orthonormal.  
        \item We verify $\Conj{T}T=\Identity$.
    \end{itemize}
    \item Matrix $T$ maps the computational basis to eigenstates of our desired basis.
    \item The matrix's inverse performs the mapping in the other direction.
    \item Because the matrix is unitary, \Conj{T} is its inverse.
\end{itemize}
\end{frame}


\begin{frame}{Visual interpretation for a single qubit}{Measurement in the \NamedGate{X} (Hadamard) basis}
\Vskip{-4em}\TwoColumns{%
\only<1-8>{\begin{itemize}
    \item<1-> Consider the observable \PauliX{} with eigenvectors \visible<2->{\ColorOne{\ket{+}}} \visible<3->{and \ColorTwo{\ket{-}}.}
    \item<4-> If we had an \NamedGate{X} measurement device, then the measured qubit would collapse to~\ColorOne{\ket{+}} \visible<5->{ or~\ColorTwo{\ket{-}}}
    \item<6-> The detector reports the corresponding \ColorThree{eigenvalue}, revealing the qubit's collapsed state.
    \item<7-> \PauliX{} is both a unitary gate and a measurement operator.
\end{itemize}}%
\only<9-13>{%
\begin{itemize}
    \item<9-> Thanks to linearity we can map the \PauliX{} basis to the computational basis.
    \item<10-> Given state \QState{}, we require a transformation that maps
    \begin{center}
    \begin{tabular}{r@{$\mapsto$}l}
       \ColorOne{\ket{+}} & \QZero{} \\
       \ColorTwo{\ket{-}} & \QOne{}
    \end{tabular}\end{center}
    \item<11-> The \ColorFive{\Hadamard{}} gate performs this mapping.
    \item<12-> Measurement in the computational basis now yields the correct result.
    \item<13-> However, the qubit collapses to~\ket{0} or~\ket{1}; neither is a valid state after measurement in the \PauliX{} basis.
\end{itemize}
}%
\only<14->{%
\begin{itemize}
    \item<14-> Recall \Hadamard{} is its own inverse.
    \item<15-> Thus, after we measure, we apply another \Hadamard{} operator to obtain the correct state for further computation, if necessary.
    \item<16-> This is demonstrated in the \texttt{MeasureH} notebook.
\end{itemize}
}
}{%
\Vskip{-2em}
\begin{center}
\begin{TIKZP}
\alert<8-9>{\draw[->,thick] (0,0) -- (1,0) node[right] {\ \ket{1}};}
\alert<8-9>{\draw[->,thick] (0,0) -- (0,1) node[above] {\ket{0}};}
\textcolor<2,4,10->{\RCone}{\draw[->,thick] (0,0) -- (45:1) node[above] {\ket{+}};}
\textcolor<3,5,10->{\RCtwo}{\draw[->,thick] (0,0) -- (-45:1) node[below] {\ket{-}};}
\end{TIKZP}
\end{center}
\Vskip{-2.5em}\only<1-6>{\begin{alignat*}{5}
    \visible<2->{\XMatrix{} & \textcolor<2>{\RCone}{\PHad{}} & \mbox{ = } & \ColorThree{+1}  & \textcolor<2,4>{\RCone}{\PHad{}}  \\}
    \visible<3->{\XMatrix{} & \textcolor<3>{\RCtwo}{\PHadm{}} & \mbox{ = } & \ColorThree{-1}  & \textcolor<3,5>{\RCtwo}{\PHadm{}} }
\end{alignat*}}%
\only<8>{%
\SmallSkip{}

If we can measure only in the \alert{computational basis}, how do we perform measurements in the \PauliX{} basis?
}%
\only<11-14>{%
\begin{align*}
    \ColorFive{\HMatrix{}}\ColorOne{\PPlus{}} &= \PZero{} \\
    \ColorFive{\HMatrix{}}\ColorTwo{\PMinus{}} &= \POne{}
\end{align*}
}%
\only<15->{%
\begin{align*}
    \ColorFive{\HMatrix{}}\PZero{} &= \ColorOne{\PPlus{}} \\
    \ColorFive{\HMatrix{}}\POne{} &= \ColorTwo{\PMinus{}}
\end{align*}
}
}
    
\end{frame}

\begin{frame}{Example using two qubits}{Translation to and from the \PauliX{} basis}

\TwoUnequalColumns{0.6\textwidth}{0.4\textwidth}{%
\Vskip{-3em}\begin{itemize}
    \item<1-> A proposed mapping maintains eigenvalues:
    \begin{center}
        \begin{tabular}{cc@{\ $\rightarrow$\ }c}
        \multicolumn{2}{c}{Our basis} &
        \multicolumn{1}{c}{\PauliZ{} basis} \\
        Eigenstate & Eigenvalue & Eigenstate \\ \hline
        \ColorOne{\ket{++}} & $+1$ & \ket{00} \\
        \ColorTwo{\ket{+-}} & $-1$ & \ket{01} \\
        \ColorThree{\ket{-+}} & $-1$ & \ket{10} \\
        \ColorFour{\ket{--}} & $+1$ & \ket{11}
         \end{tabular}
    \end{center}
    \item<2-> $T$ is comprised of our eigenstates.
    \item<4-> As expected, \TensProd{\PauliX}{\PauliX} is the measurement operator, but we do not use it.
    \item<5-> We only need \Conj{T} (and $T$) to transform from (and into) our basis.
    
\end{itemize}
}{%
\visible<2->{\[T =
\frac{1}{2}\begin{pmatrix*}[r]
\alt<2>{\ColorOne{\ket{++}} \ColorTwo{\ket{+-}}\ColorThree{\ket{-+}}\ColorFour{\ket{--}}}{%
\ColorOne{1} & \ColorTwo{1} & \ColorThree{1} & \ColorFour{1} \\
\ColorOne{1} & \ColorTwo{-1} & \ColorThree{1} & -\ColorFour{1} \\
\ColorOne{1} & \ColorTwo{1} & \ColorThree{-1} & \ColorFour{-1} \\
\ColorOne{1} & \ColorTwo{-1} & \ColorThree{-1} & \ColorFour{1}}
\end{pmatrix*}
\]}\SmallSkip{}
\visible<4->{\[
\TensProd{\PauliX}{\PauliX} = 
\begin{pmatrix*}[r]
0 & 0 & 0 & 1 \\
0 & 0 & 1 & 0 \\
0 & 1 & 0 & 0 \\
1 & 0 & 0 & 0
\end{pmatrix*}
\]}
}
    
\end{frame}

\begin{frame}{Computations with measurements in another basis}{Transform$\ldots$measure$\ldots$untransform}

\TwoUnequalColumns{0.372\textwidth}{0.628\textwidth}{%
\adjustbox{scale=0.8,valign=t}{%
\begin{quantikz}
\lstick{\QZero{}} & \meter{\ColorOne{\ket{+}/\ket{-}}} & \gate{\alert<12-14>{\PauliZ}} &   \meter{\ColorTwo{\ket{+}/\ket{-}}}&\qw
\end{quantikz}}%
}{%
\visible<9->{%
\adjustbox{scale=0.77,valign=t}{%
\begin{quantikz}
\lstick{\QZero{}} & \gate{\visible<9->{\ColorOne{\Conj{T}}}} & \meter{\ColorOne{\visible<10->{\QZero{}/\QOne{}}}} & \gate{\visible<11->{\ColorOne{T}}} & \gate{\alert<12-14>{\PauliZ}} &   \gate{\ColorTwo{\visible<17->{\Conj{T}}}} & \meter{\ColorTwo{\visible<18->{\QZero{}/\QOne{}}}} & \gate{\ColorTwo{\visible<19->{\ColorTwo{T}}}} &\qw
\end{quantikz}}}%
}
\BigSkip{}
\only<1-6>{%
\begin{itemize}
    \item<1-> Our desired circuit is as shown on the left, with measurements in the \PauliX{} basis.
    \item<2-> The \ColorOne{first measurement} collapses state into \ColorOne{\ket{+}} or \ColorOne{\ket{-}} with equal likelihood.
    \item<3-> The \PauliZ{} gate flips the current state from~\ket{+} to~\ket{-}, or \textit{vice~versa} (recall the Bloch sphere).
    \item<4-> The \ColorTwo{second measurement} reports that state in the \PauliX{} basis as~\ColorTwo{\ket{+}} or~\ColorTwo{\ket{-}}.  The only possible measurements we can see are \ket{\ColorOne{+}\ColorTwo{-}} and \ket{\ColorOne{-}\ColorTwo{+}}.
    \item<5-> However, lacking ability to measure in the \PauliX{} basis, we must measure in the computational basis, suitably transforming before \emph{and after} the measurement.
    \item<6-> If we fail to restore state after the first measurement, the subsequent \PauliZ{} gate produces the wrong output.  Let's examine how that could happen.
\end{itemize}}%
\only<7->{%
\TwoUnequalColumns{0.65\textwidth}{0.35\textwidth}{%
\only<7-8>{%
\begin{itemize}
    \item Here is the matrix \Hadamard{} our transformation from the computational basis into \PauliX{}: $\PauliZ\mapsto \PauliX$. Each column is an eigenvector of \PauliX.
    \item<8-> It happens to be its own inverse, but let's regard~\Conj{T} properly as transforming $\PauliX\mapsto\PauliZ$.
\end{itemize}}%
\only<9-11>{%
 \begin{itemize}
     \item<9-> We wish to measure our initial state in the \PauliX{} basis, so we first transform $\PauliX\mapsto\PauliZ$ using~\ColorOne{\Conj{T}}.
     \item<10-> Now we \ColorOne{measure} in the computational (\PauliZ) basis.  We have arranged $T$ so that:
     \begin{center}
         \begin{tabular}{cc}
           Observe &  Interpret \\\hline
           \QZero & \ket{+} \\
           \QOne  & \ket{-}
         \end{tabular}
     \end{center}
     \item<11-> Now we transform $\PauliZ\mapsto\PauliX$ back to our basis using~\ColorOne{$T$}.  This restores the state to either~\ket{+} or~\ket{-} based on the measurement.
 \end{itemize}
}%
\only<12-15>{%
Ahead is our \alert<12-14>{\PauliZ{}} gate.  It acts on states of interest as follows:
\Vskip{-2.5em}\begin{center}
    \begin{tabular}{r@{$\rightarrow$}l|r@{$\rightarrow$}l}
         \QZero &  No change &
         \QOne  &  No change \\
         \ket{+} & \ket{-} & \ket{-} & \ket{+}
    \end{tabular}
\end{center}
\begin{itemize}
    \item<13-> The computational-basis measurement leaves us in state \QZero{} or \QOne{}.
    \item<14-> If we fail to transform back to \PauliX{}, the \alert<12-14>{\PauliZ{}} gate will have no effect on state.
    \item<15-> \ColorOne{$T$} is necessary to transform the standard-basis measurement into our \PauliX{} basis.
\end{itemize}
}%
\only<16->{%
\begin{itemize}
    \item<16-> Our \ColorTwo{second measurement} should confirm that the state switched in the \PauliX{} basis.
    \item<17-> We prepare for the measurement using \ColorTwo{\Conj{T}} to effect $\PauliX\mapsto\PauliZ$.
    \item<18-> We measure in the computational basis, using the previous table to interpret the outcome.
    \item<19-> We restore state using \ColorTwo{$T$} to effect $\PauliZ\mapsto\PauliX$. We don't really need that last gate because the computation is over.
\end{itemize}
}
}{%
\[T = \HMatrix \visible<8->{=\Conj{T}}\]
}
}
    
\end{frame}

