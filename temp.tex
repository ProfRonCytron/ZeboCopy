\SetTitle{99}{temp}{faster}{99}

\begin{frame}{Reconstructing the computational basis}{A single qubit}

\TwoUnequalColumns{0.6\textwidth}{0.4\textwidth}{%
\only<1-2>{\begin{itemize}
    \item<1-> We have a device that performs measurements, collpasing state as follows with the associated eigenvalues:
       \begin{center}
        \begin{tabular}{cc}
        Eigentate & Eigenvalue \\
        \ColorOne{\QZero} & \ColorThree{$+1$} \\
        \ColorTwo{\QOne} & \ColorFour{$-1$} 
        \end{tabular}\end{center}
    \item<2-> The matrix $T$ containing those eigenstates as columns transforms the current basis into the computational basis, which is  the identity matrix since the bases are the same.


\end{itemize}}
\only<3->{\begin{itemize}
    \item The associated \emph{measurement} operator~$M$ is formed as follows:
    \[
    M = T\times D \times \Conj{T}
    \]
    where $D$ is a diagonal matrix containing the eigenvalues in order:
    \begin{align*}
        D =& \SQBG{\relax}{\ColorThree{+1}}{0}{0}{\ColorFour{-1}} \\
        \visible<4->{M = & \ColorFive{\ZMatrix}}
    \end{align*}
\end{itemize}}
}{%
\visible<4->{\begin{align*}
    \ColorFive{\ZMatrix{}}\ColorOne{\PZero{}} &= \ColorThree{+1}\PZero{} \\
    \ColorFive{\ZMatrix{}}\ColorTwo{\POne{}} &= \ColorFour{-1}\POne{}
\end{align*}}
\only<1-4>{%
\SmallSkip{}
\visible<2->{\[T =
\IMatrix
\]}}%
\only<5->{%

$M$ is our familiar \ColorFive{Z} matrix, which we have been claiming as our measurement operator for the computational basis.
}
}
    
\end{frame}

\begin{frame}{Reconstructing the computational basis}{Two qubits}
\TwoUnequalColumns{0.6\textwidth}{0.4\textwidth}{%
\begin{itemize}
    \item<1-> If \PauliZ{} is our operator for one qubit, then we expect \TensProd{\PauliZ}{\PauliZ} to be the measurement operator for two qubits.
    \item<2-> Its eigenstates and their eigenvalues are:
    \begin{center}
        \begin{tabular}{cc}
        State & Eigenvalue \\
        \ket{00} & $+1$ \\
        \ket{01} & $-1$ \\
        \ket{10} & $-1$ \\
        \ket{11} & $+1$
         \end{tabular}
    \end{center}
    \item<3-> The matrix of its eigenvalues is again \Identity.
\end{itemize}
}{%
\[
\TensProd{\PauliZ}{\PauliZ} = 
\begin{pmatrix*}[r]
1 & 0 & 0 & 0 \\
0 & -1 & 0 & 0 \\
0 & 0 & -1 & 0 \\
0 & 0 & 0 & 1
\end{pmatrix*}
\]
\SmallSkip\visible<3->{\[T =
\begin{pmatrix*}[r]
1 & 0 & 0 & 0 \\
0 & 1 & 0 & 0 \\
0 & 0 & 1 & 0 \\
0 & 0 & 0 & 1
\end{pmatrix*}
\]}
}
\end{frame}

\begin{frame}{Generally}{For $n$ qubits}
\begin{theorem}
In the computational basis of an $n$-qubit system, the operator \TensSupProd{\PauliZ}{n} has $2^n$ eigenstates: $\ket{x}\ |\ \AllBits{x}{n}$.
\SmallSkip{}
Eigenstate $\ket{x}$ has eigenvalue $\pm 1$ that is consistent with the parity of $x$:  even-parity states have eigenvalue $+1$ and odd-parity states have eigenvalue $-1$.
\end{theorem}
    
\end{frame}

\begin{frame}{Mapping from other bases}{Overview}

\begin{itemize}
    \item We consider a system of $n$ qubits.
    \item We find $2^n$ orthonormal eigenstates in our desired basis. These are the collapsed outcomes we expect from measurement in that basis.
    \item Because these will eventually map to the computational basis, we must determine that half of our eigenstates have eigenvalue~$+1$ and the other half~$-1$.
    \item We form a matrix $T$ whose columns are those eigenstates.  The matrix must be unitary:
    \begin{itemize}
        \item It is halfway there, as the columns are orthonormal.  
        \item If the rows are not orthonormal, then we must use other eigenstates.
    \end{itemize}
    \item This matrix maps the comuputational basis to eigenstates of our desired basis.
    \item The matrix's inverse performs the mapping in the other direction.
    \item Because the matrix is unitary, its conjugate transpose is its inverse.
\end{itemize}
\end{frame}

\begin{frame}{Translation to and from the \PauliX{} basis}{An example where we already know the outcome}

\TwoUnequalColumns{0.6\textwidth}{0.4\textwidth}{%
\Vskip{-3em}\begin{itemize}
    \item Our orthonormal eigenstates and their eigenvalues are:
    \begin{center}
        \begin{tabular}{cc}
        State & Eigenvalue \\
        \ket{++} & $+1$ \\
        \ket{+-} & $-1$ \\
        \ket{-+} & $-1$ \\
        \ket{--} & $+1$
         \end{tabular}
    \end{center}
    \item The resulting matrix $T$ is shown to the right, and its inverse is its transpose.
    \item The operator that produces these eigenvalues is shown as well.  Unsurprisingly it is \TensProd{\PauliX}{\PauliX}, which is its own inverse.
    
\end{itemize}
}{%
\[T =
\frac{1}{2}\begin{pmatrix*}[r]
1 & 1 & 1 & 1 \\
1 & -1 & 1 & -1 \\
1 & 1 & -1 & -1 \\
1 & -1 & -1 & 1
\end{pmatrix*}
\]\SmallSkip{}
\[
\TensProd{\PauliX}{\PauliX} = 
\begin{pmatrix*}[r]
0 & 0 & 0 & 1 \\
0 & 0 & 1 & 0 \\
0 & 1 & 0 & 0 \\
1 & 0 & 0 & 0
\end{pmatrix*}
\]
}
    
\end{frame}