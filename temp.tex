\SetTitle{99}{temp}{faster}{99}

\section*{Magic square}
{
\def\MPC#1{\fbox{\hbox to 5ex{\hss\mbox{#1}\vrule width 0pt height 3ex depth 2ex\hss}}}
\def\XP#1{\hbox to 2ex{\hss#1\hss 1}}%
\def\PP{%
\XP{$+$}}%
\def\MM{%
\XP{$-$}}%
\def\MPSquare#1#2#3#4#5#6#7#8#9{%
{%
    \setlength{\tabcolsep}{0pt}%
    \adjustbox{valign=t}{\begin{tabular}{rrccc}
   \multicolumn{1}{c}{ } & \multicolumn{1}{c}{ } & \multicolumn{3}{c}{$\downarrow\ $\ColorTwo{Bob}$\ \downarrow$} \\
    & & 1 & 2 & 3 \\
        $\rightarrow\ $ & \ 1\ \ & \MPC{#1} & \MPC{#2} & \MPC{#3} \\
        \mbox{\ColorOne{Alice} }& \ 2\ \ & \MPC{#4} & \MPC{#5} & \MPC{#6} \\
        $\rightarrow\ $&\ 3\ \ & \MPC{#7} & \MPC{#8} & \MPC{#9} \\
    \end{tabular}}}}
\def\VS#1#2{\visible<#1>{#2}}
\begin{frame}{Introduction to the \href{https://en.wikipedia.org/wiki/Quantum_pseudo-telepathy\#The_Mermin\%E2\%80\%93Peres_magic_square_game}{Mermin--Peres magic square}}{This game is due to \href{https://en.wikipedia.org/wiki/N._David_Mermin}{David Mermin} and \href{https://en.wikipedia.org/wiki/Asher_Peres}{Asher Peres}}

\begin{itemize}[<+->]
    \item With CHSH, a quantum approach is probabilistically better than any classical approach but does not guarantee to win the game.
    \item The Mermin--Peres magic square is a puzzle that Alice and Bob can win at best $\frac{8}{9}$ of the time, but a quantum approach on a reliable quantum computer can always win.
    \item Alice and Bob share two EPR pairs, each initially in the state $\frac{\ket{00}+\ket{11}}{\sqrt{2}}$.
    \item Variations of the game we consider also go by the same name, and they are equivalent in difficulty and in the approach to a quantum-based solution.
    \item We follow the game as specified in the Wikipedia \href{https://en.wikipedia.org/wiki/Quantum_pseudo-telepathy\#The_Mermin\%E2\%80\%93Peres_magic_square_game}{article}.
\end{itemize}
    
\end{frame}

\begin{frame}{How is the game played?}
\Vskip{-3em}\TwoUnequalColumns{0.5\textwidth}{0.5\textwidth}{%
\only<1-6>{%
\begin{itemize}
    \item<1-> After enjoying some time together, Alice and Bob separate and cannot communicate with each other.
    \item<2-> A referee chooses a \textcolor{\RCone}{row number for Alice (2)} and a \textcolor{\RCtwo}{column number for Bob (1)}.
    \item<3-> Alice must respond with an entry for each cell of \textcolor{\RCone}{her designated row}, choosing~\PP{} or~\MM{}.
    \item<6-> The product of \textcolor{\RCone}{her entries} must be~\PP{}.
\end{itemize}}
\only<7->{%
\begin{itemize}
    \item<7-> Bob must do the same for \textcolor{\RCtwo}{his designated column}.
    \item<8-> Where the row and column \textcolor{Purple}{intersect}, Alice and Bob must agree on the value, or they lose.
    \item<10-> The product of \textcolor{\RCtwo}{his entries} must be~\MM{}.
\end{itemize}
}
}{%
\begin{center}
\MPSquare{\VS{7-}{\textcolor{\RCtwo}{\PP}}}{ }{ }{\only<3-7>{\VS{3-}{\textcolor{\RCone}{\PP}}}\only<8->{\VS{3-}{\textcolor{Purple}{\PP}}}}{\VS{4-}{\textcolor{\RCone}{\MM}}}{\VS{5-}{\textcolor{\RCone}{\MM}}}{\VS{9-}{\textcolor{\RCtwo}{\MM}}}{ }{ }
\end{center}
}%
\end{frame}

\begin{frame}{Best classical solution}{They can agree to respond as shown on this slide, winning $8$ of $9$ games on average}

\Vskip{-3em}\TwoUnequalColumns{0.6\textwidth}{0.4\textwidth}{%
\only<1-6>{%
\begin{itemize}[<+->]
    \item Before separating, Alice and Bob can agree to respond as shown here.
    \item We can verify that for the first two rows and columns, they always agree on the common value and the products are correct.
    \item For row or column three, Alice and Bob would have to complete their response in a way that creates the correct product.
    \begin{description}
      \item[\ColorOne{Alice}] must respond \ColorOne{\MM{}} so that her row's product is \PP{}.
      \item[\ColorTwo{Bob}] must respond \ColorTwo{\PP{}} so that his column's product is \MM{}.
    \end{description}
\end{itemize}}%
\only<7->{%
\begin{itemize}
    \item<7-> While other static solutions are possible, each would have at least one square where Alice and Bob cannot agree classically on an entry.
    \item<8-> As with previous games, we will use measurements of entangled qubits to influence choices made by \textit{incommunicado} Alice and Bob.
    \item<9-> With quantum computing, they can always win.
    \item<10-> But first let's rule out the possibility of any always-win classical solution.
\end{itemize}
}
}{%
\Vskip{-3em}\begin{center}
\MPSquare{\PP{}}{\PP{}}{\PP{}}{\PP{}}{\MM{}}{\MM{}}{\MM{}}{\PP{}}{\only<1-3,6-8>{\alert{\textbf{?}}}\only<4>{\ColorOne{\MM{}}}\only<5>{\ColorTwo{\PP{}}}}
\end{center}
\visible<6->{%
\alert{They lose only when \emph{both} the third row and column are designated.  Is there a way to win always?}}
}
    
\end{frame}

\begin{frame}{Proof of no static solution}{We cannot place values in each cell that always work}

\Vskip{-3em}\TwoUnequalColumns{0.6\textwidth}{0.4\textwidth}{%
\Vskip{-3em}\begin{itemize}
    \item<1-> Each row's product is~\PP{} so the three rows' products is also~\PP{}.
    \item<2-> Each column's product is~\MM{} so the three columns' products is also~\MM{}.
    \item<3-> This is a contradiction.
\end{itemize}
\begin{align*}
    \visible<1>{a b c = d e f = g h k &= \PP{} \\}
    \visible<1,3>{\ColorOne{a b c \times d e f \times g h k} & = \PP{} \\}
    \visible<2>{a d g = b e h = c f k &= \MM{} \\}
    \visible<2->{\ColorTwo{a d g \times b e h \times c f k} & = \MM{} \\}
    \visible<3->{\ColorOne{abcdefghk} & \neq \ColorTwo{abcdefghk}}
\end{align*}
\visible<3->{\Vskip{-4em}\QED}

}{%
\Vskip{-3em}\begin{center}
\MPSquare{$a$}{$b$}{$c$}{$d$}{$e$}{$f$}{$g$}{$h$}{$k$}
\end{center}
}
    
\end{frame}

\begin{frame}{Variaton 1}{All products are the same except one}
\Vskip{-5em}\TwoUnequalColumns{0.53\textwidth}{0.47\textwidth}{%
\begin{itemize}[<+->]
    \item In the version we consider, we require the product of each row's entries to be \PP{} and the product of each column's entries to be \MM{}.
    \item In another version, the product of each row's and each column's entries is \PP{}, \emph{except} the product of the third column's entries is \MM{}.
    \item A solution to the first form can be reduced from a solution to the second form by always negating the values as shown.
    \item The proof of no static solution is the same.
\end{itemize}
}{%
\Vskip{-1em}\begin{center}
\MPSquare{ }{ }{ }{ }{ }{ }{\visible<3->{\alert{$-$}}}{\visible<3->{\alert{$-$}}}{ }
\end{center}
\visible<4->{%
This causes columns~1 and~2 to have product~\MM{}.  The third row still has a positive product because the two negations cancel.
}
}
\end{frame}

\begin{frame}{Variation 2}{Using sums instead of products}
\Vskip{-3em}\TwoUnequalColumns{0.6\textwidth}{0.4\textwidth}{%
\begin{itemize}
    \item<1-> Here is the original formulation
    \item<2-> In this variation, \ColorOne{Alice}'s rows must have even parity, and \ColorTwo{Bob}'s columns must have odd parity.
    \item<3-> The \PP{} and \MM{} of the original version are mapped to $0$ and $1$, respectively.
    \item<4-> Of course there is still no solution classically for every square.
\end{itemize}
}{%
\Vskip{-3em}\begin{center}
\alt<1>{\MPSquare{\PP{}}{\PP{}}{\PP{}}{\PP{}}{\MM{}}{\MM{}}{\MM{}}{\PP{}}{\alert{?}}}{%
\MPSquare{$0$}{$0$}{$0$}{$0$}{$1$}{$1$}{$1$}{$0$}{\alert{?}}
}
\end{center}
}

\end{frame}

{
\def\M#1#2{\mbox{#1{\ensuremath{\mathcal{M}_{#2}}}}}
\def\Mone{\M{\ColorThree}{1}}
\def\Mtwo{\M{\ColorFour}{2}}
\def\Mthree{\M{\ColorFive}{3}}
\begin{frame}{Successive measurements}{These are reliable if there is a common eigenbasis}
\Vskip{-4.5em}\TwoUnequalColumns{0.6\textwidth}{0.4\textwidth}{%
\only<1-8>{%
\begin{itemize}
    \item<1-> The boxes represent measurement operators applied to the two-qubit circuit.
    \item<2-> Recall that the operators are implemented by 
    \Vskip{-2em}\begin{enumerate}
        \item transformation into \PauliZ{}
        \item measurement in the \PauliZ{} basis
        \item transformation out of \PauliZ{}
    \end{enumerate}
    but to avoid clutter we show this using the measurement operators.
    \item<3-> After \Mone, the quantum system's state must be an eigenstate of \Mone.
    \item<4-> If \Mtwo{} and \Mthree{} share an eigenbasis with \Mone, then the outcome of their measurements is determined solely by \Mone.
\end{itemize}}%
\only<9-14>{%
\begin{itemize}
    \item<9-> Suppose we change the sequence of measurements, so that \Mtwo{} is first.
    \item<10-> While \Mtwo's results could be unpredictable, the resulting eigenstate \QState{} \emph{completely determines} the outcomes of \Mone{} and \Mthree.
    \item<11-> Suppose we prepare eigenstate~\QState{} by measuring in the common eigenbasis.  Then \Mtwo's outcome is also determined.
    \item<12-> Each outcome is $\pm 1$: the eigenvalue associated with \QState{} for each operator.
    \item<13-> Let the cummulative result of the three operators be the \emph{product} of their measured eigenvalues.
\end{itemize}
}%
\only<15->{%
\begin{itemize}
    \item<15-> We regard the puzzle's solution in two ways:
    \begin{itemize}
        \item<16-> A single measurement of the two qubits is performed to obtain eigenstate \alert<16>{\QState{}}, which determines the outcome of \Mone, \Mtwo, and \Mthree.  Those eigenvalues' product will yield the desired result.
        \item<17-> Three successive measurements are made, but the \alert<17>{first outcome} determines the other two outcomes.  The product of the three eigenvalues will also be as desired.
    \end{itemize}
    \item<18-> Linear algebra teaches that if each pair of operators in \Set{\Mone, \Mtwo, \Mthree} \emph{commutes}, then there is a common eigenbasis for the three operators.
\end{itemize}
}
}{%
\begin{center}
\adjustbox{scale=0.75,valign=t}{%
\begin{quantikz}[row sep=-12pt] \lstick[wires=2]{\mbox{\visible<11-14,16>{\alert<16>{\QState{}}}}}\qw & \gate[wires=2]{\alt<9-14>{\Mtwo}{\Mone}}\slice{\mbox{\visible<1-10,17->{\alert<17>{\QState{}}}}} & \qw & \gate[wires=2]{\alt<9-14>{\Mone}{\Mtwo}} & \qw & \gate[wires=2]{\Mthree} & \qw \\
\qw & \qw & \qw & \qw & \qw & \qw & \qw
\end{quantikz}}\end{center}%
\only<1-8>{%
\begin{itemize}
    \item<5-> Suppose the outcome of \Mone{} is \QState{}, an eigenstate of \Mtwo{}.
    \item<6-> Then $\Mtwo\QState{}=\pm 1\QState{}$, an eigenstate of \Mthree.
    \item<7-> Then $\Mthree\QState{}=\pm 1\QState{}$.
\end{itemize}
\SmallSkip{}\visible<8->{%
The results of \Mtwo{} and \Mthree{} are determined by \QState{}, the result of~\Mone.}
}%
\only<14>{%
\SmallSkip{}We can choose operators such that for any eigenstate~\QState{} common to the three operators, we obtain products as follows: 
    \begin{itemize}
      \item \ColorOne{\PP{}} for \ColorOne{Alice's rows}
      \item \ColorTwo{\MM{}} for \ColorTwo{Bob's columns}
    \end{itemize}
}%
\only<18->{%
\SmallSkip{}A pair of operators \Set{A,B} commutes if
\[ \left(A\times B\right) - \left(B\times A\right) = \ZeroMatrix \]
where \ZeroMatrix{} is the all-zero matrix.}%
}
    
\end{frame}

\def\B#1{\hbox to 1.2em{\hss\ensuremath{#1}}}
\begin{frame}{Example of three commuting operators}

\Vskip{-5em}\TwoUnequalColumns{0.455\textwidth}{0.545\textwidth}{%
\begin{itemize}
    \item<1-> We need to find an eigenbasis common to all three operators.
    \item<2-> \texttt{MATLAB} provides this common eigenbasis for \Mone{} and \Mtwo{} \visible<3->{and their eigenvalues.}
    \item<4-> Here are their product states after normalization.
\end{itemize}
\Vskip{-3em}\begin{center}\small
    \begin{tabular}{cccc}
    \alert<5>{\QState{1}} & \alert<6>{\QState{2}} & \alert<7>{\QState{3}} & \alert<8>{\QState{4}} \\
     \visible<3->{$1$ & $-1$ & $-1$ & $1$ \\}
     \DQB{-1}{0}{-1}{0}
           & \DQB{-1}{0}{1}{0}
           & \DQB{0}{-1}{0}{-1}
           & \DQB{0}{-1}{0}{1} \\[2em]
        \visible<4->{   \ket{\ColorOne{+}\ColorTwo{0}} & \ket{\ColorOne{-}\ColorTwo{0}}& \ket{\ColorOne{+}\ColorTwo{1}} &\ket{\ColorOne{-}\ColorTwo{1}}}
    \end{tabular}
\end{center}
}{%
\only<1-4>{%
\begin{align*}
    \Mone =& \ColorThree{\TensProd{\Identity}{\PauliZ}} =& \ColorThree{\begin{pmatrix*}[r]
     \B{1}  &   \B{0}  &   \B{0}   &  \B{0} \\
     \B{0}  &  \B{-1}  &   \B{0}   &  \B{0} \\
     \B{0}  &   \B{0}  &   \B{1}   &  \B{0} \\
     \B{0}  &   \B{0}  &   \B{0}   & \B{-1}
    \end{pmatrix*}} \\
    \Mtwo =& \ColorFour{\TensProd{\PauliX}{\Identity}} =& \ColorFour{\begin{pmatrix*}[r]
     \B{0}   &  \B{0}  &   \B{1}   &  \B{0} \\
     \B{0}   &  \B{0}  &   \B{0}   &  \B{1} \\
     \B{1}   &  \B{0}  &   \B{0}   &  \B{0} \\
     \B{0}   &  \B{1}  &   \B{0}   &  \B{0}
    \end{pmatrix*}} \\
    \Mthree =& \ColorFive{\TensProd{\PauliX}{\PauliZ}} =& \ColorFive{\begin{pmatrix*}[r]
     \B{0} &     \B{0} &     \B{1} &     \B{0} \\
     \B{0} &     \B{0} &     \B{0} &    \B{-1} \\
     \B{1} &     \B{0} &     \B{0} &     \B{0} \\
     \B{0} &    \B{-1} &     \B{0} &     \B{0}
    \end{pmatrix*}}
\end{align*}}
\only<5-6>{%
\begin{align*}
\DQB{-1}{0}{-1}{0} &\equiv \RootTwo{}\DQB{1}{0}{1}{0} &=\TensProd{\ColorOne{\PPlus}}{\ColorTwo{\PZero}} \\[2em]
\visible<6->{\DQB{-1}{0}{1}{0} &\equiv \RootTwo{}\DQB{1}{0}{-1}{0} &= \TensProd{\ColorOne{\PMinus}}{\ColorTwo{\PZero}}}
\end{align*}
}%
\only<7-8>{%
\begin{align*}
\DQB{0}{-1}{0}{-1} &\equiv \RootTwo{}\DQB{0}{1}{0}{1} &=\TensProd{\ColorOne{\PPlus}}{\ColorTwo{\POne}} \\[2em]
\visible<8->{\DQB{0}{-1}{0}{1} &\equiv \RootTwo{}\DQB{0}{1}{0}{-1} &= \TensProd{\ColorOne{\PMinus}}{\ColorTwo{\POne}}}
\end{align*}
}%
\only<9->{%
\Vskip{3em}
T and \Conj{T}
}
}
    
\end{frame}
}

\begin{frame}{Quantum-based solution}{We use two EPR pairs}

\begin{itemize}[<+->]
    \item Alice and Bob begin with two EPR pairs, each in the state $\frac{\ket{00}+\ket{11}}{\sqrt{2}}$.
    \item When they separate, they take one half of each pair with them.
    \item Depending on the row (for Alice) and the column (for Bob), they will carry out measurements based on a table, in a basis related to their particular row or column.
    \item We will prove that they
    \begin{itemize}
        \item agree on a value at the cell intersecting the row and column, and
        \item obtain the necessary row and column products.
    \end{itemize}
\end{itemize}
    
\end{frame}

}
