\SetTitle{99}{temp}{faster}{99}

\section*{Magic square}
{
\def\MPC#1{\fbox{\hbox to 5ex{\hss\mbox{#1}\vrule width 0pt height 3ex depth 2ex\hss}}}
\def\XP#1{\hbox to 3ex{\hss#1\hss 1}}%
\def\PP{%
\XP{$+$}}%
\def\MM{%
\XP{$-$}}%
\def\MPSquare#1#2#3#4#5#6#7#8#9{%
{%
    \setlength{\tabcolsep}{0pt}%
    \adjustbox{valign=t}{\begin{tabular}{rrccc}
   \multicolumn{1}{c}{ } & \multicolumn{1}{c}{ } & \multicolumn{3}{c}{$\downarrow\ $\ColorTwo{Bob}$\ \downarrow$} \\
    & & 1 & 2 & 3 \\
        $\rightarrow\ $ & \ 1\ \ & \MPC{#1} & \MPC{#2} & \MPC{#3} \\
        \mbox{\ColorOne{Alice} }& \ 2\ \ & \MPC{#4} & \MPC{#5} & \MPC{#6} \\
        $\rightarrow\ $&\ 3\ \ & \MPC{#7} & \MPC{#8} & \MPC{#9} \\
    \end{tabular}}}}
\def\VS#1#2{\visible<#1>{#2}}
\begin{frame}{Introduction to the \href{https://en.wikipedia.org/wiki/Quantum_pseudo-telepathy\#The_Mermin\%E2\%80\%93Peres_magic_square_game}{Mermin--Peres magic square}}{This game is due to \href{https://en.wikipedia.org/wiki/N._David_Mermin}{David Mermin} and \href{https://en.wikipedia.org/wiki/Asher_Peres}{Asher Peres}}

\begin{itemize}[<+->]
    \item With CHSH, a quantum approach is probabilistically better than any classical approach but does not guarantee to win the game.
    \item The Mermin--Peres magic square is a puzzle that Alice and Bob can win at best $\frac{8}{9}$ of the time, but a quantum approach on a reliable quantum computer can always win.
    \item Alice and Bob share two EPR pairs, each initially in the state $\frac{\ket{00}+\ket{11}}{\sqrt{2}}$.
    \item Variations of the game we consider also go by the same name, and they are equivalent in difficulty and in the approach to a quantum-based solution.
    \item We follow the game as specified in the Wikipedia \href{https://en.wikipedia.org/wiki/Quantum_pseudo-telepathy\#The_Mermin\%E2\%80\%93Peres_magic_square_game}{article}.
\end{itemize}
    
\end{frame}

\begin{frame}{How is the game played?}
\Vskip{-3em}\TwoUnequalColumns{0.5\textwidth}{0.5\textwidth}{%
\only<1-6>{%
\begin{itemize}
    \item<1-> After enjoying some time together, Alice and Bob separate and cannot communicate with each other.
    \item<2-> A referee chooses a \textcolor{\RCone}{row number for Alice (2)} and a \textcolor{\RCtwo}{column number for Bob (1)}.
    \item<3-> Alice must respond with an entry for each cell of \textcolor{\RCone}{her designated row}, choosing~\PP{} or~\MM{}.
    \item<6-> The product of \textcolor{\RCone}{her entries} must be~\PP{}.
\end{itemize}}
\only<7->{%
\begin{itemize}
    \item<7-> Bob must do the same for \textcolor{\RCtwo}{his designated column}.
    \item<8-> Where the row and column \textcolor{Purple}{intersect}, Alice and Bob must agree on the value, or they lose.
    \item<10-> The product of \textcolor{\RCtwo}{his entries} must be~\MM{}.
\end{itemize}
}
}{%
\begin{center}
\MPSquare{\VS{7-}{\textcolor{\RCtwo}{\PP}}}{ }{ }{\only<3-7>{\VS{3-}{\textcolor{\RCone}{\PP}}}\only<8->{\VS{3-}{\textcolor{Purple}{\PP}}}}{\VS{4-}{\textcolor{\RCone}{\MM}}}{\VS{5-}{\textcolor{\RCone}{\MM}}}{\VS{9-}{\textcolor{\RCtwo}{\MM}}}{ }{ }
\end{center}
}%
\end{frame}

\begin{frame}{Best classical solution}{They can agree to respond as shown on this slide, winning $8$ of $9$ games on average}

\Vskip{-3em}\TwoUnequalColumns{0.6\textwidth}{0.4\textwidth}{%
\only<1-6>{%
\begin{itemize}[<+->]
    \item Alice and Bob can agree before separating to respond as shown in the square here.
    \item We can verify that for the first two rows and columns, they always agree on the common value and the products are correct.
    \item For row or column three, Alice and Bob would have to complete their response in a way that creates the correct product.
    \begin{description}
      \item[Alice] must respond \MM{} so her row's product is \PP{}.
      \item[Bob] must respond \PP{} so his column's product is \MM{}.
    \end{description}
\end{itemize}}%
\only<7->{%
\begin{itemize}
    \item<7-> While other static solutions are possible, each would have at least one square where Alice and Bob cannot agree classically on an entry.
    \item<8-> As with previous games, we will use measurements of entangled qubits to influence choices made by \textit{incommunicado} Alice and Bob.
    \item<9-> On a perfect quantum computer, they can always win.
\end{itemize}
}
}{%
\Vskip{-3em}\begin{center}
\MPSquare{\PP{}}{\PP{}}{\PP{}}{\PP{}}{\MM{}}{\MM{}}{\MM{}}{\PP{}}{\only<1-3,6-8>{\alert{\textbf{?}}}\only<4>{\MM{}}\only<5>{\PP{}}}
\end{center}
\visible<6->{%
\alert{They only lose when \emph{both} the third row and column are designated because they can't agree.}}
}
    
\end{frame}

\begin{frame}{Proof of no static solution}{We cannot place values in each cell that always work}

\Vskip{-3em}\TwoUnequalColumns{0.6\textwidth}{0.4\textwidth}{%
\Vskip{-3em}\begin{itemize}
    \item<1-> Each row's product is~\PP{} so the three rows' products is also~\PP{}.
    \item<2-> Each column's product is~\MM{} so the three columns' products is also~\MM{}.
    \item<3-> This is a contradiction.
\end{itemize}
\begin{align*}
    \visible<1>{a b c = d e f = g h k &= \PP{} \\}
    \visible<1,3>{a b c \times d e f \times g h k & = \PP{} \\}
    \visible<2>{a d g = b e h = c f k &= \MM{} \\}
    \visible<2->{a d g \times b e h \times c f k & = \MM{} \\}
    \visible<3->{abcdefghk & \neq abcdefghk}
\end{align*}

}{%
\Vskip{-3em}\begin{center}
\MPSquare{$a$}{$b$}{$c$}{$d$}{$e$}{$f$}{$g$}{$h$}{$k$}
\end{center}
}
    
\end{frame}

\begin{frame}{Variations}{They are equivalent}
\Vskip{-5em}\TwoUnequalColumns{0.53\textwidth}{0.47\textwidth}{%
\begin{itemize}[<+->]
    \item In the version we consider, we require the product of each row's entries to be \PP{} and the product of each column's entries to be \MM{}.
    \item In another version, the product of each row's and each column's entries is \PP{}, \emph{except} the product of the third column's entries is \MM{}.
    \item A solution to the first form can be reduced from a solution to the second form by always negating the values as shown.
    \item The proof of no static solution is the same.
\end{itemize}
}{%
\Vskip{-1em}\begin{center}
\MPSquare{ }{ }{ }{ }{ }{ }{\visible<3->{\alert{$-$}}}{\visible<3->{\alert{$-$}}}{ }
\end{center}
\visible<4->{%
This causes columns~1 and~2 to have product~\MM{}.  The third row still has a positive product because the two negations cancel.
}
}
\end{frame}


\begin{frame}{Quantum-based solution}{We use two EPR pairs}

\begin{itemize}[<+->]
    \item Alice and Bob begin with two EPR pairs, each in the state $\frac{\ket{00}+\ket{11}}{\sqrt{2}}$.
    \item When they separate, they take one half of each pair with them.
    \item Depending on the row (for Alice) and the column (for Bob), they will carry out measurements based on a table, in a basis related to their particular row or column.
    \item We will prove that they
    \begin{itemize}
        \item agree on a value at the cell intersecting the row and column, and
        \item obtain the necessary row and column products.
    \end{itemize}
\end{itemize}
    
\end{frame}

}
