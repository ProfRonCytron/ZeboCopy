\SetTitle{99}{temp}{faster}{99}



\begin{frame}{Analysis}{Probability of success in a given iteration}

\begin{itemize}
    \item We regard an $n$-qubit basis vector interchangeably as an $n$-bit string.
    \item Consider a set $T=\EmptySet{}$ initially, and an $n$-qubit instance of Simon's problem.
    \item We define the \href{https://en.wikipedia.org/wiki/Linear_span}{span} of $T$, \Span{T}, as the set of $n$-bit basis vectors that are obtainable by addition (\Xor{}{}) of elements in~$T$.  
    \item Note that $\Forall{T}{\TensSupProd{0}{n} \in \Span{T}}$, as a trivial, zero sum.
    \item We shall add an $n$-bit basis vector to $T$ only if it is not already in \Span{T}.
    \item We can then claim 
    \[\Mag{\Span{T}} = 2^\Mag{T}\]
\end{itemize}
    
\end{frame}

\begin{frame}{Illustration of theorem using our example}{To build intuition}
\begin{itemize}
    \item<1-> $T$ is initially empty.  We add:
    \begin{itemize}
    \item<2-> 0101
    \item<3-> 1010
    \item<4-> 0001
    \end{itemize}
\end{itemize}

\begin{align*}
T &= \Set{
\visible<2->{0101}
\visible<3->{, 1010}
\visible<4->{, 0001}
} \\
\Span{T} &= \Set{
\TensSupProd{0}{n}%
\visible<2->{, 0101}
\visible<3->{, 1010, 1111}
\visible<4->{, 0001, 0100, 1011, 1110}
} \\
\Mag{\Span{T}} &= 2^\Mag{T} = \alt<1>{1}{\alt<2>{2}{\alt<3>{4}{8}}}
\end{align*}

\end{frame}

\begin{frame}{Proof of our claim}{By induction on the size of $T$}

\begin{lemma}
In Simon's problem, if $T=\Set{w_{1}, w_{2},\ldots,w_{n}}$, where the $w_{i}$ are linearly independent, then $\Mag{\Span{T}}=2^\Mag{T}$.
\end{lemma}
\begin{itemize}
    \item The lemma is true initially, because $\Mag{T}=0$ and $\Span{T}=\Set{\TensSupProd{0}{n}}$.
    \item Assume $T=\Set{w_{1},w_{2},\ldots,w_{k-1}}$ so that $\Mag{\Span{T}}=2^{k-1}$
    \item Consider the addition of the next observation $w_{k}\not=\TensSupProd{0}{n}$ to $T$, where $w_k$ is linearly independent of all strings in~$T$.
    \item Then \Forall{w_{i}\in \Union{T}{\TensSupProd{0}{n}}}{\mbox{$\Xor{w_{i}}{w_{k}}\not\in\Span{T}$ but $\Xor{w_{i}}{w_{k}}\in \Span{\Union{T}{\Set{w_{k}}}}$}}
    \item Thus, adding $w_{k}$ to $T$ doubles its span
    \item $\Mag{\Span{\Union{T}{\Set{w_{k}}}}}=2^k$ \QED{}
\end{itemize}
    
\end{frame}

\begin{frame}{Analysis}{How likely are we to obtain $n-$ independent observations?}
    
\end{frame}