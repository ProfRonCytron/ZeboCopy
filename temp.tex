\SetTitle{99}{temp}{faster}{99}

\begin{frame}{Measurement in a basis other than the computational basis}{Overview}

\begin{itemize}[<+->]
    \item Qiskit, in emulation and on the available hardware, performs measurements only in the computational basis.
    \item Recall that \QZero{} and \QOne{} are the North and South poles of the Bloch sphere, respectively.
    \item Some computations require measuring in a \emph{different} basis.
    \item How are we to achieve that?
    \item We can rotate from our desired basis to the computational basis.
    \item We then measure in the computational basis, which causes collapse upon measurement to~\QZero{} or~\QOne{}
    \item If there is more to the computation, then we must rotate back so that we are in the corresponding eigenstate of our desired basis.
\end{itemize}
    
\end{frame}

\begin{frame}{Example}{Measurement in the \NamedGate{X} (Hadamard) basis}
\Vskip{-4em}\TwoColumns{%
\only<1-8>{\begin{itemize}
    \item<1-> Consider the observable \PauliX{} with eigenvectors \visible<2->{\ColorOne{\ket{+}}} \visible<3->{and \ColorTwo{\ket{-}}.}
    \item<4-> If we had an \NamedGate{X} measurement device, then the measured qubit would collapse to~\ColorOne{\ket{+}} \visible<5->{ or~\ColorTwo{\ket{-}}}
    \item<6-> The detector reports the corresponding \ColorThree{eigenvalue}, revealing the qubit's collapsed state.
    \item<7-> \PauliX{} is both a unitary gate and a measurement operator.
\end{itemize}}%
\only<9-13>{%
\begin{itemize}
    \item<9-> Thanks to linearity we can map the \PauliX{} basis to the computational basis.
    \item<10-> Given state \QState{}, we require a transformation that maps
    \begin{center}
    \begin{tabular}{r@{$\mapsto$}l}
       \ColorOne{\ket{+}} & \QZero{} \\
       \ColorTwo{\ket{-}} & \QOne{}
    \end{tabular}\end{center}
    \item<11-> The \ColorFive{\Hadamard{}} gate performs this mapping.
    \item<12-> Measurement in the computational basis now yields the correct result.
    \item<13-> However, the qubit collapses to~\ket{0} or~\ket{1}; neither is a valid state after measurement in the \PauliX{} basis.
\end{itemize}
}%
\only<14->{%
\begin{itemize}
    \item<14-> Recall \Hadamard{} is its own inverse.
    \item<15-> Thus, after we measure, we apply another \Hadamard{} operator to obtain the correct state for further computation, if necessary.
    \item<16-> This is demonstrated in the \texttt{MeasureH} notebook.
\end{itemize}
}
}{%
\Vskip{-2em}
\begin{center}
\begin{TIKZP}
\alert<8-9>{\draw[->,thick] (0,0) -- (1,0) node[right] {\ \ket{1}};}
\alert<8-9>{\draw[->,thick] (0,0) -- (0,1) node[above] {\ket{0}};}
\textcolor<2,4,10->{\RCone}{\draw[->,thick] (0,0) -- (45:1) node[above] {\ket{+}};}
\textcolor<3,5,10->{\RCtwo}{\draw[->,thick] (0,0) -- (-45:1) node[below] {\ket{-}};}
\end{TIKZP}
\end{center}
\Vskip{-2.5em}\only<1-6>{\begin{alignat*}{5}
    \visible<2->{\XMatrix{} & \textcolor<2>{\RCone}{\PHad{}} & \mbox{ = } & \ColorThree{+1}  & \textcolor<2,4>{\RCone}{\PHad{}}  \\}
    \visible<3->{\XMatrix{} & \textcolor<3>{\RCtwo}{\PHadm{}} & \mbox{ = } & \ColorThree{-1}  & \textcolor<3,5>{\RCtwo}{\PHadm{}} }
\end{alignat*}}%
\only<8>{%
\SmallSkip{}

If we can measure only in the \alert{computational basis}, how do we perform measurements in the \PauliX{} basis?
}%
\only<11-14>{%
\begin{align*}
    \ColorFive{\HMatrix{}}\ColorOne{\PPlus{}} &= \PZero{} \\
    \ColorFive{\HMatrix{}}\ColorTwo{\PMinus{}} &= \POne{}
\end{align*}
}%
\only<15->{%
\begin{align*}
    \ColorFive{\HMatrix{}}\PZero{} &= \ColorOne{\PPlus{}} \\
    \ColorFive{\HMatrix{}}\POne{} &= \ColorTwo{\PMinus{}}
\end{align*}
}
}
    
\end{frame}