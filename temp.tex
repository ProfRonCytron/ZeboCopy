\SetTitle{99}{temp}{faster}{99}


{
\def\T#1{\TwoSupOp{\QZero}{\ExpPhase{2\pi\left(2^{#1}\W{}\right)}\QOne}{+}}
\def\N{\ColorOne{\ensuremath{n}}}\def\W{\ColorTwo{\ensuremath{w}}}\def\Y{\ColorFour{y}}
\begin{frame}{Our theorem}{Needs a proof, but first a restatement}
    \Vskip{-3em}\begin{align*}
 \RootTwoN{\N}\alt<3->{\alert<3>{\SumBV{\Y}{\N}}}{\SumPH{\Y}{\N}} \ExpPhase{2\pi \W{}\Y} \ket{\Y} =& \T{\N{}-1}  \\
 \TensOp{} & \T{\N{}-2} \\
 \TensOp{} & \cdots \\
 \TensOp{} & \T{\N{}-\N{}} 
\end{align*}
\only<1-2>{%
\begin{itemize}
    \item<1-> The use of \Y{} as the variable for the summation and as \ket{\Y} requires us to view \Y{} as a \href{https://en.wikipedia.org/wiki/Number}{number}, taking on the sequence $0, 1, \ldots,2^{\N}-1$,
    \item<2-> For the purposes of our proof it is more useful to view \Y{} as a base-2 \href{https://en.wikipedia.org/wiki/Numeral}{numeral}:  a string of \N{}~bits.
\end{itemize}}%
\only<3-4>{%
\begin{itemize}
    \item The summation in our theorem changes as shown here.
    \item The summation is now over all bit strings of length~$n$.  These must be interpreted numerically in the exponent.
\end{itemize}
}%
\only<5->{%
\begin{itemize}
    \item<5-> In truth this is more accurate, as the tensor products on the right pile up bit strings, not integers.
\end{itemize}}
\end{frame}

\def\ThLeftTerm#1{%
\ensuremath{\ExpPhase{2\pi \W{} #1} \ket{#1}}}
\def\ThLeft#1{%
\ensuremath{\RootTwoN{#1}\SumBV{\Y}{#1} \ThLeftTerm{\Y}}}
\def\Th#1{%
\begin{align*}
P(#1):\ \ThLeft{#1} =& \T{#1-1}  \\
 \TensOp{} & \T{#1-2} \\
 \TensOp{} & \cdots \\
 \TensOp{} & \T{#1-#1} 
\end{align*}
}
\def\ThTop#1{%
\fbox{\adjustbox{valign=t, width=0.95\textwidth}{
$\ThLeft{#1} = \T{#1-1}\TensOp{}\T{#1-2}\TensOp{}\cdots\TensOp\T{#1-#1}$}}}
\begin{frame}{Theorem}{And proof}
\Vskip{-3em}\begin{theorem}
\Th{\N}
\Forall{n\geq 1}{P(n)}
\end{theorem}
Proof by induction.  We first show $P(1)$ and then show $\Implies{P(k-1)}{P(k)}$.
    
\end{frame}

\begin{frame}{Base case}{$n=1$}
\Vskip{-4em}\begin{center}\ThTop{\N}\end{center}
\SmallSkip{}
\begin{Reasoning}
\Reason{1}{Expansion of sum}
\Reason{2}{Simplify constants in exponents}
\Reason{3}{To satisfy the tensor-factored form in the theorem}
\end{Reasoning}

\ScrollProof{1}{3}{%
   \Next{\Three}{ \ThLeft{1} &= \TwoSupOp{\ThLeftTerm{0}}{\ThLeftTerm{1}}{+} \\}
    \Next{\Two}{           &= \TwoSupOp{\QZero}{\ExpPhase{2\pi\W}\QOne}{+} \\}
      \Next{\One}{         &= \T{\ColorOne{1-1}}}
}
\QED{}
    
\end{frame}
}


