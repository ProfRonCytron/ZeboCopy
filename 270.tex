\SetTitle{27}{Grover's Algorithm}{Analysis}{27}

\begin{Grover}

\section*{Setup}
\begin{frame}{Setup}{For the math of each iteration}
\begin{itemize}
    \item Our function's inputs are $n$ bits wide, with $N=2^{n}$
    \item Our function has just one secret (special input value),  \W.
    \item \S{} is the uniform superposition of all possible function inputs:
    \[ \S = \RootN{N} \SumBV{x}{n} \ket{x} = \Hadamard\left(\TensSupProd{0}{n}\right)\]
    \item Throughout
    \begin{description}
      \item[\A] represents the amplitude on our special value \W.
    \item[\B] represents the amplitude on each of the $N-1$ other states.
    \end{description}
    \item We therefore require
    \[ \Prob{\A}  + (N-1)\,\Prob{\B} = 1\]
    \item Initially $\A=\B=\RootN{N}$, so we begin at \S.
\end{itemize}
\end{frame}

\section*{Amplitudes}

\begin{frame}{Analysis of amplitudes in an arbitrary iteration}{With $\alpha$ amplitude on \ket{w} and $\beta$ amplitude on all other values}

\begin{Reasoning}
\Reason{1}{Prior to step 1, we have \A{} amplitude on \W, and \B{} amplitude on all other states}%
\Reason{2}{In step 1, phase kickback causes the oracle $U_{f}$ to reflect amplitude only on \W}%
\Reason{3}{Completing the sum}%
\Reason{4}{\Hadamard(\TensSupProd{0}{n}) = \RootN{N}\SumBV{x}{n}\ket{x}}%
\Reason{6}{Collect terms in preparation for the upcoming \Hadamard{}}%
\Reason{7}{Collect the coefficients for \W}%
\Reason{8}{Apply \Hadamard{} to both sides}%
\Reason{9}{\Hadamard{} is its own inverse}%
\Reason{12}{Recall $\Hadamard\left(\W\right) = \RootN{N} \NHadamard{\ColorThree{w}}{n}{x}$}%
\Reason{13}{Rearrange. Note that \ket{\TensSupProd{0}{n}} appears in the summation}%
\Reason{14}{Collect coefficients for \ket{\TensSupProd{0}{n}}. $\B\,\sqrt{N}-\B\RootN{N} - \A\RootN{N}=\RootUN{\B(N-1)-\A}{N}$}%
\Reason{16}{We next negate all states except \ket{\TensSupProd{0}{n}}}%
\Reason{17}{We denote the reflection of all states except \ket{\TensSupProd{0}{n}} using $R$}%
\Reason{18}{Complete the summation}%
\Reason{19}{We next collect coefficients of \ket{\TensSupProd{0}{n}}}%
\Reason{21}{Collected coefficients of \ket{\TensSupProd{0}{n}}}%
\Reason{22}{Preparing for last \Hadamard}
\Reason{24}{Final \Hadamard, returning to the $\R\times\W$ basis}%
\Reason{25}{Recall $\S{}=\Hadamard\left(\ket{\TensSupProd{0}{n}}\right)$ and $\RootN{N}\Hadamard\left(\NHadamard{\ColorThree{w}}{n}{x}\right)=\W$}%
\Reason{27}{\S{} includes \W{}.  We next combine amplitudes for \W}%
\Reason{28}{Expanding \S}%
\Reason{32}{Pulling out the amplitude on \W{} from the summation}%
\Reason{33}{Collecting amplitudes for \W{}}%
\Reason{34}{The parenthesized expressions become our new \B{} and \A{} in the next iteration}%
\end{Reasoning}

\ScrollProof{1}{5}{%
    \Next{\Four}{\D &= +\A\W +  \B\sum_{\AllBits{x}{n},\ x\neq w} \ket{x}\\}
    \Next{\Three}{\F = U_{f}(\D) &= -\A\W + \B \sum_{\AllBits{x}{n},\ x\neq w} \ket{x}\\}
   \Next{\Two}{ &=  -\A\W + \B\left(-\W +  \SumBV{x}{n}\ket{x}\right) \\}
    \Last{&=  -\A\W + \B\left(-\W +  \sqrt{N}\ \Hadamard\left(\ket{\TensSupProd{0}{n}}\right)\right)}
}%
\ScrollProof{6}{10}{%
  \Next{\Four}{\F &=  -\A\W + \B\left(-\W +  \sqrt{N}\ \Hadamard\left(\ket{\TensSupProd{0}{n}}\right)\right) \\}
  \Next{\Three}{ &= \left(-\A-\B\right)\W + \B\sqrt{N}\  \Hadamard\left(\ket{\TensSupProd{0}{n}}\right)\\}
  \Next{\Two}{\alert<8>{\Hadamard}(\F) &= \left(-\A-\B\right)\alert<8>{\Hadamard}\left(\W\right) + \B\sqrt{N}\  \ \alert<9>{\alert<8>{\Hadamard}\left(\Hadamard\left(\ket{\TensSupProd{0}{n}}\right)\right)}\\}
  \Last{&=\left(-\A-\B\right)\Hadamard\left(\W\right) + \B\,\sqrt{N}\, \alert<9>{\ket{\TensSupProd{0}{n}}}}
}%
\ScrollProof{11}{15}{%
  \Next{\Four}{\Hadamard(\F)&=\left(-\A-\B\right)\alert<12>{\Hadamard\left(\W\right)} + \B\,\sqrt{N}\, \ket{\TensSupProd{0}{n}} \\}
%%
  \Next{\Three}{ &= \left(-\A-\B\right)\alert<12>{\RootN{N}\NHadamard{\ColorThree{w}}{n}{x}} + \B\,\sqrt{N}\, \ket{\TensSupProd{0}{n}}\\}
  \Next{\Two}{ &= \B\,\sqrt{N}\, \ket{\TensSupProd{0}{n}} - \left(\A+\B\right)\alert<12>{\RootN{N}\alert<13>{\NHadamard{\ColorThree{w}}{n}{x}}}\\}
  \Last{&= \left(\RootUN{\B(N-1)-\A}{N}\right) \ket{\TensSupProd{0}{n}} - \left(\A+\B\right)\RootN{N}\NHadamardS{\ColorThree{w}}{n}{x}{\sum_{\alert<14>{x\neq\TensSupProd{0}{n}}}}}
}%
\ScrollProof{16}{19}{%
 \Next{\Three}{\Hadamard(\F) &=\left(\RootUN{\B(N-1)-\A}{N}\right) \ket{\TensSupProd{0}{n}} \alert<17>{-} \RootUN{\A+\B}{N}\NHadamardS{\ColorThree{w}}{n}{x}{\sum_{\alert<14>{x\neq\TensSupProd{0}{n}}}} \\}
 \Next{\Two}{\alert<17>{R}\left(\Hadamard(\F)\right) &=\left(\RootUN{\B(N-1)-\A}{N}\right) \ket{\TensSupProd{0}{n}} \alert<17>{+} \RootUN{\A+\B}{N}\NHadamardS{\ColorThree{w}}{n}{x}{\sum_{\alert<14>{\alert<18>{x\neq\TensSupProd{0}{n}}}}}  \\}
 %% left off here
 \Last{ &=\left(\RootUN{\B(N-1)-\A}{N}\right) \ket{\TensSupProd{0}{n}} \alert<17>{+} \RootUN{\A+\B}{N}\NHadamardS{\ColorThree{w}}{n}{x}{\sum_{\alert<14>{\alert<18>{\AllBits{x}{n}}}}}\\ & \hbox to 11em{\hss} \alert<18>{- \RootUN{\A+\B}{N}\ket{\TensSupProd{0}{n}}}
}}%
\ScrollProof{20}{22}{%
  \Next{\Two}{R\left(\Hadamard(\F)\right) &= \left(\RootUN{\B(N-1)-\A}{N}\right) \ket{\TensSupProd{0}{n}} + \RootUN{\A+\B}{N}\NHadamardS{\ColorThree{w}}{n}{x}{\sum_{\AllBits{x}{n}}}\\ & \hbox to 11em{\hss} - \RootUN{\A+\B}{N}\ket{\TensSupProd{0}{n}} \\[1em]}
  \Last{ &= \left(\RootUN{\B\,(N-2)-2\A}{N}\right) \ket{\TensSupProd{0}{n}} +\RootUN{\A+\B}{N}\NHadamard{\ColorThree{w}}{n}{x}}
}
\ScrollProof{23}{26}{%
  \Next{\Three}{R\left(\Hadamard(\F)\right) &=\left(\RootUN{\B\,(N-2)-2\A}{N}\right) \ket{\TensSupProd{0}{n}} +\RootUN{\A+\B}{N}\NHadamard{\ColorThree{w}}{n}{x} \\}
  \Next{\Two}{\alert<24>{\Hadamard}\left(R\left(\Hadamard(\F)\right)\right) &=\left(\RootUN{\B\,(N-2)-2\A}{N}\right) \alert<24>{\Hadamard}\left(\ket{\TensSupProd{0}{n}}\right) \\& \hbox to 12em{\hss}+\RootUN{\A+\B}{N}\alert<24>{\Hadamard}\left(\NHadamard{\ColorThree{w}}{n}{x}\right) \\}
  \Last{ &= \RootUN{\B\,(N-2)-2\A}{N}\S+\left(\A+\B\right)\W=\D }
}%
\ScrollProof{27}{30}{%
\Next{\Three}{\D &= \RootUN{\B\,(N-2)-2\A}{N}\S+\left(\A+\B\right)\W \\}
\Next{\Two}{ &=  \RootUN{\B\,(N-2)-2\A}{N}\RootN{N}\SumBV{x}{n} \ket{x} + \left(\A+\B\right)\W \\}
\Last{ &=  \frac{\B\,(N-2)-2\A}{N}\SumBV{x}{n} \ket{x} + \left(\A+\B\right)\W}
}
\ScrollProof{31}{35}{%
\Next{\Four}{\D  &=  \frac{\B\,(N-2)-2\A}{N}\SumBV{x}{n} \ket{x} + \left(\A+\B\right)\W \\}
\Next{\Three}{ &=  \frac{\B\,(N-2)-2\A}{N}\hbox to -1em{\hss}\sum_{\AllBits{x}{n},\ x\neq w} \hbox to -2em{\hss}\ket{x} + \left(\A+\B\right)\W +\frac{\B\,(N-2)-2\A}{N}\W \\}
\Next{\Two}{& = \frac{\B\,(N-2)-2\A}{N}\hbox to -1em{\hss}\sum_{\AllBits{x}{n},\ x\neq w} \hbox to -2em{\hss}\ket{x} + \frac{N(2\B+\A)-2(\A+\B)}{N}\W \\}
\Last{&=  \left(\B\,\left(1-\frac{2}{N}\right)-\frac{2\A}{N}\right)\hbox to -1em{\hss}\sum_{\AllBits{x}{n},\ x\neq w} \hbox to -2em{\hss}\ket{x} + \left((2\B+\A)-\frac{2}{N}(\A+\B)\right)\W}
}
\end{frame}


\begin{frame}{State $\D_{1}$ reached after the first iteration}{Provides a formula for all iterations}
\begin{align*}
\D_{1} &=  \left(\B\,\left(1-\frac{2}{N}\right)-\frac{2\A}{N}\right)\hbox to -1em{\hss}\sum_{\AllBits{x}{n},\ x\neq w} \hbox to -2em{\hss}\ket{x} + \left((2\B+\A)-\frac{2}{N}(\A+\B)\right)\W \\
&= \B_{1} \hbox to -1em{\hss}\sum_{\AllBits{x}{n},\ x\neq w} \hbox to -2em{\hss}\ket{x} + \A_{1} \W
\end{align*}
\begin{itemize}
  \item $\D_{1}$ is the state (above), obtained at the end of the first iteration.
  \item By linearity, this same effect happens each iteration.
  \item We thus extend this result to define the results of iteration $i+1$ in terms of
  \begin{description}
     \item[$\A_{i}$] the amplitude on \W{} at the end of iteration $i$, and
     \item[$\B_{i}$] the amplitude on all other states at the end of iteration $i$.
  \end{description}
\end{itemize}
\end{frame}

\subsection*{Recurrence}

\begin{frame}{Recurrence defining the state at the end of iteration $i+1$}{Valid for $i\geq 0$;  We need the formula for $\A_{i+1}$ later}
\Vskip{-3em}\[ \D_{i+1} = \B_{i+1} \hbox to -1em{\hss}\sum_{\AllBits{x}{n},\ x\neq w} \hbox to -2em{\hss}\ket{x} + \A_{i+1} \W\]
where \visible<2->{\alert<2-3>{(approximations are for large $N$)}}
\begin{align*}
    \B_{i+1} &= \B_{i}\,\left(1-\frac{2}{\alert<2>{N}}\right)-\frac{2\A_{i}}{\alert<2>{N}} && \alert<3>{\visible<3->{\approx \B_{i}}} \\
    \A_{i+1} &= (2\B_{i}+\A_{i})-\frac{2}{\alert<2>{N}}(\A_{i}+\B_{i}) && \alert{\visible<3->{\approx 2\B_{i}+\A_{i}}}
\end{align*}
with
\Vskip{-3em}\begin{align*}
    \B_{0} = \B =\RootN{N} &&
    \A_{0} = \A =\RootN{N}
\end{align*}
\visible<4->{The amplitude $\A_{i}$ for \W{} at the end of iteration $0, 1, 2, \ldots, i$ is approximately \[\RootUN{1}{N}, \RootUN{3}{N}, \RootUN{5}{N},\ldots, \RootUN{2i+1}{N}\]}
\end{frame}



\begin{frame}{Example of amplitudes for $n=5$ bits}{Using \href{https://docs.google.com/spreadsheets/d/1JCp9PKLTQ7q9jdvRwj-gXxL9jYvu1O-1wA2OnnEBB3s/edit?usp=sharing}{this} Google sheet}

\Vskip{-4em}\begin{center}
\begin{Pixture}[width=0.5\textwidth]{27}{5bitGrover.png}
\end{Pixture}
\end{center}
\begin{itemize}
    \item At first, \A{} increases linearly, but at some point \B{} becomes negative, diminishing \A{}.
    \item How many iterations should we perform to have reasonable probability of seeing \W?
\end{itemize}
    
\end{frame}


\begin{frame}{How many iterations?}{Empirical determination}

\begin{itemize}
    \item We can use the recurrence to compute how many iterations are needed, for example, by a \href{https://docs.google.com/spreadsheets/d/1JCp9PKLTQ7q9jdvRwj-gXxL9jYvu1O-1wA2OnnEBB3s/edit?usp=sharing}{Google sheet}.
    \item We need only reach a point where the probability of seeing \W{} is 10\% or better.
   \begin{itemize}\item Then, with 40 trials, we are $98\%$ sure of finding \W.\end{itemize}
     \item From the sequence that approximates $\A_{i}$
    \(\RootUN{1}{N}, \RootUN{3}{N}, \RootUN{5}{N},\ldots, \RootUN{2i+1}{N}\)
    we can see we need $\Theta(\sqrt{N})$ iterations to obtain an observable answer for any $N$.
       \item Computing such a sheet is covered by this complexity.
    \item We will, however, also determine an appropriate number of iterations mathematically.
\end{itemize}
    
\end{frame}

\subsection*{Average}

\begin{frame}{Another view of \A{} in terms of the average amplitude}{As measured after reflection and before diffusion}
\begin{itemize}
    \item The average amplitude \alert{after the oracle step} (reflection about \R{}) and \alert{prior to the diffusion step} (reflection about \S{}) is
    \[ \mu_{i} = \frac{(N-1)\B_{i} -\A_{i}}{N} \]
    because \W{} has amplitude $-\A_{i}$ and all other states have amplitude $\B_{i}$.
    \item We next show that $\A_{i+1}$ and $\A_{i}$ are related by $2\mu_{i} +\A_{i}$, namely:
    \[\A_{i+1} = 2\mu_{i} + \A_{i}
    \]
    \item In other words, the amplitude on \W{} increases in step $i+1$ when $\mu_{i}$ is positive.
\end{itemize}

\end{frame}


\begin{frame}{Lemma: $\A_{i+1} = 2\mu_{i} + \A_{i}$}
\begin{Reasoning}
\Reason{1}{Definition of average, computed after reflection but before diffusion}%
\Reason{3}{Common denominator}%
\Reason{4}{Rearrange}%
\Reason{7}{Apply division}%
\Reason{8}{From recurrence presented earlier (see Recurrence): the formula for $\A_{i+1}$}%
\Reason{9}{QED}%
\end{Reasoning}
\ScrollProof{1}{5}{%
\Next{\Four}{\mu_{i} &= \frac{(N-1)\B_{i} -\A_{i}}{N}\\}
\Next{\Three}{2\mu_{i}+\A_{i} &= 2\left(\frac{(N-1)\B_{i} -\A_{i}}{N}\right) + \A_{i}\\}
\Next{\Two}{&= \frac{2(N-1)\B_{i} -2\A_{i} +N\A_{i}}{N} \\}
\Last{&= \frac{(2N\B_{i}+N\A_{i}) -2(\A_{i}+\B_{i})}{N}}
}
\ScrollProof{6}{9}{%
\Next{\Four}{2\mu_{i}+\A_{i} &=\frac{(2N\B_{i}+N\A_{i}) -2(\A_{i}+\B_{i})}{N} \\}
\Next{\Three}{&= (2\B_{i}+\A) -\frac{2}{N}(\A_{i}+\B_{i}) \\}
\Next{\Two}{&= \A_{i+1} \\[2em]}
\Next{\One}{\A_{i+1} &= 2\mu_{i} + \A_{i}}
}
\end{frame}

\begin{frame}{Outline of analysis of \A}{Hang in there}
Given a sufficiently large problem, $N\geq 16$:
\begin{enumerate}
   \item If $\A{} \leq \frac{1}{2}$, then \A{} increases each iteration by at least \RootN{N}
    \item \A{} increases each iteration by at most \RootUN{2}{N}
        \item If we perform at most $\frac{\sqrt{N}}{8}$ iterations, then \A{} never exceeds $\frac{1}{2}$
        \item $\frac{\sqrt{N}}{8}$ iterations suffice to discover \W{}, with high probability, in a reasonable number of shots

\end{enumerate}
\end{frame}

\subsection*{Lower bound}

\begin{frame}{(1) Theorem: \A{} increases each iteration by at least \RootN{N}}{If \A{} is not already too large, and we have a reasonably large problem to solve}
\Vskip{-3.7em}\begin{theorem}
\[\Implies{\And{\A_{i}\leq \frac{1}{2}}{N\geq 4}}{{\A_{i+1}\geq \A_{i}+\RootN{N}}}\]
\end{theorem}
\begin{Reasoning}
\Reason{1}{Outcome probabilities sum to 1; both $\A_{i}$ and $\B_{i}$ are real, so we can drop $\Mag{\cdots}$}%
\Reason{3}{Applying hypothesis $\A_{i}\leq\frac{1}{2}$}
\Reason{16}{From hypothesis: $-\A_{i}\geq -\frac{1}{2}$}%
\Reason{24}{Hypothesis $\Implies{N\geq 4}{\sqrt{3N-3}-1 \geq \sqrt{N}}$; they are equal when $N=4$}%
\Reason{30}{QED}%
\Reason{32-33}{Corollary for $t$ iterations}%
\end{Reasoning}
\ScrollProof{1}{5}{%
\Next{\Four}{\Square{\A_{i}} + (N-1)\Square{\B_{i}} &= 1\\}
\Next{\Three}{\Square{\A_{i}}  &= 1 -(N-1)\Square{\B_{i}} \\}
\Next{\Two}{\Square{\left(\frac{1}{2}\right)} \geq \Square{\A_{i}}  &= 1 -(N-1)\Square{\B_{i}} \\}
\Last{\frac{1}{4} & \geq 1 -(N-1)\Square{\B_{i}} \\}
}%
\ScrollProof{6}{10}{%
\Next{\Four}{\frac{1}{4} & \geq 1 -(N-1)\Square{\B_{i}} \\}
\Next{\Three}{(N-1)\Square{\B_{i}} & \geq 1-\frac{1}{4}\\}
\Next{\Two}{(N-1)\Square{\B_{i}} & \geq \frac{3}{4}\\}
\Last{\B_{i} & \geq \sqrt{\frac{3}{4(N-1)}}}
}%
\ScrollProof{11}{14}{%
\Next{\Three}{\alert<13>{\B_{i}} & \alert<13>{\geq \sqrt{\frac{3}{4(N-1)}}}\\}
\Next{\Two}{\mu_{i} &=\frac{-\A_{i} + (N-1)\alert<13>{\B_{i}}}{N} \\}
\Last{& \geq \frac{-\A_{i}+(N-1)\alert<13>{\sqrt{\frac{3}{4(N-1)}}}}{N}}
}%
\ScrollProof{15}{18}{%
\Next{\Three}{\mu_{i} & \geq \frac{-\A_{i}+(N-1)\sqrt{\frac{3}{4(N-1)}}}{N}\\}
\Next{\Two}{ & \geq \frac{\alert<16>{-\frac{1}{2}}+(N-1)\sqrt{\frac{3}{\alert<17>{4}(N-1)}}}{N} \\}
\Last{ & \geq \frac{\alert<16>{-\frac{1}{2}}+\alert<17>{\frac{1}{2}}(N-1)\sqrt{\frac{3}{(N-1)}}}{N}}
}%
\ScrollProof{19}{22}{%
\Next{\Three}{\mu_{i} &\geq \frac{-\frac{1}{2}+\alert<17>{\frac{1}{2}}(N-1)\sqrt{\frac{3}{(N-1)}}}{N} \\}
\Next{\Two}{& \geq \frac{1}{2}\frac{-1 + \sqrt{3N-3}}{N} \\}
\Last{& \geq \frac{1}{2}\frac{\sqrt{3N-3}-1}{N}}
}
\ScrollProof{23}{26}{%
\Next{\Three}{\mu_{i} &\geq \frac{1}{2}\frac{\sqrt{3N-3}-1}{N}\\}
\Next{\Two}{& \geq \frac{1}{2}\frac{\sqrt{N}}{N} \\}
\Last{& \geq \frac{1}{2}\RootN{N}}
}%
\ScrollProof{27}{31}{%
\Next{\Four}{\mu_{i} & \geq \frac{1}{2}\RootN{N}\\}
\Next{\Three}{\A_{i+1} &= 2\mu_{i} + \A_{i} \\}
\Next{\Two}{ & \geq 2\left(\frac{1}{2}\RootN{N}\right) + \A_{i} \\[1.2em]}
\Last{\A_{i+1} &\geq \A_{i} + \RootN{N}}
}%
\ScrollProof{32}{33}{%
\Next{\Two}{\A_{i+1} &\geq \A_{i} + \RootN{N} \\}
\Next{\One}{\A_{t} &\geq \A_{0} + \RootN{N}t\\
  & \geq \RootN{N} + \RootN{N}t\\&\geq \RootUN{t+1}{N}}
}
\end{frame}


\begin{frame}{Outline of analysis of \A}{Hang in there}
Given a sufficiently large problem, $N\geq 16$:
\begin{enumerate}
    \item If $\A{} \leq \frac{1}{2}$, then \A{} increases each iteration by at least \RootN{N}
    \alert{\item \A{} increases each iteration by at most \RootUN{2}{N}}
        \item If we perform at most $\frac{\sqrt{N}}{8}$ iterations, then \A{} never exceeds $\frac{1}{2}$
        \item $\frac{\sqrt{N}}{8}$ iterations suffice to discover \W{}, with high probability, in a reasonable number of shots

\end{enumerate}
\end{frame}


\subsection*{Upper bound}

\begin{frame}{(2) Theorem: \A{} increases each iteration by at most \RootUN{2}{N}}
\Vskip{-3.8em}\begin{theorem}
    \[ \Implies{\A_{i}\geq 0}{\A_{i+1} \leq \A_{i}+\RootUN{2}{N}} \]
\end{theorem}
\begin{Reasoning}
\Reason{1}{Definition of $\mu_{i}$}%
\Reason{2}{$\A_{i} \geq 0$}%
\Reason{5}{Use bound on $\mu_{i}$}%
\Reason{6}{Use bound on $\B_{i}$}%
\Reason{9}{Simplify}%
\Reason{10}{$\sqrt{N}>\sqrt{N-1}$}%
\Reason{11}{Rearrange and QED}%
\Reason{14-15}{Corollary for $t$ iterations of Grover}
\end{Reasoning}
\only<1-7>{%
\TwoColumns{%
\only<1-3>{\ScrollProof{1}{3}{%
\Next{\Two}{\mu_{i} &= \frac{-\A_{i} + (N-1)\B_{i}}{N}\\}
\Last{\mu_{i}& \leq \frac{N-1}{N}\B_{i}\\}
}}%
\only<4-7>{%
\[ \mu_{i} \leq \frac{N-1}{N}\B_{i} \]
}
}{%
\only<1-3>{\ScrollProof{1}{3}{%
\Next{\Two}{\Square{\A_{i}} + (N-1)\Square{\B_{i}} &= 1\\
(N-1)\Square{\B_{i}} & \leq 1 \\}
\Last{\B_{i}\leq \frac{1}{\sqrt{N-1}}}
}}%
\only<4-7>{%
\[ \B_{i} \leq \frac{1}{\sqrt{N-1}} \]
}
}}%
\Vskip{-2em}\ScrollProof{4}{7}{%
\Next{\Three}{\A_{i+1} &= 2\mu_{i} + \A_{i} \\}
\Next{\Two}{& \leq 2\left(\frac{N-1}{N}\B_{i}\right) + \A_{i}\\}
\Last{\A_{i+1}& \leq 2\left(\frac{N-1}{N}\RootN{N-1}\right) + \A_{i}}
}%
\ScrollProof{8}{12}{%
\Next{\Four}{\A_{i+1}& \leq 2\left(\frac{N-1}{N}\RootN{N-1}\right) + \A_{i} \\}
\Next{\Three}{& \leq 2\frac{\sqrt{N-1}}{N} + \A_{i}\\}
\Next{\Two}{& \leq 2\frac{\sqrt{N}}{N} + \A_{i} \\}
\Last{\A_{i+1} &\leq \A_{i} + \RootUN{2}{N}}
}%
\ScrollProof{13}{15}{%
\Next{\Three}{\A_{i+1} &\leq \A_{i} + \RootUN{2}{N} \\}
\Next{\Two}{\A_{t}  &\leq \A_{0} + \RootUN{2}{N}t\\}
\Next{\One}{\A_{t}  &\leq \RootN{N} + \RootUN{2}{N}t}
}
\end{frame}

\begin{frame}{Outline of analysis of \A}{Hang in there}
Given a sufficiently large problem, $N\geq 16$:
\begin{enumerate}
    \item If $\A{} \leq \frac{1}{2}$, then \A{} increases each iteration by at least \RootN{N}
   \item \A{} increases each iteration by at most \RootUN{2}{N}
   \alert{\item If we perform at most $\frac{\sqrt{N}}{8}$ iterations, then \A{} never exceeds $\frac{1}{2}$}
        \item $\frac{\sqrt{N}}{8}$ iterations suffice to discover \W{}, with high probability, in a reasonable number of shots
\end{enumerate}
\end{frame}

\subsection*{Iterations}

\begin{frame}{(3) Theorem:  $\Implies{N\geq 16}{\A_{\frac{\sqrt{N}}{8}} \leq \frac{1}{2}}$}
\begin{Reasoning}
\Reason{1}{Corollary for $t$ iterations of Grover}%
\Reason{2-3}{Consider $t=\frac{\sqrt{N}}{8}$}
\Reason{4}{$N\geq 16\ldots$QED}
\end{Reasoning}
\ScrollProof{1}{4}{%
\Next{\Four}{\A_{t}  &\leq \RootN{N} + \RootUN{2}{N}t \\}
\Next{\Three}{\A_{\frac{\sqrt{N}}{8}}&\leq \RootN{N} + \RootUN{2}{N}\frac{\sqrt{N}}{8} \\}
\Next{\Two}{& \leq \RootN{N} + \frac{1}{4} \\}
\Next{\One}{& \leq \frac{1}{4} + \frac{1}{4} \\ &\leq \frac{1}{2}}
}
    
\end{frame}

\begin{frame}{Outline of analysis of \A}{Hang in there}
Given a sufficiently large problem, $N\geq 16$:
\begin{enumerate}
    \item If $\A{} \leq \frac{1}{2}$, then \A{} increases each iteration by at least \RootN{N}
   \item \A{} increases each iteration by at most \RootUN{2}{N}
   \item If we perform at most $\frac{\sqrt{N}}{8}$ iterations, then \A{} never exceeds $\frac{1}{2}$
        \alert{\item $\frac{\sqrt{N}}{8}$ iterations suffice to discover \W{}, with high probability, in a reasonable number of shots}
\end{enumerate}
\end{frame}

\begin{frame}{(4) Theorem: $\Implies{t=\frac{\sqrt{N}}{8}\mbox{ iterations}}{\A_{t}\geq 0.1}$}
\begin{Reasoning}
\Reason{1}{case $N<16$}%
\Reason{2}{case QED: we do not iterate at all}%
\Reason{3}{case $N\geq 16$, using our lower bound corollary}%
\Reason{5}{Substituting $t$}%
\Reason{6}{case QED}%
\end{Reasoning}
\ScrollProof{1}{2}{%
\Next{\Two}{N<16 &\rightarrow \A_{0} \geq \RootN{16}\\}
\Next{\One}{ &\rightarrow \A_{0} \geq \frac{1}{4}}
}%
\ScrollProof{3}{7}{%
\Next{\Four}{\And{N\geq 16}{t=\frac{\sqrt{N}}{8}}\rightarrow \A_{t} &\geq \RootUN{t+1}{N} \\}
\Next{\Three}{& \geq \RootUN{t}{N} \\}
\Next{\Two}{&\geq \RootN{N}\frac{\sqrt{N}}{8}\\}
\Last{\A_{\frac{\sqrt{N}}{8}}&\geq \frac{1}{8}\geq \frac{1}{10}}
}%
\end{frame}

\begin{frame}{Conclusion}
\Vskip{-3em}\[ \A_{\frac{\sqrt{N}}{8}}\geq  \frac{1}{10} \]
\begin{itemize}
    \item If $\A{} \geq 0.1$, then the probability of measuring \W{} is at least~$0.1^{2}=0.01$
    \item Let a single \emph{run} of Grover perform $\frac{\sqrt{N}}{8}$ iterations as described
    \item In one complete run of Grover, the probability of \emph{not} measuring \W{} is $0.99$.
    \item In $k$ runs, the probability of not seeing \W{} is $\left(\frac{99}{100}\right)^{k}$
    \item The probability of measuring \W{} sometime in $k$ runs is then \( 1-\left(\frac{99}{100}\right)^{k} \)
    \item If we perform $110$~runs then we measure \W{} with probability~$\approx\frac{2}{3}$
    \item In practice, we would keep running until we find \W{}, with high confidence it will show up in hundreds of tries.
\end{itemize}
\end{frame}
\end{Grover}

\begin{frame}{Summary}{What have we learned?}

\begin{itemize}
    \item Grover's algorithm is astounding in that it can find a secret element among $2^{n}=N$ possibilities.
    \item Performing that task classically would require looking at \emph{all} elements, taking~$\Theta(N)$ time.
    \item Grover can perform this task in~$\Theta(\sqrt{N})$ time.
    \item While this is not an exponential improvement, it has prompted cryptographic systems to use longer keys.
    \item Executing Grover on actual hardware yields surprisingly poor results, as of this writing.
\end{itemize}
\end{frame}
