\SetTitle{14}{The Mermin--Peres game}{Solution and Proofs}{14}

\section*{Solution}

\begin{frame}{Quantum-based solution}{We use two EPR pairs}

\Vskip{-4em}\begin{center}
\adjustbox{valign=t}{%
\begin{quantikz} 
\lstick{\QZero}\qw &  \gate{\Hadamard} & \ctrl{1} & \qw\rstick{\ColorOne{\mbox{$a_1$}}} \\
\lstick{\QZero}\qw & \qw & \targ{} & \qw \rstick{\ColorTwo{\mbox{$b_1$}}} 
\end{quantikz}}%
\hbox to 3ex{\hss}%
\adjustbox{valign=t}{%
\begin{quantikz} 
\lstick{\QZero}\qw &  \gate{\Hadamard} & \ctrl{1} & \qw\rstick{\ColorOne{\mbox{$a_2$}}} \\
\lstick{\QZero}\qw & \qw & \targ{} & \qw\rstick{\ColorTwo{\mbox{$b_2$}}} 
\end{quantikz}}\end{center}%

\begin{itemize}[<+->]
    \item Two EPR pairs are created, each in state \TwoSup{00}{11}.
    \item \ColorOne{Alice} receives~\ColorOne{\QState{a}=\ket{a_{1}a_{2}}} and \ColorTwo{Bob} receives~\ColorTwo{\QState{b}=\ket{b_{1}b_{2}}}
    \item Depending on the row (for Alice) and the column (for Bob), they will carry out measurements based on a table, in a basis related to their particular row or column.
    \item We must prove that they
    \begin{itemize}
        \item agree on a value at the cell intersecting the row and column, and
        \item obtain the necessary row and column products.
    \end{itemize}
\end{itemize}
    
\end{frame}

\begin{frame}{Solution}{Expressed in measurement operators}

\Vskip{-4.5em}\TwoUnequalColumns{0.6\textwidth}{0.4\textwidth}{%
\only<1-4>{%
\begin{itemize}
    \item<1-> Each square shows the measurement operators used by a player on two qubits.  
    \item<2-> The \alert<2>{\Identity{}} measurement operator has an eigenvalue of $+1$, regardless of its input.
    \item<3-> The \alert<3>{bottom left two entries} are marked so that their outcome is complimented.
    \item<4-> The result reported in a given square by a player is the eigenvalue associated with the two-qubit measurement performed by \ColorOne{her} or \ColorTwo{him}, complimented in \alert<4>{these} two squares.
\end{itemize}}
\only<5->{%
\begin{itemize}
   \item<5-> Each square's operator is already expressed as a product of two gates.
   \item<6-> A two-qubit measurement can be performed separately on each qubit\visible<7->{, \emph{unless} the input or measurement basis is entangled.}
   \item<8-> A player's qubits are not initially entangled with each other!  Each is entangled with one of the \emph{other player's} qubits.
   \item<9-> When measured separately, the overall square's result is the product of the measurements' eigenvalues\visible<10->{, \alert<10>{possibly complimented}.}
\end{itemize}
}
}{%
\begin{center}
    \begin{MPSquareEnv}
    \def\MyMarkupSquareComp####1{\alert<3,4,10>{####1}}
    \def\MyMarkupSquareI####1{\alert<2>{####1}}
    \MPSoln{}
    \end{MPSquareEnv}
\end{center}
}

\end{frame}




\begin{frame}{Three possible proof approaches}{We do two of them here}

\TwoUnequalColumns{0.6\textwidth}{0.4\textwidth}{%
\only<1-2>{Obligation:
\begin{itemize}
    \item<1-> We must show that for any row, \ColorOne{Alice}'s results multiply to~$+1$ and \ColorTwo{Bob}'s results multiply to~$-1$.
    \item<2-> We must show that for the square in common for a given round of the game, the players agree on the value there.
\end{itemize}}%
\only<3->{%
\begin{enumerate}
    \item<3-> We can show algebraically that the products in the \ColorOne{rows} are~$+1$ and the products in the \ColorTwo{columns} are~$-1$.
    \begin{itemize}
        \item This results at the end are correct.
        \item But no square-by-square account is given.
    \end{itemize}
    \item<4-> Based on a \ColorOne{row} (or \ColorTwo{column}), one two-qubit measurement is made in a prescribed basis.  That result determines what is reported in each square in that \ColorOne{row} (or \ColorTwo{column}).
    \item<5-> \ColorOne{Alice} and \ColorTwo{Bob} make and report a measurement in each square of their \ColorOne{row} or \ColorTwo{column}.
\end{enumerate}}
}{%
\begin{center}
\begin{MPSquareEnv}
    \MPSoln{}
\end{MPSquareEnv}
\end{center}
}
\end{frame}

\section*{Algebraic proof}

\begin{frame}{Algebraic proof}{A sequence of measurements would culminate in the correct result}

Recall ($\times$ and $\cdot$ both mean matrix multiply):
\[ (\TensProd{A}{D}) \times (\TensProd{B}{E}) = \TensProd{(A\cdot B)}{(D\cdot E)} \]
and this extends to three product terms:
\[
(\TensProd{A}{D}) \times (\TensProd{B}{E}) \times (\TensProd{C}{F}) = \TensProd{(A\cdot B\cdot C)}{(D\cdot E\cdot F)}
\]
We consider each \ColorOne{row} and \ColorTwo{column} of the solution and show that
\begin{itemize}
    \item The product matrix of each row is \Identity, whose measurement always yields $+1$ regardless of input state.
    \item The product matrix of each column is $-\Identity$ (or \Not{\Identity}), whose measurement always yields $-1$ regardless of input state.
\end{itemize}
When operators are applied left-to-right in a circuit, their associated matrices are multipled from right-to-left.

\end{frame}

\begin{frame}{Rows}{Algebraic proof}
\Vskip{-3em}\TwoUnequalColumns{0.68\textwidth}{0.32\textwidth}{%
\begin{Reasoning}
\Reason{1}{Row 1}%
\Reason{2}{Row 2}%
\Reason{3}{Row 3}%
\end{Reasoning}
\begin{align*}
(\ColorThree{\TensProd{\PauliZ}{\PauliZ}})\cdot (\ColorFour{\TensProd{\Identity}{\PauliZ}})\cdot (\ColorFive{\TensProd{\PauliZ}{\Identity}}) =& \TensProd{\ColorThree{\PauliZ}\ColorFour{\Identity}\ColorFive{\PauliZ}}{\ColorThree{\PauliZ}\ColorFour{\PauliZ}\ColorFive{\Identity}} \\
=& \TensProd{\Identity}{\Identity} \\
\visible<2->{(\ColorThree{\TensProd{\PauliX}{\PauliX}})\cdot (\ColorFour{\TensProd{\PauliX}{\Identity}})\cdot (\ColorFive{\TensProd{\Identity}{\PauliX}}) =& \TensProd{\ColorThree{\PauliX}\ColorFour{\PauliX}\ColorFive{\Identity}}{\ColorThree{\PauliX}\ColorFour{\Identity}\ColorFive{\PauliX}} \\
=& \TensProd{\Identity}{\Identity}} \\
\visible<3->{(\ColorThree{\TensProd{\PauliY}{\PauliY}})\cdot (\alert{\Not{\ColorFour{\TensProd{\PauliX}{\PauliZ}}}})\cdot (\alert{\Not{\ColorFive{\TensProd{\PauliZ}{\PauliX}}}}) =& \alert{- -} (\TensProd{\ColorThree{\PauliY}\ColorFour{\PauliX}\ColorFive{\PauliZ}}{\ColorThree{\PauliY}\ColorFour{\PauliZ}\ColorFive{\PauliX}}) \\
=& \TensProd{\SQBG{\relax}{-\NiceI}{0}{0}{-\NiceI}}{\SQBG{\relax}{\NiceI}{0}{0}{\NiceI}} \\
=& \TensProd{\Identity}{\Identity}}
\end{align*}
}{%
\begin{center}
\adjustbox{scale=0.7}{\begin{MPSquareEnv}
    \def\TpRt####1{\textcolor<1>{\RCthree}{####1}}%
    \def\TpCt####1{\textcolor<1>{\RCfour}{####1}}%
    \def\TpLf####1{\textcolor<1>{\RCfive}{####1}}%
    \def\MdRt####1{\textcolor<2>{\RCthree}{####1}}%
    \def\MdCt####1{\textcolor<2>{\RCfour}{####1}}%
    \def\MdLf####1{\textcolor<2>{\RCfive}{####1}}%
    \def\BtRt####1{\textcolor<3>{\RCthree}{####1}}%
    \def\BtCt####1{\textcolor<3>{\RCfour}{####1}}%
    \def\BtLf####1{\textcolor<3>{\RCfive}{####1}}%
    \MPSoln{}
\end{MPSquareEnv}}
\end{center}
We perform operations left-to-right in a row.  Note that the matrices multiply in the opposite order.
}
\end{frame}

\begin{frame}{Columns}{Algebraic proof}
\Vskip{-3em}\TwoUnequalColumns{0.71\textwidth}{0.29\textwidth}{%
\begin{Reasoning}
\Reason{1}{Column 1}%
\Reason{2}{Column 2}%
\Reason{3}{Column 3}%
\end{Reasoning}
\begin{align*}
(\alert{\Not{\ColorThree{\TensProd{\PauliZ}{\PauliX}}}})\cdot (\ColorFour{\TensProd{\Identity}{\PauliX}})\cdot (\ColorFive{\TensProd{\PauliZ}{\Identity}}) =& \alert{-} \TensProd{\ColorThree{\PauliZ}\ColorFour{\Identity}\ColorFive{\PauliZ}}{\ColorThree{\PauliX}\ColorFour{\PauliX}\ColorFive{\Identity}} \\
=& \alert{-}(\TensProd{\Identity}{\Identity}) \\
\visible<2->{(\alert{\Not{\ColorThree{\TensProd{\PauliX}{\PauliZ}}}})\cdot (\ColorFour{\TensProd{\PauliX}{\Identity}})\cdot (\ColorFive{\TensProd{\Identity}{\PauliZ}}) =& \alert{-}\TensProd{\ColorThree{\PauliX}\ColorFour{\PauliX}\ColorFive{\Identity}}{\ColorThree{\PauliZ}\ColorFour{\Identity}\ColorFive{\PauliZ}} \\
=& \alert{-}(\TensProd{\Identity}{\Identity})} \\
\visible<3->{(\ColorThree{\TensProd{\PauliY}{\PauliY}})\cdot (\ColorFour{\TensProd{\PauliX}{\PauliX}})\cdot (\ColorFive{\TensProd{\PauliZ}{\PauliZ}}) =& (\TensProd{\ColorThree{\PauliY}\ColorFour{\PauliX}\ColorFive{\PauliZ}}{\ColorThree{\PauliY}\ColorFour{\PauliX}\ColorFive{\PauliZ}}) \\
=& \TensProd{\SQBG{\relax}{-\NiceI}{0}{0}{-\NiceI}}{\SQBG{\relax}{-\NiceI}{0}{0}{-\NiceI}} \\
=& -(\TensProd{\Identity}{\Identity})}
\end{align*}
}{%
\begin{center}
\adjustbox{scale=0.6}{\begin{MPSquareEnv}
    \def\BtLf####1{\textcolor<1>{\RCthree}{####1}}%
    \def\MdLf####1{\textcolor<1>{\RCfour}{####1}}%
    \def\TpLf####1{\textcolor<1>{\RCfive}{####1}}%
    \def\BtCt####1{\textcolor<2>{\RCthree}{####1}}%
    \def\MdCt####1{\textcolor<2>{\RCfour}{####1}}%
    \def\TpCt####1{\textcolor<2>{\RCfive}{####1}}%
    \def\BtRt####1{\textcolor<3>{\RCthree}{####1}}%
    \def\MdRt####1{\textcolor<3>{\RCfour}{####1}}%
    \def\TpRt####1{\textcolor<3>{\RCfive}{####1}}%
    \MPSoln{}
\end{MPSquareEnv}}
\end{center}
We perform operations top-down in a column.  Note that the matrices multiply in the opposite order.
}
\end{frame}

\begin{frame}{Summary}{Up to this point}
\begin{itemize}[<+->]
    \item The results we have obtained hold for \emph{any} initial state.
    \item The game's initial states, each \TwoSup{00}{11}, are special because they cause \Alice{} and \Bob{} to agree on a value in their common square.
    \item For \Alice{}, the product of measurements along any row is~$+1$.
    \item For \Bob{}, the product of measurements along any column is~$-1$.
    \item While the final results are confirmed, the proof is not instructive concerning the three measurements each player is supposed to report.
    \item This is akin to deciding not to play the game at all, because they know they should win, if only they knew how to play the game.
\end{itemize}
    
\end{frame}

\section*{One measurement each}

\begin{frame}{Actually playing the game}{Making only one measurement}

\Vskip{-3em}\TwoUnequalColumns{0.6\textwidth}{0.4\textwidth}{%
Idea based on analysis from \href{http://philsci-archive.pitt.edu/18398/}{this paper}:
\only<1-5>{%
\begin{itemize}
    \item<1-> \Alice{} measures her two qubits \emph{once}, in a basis prescribed for her row.
    \item<3-> However, each operator may have a \emph{different eigenvalue} associated with \QState{}.
    \item<4-> \Alice{} reports the eigenvalue for each operator in her row, knowing their product is~\ColorOne{$+1$}.
    \item<5-> \Bob{} uses a different basis prescribed for his column, with his eigenvalues multiplying to~\ColorTwo{$-1$}.
\end{itemize}}
\only<6-10>{%
\begin{itemize}
    \item<6-> \Alice{} knows the operators of her rows.
    \item<7-> She uses \MATLAB{} to compute a row's operators' common eigenbasis \NamedBasis{B} \Set{\QState{0},\QState{1},\QState{2},\QState{3}}.
    \item<8-> The matrix (gate) $T$ that transforms the standard basis to \NamedBasis{B} has those states as its columns.
    \item<9-> The gate \Conj{T} transforms from \NamedBasis{B} to the standard basis.
    \item<10-> Because we perform only one measurement, we need only \Conj{T} followed by a measurement in the standard basis.
\end{itemize}}%
\only<11-13>{%
\begin{itemize}
    \item<11-> For each row, \Alice{} has a table that provides the eigenvalue for each square.
    \item<12-> The algebraic proof guarantees that those will multiply to~$+1$.
    \item<13-> The proof holds for \emph{any} initial state, but measurement in \NamedBasis{B} causes \Alice{} to report the same result as \Bob{} in their shared square.
    \end{itemize}}
    \only<14->{%
    \begin{itemize}
    \item<14-> Recall that each of \Alice's qubits is entangled with one of \Bob's qubits.  Each such pair is in the state \TwoSup{00}{11}.
    \item<15-> As a example where careless measurements cause problems, suppose
    \begin{itemize}
        \item \Alice{} has row 1 and measures her first qubit in the standard (\PauliZ) basis.
        \item \Bob{} has column 1 but measures his first qubit (thanks to Alice, now in state~\QZero{} or~\QOne{}) in the~\PauliX{} basis.
    \end{itemize}
    \item<16->\Bob{} is equally likely to see \QZero{} or \QOne{}, which could contradict \Alice's measurement.
\end{itemize}}
}{%
\begin{center}
\begin{MPSquareEnv}
    \def\TpLf####1{\textcolor<14-16>{\RCfive}{####1}}
    \MPSoln{}
\end{MPSquareEnv}
\end{center}
}
\end{frame}

\begin{frame}{The details}{How to interpret the upcoming slides}
%%
%% beamer bug is that a slide 3 is generated
%%
\only<1-2>{%
\Vskip{-3em}\TwoUnequalColumns{0.6\textwidth}{0.4\textwidth}{%
\begin{MPbasis}{Op1}{Op2}{Op3}
\def\A{\Col{\cdot}{\cdot}{\cdot}{\cdot}}
\def\B{\A}
\def\C{\A}
\def\D{\A}
\def\EigA{\Row{ }{ }{ }}
\def\EigB{\EigA}
\def\EigC{\EigA}
\def\EigD{\EigA}
\only<1>{\Bases{\relax}}
\only<2>{\Vskip{-3em}\Eigs}
    
\end{MPbasis}
\only<1>{%
\begin{itemize}
    \item The matrix $T$ is formed from the common eigenbasis of three operators in a given \ColorOne{row} or \ColorTwo{column}.
    \item \MATLAB{} can verify that $T$ is unitary and that each column is an eigenvector for each of the three operators.
\end{itemize}}%
\only<2>{%
\begin{itemize}
    \item A table of eigenvalues is given for each of the three operators and each of the four basis states.
    \item \MATLAB{} can verify the entries are correct.
    \item The entries for a given basis state multiply to~$+1$ for \Alice{} and to~$-1$ for \Bob. 
\end{itemize}
}
}{%
\begin{center}
\begin{MPSquareEnv}
    \MPSoln{}
\end{MPSquareEnv}
\end{center}
}
}
\end{frame}

\begin{frame}{Alice}{Row 1, using the \TensProd{\PauliZ}{\PauliZ} basis}
\TwoUnequalColumns{0.63\textwidth}{0.37\textwidth}{%
\begin{MPbasis}{\TensProd{\PauliZ}{\Identity}}{\TensProd{\Identity}{\PauliZ}}{\TensProd{\PauliZ}{\PauliZ}}
\def\A{\Col{1}{0}{0}{0}}
\def\B{\Col{0}{1}{0}{0}}
\def\C{\Col{0}{0}{1}{0}}
\def\D{\Col{0}{0}{0}{1}}
\def\EigA{\Row{\PP}{\PP}{\PP}}
\def\EigB{\Row{\PP}{\MM}{\MM}}
\def\EigC{\Row{\MM}{\PP}{\MM}}
\def\EigD{\Row{\MM}{\MM}{\PP}}
\only<1-5>{\Bases{}}

\Eigs
\end{MPbasis}%
\only<6->{%

\begin{itemize}
    \item If \Alice{} measures binary value $b$ then she uses the table, row~$b$, to report her results.
    \item For example, if \Alice{} measures \alert{\ket{10}}, then she reports the results in row \ColorFive{\QState{2}} for the three squares, left to right.  And of course, their product is $+1$.
\end{itemize}

}
}{%
\Vskip{-2em}
\begin{center}
\adjustbox{scale=0.8}{\MPCircuit}
\end{center}
\begin{center}
\adjustbox{scale=0.7}{\begin{MPSquareEnv}
    \def\TpLf####1{\ColorOne{####1}}
    \def\TpCt####1{\TpLf{####1}}
    \def\TpRt####1{\TpLf{####1}}
    \MPSoln{}
\end{MPSquareEnv}}
\end{center}}
    
\end{frame}

\begin{frame}{Alice}{Row 2, using the \TensProd{\PauliX}{\PauliX} basis}
\TwoUnequalColumns{0.63\textwidth}{0.37\textwidth}{%
\Vskip{-2em}\begin{MPbasis}{\TensProd{\Identity}{\PauliX}}{\TensProd{\PauliX}{\Identity}}{\TensProd{\PauliX}{\PauliX}}
\def\A{\Col{1}{1}{1}{1}}
\def\B{\Col{1}{-1}{1}{-1}}
\def\C{\Col{1}{1}{-1}{-1}}
\def\D{\Col{1}{-1}{-1}{1}}
\def\EigA{\Row{\PP}{\PP}{\PP}}
\def\EigB{\Row{\MM}{\PP}{\MM}}
\def\EigC{\Row{\PP}{\MM}{\MM}}
\def\EigD{\Row{\MM}{\MM}{\PP}}
\def\AltA{\ket{++}}
\def\AltB{\ket{+-}}
\def\AltC{\ket{-+}}
\def\AltD{\ket{--}}
\Bases{\frac{1}{2}}

\only<1-5>{\Eigs}
\end{MPbasis}
\only<6->{%
\begin{itemize}
    \item Note $T=\TensProd{\Hadamard}{\Hadamard}$ and $T=\Conj{T}$.
    \item This is as expected to map $(\TensProd{\PauliZ}{\PauliZ})\mapsto(\TensProd{\PauliX}{\PauliX})$
\end{itemize}
}
}{%
\Vskip{-2em}
\begin{center}
\adjustbox{scale=0.8}{\MPCircuit}
\end{center}
\begin{center}\adjustbox{scale=0.7}{%
\begin{MPSquareEnv}
    \def\MdLf####1{\ColorOne{####1}}
    \def\MdCt####1{\MdLf{####1}}
    \def\MdRt####1{\MdLf{####1}}
    \MPSoln{}
\end{MPSquareEnv}}
\end{center}}
    
\end{frame}

\begin{frame}{Alice}{Row 3, using an entangled basis}
\TwoUnequalColumns{0.63\textwidth}{0.37\textwidth}{%
\begin{MPbasis}{\Not{\TensProd{\PauliZ}{\PauliX}}}{\Not{\TensProd{\PauliX}{\PauliZ}}}{\TensProd{\PauliY}{\PauliY}}
\def\A{\Col{1}{1}{1}{-1}}
\def\B{\Col{1}{1}{-1}{1}}
\def\C{\Col{1}{-1}{1}{1}}
\def\D{\Col{-1}{1}{1}{1}}
\def\EigA{\Row{\MM}{\MM}{\PP}}
\def\EigB{\Row{\MM}{\PP}{\MM}}
\def\EigC{\Row{\PP}{\MM}{\MM}}
\def\EigD{\Row{\PP}{\PP}{\PP}}
\Bases{\frac{1}{2}}

\only<1-5>{\Eigs}
\end{MPbasis}
\only<6->{%

The columns of $T$ can also be expressed as:
\begin{center}\begin{tabular}{cc}
\ColorThree{\QState{0}} & \TwoSup{0+}{1-} \\
\ColorFour{\QState{1}} & \TwoSupOp{\ket{0+}}{\ket{1-}}{-} \\
\ColorFive{\QState{2}} & \TwoSup{1+}{0-} \\
\ColorSix{\QState{3}} & \TwoSupOp{\ket{1+}}{\ket{0-}}{-}
\end{tabular}\end{center}
}
}{%
\Vskip{-2em}
\begin{center}
\adjustbox{scale=0.8}{\MPCircuit}
\end{center}
\begin{center}\adjustbox{scale=0.7}{%
\begin{MPSquareEnv}
    \def\BtLf####1{\ColorOne{####1}}
    \def\BtCt####1{\BtLf{####1}}
    \def\BtRt####1{\BtLf{####1}}
    \MPSoln{}
\end{MPSquareEnv}}
\end{center}}
    
\end{frame}

\begin{frame}{Bob}{Column 1, using basis \TensProd{\PauliZ}{\PauliX}}
\TwoUnequalColumns{0.63\textwidth}{0.37\textwidth}{%
\begin{MPbasis}{\TensProd{\PauliZ}{\Identity}}{\TensProd{\Identity}{\PauliX}}{\Not{\TensProd{\PauliZ}{\PauliX}}}
\def\A{\Col{1}{1}{0}{0}}
\def\B{\Col{1}{-1}{0}{0}}
\def\C{\Col{0}{0}{1}{1}}
\def\D{\Col{0}{0}{1}{-1}}
\def\EigA{\Row{\PP}{\PP}{\MM}}
\def\EigB{\Row{\PP}{\MM}{\PP}}
\def\EigC{\Row{\MM}{\PP}{\PP}}
\def\EigD{\Row{\MM}{\MM}{\MM}}
\def\AltA{\ket{0+}}
\def\AltB{\ket{0-}}
\def\AltC{\ket{1+}}
\def\AltD{\ket{1-}}
\Bases{\RootTwo{}}

\only<1-5>{\Eigs}
\end{MPbasis}
\only<6->{%

\begin{itemize}
    \item Note $T=\TensProd{\Identity}{\Hadamard}$ and $T=\Conj{T}$.
    \item This is as expected to map $(\TensProd{\PauliZ}{\PauliZ})\mapsto(\TensProd{\PauliZ}{\PauliX})$
\end{itemize}
}
}{%
\Vskip{-2em}
\begin{center}
\adjustbox{scale=0.8}{\MPCircuit}
\end{center}
\begin{center}\adjustbox{scale=0.7}{%
\begin{MPSquareEnv}
    \def\TpLf####1{\ColorTwo{####1}}
    \def\MdLf####1{\TpLf{####1}}
    \def\BtLf####1{\TpLf{####1}}
    \MPSoln{}
\end{MPSquareEnv}}
\end{center}}
    
\end{frame}

\begin{frame}{Bob}{Column 2, using basis \TensProd{\PauliZ}{\PauliX}}
\TwoUnequalColumns{0.63\textwidth}{0.37\textwidth}{%
\begin{MPbasis}{\TensProd{\Identity}{\PauliZ}}{\TensProd{\PauliX}{\Identity}}{\Not{\TensProd{\PauliX}{\PauliZ}}}
\def\A{\Col{1}{0}{1}{0}}
\def\B{\Col{0}{1}{0}{1}}
\def\C{\Col{1}{0}{-1}{0}}
\def\D{\Col{0}{1}{0}{-1}}
\def\EigA{\Row{\PP}{\PP}{\MM}}
\def\EigB{\Row{\MM}{\PP}{\PP}}
\def\EigC{\Row{\PP}{\MM}{\PP}}
\def\EigD{\Row{\MM}{\MM}{\MM}}
\def\AltA{\ket{+0}}
\def\AltB{\ket{+1}}
\def\AltC{\ket{-0}}
\def\AltD{\ket{-1}}
\Bases{\RootTwo{}}

\only<1-5>{\Eigs}
\end{MPbasis}
\only<6->{%

\begin{itemize}
    \item Note $T=\TensProd{\Hadamard}{\Identity}$ and $T=\Conj{T}$.
    \item This is as expected to map $(\TensProd{\PauliZ}{\PauliZ})\mapsto(\TensProd{\PauliX}{\PauliZ})$
\end{itemize}
}
}{%
\Vskip{-2em}
\begin{center}
\adjustbox{scale=0.8}{\MPCircuit}
\end{center}
\begin{center}\adjustbox{scale=0.7}{%
\begin{MPSquareEnv}
    \def\TpCt####1{\ColorTwo{####1}}
    \def\MdCt####1{\TpCt{####1}}
    \def\BtCt####1{\TpCt{####1}}
    \MPSoln{}
\end{MPSquareEnv}}
\end{center}}
    
\end{frame}

\begin{frame}{Bob}{Column 3, using an entangled basis}
\TwoUnequalColumns{0.63\textwidth}{0.37\textwidth}{%
\begin{MPbasis}{\TensProd{\PauliZ}{\PauliZ}}{\TensProd{\PauliX}{\PauliX}}{\TensProd{\PauliY}{\PauliY}}
\def\A{\Col{1}{0}{0}{1}}
\def\B{\Col{1}{0}{0}{-1}}
\def\C{\Col{0}{1}{1}{0}}
\def\D{\Col{0}{1}{-1}{0}}
\def\EigA{\Row{\PP}{\PP}{\MM}}
\def\EigB{\Row{\PP}{\MM}{\PP}}
\def\EigC{\Row{\MM}{\PP}{\PP}}
\def\EigD{\Row{\MM}{\MM}{\MM}}
\Bases{\RootTwo{}}

\only<1-5>{\Eigs}
\end{MPbasis}
\only<6->{%

The columns of $T$ can also be expressed as:
\begin{center}\begin{tabular}{cc}
\ColorThree{\QState{0}} & \TwoSup{00}{11} \\
\ColorFour{\QState{1}} & \TwoSupOp{\ket{00}}{\ket{11}}{-} \\
\ColorFive{\QState{2}} & \TwoSup{01}{10} \\
\ColorSix{\QState{3}} & \TwoSupOp{\ket{01}}{\ket{10}}{-}
\end{tabular}\end{center}
}
}{%
\Vskip{-2em}
\begin{center}
\adjustbox{scale=0.8}{\MPCircuit}
\end{center}
\begin{center}\adjustbox{scale=0.7}{%
\begin{MPSquareEnv}
    \def\TpRt####1{\ColorTwo{####1}}
    \def\MdRt####1{\TpRt{####1}}
    \def\BtRt####1{\TpRt{####1}}
    \MPSoln{}
\end{MPSquareEnv}}
\end{center}}
    
\end{frame}


\begin{frame}{Scenario with the square: Row 1 and Column 3}{Alice's row number tells her what basis to use for measurement.\\Independently, Bob's column number tells him what basis to use.}
\begin{MPScene}
\TwoUnequalColumns{0.7\textwidth}{0.3\textwidth}{%
\Vskip{-4em}\begin{center}
\adjustbox{valign=t,scale=0.8}{%
\begin{quantikz}
\lstick{\ColorOne{\QZero}}\qw &  \gate{\Hadamard} & \ctrl{1} & \qw\rstick{\AoneColor{\Aone\visible<4->{=\QZero}}} \\
\lstick{\ColorTwo{\QZero}}\qw & \qw & \targ{} & \qw \rstick{\BoneColor{\Bone\visible<4->{=\QZero}}}
\end{quantikz}}%
\hbox to 3ex{\hss}%
\adjustbox{valign=t,scale=0.8}{%
\begin{quantikz}
\lstick{\ColorOne{\QZero}}\qw &  \gate{\Hadamard} & \ctrl{1} & \qw\rstick{\AtwoColor{\Atwo\visible<5->{=\QZero}}} \\
\lstick{\ColorTwo{\QZero}}\qw & \qw & \targ{} & \qw\rstick{\BtwoColor{\Btwo\visible<5->{=\QZero}}}
\end{quantikz}}\end{center}%
\visible<1->{
\Vskip{-2em}\begin{ProtocolDialog}{0.25\textwidth}{0.375\textwidth}{0.375\textwidth}
\visible<2->{\All{{\small\ColorOne{Row 1} \visible<3->{\ColorTwo{Col 1}}}}{Measure in \TensProd{\AoneColor{\PauliZ}}{\AtwoColor{\PauliZ}}}{\visible<3->{Measure \NamedGate{Bell}(\BoneColor{$b_1$}, \BtwoColor{$b_2$})}}}
\visible<4->{\AliceBob{Measure \AoneColor{\Aone{}=\QZero}}{\alert{Collapse} \BoneColor{\Bone{}=\QZero}}}
\visible<5->{\AliceBob{Measure \AtwoColor{\Atwo{}=\QZero}}{\alert{Collapse} \BtwoColor{\Btwo{}=\QZero}}}
\visible<7->{\AliceBob{\ReportCommon{\AoneColor{1}\times \IColor{1}}}{$\ket{\BoneColor{0}\BtwoColor{0}}$ from \NamedGate{Bell} to \NamedGate{ZZ}}}
\visible<9->{\AliceBob{\Report{\IColor{1}\times \AtwoColor{1}}}{is $\RootTwo(\ket{00}+\ket{01})$}}
\visible<11->{\AliceBob{\Report{\AoneColor{1}\times \AtwoColor{1}}}{Bob measures \Ambig{\ket{\BoneColor{0}\BtwoColor{1}}}}}
\visible<13->{\Bob{\ReportCommon{1}}}
\visible<15->{\Bob{\Report{-1}}}
\visible<17->{\Bob{\Report{1}}}
\end{ProtocolDialog}}
}{%
\Vskip{-3em}\begin{itemize}\small
    \item \Alice{} has \Aone{} and \Atwo.
    \item \MakeSameWidth[l]{Alice}{\Bob{}} has \Bone{} and \Btwo.
    \item $\QZero{} = +1$
    \item $\QOne{} = -1$
    \item \Ovalbox{$n$} random result~$n$
    \item \ReportCommon{\mbox{agreement}}
\end{itemize}
\Vskip{-3.5em}\begin{center}\adjustbox{scale=0.75}{%
\hskip -4em\begin{MPSquareEnv}
    \def\TpLf####1{\only<2>{\ColorOne{####1}}\only<6>{\ColorOne{\TensProd{\AoneColor{\PauliZ}}{\AtwoColor{\Identity}}}}\only<1,3-5,12->{####1}\only<7-11>{\ColorOne{\Answer{+1}}}}
    \def\TpCt####1{\only<8>{\ColorOne{\TensProd{\AoneColor{\Identity}}{\AtwoColor{\PauliZ}}}}\only<2>{\ColorOne{####1}}\only<1,3-7,12->{####1}\only<9-11>{\ColorOne{\Answer{+1}}}}
    \def\TpRt####1{\only<12>{\ColorTwo{\TensProd{\BoneColor{\PauliZ}}{\BtwoColor{\PauliZ}}}}\only<2>{\ColorOne{####1}}\only<3>{\ColorTwo{####1}}\only<1,4-9>{####1}\only<10>{\ColorOne{\TensProd{\AoneColor{\PauliZ}}{\AtwoColor{\PauliZ}}}}\only<11>{\ColorOne{\Answer{+1}}}\only<13->{\ColorTwo{\Answer{+1}}}}
    \def\MdRt####1{\only<14>{\ColorTwo{\TensProd{\BoneColor{\PauliX}}{\BtwoColor{\PauliX}}}}\only<3>{\ColorTwo{####1}}\only<1-2,4-13>{####1}\only<15->{\ColorTwo{\Answer{-1}}}}
    \def\BtRt####1{\only<16>{\ColorTwo{\TensProd{\BoneColor{\PauliY}}{\BtwoColor{\PauliY}}}}\only<3>{\ColorTwo{####1}}\only<1-2,4-15>{####1}\only<17->{\ColorTwo{\Answer{+1}}}}
    \MPSoln{}
\end{MPSquareEnv}}
\end{center}
}
\end{MPScene}
\end{frame}


\section*{Summary}

\begin{frame}{Summary}{What have we learned?}
\begin{itemize}[<+->]
    \item With quantum computation, the two games we study can be won more often than is classically possible.
    \item For Mermin--Peres, we have an algebraic proof that the players report correct answers.
    \item We also have an implementation where a single measurement is made by each player, who can then respond with a value for each square.
    \item We did not prove that the players compute the same result for their common square.
    \item We did not develop an implementation where the players make three measurements and report each separately.
    \item The above tasks may be a project or homework in the future.
\end{itemize}
\end{frame}

