\SetTitle{8}{Universal quantum gates}{Several approaches, culminating in the qiskit U gate}{08}

\begin{frame}{Universality}{Some initial thoughts}

\begin{itemize}[<+->]
    \item We captured the essence of a single qubit using two real rotational parameters $\phi$ and $\theta$.  Those parameters describe points on the unit sphere, each point corresponding to a quantum state.
    \item A quantum gate is a mapping from states to states.
    \item Because such gates must have inverses, the mapping established by such gates must be \href{https://en.wikipedia.org/wiki/Bijection}{bijective}.
    \item We can think of a gate as a permutation of states, or points, on the Bloch sphere.
    \item How many parameters does it take to characterize an arbitrary single-qubit quantum gate?
\end{itemize}

    
\end{frame}

\begin{frame}{Quantum gate}{Properties}

\TwoUnequalColumns{0.4\textwidth}{0.6\textwidth}{%
We begin with an arbitrary unitary matrix $U$ to represent a quantum gate for a single qubit:
\[
U=\SQBG{\relax}{a}{c}{b}{d}
\]
}{%
\begin{itemize}[<+->]
    \item $a, b, c, d$ are complex
    \item So, apparently 8 parameters
    \item $U$ is invertible and $\Inverse{U}=\Conj{U}$
    \item The columns form an orthonormal basis
    \item The rows form an orthonormal basis
\end{itemize}}
\BigSkip{}
\visible<+->{%
These constraints will take us from 8 parameters to only 3.}
    
\end{frame}

\begin{frame}{The Qiskit U gate}{Theorem statement}
\begin{theorem}
   Up to a global phase, any $2\times 2$ unitary matrix can be expressed as
    \[
    U(\theta, \phi, \lambda) = 
    \SQBG{\relax}{\cos(\theta/2)}{-\ExpPhase{\lambda}\sin(\theta/2)}{\ExpPhase{\phi}\sin(\theta/2)}{\ExpPhase{(\phi+\lambda)}\cos(\theta/2)}
    \] where
    $0 \leq \theta,\phi,\lambda < 2\pi$
    \end{theorem}
    For example, the Hadamard gate \HMatrix{} is $U(\pi/2, 0, \pi)$, which yields
    \[
    \SQBG{\relax}{\cos \pi/4}{-\ExpPhase{\pi}\sin \pi/4}{\ExpPhase{0}\sin \pi/4}{\ExpPhase{\pi}\cos pi/4} = \HMatrix{}
    \]

\end{frame}

\begin{frame}{Proof}{We begin with an arbitrary unitary matrix $U$ with complex entries.}
\TwoColumns{%
$U = \SQBG{\relax}{a}{b}{c}{d}$ so
$\Conj{U} = \SQBG{\relax}{\Conj{a}}{\Conj{c}}{\Conj{b}}{\Conj{d}}$

\MedSkip{}
Because $U$ is unitary:
\begin{itemize}
\item<1-> Each column of $U$ (row of \Conj{U}) is \href{https://en.wikipedia.org/wiki/Orthonormality}{orthonormal}:
\begin{align*}
    \Conj{a}a + \Conj{c}c = & \Prob{a} + \Prob{c} = 1 \\
    \Conj{b}b + \Conj{d}d = & \Prob{b} + \Prob{d} = 1 
\end{align*}
\item<2-> $\Implies{\Prob{a} + \Prob{c} = 1}{\Mag{a} \leq 1}$
\end{itemize}
\visible<5>{%
\alert<5>{%
We use $\theta/2$ to accommodate $0\leq \theta < 2\pi$}}
}{%
\Vskip{-5em}\begin{center}
\begin{TIKZP}[scale=0.45]
\UnitComplexCircle{}
\end{TIKZP}
\end{center}
\visible<3->{%
Because $\Mag{a}\leq 1$, $a$ is some point on or in the unit
complex circle.
\begin{itemize}
    \item<4-> Let $\ExpPhase{\phi_{a}}$ be a point on the circle.
    \item<5-> We can then scale it by \alert<5>{$\cos(\theta/2)$} to obtain any point on or in the circle.
    \item<6-> We thus obtain
    \[
    a = \ExpPhase{\phi_{a}}\cos(\theta/2)
    \]
\end{itemize}}
}
\end{frame}

\begin{frame}{Proof (continued)}{The other entries follow from the requirement that each row and column is \href{https://en.wikipedia.org/wiki/Unit_vector}{normal}.}
\[
    U=\SQBG{\relax}{a}{b}{c}{d} = 
    \SQBG{\relax}{\ExpPhase{\phi_{a}}\cos(\theta/2)}{\visible<4->{\ExpPhase{\phi_{b}}\sin(\theta/2)}}{\visible<6->{\ExpPhase{\phi_{c}}\sin(\theta/2)}}{\visible<7->{\ExpPhase{\phi_{d}}\cos(\theta/2)}}
    \]
\TwoColumns{%
\begin{itemize}
    \item<2-> Each row of $U$ is orthonormal, so 
    $\Prob{a}+\Prob{b}=1$
    \item<3-> $\Prob{b}=\sin^{2}(\theta/2)$
    \item<4-> We can let $b=\ExpPhase{\phi_{b}}\sin(\theta/2)$
    \item<5-> $\Prob{a}+\Prob{c}=1$
    \item<6-> We can let $c=\ExpPhase{\phi_{c}}\sin(\theta/2)$
    \item<7-> Same for $d$
\end{itemize}
}{%
\begin{itemize}
    \item<3-> $\Mag{\ExpPhase{\theta}}=1$ for all $\theta$
    \item<4-> Yields the correct magnitude with an arbitrary phase $\phi_{b}$
    \item<5-> $c$ and $d$ follow in the same way
    \item<7-> We now have 5 parameters instead of 8
\end{itemize}
}
\end{frame}

\begin{frame}{Proof (continued)}{We can simplify by applying orthogonality of $U$'s columns.}
\Vskip{-3em}\[
    U=\SQBG{\relax}{a}{b}{c}{d} = 
    \SQBG{\relax}{\textcolor{OrangeRed}{\ExpPhase{\phi_{a}}\cos(\theta/2)}}{\textcolor{NavyBlue}{\ExpPhase{\phi_{b}}\sin(\theta/2)}}{\textcolor{OliveGreen}{\ExpPhase{\phi_{c}}\sin(\theta/2)}}{\textcolor{Brown}{\ExpPhase{\phi_{d}}\cos(\theta/2)}}
    \]
    
\TwoColumns{%
\begin{itemize}
    \item<1-> The columns of $U$ are orthogonal
    \item<2-> $\CQB{\Conj{\textcolor{OrangeRed}{a}}}{\Conj{\textcolor{OliveGreen}{c}}} \SQB{\textcolor{NavyBlue}{b}}{\textcolor{Brown}{d}}=\Conj{\textcolor{OrangeRed}{a}}\textcolor{NavyBlue}{b} + \Conj{\textcolor{OliveGreen}{c}}\textcolor{Brown}{d}=0$
\end{itemize}
}{%
\begin{itemize}
        \item<3-> Let's plug in the matrix values
        \item<4-> Recall the conjugate of $\ExpPhase{x}$ is $\ExpNegPhase{x}$
\end{itemize}
}
\visible<4->{%
\MedSkip{}\[
\textcolor{OrangeRed}{\ExpNegPhase{\phi_a}\cos(\theta/2)}\,\textcolor{NavyBlue}{\ExpPhase{\phi_b}\sin(\theta/2)}
+
\textcolor{OliveGreen}{\ExpNegPhase{\phi_c}\cos(\theta/2)}\,\textcolor{Brown}{\ExpPhase{\phi_d}\sin(\theta/2)}
= 0 \]}%
\visible<5->{\[
\cos(\theta/2)\sin(\theta/2)\left[
\ExpPhase{(\phi_{b}-\phi_{a})} +
\ExpPhase{(\phi_{d}-\phi_{c})} \right]=0
\]}%
\visible<6->{%
The value of $\theta$ must be arbitrary, so we can only get $0$ when
\[
\ExpPhase{(\phi_{b}-\phi_{a})} = - \ExpPhase{(\phi_{d}-\phi_{c})}
\]}
\end{frame}
\begin{frame}{Proof (continued)}{We can now relate the phases.}

\Vskip{-3em}\[
    U=\SQBG{\relax}{a}{b}{c}{d} = 
    \SQBG{\relax}{\textcolor{OrangeRed}{\ExpPhase{\phi_{a}}\cos(\theta/2)}}{\textcolor{NavyBlue}{\ExpPhase{\phi_{b}}\sin(\theta/2)}}{\textcolor{OliveGreen}{\ExpPhase{\phi_{c}}\sin(\theta/2)}}{\textcolor{Brown}{\ExpPhase{\phi_{d}}\cos(\theta/2)}}
    \]
    
\begin{align*}
\ExpPhase{(\phi_{b}-\phi_{a})} &= - \ExpPhase{(\phi_{d}-\phi_{c})} \\
 & = \ExpPhase{\pi}\,\ExpPhase{(\phi_{d}-\phi_{c})} \\
 & = \ExpPhase{(\phi_{d}-\phi_{c}+\pi)} \\
 \textcolor{Brown}{\phi_{d}} &= \phi_{b}+\phi_{c}-\phi_{a} -\pi \\
 &= \phi_{b}+\phi_{c}-\phi_{a} -\pi + 2\pi \\
 &= (\phi_{b}+\pi) + \phi_{c}-\phi_{a}
\end{align*}
We next eliminate $\phi_{d}$ in favor of the other phases.
\end{frame}

\begin{frame}{Proof (continued)}{Now 4 parameters, but soon only 3}
\Vskip{-3em}\[
    U=\SQBG{\relax}{a}{b}{c}{d} = 
    \SQBG{\relax}{\textcolor{OrangeRed}{\ExpPhase{\phi_{a}}\cos(\theta/2)}}{\textcolor{NavyBlue}{\ExpPhase{\phi_{b}}\sin(\theta/2)}}{\textcolor{OliveGreen}{\ExpPhase{\phi_{c}}\sin(\theta/2)}}{\textcolor{Brown}{\ExpPhase{\phi_{d}}\cos(\theta/2)}}
    \]
    \[
    \phi_{d} = (\phi_{b}+\pi) + \phi_{c}-\phi_{a}
    \]
    \begin{align*}
    \visible<2->{
    U&= 
\SQBG{%
    \phantom{\ExpPhase{\phi_a}}}{%
    \textcolor{Black}{\ExpPhase{\phi_{a}}\cos(\theta/2)}}{%
    \textcolor{Black}{\ExpPhase{\phi_{b}}\sin(\theta/2)}}{%
    \textcolor{Black}{\ExpPhase{\phi_{c}}\sin(\theta/2)}}{%
    \textcolor{Black}{\ExpPhase{((\phi_{b}+\pi) + \phi_{c}-\phi_{a})}\cos(\theta/2)}} \\[1.5em]}
    \visible<3->{
    &\only<3-4>{=}\only<5->{\equiv} \SQBG{%
    \visible<3-4>{\ExpPhase{\phi_a}}}{%
    \textcolor{Black}{\cos(\theta/2)}}{%
    \textcolor{Black}{\ExpPhase{(\phi_{b}-\phi_{a})}\sin(\theta/2)}}{%
    \textcolor{Black}{\ExpPhase{(\phi_{c}-\phi_{a})}\sin(\theta/2)}}{%
    \textcolor{Black}{\ExpPhase{((\phi_{b}+\pi) + \phi_{c}-2\phi_{a})}\cos(\theta/2)}}}
    \end{align*}
\visible<4->{
The coefficient $\ExpPhase{\phi_{a}}$ is a \emph{global} phase and can therefore be \visible<5->{ignored.}}
    
\end{frame}
\begin{frame}{Conclusion of Proof}{We now have a general unitary matrix specified using only three parameters.}
\def\Phii{\textcolor{OrangeRed}{\phi}}
\def\Lambdaa{\textcolor{NavyBlue}{\lambda}}
\[
U = \SQBG{%
    \relax}{%
    \textcolor{Black}{\cos(\theta/2)}}{%
    \ExpPhase{(\textcolor<2->{NavyBlue}{\phi_{b}-\phi_{a}-\pi}+\pi)}\sin(\theta/2)}{%
    \ExpPhase{\textcolor<2->{OrangeRed}{(\phi_{c}-\phi_{a})}}\sin(\theta/2)}{%
    \textcolor{Black}{\ExpPhase{((\phi_{b}+\pi) + \phi_{c}-2\phi_{a})}\cos(\theta/2)}}
    \]
\TwoColumns{%

\Vskip{-3.5em}\begin{align*}
     \visible<2->{\mbox{Let } \Phii &= \textcolor<2>{OrangeRed}{\phi_{c}-\phi_{a}} \\
     \mbox{Let } \Lambdaa &= \textcolor{NavyBlue}{\phi_{b}-\phi_{a}-\pi}}
\end{align*}
\Vskip{-3em}
\only<5>{%
\begin{align*}
     \visible<5->{%
     b' &= \ExpPhase{(\Lambdaa+\pi)}\sin(\theta/2) \\
     &= \ExpPhase{\pi}\ExpPhase{\Lambdaa}\sin(\theta/2) \\
     &= -\ExpPhase{\Lambdaa}\sin(\theta/2)}
\end{align*}}
\only<6->{%
\begin{align*}
    d' &= \ExpPhase{(\phi_{b}+\pi+\phi_{c}-2\phi_{a})}\cos(\theta/2) \\
       &= \ExpPhase{(\Lambdaa+\Phii+2\pi)}\cos(\theta/2) \\
       &= \ExpPhase{(\Lambdaa+\Phii)}\cos(\theta/2)
\end{align*}
}
}{%
\visible<3->{%
\[
U = \SQBG{}{%
\only<3>{a'}%
\only<4->{\cos(\theta/2)}}{%
\only<3-4>{b'}%
\only<5->{-\ExpPhase{\Lambdaa}\sin(\theta/2)}}{%
\only<3>{c'}
\only<4->{\ExpPhase{\Phii}\sin(\theta/2)}}{%
\only<3-5>{d'}%
\only<6->{\ExpPhase{(\Lambdaa+\Phii)}\cos(\theta/2)}}
\]}
\only<4-5>{
\begin{align*}
a' &= \cos(\theta/2) \\
    c' &= \ExpPhase{\Phii}\sin(\theta/2)
\end{align*}}%
\only<7->{%

We need just these three parameters to determine a quantum gate: $\theta, \Phii, \Lambdaa$. \QED{}
}
}

    
\end{frame}

\begin{frame}{Importance of this result}{The qiskit \NamedGate{U} gate}
\begin{theorem}
   Up to a global phase, any $2\times 2$ unitary matrix can be expressed as
    \[
    U(\theta, \phi, \lambda) = 
    \SQBG{\relax}{\cos(\theta/2)}{-\ExpPhase{\lambda}\sin(\theta/2)}{\ExpPhase{\phi}\sin(\theta/2)}{\ExpPhase{(\phi+\lambda)}\cos(\theta/2)}
    \] where
    $0 \leq \theta,\phi,\lambda < 2\pi$
    \end{theorem}
\begin{itemize}[<+->]
   \item A qubit has a given probability of measuring \QZero{} or \QOne{}, based on its amplitude.   The \NamedGate{U} gate affects such measurements of that qubit using the parameter~$\theta$. 
   \item The qiskit environment offers the \href{https://qiskit.org/documentation/stubs/qiskit.circuit.library.UGate.html}{\NamedGate{U}} gate, which is compiled into the necessary actions on a quantum device to realize the gate.
\end{itemize}


    
\end{frame}

\begin{frame}{Some theoretical results}{From \Kaye{}}
\end{frame}

\begin{frame}{Universality via Pauli gates}
    
\end{frame}