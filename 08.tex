\SetTitle{8}{Universal quantum gates}{Several approaches, culminating in the qiskit U gate}{08}

\begin{frame}{Universality}{Some initial thoughts}

\begin{itemize}[<+->]
    \item We captured the essence of a single qubit using two real rotational parameters $\phi$ and $\theta$.  Those parameters describe points on the unit sphere, each point corresponding to a quantum state.
    \item A quantum gate is a mapping from states to states.
    \item Because such gates must have inverses, the mapping established by such gates must be \href{https://en.wikipedia.org/wiki/Bijection}{bijective}.
    \item We can think of a gate as a permutation of states, or points, on the Bloch sphere.
    \item How many parameters does it take to characterize an arbitrary single-qubit quantum gate?
\end{itemize}

    
\end{frame}

\begin{frame}{Quantum gate}{Properties}

\TwoUnequalColumns{0.4\textwidth}{0.6\textwidth}{%
\begin{center}
\SQBG{\relax}{a}{c}{b}{d}
\end{center}
}{%

\begin{itemize}[<+->]
    \item $a, b, c, d$ are complex
    \item So, apparently 8 parameters
    \item $U$ is invertible and $\Inverse{U}=\Conj{U}$
    \item The columns form an orthonormal basis
    \item The rows form an orthonormal basis
\end{itemize}}
\BigSkip{}
\visible<+->{%
These constraints will take us from 8 parameters to only 3.}
    
\end{frame}



\begin{frame}{Some theoretical results}{From \Kaye{}}
\end{frame}

\begin{frame}{Universality via Pauli gates}
    
\end{frame}

\begin{frame}{The Qiskit U gate}{Theorem statement}
\begin{theorem}
   Up to a global phase, any $2\times 2$ unitary matrix can be expressed as
    \[
    U(\theta, \phi, \lambda) = 
    \SQBG{\relax}{\cos(\theta/2)}{-\ExpPhase{\lambda}\sin(\theta/2)}{\ExpPhase{\phi}\sin(\theta/2)}{\ExpPhase{(\phi+\lambda)}\cos(\theta/2)}
    \] where
    $0 \leq \theta,\phi,\lambda < 2\pi$
    \end{theorem}
    For example, the Hadamard gate \HMatrix{} is $U(\pi/2, 0, \pi)$, which yields
    \[
    \SQBG{\relax}{\cos \pi/4}{-\ExpPhase{\pi}\sin \pi/4}{\ExpPhase{0}\sin \pi/4}{\ExpPhase{\pi}\cos pi/4} = \HMatrix{}
    \]

\end{frame}

\begin{frame}{Proof}{We begin with an arbitrary unitary matrix $U$ with complex entries.}
\TwoColumns{%
$U = \SQBG{\relax}{a}{b}{c}{d}$ so
$\Conj{U} = \SQBG{\relax}{\Conj{a}}{\Conj{c}}{\Conj{b}}{\Conj{d}}$

\MedSkip{}
Because $U$ is unitary:
\begin{itemize}
\item<1-> Each column of $U$ (row of \Conj{U}) is \href{https://en.wikipedia.org/wiki/Orthonormality}{orthonormal}:
\begin{align*}
    \Conj{a}a + \Conj{c}c = & \Prob{a} + \Prob{c} = 1 \\
    \Conj{b}b + \Conj{d}d = & \Prob{b} + \Prob{d} = 1 
\end{align*}
\item<2-> $\Implies{\Prob{a} + \Prob{c} = 1}{\Mag{a} \leq 1}$
\end{itemize}
\visible<5>{%
\alert<5>{%
We use $\theta/2$ to accommodate $0\leq \theta < 2\pi$}}
}{%
\Vskip{-5em}\begin{center}
\begin{TIKZP}[scale=0.45]
\UnitComplexCircle{}
\end{TIKZP}
\end{center}
\visible<3->{%
Because $\Mag{a}\leq 1$, $a$ is some point on or in the unit
complex circle.
\begin{itemize}
    \item<4-> Let $\ExpPhase{\phi_{a}}$ be a point on the circle.
    \item<5-> We can then scale it by \alert<5>{$\cos(\theta/2)$} to obtain any point on or in the circle.
    \item<6-> We thus obtain
    \[
    a = \ExpPhase{\phi_{a}}\cos(\theta/2)
    \]
\end{itemize}}
}
\end{frame}

\begin{frame}{Proof (continued)}{The other entries follow from the requirement that each row and column is \href{https://en.wikipedia.org/wiki/Unit_vector}{normal}.}
\[
    U=\SQBG{\relax}{a}{b}{c}{d} = 
    \SQBG{\relax}{\ExpPhase{\phi_{a}}\cos(\theta/2)}{\visible<4->{\ExpPhase{\phi_{b}}\sin(\theta/2)}}{\visible<6->{\ExpPhase{\phi_{c}}\sin(\theta/2)}}{\visible<7->{\ExpPhase{\phi_{d}}\cos(\theta/2)}}
    \]
\TwoColumns{%
\begin{itemize}
    \item<2-> Each row of $U$ is orthonormal, so 
    $\Prob{a}+\Prob{b}=1$
    \item<3-> $\Prob{b}=\sin^{2}(\theta/2)$
    \item<4-> We can let $b=\ExpPhase{\phi_{b}}\sin(\theta/2)$
    \item<5-> $\Prob{a}+\Prob{c}=1$
    \item<6-> We can let $c=\ExpPhase{\phi_{c}}\sin(\theta/2)$
    \item<7-> Same for $d$
\end{itemize}
}{%
\begin{itemize}
    \item<3-> $\Mag{\ExpPhase{\theta}}=1$ for all $\theta$
    \item<4-> Yields the correct magnitude with an arbitrary phase $\phi_{b}$
    \item<5-> $c$ and $d$ follow in the same way
    \item<7-> We now have 5 parameters instead of 8
\end{itemize}
}
\end{frame}

\begin{frame}{Proof (continued)}{We can simplify by applying orthogonality of $U$'s columns.}
\[
    U=\SQBG{\relax}{a}{b}{c}{d} = 
    \SQBG{\relax}{\ExpPhase{\phi_{a}}\cos(\theta/2)}{\ExpPhase{\phi_{b}}\sin(\theta/2)}{\ExpPhase{\phi_{c}}\sin(\theta/2)}{\ExpPhase{\phi_{d}}\cos(\theta/2)}
    \]
    
\TwoColumns{%
\begin{itemize}
    \item The columns of $U$ are orthogonal
    \item 
\end{itemize}
}{%
}
\end{frame}